\makeatletter
\let\pstdivide\pst@divide
\makeatother

\makeatletter
\def\pst@mymult#1#2#3{%
  \begingroup%
  \pst@cntm=#1\pst@cntn=#2\relax%
  \pst@cnto=\pst@cntm%
  \multiply\pst@cnto by \pst@cntn%
  \edef\value{\endgroup\def\noexpand#3{\number\pst@cnto}}\value%
}
\let\pstmymultiply\pst@mymult
\makeatother

%for error bars
\makeatletter
\SpecialCoor
\def\errorLine{\@ifnextchar[{\pst@errorLine}{\pst@errorLine[]}}
\def\pst@errorLine[#1](#2)#3#4{{%
    \ifx#1\empty\else\psset{#1}\fi
    \pst@getcoor{#2}\pst@tempb
    \def\@errorMin{#3}
    \def\@errorMax{#4}
    \psline{|-|}%
    (!%
        /yDot \pst@tempb exch pop \pst@number\psyunit div def
        /xDot \pst@tempb pop \pst@number\psxunit div def
        xDot yDot \@errorMin\space add%
    )(!%
        /yDot \pst@tempb exch pop \pst@number\psyunit div def
        /xDot \pst@tempb pop \pst@number\psxunit div def
        xDot yDot \@errorMax\space add%
    )
}}
%
\def\GetCoordinates#1{\expandafter\GetCoordinates@i#1}
\def\GetCoordinates@i #1{\GetCoordinates@ii#1}
\def\GetCoordinates@ii#1 #2 #3 #4 #5 #6 #7 #8 {%
    \DoCoordinate{#2}{#4}%
    \errorLine[linewidth=1pt](#2,#4){#6}{#8}% <<<<<
    \@ifnextchar D{\GetCoordinates@ii}{}%
}
\makeatother

\newsavebox\myboxA
\newsavebox\myboxB
\newlength\mylenA

\newcommand*\xoverline[2][0.75]{%
    \sbox{\myboxA}{$\m@th#2$}%
    \setbox\myboxB\null% Phantom box
    \ht\myboxB=\ht\myboxA%
    \dp\myboxB=\dp\myboxA%
    \wd\myboxB=#1\wd\myboxA% Scale phantom
    \sbox\myboxB{$\m@th\overline{\copy\myboxB}$}%  Overlined phantom
    \setlength\mylenA{\the\wd\myboxA}%   calc width diff
    \addtolength\mylenA{-\the\wd\myboxB}%
    \ifdim\wd\myboxB<\wd\myboxA%
       \rlap{\hskip 0.5\mylenA\usebox\myboxB}{\usebox\myboxA}%
    \else
        \hskip -0.5\mylenA\rlap{\usebox\myboxA}{\hskip 0.5\mylenA\usebox\myboxB}%
    \fi}
\makeatother



% for inserting comments on the text that are printable and easily
% removed whenever necessary

\providecommand{\abs}[1]{\lvert#1\rvert}
\providecommand{\norm}[1]{\left\lVert#1\right\rVert}

%\newcommand{\meira}[1]{{\bf MEIRA: #1}}
%\newcommand{\zaki}[1]{{\bf ZAKI: #1}}
\newcommand{\meira}[1]{}
\newcommand{\zaki}[1]{}
%\newcommand{\comment}[1]{}
\newcommand{\TimeStamp}{{\bf\color{red}{Last Modified: \timestamp}}}

% for generating draft versions and final version of the graphs
% draft version
\newcommand{\draftgraph}[1]{#1}
\newcommand{\finalgraph}[1]{}
% final version
%\newcommand{\draftgraph}[1]{}
%\newcommand{\finalgraph}[1]{#1}

%for algorithms style
\DontPrintSemicolon

% after stop using svmono, we had to define.

\newcommand{\sectionstar}[1]{\section{$\star$ #1}}
\newcommand{\subsectionstar}[1]{\subsection{$\star$ #1}}
\newcommand{\subsubsectionstar}[1]{\subsubsection{$\star$ #1}}
\newcommand{\paragraphstar}[1]{\paragraph{$\star$ #1}}


%\newtheorem{theorem}{Theorem}[chapter]
%\newtheorem{lemma}[theorem]{Lemma}
%\newtheorem{proposition}[theorem]{Proposition}
%\newtheorem{corollary}[theorem]{Corollary}
%\newtheorem{fact}[theorem]{Fact}

%\newenvironment{proof}[1][Proof]{\begin{trivlist}
%  \item[\hskip \labelsep {\bfseries #1}]}{\end{trivlist}}
%\newenvironment{definition}[1][Definition]{\begin{trivlist}
%  \item[\hskip \labelsep {\bfseries #1}]}{\end{trivlist}}
%\newenvironment{remark}[1][Remark]{\begin{trivlist}
%  \item[\hskip \labelsep {\bfseries #1}]}{\end{trivlist}}

%\newcommand{\qed}{\nobreak \ifvmode \relax \else
%  \ifdim\lastskip<1.5em \hskip-\lastskip
%  \hskip1.5em plus0em minus0.5em \fi \nobreak
%  \vrule height0.75em width0.5em depth0.25em\fi}
%\newcommand{\qed}{\hfill \mbox{\raggedright \rule{.07in}{.1in}}}

\newcommand{\download}{{\LARGE \ComputerMouse}}
%\newcommand{\download}{\MVAt}

%%EXAMPLE%%

%\theoremstyle{plain}
%\theoremseparator{.}
%\theoremsymbol{}
%\theorembodyfont{\upshape}
%
%%%\def\exf{\fontfamily{opl}\selectfont\fontseries{b}\fontshape{n}\fontsize{10}{10}\selectfont}

%\newtheorem{myexample}{Example}[chapter]
%\theoreminframepreskip{0pt}
%\theoreminframepostskip{0pt}
%\def\theoremframecommand{%
%    \psframebox[fillstyle=solid,fillcolor=gray!10,linecolor=gray!10]}
%\newshadedtheorem{example}[myexample]{Example}
%
%\theoremstyle{nonumberplain}
%\theoremseparator{}
%\theoreminframepreskip{0pt}
%\theoreminframepostskip{0pt}
%\def\theoremframecommand{%
%    \psframebox[fillstyle=solid,fillcolor=gray!10,linecolor=gray!10]}
%\newshadedtheorem{conexample}{}


%%%\makeatletter
%%%\newtheoremstyle{common}
%%%    {12pt plus 3\p@}% above space (default)
%%%    {12pt plus 3\p@}% below space
%%%    {\rm}% body
%%%    {0em}% indent
%%%    {\bfseries}% head
%%%    {}% punct
%%%    {0.5em}% space
%%%    {}% custom
%%%\theoremstyle{common}
%%%
%%%\makeatother
%%%
%%%\newtheorem{example}[myexample]{Example}

%\newcounter{examp}
%\setcounter{examp}{0}
%\renewcommand{\theexamp}{\thechapter.\arabic{examp}}

% \def\exf{\fontfamily{opl}\fontseries{b}\fontshape{n}\selectfont\fontsize{10}{10}\selectfont}

%\makeatletter
%\newtheoremstyle{tintthit}
%    {12pt plus 3\p@}% above space (default)
%    {12pt plus 3\p@}% below space
%    {\rm}% body
%    {0em}% indent
%    {\bfseries}% head
%    {}% punct
%    {0.5em}% space
%    {}% custom
%\theoremstyle{tintthit}
%
%\newtheorem{ohtheorem}{Example}[chapter]%
%
%\newenvironment{example}{\vskip 12pt plus 3pt minus 1pt\fboxsep10pt\fboxrule0.5pt\noindent\begin{Sbox}\begin{minipage}{366pt}\begin{ohtheorem}}{\end{ohtheorem}\end{minipage}\end{Sbox}\fboxrule\z@\fboxsep3\p@\fcolorbox{ctinta}{ctinta}{\TheSbox}\vskip 12pt plus 3pt minus 1pt}
%
%\makeatother



%\newtheoremstyle{example_contd}
%{\topsep} {\topsep}%
%{\upshape}% Body font
%{}% Indent amount (empty = no indent, \parindent = para indent)
%{\bfseries\scshape}% Thm head font
%{}% Punctuation after thm head
%{}
%{\thmname{#1}\thmnumber{ #2}\thmnote{#3}\enspace(continued)}% Thm head spec
%
%\theoremstyle{example_contd}
%\newtheorem{example}[myexample]{Example}
%\newtheorem*{example_contd}{Example}

%%%% EDA %%%%%%%%%%%%%
\newcommand{\dsum}{\displaystyle \sum}

\newcommand{\setR}{{\mathord{\mathbb R}}}
\newcommand{\setD}{{\mathord{\mathbb D}}}
\newcommand{\setI}{{\mathord{\mathbb I}}}
\newcommand{\setN}{{\mathord{\mathbb N}}}
\newcommand{\area}{{\mathord{\mathbb A}}}
%\newcommand{\vol}{{\mathord{\mathbb V}}}
\def\vol{\qopname\relax{no}{vol}}
\def\area{\qopname\relax{no}{area}}

\DeclareMathOperator{\erf}{erf}
\DeclareMathOperator{\diag}{diag}
\DeclareMathOperator{\avg}{avg}
\DeclareMathOperator{\sign}{sign}
%\newcommand{\avg}{\operatorname{avg}}
%\newcommand{\vect}[1]{\mathbold{#1}}
% \newcommand{\vect}[1]{\mathbf{#1}}
\newcommand{\vect}[1]{{\boldsymbol #1}}


\newcommand{\pvalue}{{p\text{-}value}}
\newcommand{\vi}{{\vec{i}}}
\newcommand{\ba}{\vect{a}}
\newcommand{\bb}{\vect{b}}
\newcommand{\bc}{\vect{c}}
\newcommand{\bd}{\vect{d}}
\newcommand{\be}{\vect{e}}
\newcommand{\bff}{\vect{f}}
\newcommand{\bg}{\vect{g}}
\newcommand{\bh}{\vect{h}}
\newcommand{\bi}{\vect{i}}
\newcommand{\bl}{\boldsymbol l}
%%%\newcommand{\bl}{\vect{l}}
\newcommand{\bmm}{\vect{m}}
\newcommand{\bn}{\vect{n}}
\newcommand{\bo}{\vect{o}}
\newcommand{\bp}{\vect{p}}
\newcommand{\bq}{\vect{q}}
\newcommand{\br}{\vect{r}}
\newcommand{\bs}{\vect{s}}
\newcommand{\bt}{\vect{t}}
\newcommand{\bu}{\vect{u}}
\newcommand{\bv}{\vect{v}}
\newcommand{\bw}{\vect{w}}
\newcommand{\bx}{\vect{x}}
\newcommand{\by}{\vect{y}}
\newcommand{\bz}{\vect{z}}

\newcommand{\bA}{\vect{A}}
\newcommand{\bB}{\vect{B}}
\newcommand{\bC}{\vect{C}}
\newcommand{\bD}{\vect{D}}
\newcommand{\bE}{\vect{E}}
\newcommand{\bH}{\vect{H}}
\newcommand{\bI}{\vect{I}}
\newcommand{\bK}{\vect{K}}
\newcommand{\bL}{\vect{L}}
\newcommand{\bM}{\vect{M}}
\newcommand{\bN}{\vect{N}}
\newcommand{\bO}{\vect{O}}
\newcommand{\bP}{\vect{P}}
\newcommand{\bQ}{\vect{Q}}
\newcommand{\bR}{\vect{R}}
\newcommand{\bS}{\vect{S}}
\newcommand{\bT}{\vect{T}}
\newcommand{\bU}{\vect{U}}
\newcommand{\bV}{\vect{V}}
\newcommand{\bW}{\vect{W}}
\newcommand{\bX}{\vect{X}}
\newcommand{\bY}{\vect{Y}}
\newcommand{\bZ}{\vect{Z}}

\newcommand{\bone}{\mathbf{1}}
\newcommand{\bzero}{\mathbf{0}}

\newcommand{\cov}{{\bm \Sigma}}
\newcommand{\bLambda}{{\bm \Lambda}}
\newcommand{\bDelta}{{\bm \Delta}}
\newcommand{\balpha}{{\bm \alpha}}
\newcommand{\bbeta}{{\bm \beta}}
\newcommand{\bdelta}{{\bm \delta}}
\newcommand{\bomega}{{\bm \omega}}
\newcommand{\bpi}{{\bm \pi}}
\newcommand{\bsigma}{{\bm \sigma}}
\newcommand{\btheta}{{\bm \theta}}
\newcommand{\bmu}{{\bm \mu}}
\newcommand{\bepsilon}{{\bm \epsilon}}
\newcommand{\lB}{\left}
\newcommand{\rB}{\right}
\newcommand{\marr}[1]{\begin{array}{c} #1 \end{array}}
\newcommand{\matr}[1]{\begin{pmatrix} #1 \end{pmatrix}}
%alignedmatr uses mathtools aligned pmatrix
\newcommand{\amatr}[2]{%
    \begin{pmatrix*}[#1] #2 \end{pmatrix*}}

\newcommand{\hf}{\kern3pt\hat{\kern-3pt f}}
%%%\newcommand{\hf}{\hat{f}}
\newcommand{\hy}{\hat{y}}
\newcommand{\hF}{\hat{F}}
\newcommand{\hp}{\hat{p}}
\newcommand{\hbp}{\hat{\bp}}
\newcommand{\hP}{\hat{P}}
\newcommand{\hbM}{\widehat{\bM}}
\newcommand{\hbP}{\widehat{\bP}}
\newcommand{\htheta}{\hat{\theta}}
\newcommand{\hmu}{\hat{\mu}}
\newcommand{\hbmu}{\hat{\bmu}}
\newcommand{\hsigma}{\hat{\sigma}}
\newcommand{\hbsigma}{\hat{\bsigma}}
\newcommand{\hcov}{\widehat{\cov}}
\newcommand{\hrho}{\hat{\rho}}

%2nd edition
\newcommand{\truew}{\omega}
\newcommand{\truebw}{\bomega}
\newcommand{\trueb}{\beta}



\newcommand{\hrul}{\rule[1pt]{0.3in}{0.5pt}}
%%%%%%%%%%



%%%% FPM %%%%%%%%%%%%
\newcommand{\cA}{{\mathcal A}}
\newcommand{\cB}{{\mathcal B}}
\newcommand{\cC}{{\mathcal C}}
\newcommand{\cD}{{\mathcal D}}
\newcommand{\cE}{{\mathcal E}}
\newcommand{\cF}{{\mathcal F}}
\newcommand{\cG}{{\mathcal G}}
\newcommand{\cI}{{\mathcal I}}
\newcommand{\cL}{{\mathcal L}}
\newcommand{\cK}{{\mathcal K}}
\newcommand{\cM}{{\mathcal M}}
\newcommand{\cN}{{\mathcal N}}
\newcommand{\cO}{{\mathcal O}}
\newcommand{\cP}{{\mathcal P}}
\newcommand{\cQ}{{\mathcal Q}}
\newcommand{\cR}{{\mathcal R}}
\newcommand{\cS}{{\mathcal S}}
\newcommand{\cT}{{\mathcal T}}
\newcommand{\cX}{{\mathcal X}}
\newcommand{\bcX}{\vect{\cX}}
\newcommand{\cY}{{\mathcal Y}}
\newcommand{\cZ}{{\mathcal Z}}
\newcommand{\vI}{{\vect{i}}}
\newcommand{\vT}{{\vect{t}}}
\newcommand{\vc}{{\vect{c}}}
\newcommand{\Ik}{\cI^{(k)}}
\newcommand{\Fk}[1]{\cF^{(#1)}}
\newcommand{\Ck}[1]{\cC^{(#1)}}
\newcommand{\pow}[1]{2^{#1}}
\newcommand{\tup}[1]{\langle #1 \rangle}
\newcommand{\ltup}[2]{{#1\langle #2 #1\rangle}}
\newcommand{\card}[1]{|#1|}
\newcommand{\bcard}[2]{#1| #2 #1|}
\newcommand{\join}{{\times}}
\newcommand{\assign}{\leftarrow}
\newcommand{\ol}{\overline}
\newcommand{\supp}{\mathit{sup}}%%%
\newcommand{\rsupp}{\mathit{rsup}}%%%
\newcommand{\minsup}{\mathit{minsup}}%%%%
\newcommand{\conff}{\mathit{conf}}
\newcommand{\minconf}{\mathit{minconf}}

\NewDocumentCommand{\arule}{G{}}{
  \IfNoValueTF{#1}
    {\rightarrow}
    {\stackrel{#1}{\longrightarrow}}%
}


% Stuff for Algorithm2e Styles
\setlength{\algomargin}{2em}
%\SetKwInOut{Input}{Input}
%\SetKwInOut{Output}{Output}
%\SetKwInOut{Initial}{Initialization}


%SEQUENCES
\newcommand{\estring}{\epsilon}

%CLUSTERING
\newcommand{\dist}{\delta}
\newcommand{\grad}{\nabla}
\newcommand{\minpts}{minpts}


\newcommand{\domain}{domain}

\renewcommand{\approx}{\simeq}

\newcommand{\Iff}{{\it iff}}

\newcommand{\hY}{\widehatL{Y}}

\newcommand{\widehatL}[1]{\mkern 1.5mu\widehat{\mkern-1.5mu#1}\mkern 1.5mu}
\newcommand{\mX}{\overbarL{X}} %use for centered X variable
\newcommand{\overbarL}[1]{\mkern 1.5mu\overline{\mkern-4.0mu#1}\mkern 1.5mu}
\DeclareMathOperator{\proj}{proj}
\newcommand{\abD}{\tilde{\bD}}
\newcommand{\abw}{\tilde{\bw}}


\makeatletter
%adds minipage to inside of algorithms
  \newlength{\tightalgowidth}
  \newlength{\tightalgoremainder}

  % uncomment to use centered floating algorithms
  \newenvironment{tightalgo}[2][]
  {
     \setlength{\tightalgowidth}{#2}
     \setlength{\tightalgoremainder}{3pc}
     \begin{algorithm}[#1] }
  { \end{algorithm} }

  \makeatletter
  \patchcmd{\@algocf@start}{%
    \begin{lrbox}{\algocf@algobox}%
  }{%
    \rule{0.5\tightalgoremainder}{\z@}%
    \begin{lrbox}{\algocf@algobox}%
    \begin{minipage}{\tightalgowidth}%
  }{}{}
  \patchcmd{\@algocf@finish}{%
    \end{lrbox}%
  }{%
    \end{minipage}%
    \end{lrbox}%
  }{}{}
  \makeatother
\SetAlCapFnt{\sansb}
\makeatother

\newcommand{\abx}{\tilde{\bx}}
\newcommand{\abK}{\tilde{\bK}}

\DeclareMathOperator{\zscore}{z\text{-}score}
\DeclareMathOperator{\mean}{mean}
\newcommand{\mbD}{\overbarR{\bD}} %use for centered bD variable
\newcommand{\overbarR}[1]{\mkern 1.5mu\overline{#1\mkern-4.0mu}\mkern 1.5mu}
\newcommand{\mY}{\overbarL{Y}} %use for centered Y variable
\DeclareMathOperator{\ocov}{cov}
\DeclareMathOperator{\var}{var}
\DeclareMathOperator{\vspan}{span}

\newcommand{\aphi}{\tilde{\phi}}
\newcommand{\aK}{\tilde{K}}
\newcommand{\mx}{\overbarLL{x}{2.0}}
\newcommand{\my}{\overbarLL{y}{2.0}}
\newcommand{\overbarLL}[2]{\mkern 1.5mu\overline{\mkern-#2mu#1}\mkern 1.5mu}

\newcommand{\net}{\mathit{net}}%%%
\newcommand{\bnet}{\mathbf{net}}%%%

\newcommand{\myfbox}[2][linecolor=black]{%
    \psframebox[#1]{%
    \begin{tabular}{c}
        ~\\#2\\~\\
    \end{tabular}
}}

\newcommand{\myfboxB}[2][linecolor=black]{%
    \psframebox[#1]{%
    \begin{tabular}{c}
        #2
    \end{tabular}
}}
\newcommand{\myfboxD}[3][linecolor=black]{%
    \begin{tabular}{c}
    \psframebox[#1]{%
    \begin{tabular}{c}
        #2
    \end{tabular}
    }\\
    \psframebox[#1]{%
    \begin{tabular}{c}
        #3
    \end{tabular}
    }
    \end{tabular}
}

\def\mytext#1{%
    \begin{tabular}{c} #1 \end{tabular}%
}

\newcommand{\abX}{\tilde{\bX}}
\newcommand{\trueabw}{\tilde{\bomega}}
\DeclareMathOperator{\odds}{odds}
\newcommand{\abz}{\tilde{\bz}}
\newcommand{\abW}{\tilde{\bW}}
\DeclareMathOperator*{\argmax}{arg\;max}

\newcommand{\bgrad}{{\bm \grad}}
\newcommand{\bcE}{{\bm{\mathcal E}}}
\newcommand{\bF}{\vect{F}}
\newcommand{\bphi}{{\bm \phi}}
\newcommand{\bkappa}{{\bm \kappa}}

\DeclareMathOperator*{\summ}{sum} %operator with limits
\DeclarePairedDelimiter\floor{\lfloor}{\rfloor}
\newcommand{\hby}{\hat{\by}}
\newcommand{\bTheta}{{\bm \Theta}}
\newcommand{\btz}{\tilde{\bz}}
\DeclareMathOperator{\se}{se}
%\newcommand{\mbx}{\xoverline{\bx}} %use for centered bx variable
\newcommand{\mbx}{\overline{\bx}} %use for centered bx variable
\newcommand{\hmY}{\widehat{\mY}}
\newcommand{\mbX}{\xoverline{\bX}}
\newcommand{\bvepsilon}{{\bm \varepsilon}}

%%% Local Variables:
%%% mode: latex
%%% TeX-master: t
%%% End:
