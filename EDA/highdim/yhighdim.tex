\lecture{highdim}{highdim}

\date{Chapter 6: High-dimensional Data}

\begin{frame}
\titlepage
\end{frame}


\begin{frame}{High-dimensional Space}
\small

Let $\bD$ be a $n\times d$ data matrix. 
In data mining typically the data is very high dimensional.
Understanding the nature of high-dimensional space, or
{\em hyperspace}, is very important, especially because it
does not behave like the more familiar geometry in two or three
dimensions.


\medskip{\bf Hyper-rectangle:}
 The data space is a 
$d$-dimensional
{\em hyper-rectangle}
\index{hyper-rectangle}
\begin{align*}
    R_d & = \prod_{j=1}^{d} \Bigl[\min(X_{j}), \max(X_{j})\Bigr]
\end{align*}
where $\min(X_j)$ and $max(X_j)$ specify the range of $X_j$.

\medskip{\bf Hypercube:}
Assume the data is centered, and let $m$ 
denote the maximum attribute value
\begin{align*}
    m = \max_{j=1}^d \max_{i=1}^n \Bigl\{ \abs{x_{ij}} \Bigr\}
\end{align*}

The data hyperspace can be represented as a {\em hypercube}, centered
at $\bzero$, with all sides of length $l=2m$, given as
\begin{align*}
    H_d(l) = \Bigl\{\bx = (x_1, x_2, \ldots, x_d)^T \; \big|
    \; \forall i,\;
	x_i \in [-l/2, l/2] \Bigr\}
\end{align*}
The {\em unit hypercube} has all sides
of length $l=1$, and is denoted as $H_d(1)$.
\end{frame}


\begin{frame}{Hypersphere}
Assume that the data has been centered, so that
$\bmu = \bzero$.
Let $r$ denote the largest magnitude among all points:
\begin{align*}
    r = \max_i \Bigl\{ \norm{\bx_i} \Bigr\}
\end{align*}

The data hyperspace can be represented as a
$d$-dimensional {\em hyperball}
centered at $\bzero$ with radius $r$, def\/{i}ned as
\begin{align*}
  B_d(r) & = \bigl\{ \bx\; |\;\; \norm{\bx} \le r\bigr\}\\
  \text{or } B_d(r) & = \lB\{
    \bx = (x_1, x_2, \ldots, x_d) \; \big| \;
    \sum_{j=1}^{d} x_{j}^2 \le r^2
    \rB\}
\end{align*}

The surface of the hyperball is called a {\em hypersphere},
and it consists of all the points exactly at distance $r$ from
the center of the hyperball
\begin{align*}
  S_d(r) & = \bigl\{ \bx\; |\;\; \norm{\bx} = r\bigr\}\\
  \text{or } S_d(r) & = \lB\{\bx = (x_1, x_2, \ldots, x_d)
  \; \big| \;
  \sum_{j=1}^{d} (x_{j})^2 = r^2\rB\}
\end{align*}
\end{frame}



\readdata{\dataSLW}{EDA/highdim/figs/iris-slw-c.dat}
\begin{frame}{Iris Data Hyperspace: Hypercube and Hypersphere}
  \framesubtitle{$\textit{l}=4.12$ and $\textit{r}=2.19$}
\centering
\scalebox{0.8}{
\psset{dotstyle=Bo,dotscale=1.5,fillcolor=lightgray,arrowscale=2}
\psset{xAxisLabel=$X_1$: sepal length,yAxisLabel= $X_2$: sepal width,
        xAxisLabelPos={c,-0.4in},yAxisLabelPos={-0.5in,c}}
\psgraph[tickstyle=bottom,axesstyle=frame,Ox=-2,Oy=-2]%
    (-2.06,-2.06)(2.06,2.06){3in}{3in}%
\dataplot[plotstyle=dots,showpoints=true]{\dataSLW}
\psdot[dotstyle=*,dotscale=2](0,0)
\pscircle[linestyle=dashed]{2.188}
\pnode(0,0){O}
\pnode(2.06,0.75){A}
\ncline[arrowscale=2]{->}{O}{A}
\naput[nrot=:U]{$r$}
%\psline[](0,0)(2.06,0.75)
\endpsgraph
}
\end{frame}


\begin{frame}{High-dimensional Volumes}
%


{\bf Hypercube:}
The volume of a hypercube with edge length $l$ is given as
\begin{align*}
\vol(H_d(l)) = l^d
\end{align*}
%

\medskip
{\bf Hypersphere}%h3
The volume of a hyperball and its corresponding hypersphere
is identical
The volume of a hypersphere is given as
\begin{align*}
  \text{In 1 dimension: } & \vol(S_1(r)) = 2r \\
  \text{In 2 dimensions: } & \vol(S_2(r)) = \pi r^2 \\
  \text{In 3 dimensions: } & \vol(S_3(r)) = \frac{4}{3}\pi r^3 \\
\text{In $d$-dimensions: } & \vol(S_d(r))=K_d r^d =
\lB(\frac{\pi^{\frac{d}{2}}}{\Gamma\lB(\frac{d}{2}+1\rB)}\rB) r^d
\end{align*}
where
\begin{align*}
\Gamma\lB(\frac{d}{2}+1\rB) =
\begin{cases}
  {\lB(\frac{d}{2}\rB)}! & \text{if $d$ is even}\\
  \sqrt{\pi}\lB(\frac{d!!}{2^{(d+1)/2}}\rB) & \text{if $d$ is odd}
\end{cases}
\end{align*}
\end{frame}


\readdata{\dataSV}{EDA/highdim/figs/spherevol.dat}
\begin{frame}{Volume of Unit Hypersphere}
  \small
With increasing dimensionality the hypersphere volume 
f\/{i}rst increases up to a point, and then
starts to decrease, and ultimately vanishes. In particular, for the
unit hypersphere with $r=1$,
\begin{align*}
\lim_{d\to\infty} \vol(S_d(1)) =
\lim_{d\to\infty}\frac{\pi^{\frac{d}{2}}}{\Gamma(\frac{d}{2}+1)} \to 0
\end{align*}
    
\normalsize
\centerline{
\scalebox{0.6}{
    \psset{dotstyle=Bo,dotscale=1.5,fillcolor=lightgray,arrowscale=2,PointName=none}
    \psset{xAxisLabel=$d$,yAxisLabel= $\vol(S_d(1))$,
        xAxisLabelPos={c,-0.4in},yAxisLabelPos={-0.4in,c}}
    \psgraph[tickstyle=bottom,Dx=5,Dy=1,subticks=2]{->}(0,0)(55,6){4in}{2.5in}%
    \dataplot[showpoints=true]{\dataSV}
\endpsgraph
}}
\end{frame}


\begin{frame}{Hypersphere Inscribed within Hypercube}

Consider the space
enclosed within the largest hypersphere that can be accommodated within
a hypercube (which represents the dataspace).  

\medskip
The ratio of the volume of the hypersphere of radius $r$ to the
hypercube with side length $l=2r$ is given as
\begin{align*}
  \text{In 2 dimensions: } &
  \frac{\vol(S_2(r))}{\vol(H_2(2r))} = \frac{\pi r^2}{4 r^2} = \frac{\pi}{4} = 78.5\%\\
  \text{In 3 dimensions: } &
\frac{\vol(S_3(r))}{\vol(H_3(2r))} = 
\frac{\frac{4}{3} \pi r^3}{8 r^3} = \frac{\pi}{6} = 52.4\%\\
\text{In $d$ dimensions: } &
\lim_{d \to \infty} \frac{\vol(S_d(r))}{\vol(H_d(2r))} = \lim_{d \to \infty}
\frac{\pi^{d/2}}{2^d \Gamma(\frac{d}{2}+1)} \to 0
\end{align*}

As the dimensionality increases, most of the volume of
the hypercube is in the ``corners,'' whereas the center is essentially
empty. 
\end{frame}


\begin{frame}{Hypersphere Inscribed inside a Hypercube}
\centerline{
%%\captionsetup[subfloat]{captionskip=5pt}
%\subfloat[]{\label{fig:eda:hd:inscribed2d}
\scalebox{0.8}{
\psset{unit=0.75in}
\pspicture(-1.5,-1.55)(1.5,1.5)
\psaxes[labels=none,tickstyle=bottom,axesstyle=frame](-1,-1)(1,1)
\pscircle[linecolor=darkgray](0,0){1}
\uput[d](-1,-1){$-r$}
\uput[d](0,-1){0}
\uput[d](1,-1){$r$}
\endpspicture
}
\hspace{0.25in}
%\subfloat[]{\label{fig:eda:hd:inscribed3d}
\scalebox{0.8}{
\psset{viewpoint=30 30 45 rtp2xyz,Decran=60,lightsrc=viewpoint}
\begin{pspicture}(-3,-3.25)(3,3.25)
\psSolid[object=sphere, hollow, linecolor=darkgray,
    opacity=0.75,r=1,incolor=white, %fillcolor=gray!25,
    ngrid=18 18]%
\psSolid[object=parallelepiped,a=2,b=2,c=2,action=draw](0,0,0)
\multido{\ix=-1+1}{3}{%
    \psPoint(\ix\space,1,-1){X1}
    \psPoint(\ix\space,1.05,-1){X2}
    \psline(X1)(X2)}
%\uput[dr](X1){\ix}
\multido{\iy=-1+1}{3}{%
    \psPoint(1,\iy\space,-1){Y1}
    \psPoint(1.05,\iy\space,-1){Y2}
    \psline(Y1)(Y2)}
%\uput[dl](Y1){\iy}
\psPoint(1,-1,-1){Y1}
\uput[dl](Y1){$-r$}
\psPoint(1,0,-1){Y1}
\uput[dl](Y1){0}
\psPoint(1,1,-1){Y1}
\uput[dl](Y1){$r$}
\multido{\iz=-1+1}{3}{%
    \psPoint(1,-1,\iz\space){Z1}
    \psPoint(1,-1.05,\iz\space){Z2}
    \psline(Z1)(Z2)}
%\uput[l](Z1){\iz}
\end{pspicture}
}
}
\end{frame}



\begin{frame}{Conceptual View of High-dimensional Space}
  \framesubtitle{Two, three, four, and higher dimensions}

  All the volume of the hyperspace is in the corners, with the center
  being essentially empty.

  \bigskip
  \centerline{High-dimensional space looks like a rolled-up porcupine!}
  

    \centerline{
	\begin{tabular}{cccc}
    \scalebox{0.5}{
    \begin{pspicture}(-2.5,-2.5)(2.5,2.5)
    % d=2 r=0.707107 R=1.000000
\psset{unit=1in,dotsep=0.5mm,PointName=none,PointSymbol=none}
\pstGeonode(0,0){O}
\pstGeonode(0,0.707){A}
\pstCircleOA[fillstyle=solid,fillcolor=lightgray]{O}{A}
\pstGeonode(1.000,0.000){R0}
\pstGeonode(0.000,1.000){R1}
\pstGeonode(-1.000,0.000){R2}
\pstGeonode(-0.000,-1.000){R3}
\pstLineAB[linestyle=dotted]{R0}{R2}
\pstLineAB[linestyle=dotted]{R1}{R3}
\pstGeonode(0.500,0.500){r0}
\pstGeonode(-0.500,0.500){r1}
\pstGeonode(-0.500,-0.500){r2}
\pstGeonode(0.500,-0.500){r3}
\pstLineAB{r0}{R0}
\pstLineAB{r0}{R1}
\pstLineAB{r1}{R1}
\pstLineAB{r1}{R2}
\pstLineAB{r2}{R2}
\pstLineAB{r2}{R3}
\pstLineAB{r3}{R3}
\pstLineAB{r3}{R0}

    \end{pspicture}
    } &
    \scalebox{0.5}{
    \begin{pspicture}(-2.5,-2.5)(2.5,2.5)
    % d=3 r=0.440321 R=1.000000
\psset{unit=1in,dotsep=0.5mm,PointName=none,PointSymbol=none}
\pstGeonode(0,0){O}
\pstGeonode(0,0.440){A}
\pstCircleOA[fillstyle=solid,fillcolor=lightgray]{O}{A}
\pstGeonode(1.000,0.000){R0}
\pstGeonode(0.707,0.707){R1}
\pstGeonode(0.000,1.000){R2}
\pstGeonode(-0.707,0.707){R3}
\pstGeonode(-1.000,0.000){R4}
\pstGeonode(-0.707,-0.707){R5}
\pstGeonode(-0.000,-1.000){R6}
\pstGeonode(0.707,-0.707){R7}
\pstLineAB[linestyle=dotted]{R0}{R4}
\pstLineAB[linestyle=dotted]{R1}{R5}
\pstLineAB[linestyle=dotted]{R2}{R6}
\pstLineAB[linestyle=dotted]{R3}{R7}
\pstGeonode(0.407,0.169){r0}
\pstGeonode(0.169,0.407){r1}
\pstGeonode(-0.169,0.407){r2}
\pstGeonode(-0.407,0.169){r3}
\pstGeonode(-0.407,-0.169){r4}
\pstGeonode(-0.169,-0.407){r5}
\pstGeonode(0.169,-0.407){r6}
\pstGeonode(0.407,-0.169){r7}
\pstLineAB{r0}{R0}
\pstLineAB{r0}{R1}
\pstLineAB{r1}{R1}
\pstLineAB{r1}{R2}
\pstLineAB{r2}{R2}
\pstLineAB{r2}{R3}
\pstLineAB{r3}{R3}
\pstLineAB{r3}{R4}
\pstLineAB{r4}{R4}
\pstLineAB{r4}{R5}
\pstLineAB{r5}{R5}
\pstLineAB{r5}{R6}
\pstLineAB{r6}{R6}
\pstLineAB{r6}{R7}
\pstLineAB{r7}{R7}
\pstLineAB{r7}{R0}

    \end{pspicture}
    } &
    \scalebox{0.5}{
    \begin{pspicture}(-2.5,-2.5)(2.5,2.5)
    % d=4 r=0.219715 R=1.000000
\psset{unit=1in,dotsep=0.5mm,PointName=none,PointSymbol=none}
\pstGeonode(0,0){O}
\pstGeonode(0,0.220){A}
\pstCircleOA[fillstyle=solid,fillcolor=lightgray]{O}{A}
\pstGeonode(1.000,0.000){R0}
\pstGeonode(0.924,0.383){R1}
\pstGeonode(0.707,0.707){R2}
\pstGeonode(0.383,0.924){R3}
\pstGeonode(0.000,1.000){R4}
\pstGeonode(-0.383,0.924){R5}
\pstGeonode(-0.707,0.707){R6}
\pstGeonode(-0.924,0.383){R7}
\pstGeonode(-1.000,0.000){R8}
\pstGeonode(-0.924,-0.383){R9}
\pstGeonode(-0.707,-0.707){R10}
\pstGeonode(-0.383,-0.924){R11}
\pstGeonode(-0.000,-1.000){R12}
\pstGeonode(0.383,-0.924){R13}
\pstGeonode(0.707,-0.707){R14}
\pstGeonode(0.924,-0.383){R15}
\pstLineAB[linestyle=dotted]{R0}{R8}
\pstLineAB[linestyle=dotted]{R1}{R9}
\pstLineAB[linestyle=dotted]{R2}{R10}
\pstLineAB[linestyle=dotted]{R3}{R11}
\pstLineAB[linestyle=dotted]{R4}{R12}
\pstLineAB[linestyle=dotted]{R5}{R13}
\pstLineAB[linestyle=dotted]{R6}{R14}
\pstLineAB[linestyle=dotted]{R7}{R15}
\pstGeonode(0.215,0.043){r0}
\pstGeonode(0.183,0.122){r1}
\pstGeonode(0.122,0.183){r2}
\pstGeonode(0.043,0.215){r3}
\pstGeonode(-0.043,0.215){r4}
\pstGeonode(-0.122,0.183){r5}
\pstGeonode(-0.183,0.122){r6}
\pstGeonode(-0.215,0.043){r7}
\pstGeonode(-0.215,-0.043){r8}
\pstGeonode(-0.183,-0.122){r9}
\pstGeonode(-0.122,-0.183){r10}
\pstGeonode(-0.043,-0.215){r11}
\pstGeonode(0.043,-0.215){r12}
\pstGeonode(0.122,-0.183){r13}
\pstGeonode(0.183,-0.122){r14}
\pstGeonode(0.215,-0.043){r15}
\pstLineAB{r0}{R0}
\pstLineAB{r0}{R1}
\pstLineAB{r1}{R1}
\pstLineAB{r1}{R2}
\pstLineAB{r2}{R2}
\pstLineAB{r2}{R3}
\pstLineAB{r3}{R3}
\pstLineAB{r3}{R4}
\pstLineAB{r4}{R4}
\pstLineAB{r4}{R5}
\pstLineAB{r5}{R5}
\pstLineAB{r5}{R6}
\pstLineAB{r6}{R6}
\pstLineAB{r6}{R7}
\pstLineAB{r7}{R7}
\pstLineAB{r7}{R8}
\pstLineAB{r8}{R8}
\pstLineAB{r8}{R9}
\pstLineAB{r9}{R9}
\pstLineAB{r9}{R10}
\pstLineAB{r10}{R10}
\pstLineAB{r10}{R11}
\pstLineAB{r11}{R11}
\pstLineAB{r11}{R12}
\pstLineAB{r12}{R12}
\pstLineAB{r12}{R13}
\pstLineAB{r13}{R13}
\pstLineAB{r13}{R14}
\pstLineAB{r14}{R14}
\pstLineAB{r14}{R15}
\pstLineAB{r15}{R15}
\pstLineAB{r15}{R0}

    \end{pspicture}
    } &
    \scalebox{0.5}{
    \begin{pspicture}(-2.5,-2.5)(2.5,2.5)
    % d=6 r=0.050672 R=1.000000
\psset{unit=1in,dotsep=0.5mm,PointName=none,PointSymbol=none}
\pstGeonode(0,0){O}
\pstGeonode(0,0.051){A}
\pstCircleOA[fillstyle=solid,fillcolor=lightgray]{O}{A}
\pstGeonode(1.000,0.000){R0}
\pstGeonode(0.995,0.098){R1}
\pstGeonode(0.981,0.195){R2}
\pstGeonode(0.957,0.290){R3}
\pstGeonode(0.924,0.383){R4}
\pstGeonode(0.882,0.471){R5}
\pstGeonode(0.831,0.556){R6}
\pstGeonode(0.773,0.634){R7}
\pstGeonode(0.707,0.707){R8}
\pstGeonode(0.634,0.773){R9}
\pstGeonode(0.556,0.831){R10}
\pstGeonode(0.471,0.882){R11}
\pstGeonode(0.383,0.924){R12}
\pstGeonode(0.290,0.957){R13}
\pstGeonode(0.195,0.981){R14}
\pstGeonode(0.098,0.995){R15}
\pstGeonode(0.000,1.000){R16}
\pstGeonode(-0.098,0.995){R17}
\pstGeonode(-0.195,0.981){R18}
\pstGeonode(-0.290,0.957){R19}
\pstGeonode(-0.383,0.924){R20}
\pstGeonode(-0.471,0.882){R21}
\pstGeonode(-0.556,0.831){R22}
\pstGeonode(-0.634,0.773){R23}
\pstGeonode(-0.707,0.707){R24}
\pstGeonode(-0.773,0.634){R25}
\pstGeonode(-0.831,0.556){R26}
\pstGeonode(-0.882,0.471){R27}
\pstGeonode(-0.924,0.383){R28}
\pstGeonode(-0.957,0.290){R29}
\pstGeonode(-0.981,0.195){R30}
\pstGeonode(-0.995,0.098){R31}
\pstGeonode(-1.000,0.000){R32}
\pstGeonode(-0.995,-0.098){R33}
\pstGeonode(-0.981,-0.195){R34}
\pstGeonode(-0.957,-0.290){R35}
\pstGeonode(-0.924,-0.383){R36}
\pstGeonode(-0.882,-0.471){R37}
\pstGeonode(-0.831,-0.556){R38}
\pstGeonode(-0.773,-0.634){R39}
\pstGeonode(-0.707,-0.707){R40}
\pstGeonode(-0.634,-0.773){R41}
\pstGeonode(-0.556,-0.831){R42}
\pstGeonode(-0.471,-0.882){R43}
\pstGeonode(-0.383,-0.924){R44}
\pstGeonode(-0.290,-0.957){R45}
\pstGeonode(-0.195,-0.981){R46}
\pstGeonode(-0.098,-0.995){R47}
\pstGeonode(-0.000,-1.000){R48}
\pstGeonode(0.098,-0.995){R49}
\pstGeonode(0.195,-0.981){R50}
\pstGeonode(0.290,-0.957){R51}
\pstGeonode(0.383,-0.924){R52}
\pstGeonode(0.471,-0.882){R53}
\pstGeonode(0.556,-0.831){R54}
\pstGeonode(0.634,-0.773){R55}
\pstGeonode(0.707,-0.707){R56}
\pstGeonode(0.773,-0.634){R57}
\pstGeonode(0.831,-0.556){R58}
\pstGeonode(0.882,-0.471){R59}
\pstGeonode(0.924,-0.383){R60}
\pstGeonode(0.957,-0.290){R61}
\pstGeonode(0.981,-0.195){R62}
\pstGeonode(0.995,-0.098){R63}
\pstLineAB[linestyle=dotted]{R0}{R32}
\pstLineAB[linestyle=dotted]{R1}{R33}
\pstLineAB[linestyle=dotted]{R2}{R34}
\pstLineAB[linestyle=dotted]{R3}{R35}
\pstLineAB[linestyle=dotted]{R4}{R36}
\pstLineAB[linestyle=dotted]{R5}{R37}
\pstLineAB[linestyle=dotted]{R6}{R38}
\pstLineAB[linestyle=dotted]{R7}{R39}
\pstLineAB[linestyle=dotted]{R8}{R40}
\pstLineAB[linestyle=dotted]{R9}{R41}
\pstLineAB[linestyle=dotted]{R10}{R42}
\pstLineAB[linestyle=dotted]{R11}{R43}
\pstLineAB[linestyle=dotted]{R12}{R44}
\pstLineAB[linestyle=dotted]{R13}{R45}
\pstLineAB[linestyle=dotted]{R14}{R46}
\pstLineAB[linestyle=dotted]{R15}{R47}
\pstLineAB[linestyle=dotted]{R16}{R48}
\pstLineAB[linestyle=dotted]{R17}{R49}
\pstLineAB[linestyle=dotted]{R18}{R50}
\pstLineAB[linestyle=dotted]{R19}{R51}
\pstLineAB[linestyle=dotted]{R20}{R52}
\pstLineAB[linestyle=dotted]{R21}{R53}
\pstLineAB[linestyle=dotted]{R22}{R54}
\pstLineAB[linestyle=dotted]{R23}{R55}
\pstLineAB[linestyle=dotted]{R24}{R56}
\pstLineAB[linestyle=dotted]{R25}{R57}
\pstLineAB[linestyle=dotted]{R26}{R58}
\pstLineAB[linestyle=dotted]{R27}{R59}
\pstLineAB[linestyle=dotted]{R28}{R60}
\pstLineAB[linestyle=dotted]{R29}{R61}
\pstLineAB[linestyle=dotted]{R30}{R62}
\pstLineAB[linestyle=dotted]{R31}{R63}
\pstGeonode(0.051,0.002){r0}
\pstGeonode(0.050,0.007){r1}
\pstGeonode(0.049,0.012){r2}
\pstGeonode(0.048,0.017){r3}
\pstGeonode(0.046,0.022){r4}
\pstGeonode(0.043,0.026){r5}
\pstGeonode(0.041,0.030){r6}
\pstGeonode(0.038,0.034){r7}
\pstGeonode(0.034,0.038){r8}
\pstGeonode(0.030,0.041){r9}
\pstGeonode(0.026,0.043){r10}
\pstGeonode(0.022,0.046){r11}
\pstGeonode(0.017,0.048){r12}
\pstGeonode(0.012,0.049){r13}
\pstGeonode(0.007,0.050){r14}
\pstGeonode(0.002,0.051){r15}
\pstGeonode(-0.002,0.051){r16}
\pstGeonode(-0.007,0.050){r17}
\pstGeonode(-0.012,0.049){r18}
\pstGeonode(-0.017,0.048){r19}
\pstGeonode(-0.022,0.046){r20}
\pstGeonode(-0.026,0.043){r21}
\pstGeonode(-0.030,0.041){r22}
\pstGeonode(-0.034,0.038){r23}
\pstGeonode(-0.038,0.034){r24}
\pstGeonode(-0.041,0.030){r25}
\pstGeonode(-0.043,0.026){r26}
\pstGeonode(-0.046,0.022){r27}
\pstGeonode(-0.048,0.017){r28}
\pstGeonode(-0.049,0.012){r29}
\pstGeonode(-0.050,0.007){r30}
\pstGeonode(-0.051,0.002){r31}
\pstGeonode(-0.051,-0.002){r32}
\pstGeonode(-0.050,-0.007){r33}
\pstGeonode(-0.049,-0.012){r34}
\pstGeonode(-0.048,-0.017){r35}
\pstGeonode(-0.046,-0.022){r36}
\pstGeonode(-0.043,-0.026){r37}
\pstGeonode(-0.041,-0.030){r38}
\pstGeonode(-0.038,-0.034){r39}
\pstGeonode(-0.034,-0.038){r40}
\pstGeonode(-0.030,-0.041){r41}
\pstGeonode(-0.026,-0.043){r42}
\pstGeonode(-0.022,-0.046){r43}
\pstGeonode(-0.017,-0.048){r44}
\pstGeonode(-0.012,-0.049){r45}
\pstGeonode(-0.007,-0.050){r46}
\pstGeonode(-0.002,-0.051){r47}
\pstGeonode(0.002,-0.051){r48}
\pstGeonode(0.007,-0.050){r49}
\pstGeonode(0.012,-0.049){r50}
\pstGeonode(0.017,-0.048){r51}
\pstGeonode(0.022,-0.046){r52}
\pstGeonode(0.026,-0.043){r53}
\pstGeonode(0.030,-0.041){r54}
\pstGeonode(0.034,-0.038){r55}
\pstGeonode(0.038,-0.034){r56}
\pstGeonode(0.041,-0.030){r57}
\pstGeonode(0.043,-0.026){r58}
\pstGeonode(0.046,-0.022){r59}
\pstGeonode(0.048,-0.017){r60}
\pstGeonode(0.049,-0.012){r61}
\pstGeonode(0.050,-0.007){r62}
\pstGeonode(0.051,-0.002){r63}
\pstLineAB{r0}{R0}
\pstLineAB{r0}{R1}
\pstLineAB{r1}{R1}
\pstLineAB{r1}{R2}
\pstLineAB{r2}{R2}
\pstLineAB{r2}{R3}
\pstLineAB{r3}{R3}
\pstLineAB{r3}{R4}
\pstLineAB{r4}{R4}
\pstLineAB{r4}{R5}
\pstLineAB{r5}{R5}
\pstLineAB{r5}{R6}
\pstLineAB{r6}{R6}
\pstLineAB{r6}{R7}
\pstLineAB{r7}{R7}
\pstLineAB{r7}{R8}
\pstLineAB{r8}{R8}
\pstLineAB{r8}{R9}
\pstLineAB{r9}{R9}
\pstLineAB{r9}{R10}
\pstLineAB{r10}{R10}
\pstLineAB{r10}{R11}
\pstLineAB{r11}{R11}
\pstLineAB{r11}{R12}
\pstLineAB{r12}{R12}
\pstLineAB{r12}{R13}
\pstLineAB{r13}{R13}
\pstLineAB{r13}{R14}
\pstLineAB{r14}{R14}
\pstLineAB{r14}{R15}
\pstLineAB{r15}{R15}
\pstLineAB{r15}{R16}
\pstLineAB{r16}{R16}
\pstLineAB{r16}{R17}
\pstLineAB{r17}{R17}
\pstLineAB{r17}{R18}
\pstLineAB{r18}{R18}
\pstLineAB{r18}{R19}
\pstLineAB{r19}{R19}
\pstLineAB{r19}{R20}
\pstLineAB{r20}{R20}
\pstLineAB{r20}{R21}
\pstLineAB{r21}{R21}
\pstLineAB{r21}{R22}
\pstLineAB{r22}{R22}
\pstLineAB{r22}{R23}
\pstLineAB{r23}{R23}
\pstLineAB{r23}{R24}
\pstLineAB{r24}{R24}
\pstLineAB{r24}{R25}
\pstLineAB{r25}{R25}
\pstLineAB{r25}{R26}
\pstLineAB{r26}{R26}
\pstLineAB{r26}{R27}
\pstLineAB{r27}{R27}
\pstLineAB{r27}{R28}
\pstLineAB{r28}{R28}
\pstLineAB{r28}{R29}
\pstLineAB{r29}{R29}
\pstLineAB{r29}{R30}
\pstLineAB{r30}{R30}
\pstLineAB{r30}{R31}
\pstLineAB{r31}{R31}
\pstLineAB{r31}{R32}
\pstLineAB{r32}{R32}
\pstLineAB{r32}{R33}
\pstLineAB{r33}{R33}
\pstLineAB{r33}{R34}
\pstLineAB{r34}{R34}
\pstLineAB{r34}{R35}
\pstLineAB{r35}{R35}
\pstLineAB{r35}{R36}
\pstLineAB{r36}{R36}
\pstLineAB{r36}{R37}
\pstLineAB{r37}{R37}
\pstLineAB{r37}{R38}
\pstLineAB{r38}{R38}
\pstLineAB{r38}{R39}
\pstLineAB{r39}{R39}
\pstLineAB{r39}{R40}
\pstLineAB{r40}{R40}
\pstLineAB{r40}{R41}
\pstLineAB{r41}{R41}
\pstLineAB{r41}{R42}
\pstLineAB{r42}{R42}
\pstLineAB{r42}{R43}
\pstLineAB{r43}{R43}
\pstLineAB{r43}{R44}
\pstLineAB{r44}{R44}
\pstLineAB{r44}{R45}
\pstLineAB{r45}{R45}
\pstLineAB{r45}{R46}
\pstLineAB{r46}{R46}
\pstLineAB{r46}{R47}
\pstLineAB{r47}{R47}
\pstLineAB{r47}{R48}
\pstLineAB{r48}{R48}
\pstLineAB{r48}{R49}
\pstLineAB{r49}{R49}
\pstLineAB{r49}{R50}
\pstLineAB{r50}{R50}
\pstLineAB{r50}{R51}
\pstLineAB{r51}{R51}
\pstLineAB{r51}{R52}
\pstLineAB{r52}{R52}
\pstLineAB{r52}{R53}
\pstLineAB{r53}{R53}
\pstLineAB{r53}{R54}
\pstLineAB{r54}{R54}
\pstLineAB{r54}{R55}
\pstLineAB{r55}{R55}
\pstLineAB{r55}{R56}
\pstLineAB{r56}{R56}
\pstLineAB{r56}{R57}
\pstLineAB{r57}{R57}
\pstLineAB{r57}{R58}
\pstLineAB{r58}{R58}
\pstLineAB{r58}{R59}
\pstLineAB{r59}{R59}
\pstLineAB{r59}{R60}
\pstLineAB{r60}{R60}
\pstLineAB{r60}{R61}
\pstLineAB{r61}{R61}
\pstLineAB{r61}{R62}
\pstLineAB{r62}{R62}
\pstLineAB{r62}{R63}
\pstLineAB{r63}{R63}
\pstLineAB{r63}{R0}

    \end{pspicture}
    }\\
	(a) 2D &
	(b) 3D &
	(c) 4D &
	(d) $d$D
	\end{tabular}
    }
\end{frame}


\begin{frame}{Volume of a Thin Shell}

  \begin{columns}
	\column{0.5\textwidth}
The volume of a thin hypershell of width
$\epsilon$ is given as
\begin{align*}
  \vol(S_d(r,\epsilon)) & = \vol(S_d(r)) - \vol(S_d(r - \epsilon))\\
  & = K_d r^d - K_d (r -\epsilon)^d.
\end{align*}

The ratio of volume of the thin shell to the
volume of the outer sphere:
\begin{align*}
\frac{\vol(S_d(r,\epsilon))}{\vol(S_d(r))} =
\frac{K_d r^d - K_d (r-\epsilon)^d}{K_d r^d} =
1 - \lB(1-\frac{\epsilon}{r}\rB)^d
\end{align*}

As $d$ increases, we have
\begin{align*}
\lim_{d\to \infty} \frac{\vol(S_d(r,\epsilon))}{\vol(S_d(r))} =
\lim_{d\to \infty} 1 - \lB(1-\frac{\epsilon}{r}\rB)^d \to 1
\end{align*}


	\column{0.45\textwidth}
	\normalsize
\centerline{
\scalebox{0.7}{
\psset{unit=0.75in}
\pspicture(-1.5,-1.5)(1.75,1.75)
\pscircle[fillstyle=solid,fillcolor=lightgray](0,0){1.5}
\pscircle[fillstyle=solid,fillcolor=white](0,0){1.25}
\pnode(0,0){O}
\pnode(1.5,0){A}
\pnode(0.884,-0.884){B}
\pnode(1.06,-1.06){C}
\ncline[arrowscale=2]{->}{O}{A}
\naput[nrot=:U]{$r$}
\ncline[arrowscale=2]{->}{O}{B}
\nbput[nrot=:U]{$r-\epsilon$}
\ncline[]{-}{B}{C}
\nbput[nrot=:U]{$\epsilon$}
\endpspicture
}}
\end{columns}
\end{frame}



\begin{frame}{Diagonals in Hyperspace}

Consider a $d$-dimensional hypercube, with origin
${\bm 0}_d = (0_1,0_2,\ldots,0_d)$, and bounded in each dimension in
the range $[-1,1]$. Each \hbox{``corner''} of the hyperspace is a
$d$-dimensional vector of the form $(\pm 1_1, \pm 1_2, \ldots, \pm 1_d)^T$.

\smallskip
Let $\be_i = (0_1, \ldots, 1_i, \ldots, 0_d)^T$ denote the
$d$-dimensional canonical unit vector in dimension $i$,
and let $\bone$ denote the $d$-dimensional diagonal vector
$(1_1,1_2,\ldots,1_d)^T$.


\medskip
Consider the angle $\theta_d$ between the diagonal vector
$\bone$ and the f\/{i}rst axis $\be_1$, in $d$ dimensions:
\begin{align*}
  \cos\theta_d = \frac{\be_1^T \bone}{\|\be_1\| \; \| \bone\|} =
  \frac{\be_1^T \bone}{\sqrt{\be_1^T\be_1} \sqrt{\bone^T\bone}} =
  \frac{1}{\sqrt{1}\sqrt{d}}  = \frac{1}{\sqrt{d}}
\end{align*}

\medskip
As $d$ increases, we have
\begin{align*}
& \lim_{d\to \infty} \cos \theta_d =
\lim_{d\to\infty} \frac{1}{\sqrt{d}} \to 0\\
\shortintertext{which implies that}
& \lim_{d\to\infty} \theta_d \to \frac{\pi}{2} = 90^\circ
\end{align*}
\end{frame}


%\begin{figure}[b!]
\def\pshlabel#1{\scriptsize {$#1$}}
\def\psvlabel#1{\scriptsize {$#1$}}
\begin{frame}{Angle between Diagonal Vector $\bone$ and $\be_1$}
\centerline{
\begin{tabular}{cc}
%\subfloat[]{
\scalebox{0.8}{
\psset{unit=0.75in,arrowscale=1.5}
\pspicture(-1.5,-1.75)(1.5,1.5)
\psaxes[tickstyle=bottom,axesstyle=frame,Ox=-1,Oy=-1](-1,-1)(1,1)
\psline[](-1,0)(1,0)
\psline[](0,-1)(0,1)
\psline[linestyle=dotted](-1,-1)(1,1)
\psline[linestyle=dotted](-1,1)(1,-1)
\psline[linewidth=2pt]{->}(1,0)
\psline[linewidth=2pt]{->}(1,1)
\uput[ur](1,1){$\bone$}
\uput[r](1,0){$\be_1$}
%\uput[d](-1,-1){-1}
%\uput[d](0,-1){0}
%\uput[d](1,-1){1}
%\uput[l](-1,-1){-1}
%\uput[l](-1,0){0}
%\uput[l](-1,1){1}
\pstGeonode[PointSymbol=none,PointName=none](0,0){O}(1,0){B}(1,1){C}
\pstMarkAngle[MarkAngleRadius=0.25,LabelSep=0.4]{B}{O}{C}{$\theta$}
\endpspicture
} &
\scalebox{0.7}{
\definecolor{mygray}{gray}{0.95}
\psset{viewpoint=30 -35 20 rtp2xyz,Decran=60,
    lightsrc=viewpoint, opacity=0.8}
\begin{pspicture}(-3,-3.25)(3,3.25)
\psSolid[object=plan, definition=equation, action=draw*,
    fillcolor=mygray,incolor=mygray,
    args={[0 0 1 0]}, base=-1 1 -1 1]
\psSolid[object=parallelepiped,a=2,b=2,c=2,action=draw](0,0,0)
\multido{\ix=-1+1}{3}{%
    \psPoint(\ix\space,-1,-1){X1}
    \psPoint(\ix\space,-1.05,-1){X2}
    \psline(X1)(X2)\uput[dl](X1){\scriptsize $\ix$}}
\multido{\iy=-1+1}{3}{%
    \psPoint(1,\iy\space,-1){Y1}
    \psPoint(1.05,\iy\space,-1){Y2}
    \psline(Y1)(Y2)\uput[dr](Y1){\scriptsize $\iy$}}
\multido{\iz=-1+1}{3}{%
    \psPoint(-1,-1,\iz\space){Z1}
    \psPoint(-1,-1.05,\iz\space){Z2}
    \psline(Z1)(Z2)\uput[l](Z1){\scriptsize $\iz$}}
\psSolid[object=line,args=-1 0 0 1 0 0]
\psSolid[object=line,args=0 -1 0 0 1 0]
\psSolid[object=line,args=0 0 -1 0 0 1]
\psset{linestyle=dotted}
\psSolid[object=line,args=-1 -1 -1 1 1 1]
\psSolid[object=line,args=1 -1 -1 -1 1 1]
\psSolid[object=line,args=-1 1 -1 1 -1 1]
\psSolid[object=line,args=-1 -1 1 1 1 -1]
\psset{linestyle=solid,arrowscale=1.5}
\psSolid[object=vecteur, args=1 0 0, linewidth=1pt]
\psSolid[object=vecteur, args=1 1 1, linewidth=1pt]
\psPoint(0,0,0){O}
\psPoint(1,0,0){A}
\psPoint(1,1,1){C}
\psline[linewidth=2pt]{->}(O)(A)
\psline[linewidth=2pt]{->}(O)(C)
\uput[ur](C){$\bone$}
\uput[dr](A){$\be_1$}
\pstMarkAngle[MarkAngleRadius=0.6,LabelSep=0.75]{A}{O}{C}{$\theta$}
\end{pspicture}
}\\
(a) In 2D & (b) In 3D
\end{tabular}
}%}

\medskip
\small
In high dimensions all of the diagonal vectors are
perpendicular (or orthogonal) to all the coordinates axes!
Each of the $2^{d-1}$ new axes connecting pairs of 
$2^d$ corners are essentially orthogonal to all of
the $d$ principal coordinate axes! Thus, in effect, high-dimensional
space has an exponential number of orthogonal ``axes.''

\end{frame}



\begin{frame}{Density of the Multivariate Normal}

  \small

Consider the standard multivariate normal distribution
with $\bmu = \bzero$, and $\cov = \bI$ 
\begin{align*}
f(\bx) = \frac{1}{(\sqrt{2\pi})^d}\exp\lB\{-\frac{\bx^T
\bx}{2}\rB\}
\end{align*}

\medskip
The peak of the density is at the mean. 
Consider the set of points $\bx$ with density at least
$\alpha$ fraction of the density at the mean
\begin{align*}
\frac{f(\bx)}{f(\bzero)} & \ge \alpha\nonumber\\
\exp\lB\{-\frac{\bx^T\bx}{2}\rB\}  & \ge \alpha\nonumber\\
 \bx^T\bx  & \le -2 \ln (\alpha)\nonumber\\
  \sum_{i=1}^d (x_i)^2 & \le -2 \ln (\alpha)
\end{align*}

\medskip
The sum of squared IID random variables follows a chi-squared
distribution $\chi^2_d$. Thus,
\begin{align*}
P\lB(\frac{f(\bx)}{f(\bzero)} \geq \alpha\rB) 
=  F_{\chi^2_d}(-2\ln(\alpha))
\end{align*}
where $F_{\chi^2_q}$ is the CDF.

\medskip
As dimensionality increases, this probability
decreases sharply, and eventually tends to zero, that is,
\begin{equation*}
\lim_{d\to\infty}  P\lB( \bx^T\bx \le -2\ln(\alpha) \rB) \to 0
\end{equation*}
Thus, in higher dimensions the probability density around the
mean decreases very rapidly as one moves away from the mean. In
essence the entire probability mass migrates to the tail regions.
\end{frame}



\def\pshlabel#1{\scriptsize {$#1$}}
\def\psvlabel#1{\scriptsize {$#1$}}
\begin{frame}{Density Contour for $\alpha$ Fraction of the Density at
  the Mean: One Dimension}

  Let $\alpha=0.5$, then $-2\ln(0.5)=1.386$ and 
  $F_{\chi^2_1}(1.386) = 0.76$. Thus, 24\% of the density is in the
  tail regions.

  \bigskip
\centerline{
%\captionsetup[subfloat]{captionskip=10pt}
%\subfloat[]{
\psset{yunit=3in,xunit=0.3in}
\begin{pspicture}(-4,-0.05)(4,0.5)%
\psaxes[Dy=0.1,arrowscale=1.5]{->}(0,0)(-4.5,0)(4.5,0.5)
\psGauss[sigma=1]{-4.5}{4.5}%
\psline[linewidth=2pt](-1.177,0.2)(1.177,0.2)
\psset{dotscale=0.75,dotstyle=B|}
\psline[linewidth=2pt,showpoints=true](-1.177,0)(1.177,0)
\uput[ur](1.177,0.2){$\alpha=0.5$}
\end{pspicture}
}%}

\end{frame}



\begin{frame}{Density Contour for $\alpha$ Fraction of the Density at
  the Mean: Two Dimensions}

    Let $\alpha=0.5$, then $-2\ln(0.5)=1.386$ and 
  $F_{\chi^2_2}(1.386) = 0.50$. Thus, 50\% of the density is in the
  tail regions.

\bigskip
\centerline{
%\subfloat[]{
\scalebox{0.7}{
\psset{unit=0.25in}
\psset{viewpoint=30 20 30 rtp2xyz,Decran=50}
\psset{lightsrc=viewpoint,opacity=0.9,incolor=white}
\begin{pspicture}(-7,-6)(7,7.5)
%peak at 0 is 1/2pi = 0.159, taking 0.159*20/2 = 1.59 offset
\psSurface[ngrid=0.25 0.25, linewidth=0.1pt,
linecolor=lightgray,
intersectionplan ={[0 0 1 -1.59]},
intersectionlinewidth =2,
intersectioncolor=(Black),
intersectiontype=0,
transform={1 1 20 scaleOpoint3d},algebraic](-4,-4)(4,4){%
  (1/(2*pi))*e^(-1*((x^2)+(y^2))/2)}
%\psSolid[object=cylindrecreux,h=0.025,r=1.18,mode=4](0,0,0)
\axesIIID[axisnames={X_1,X_2,f(\bx)}](-4.1,-4.1,0)(5,5,4)
\multido{\ix=-4+1}{9}{%
                \psPoint(\ix\space,4,0){X1}
                \psPoint(\ix\space,4.2,0){X2}
                \psline(X1)(X2)\uput[dr](X1){$\ix$}}
\multido{\iy=-4+1}{9}{%
                \psPoint(4,\iy\space,0){Y1}
                \psPoint(4.2,\iy\space,0){Y2}
                \psline(Y1)(Y2)\uput[dl](Y1){$\iy$}}
\multido{\nz=0+1,\nl=0+0.05}{4}

\end{frame}


\def\pshlabel#1{\scriptsize {$#1$}}
\def\psvlabel#1{\scriptsize {$#1$}}
\begin{frame}{Chi-Squared Distribution: $P(f(\bx)/f(\bzero) \ge \alpha)$}

  This probability decreases
rapidly with dimensionality. For 2D, it is $0.5$. For 3D it is $0.29$,
ie., $71\%$ of the density is in the tails.
By $d=10$, it
decreases to $0.075\%$, that is, 99.925\% of the points lie in
the extreme or tail regions.


\psset{plotpoints=200,tickstyle=bottom}
\psset{xAxisLabel=$x$,yAxisLabel=$f(x)$}
  \bigskip
  \centerline{~}
\centerline{
\scalebox{0.9}{
\hspace{0.5in}
%\subfloat[$d=2$]{
\begin{pspicture}(0,-0.5)(1,5)
\begin{psgraph}[Dx=5,Dy=0.1]{->}(0,0)(16,0.55){2in}{2in}
\pscustom[fillstyle=solid,fillcolor=lightgray]{%
    \psChiIIDist[nue=2]{0.001}{1.39}%
    \psline[](1.39,0)(0.01,0)}
\psChiIIDist[nue=2,linewidth=1.5pt]{0.001}{14}
\uput[0](1,0.45){\scriptsize $F=0.5$}
\end{psgraph}
\end{pspicture}
%}
%%\hspace{-0.25in}
%\subfloat[$d=3$]{
\begin{pspicture}(0,-0.5)(1,5)
\begin{psgraph}[Dx=5,Dy=0.05]{->}(0,0)(20,0.3){2in}{2in}
\pscustom[fillstyle=solid,fillcolor=lightgray]{%
    \psChiIIDist[nue=3]{0.0001}{1.39}%
    \psline[](1.39,0)(0.01,0)}
\psChiIIDist[nue=3,linewidth=1.5pt]{0.0001}{16}
\uput[0](1,0.27){\scriptsize $F=0.29$}
\end{psgraph}
\end{pspicture}
}}
\end{frame}


\begin{frame}{Hypersphere Volume: Polar Coordinates in 2D}

  \begin{columns}
	\column{0.5\textwidth}
\centerline{
\psset{unit=0.75in}
\pspicture(-3,0)(3,3)
\psset{xAxisLabel=$X_1$,yAxisLabel= $X_2$}
\psgraph[labels=none,tickstyle=bottom,arrows=->]%
    (0,0)(-1.25,-1.25)(1.25,1.25){2in}{!}
\pscircle[](0,0){1}
\pstGeonode[PointSymbol=none,PointName=none](0,0){O}(1,0){B}(0.707,0.707){C}
\psdot[dotscale=2,dotstyle=Bo,fillcolor=lightgray](C)
\pstMarkAngle[MarkAngleRadius=0.25,LabelSep=0.4,
    arrows=->]{B}{O}{C}{$\theta_1$}
\ncline[arrowscale=2,linewidth=2pt]{->}{O}{C}
\naput[nrot=:U]{$r$}
\uput[45](C){$(x_1,x_2)$}
\psline[linestyle=dashed](0.707,0)(C)
\psline[linestyle=dashed](0,0.707)(C)
\endpsgraph
\endpspicture
}

\column{0.5\textwidth}
\small
The point $\bx=(x_1, x_2)$ in polar coordinates
\vspace*{-0.3cm}
\begin{align*}
    x_1 & = r \cos\theta_1 = r c_1\\[0.1cm]
    x_2 & = r \sin\theta_1 = r s_1
\end{align*}
\vspace*{-0.3cm}
where $r = \norm{\bx}$, and
$\cos\theta_1 = c_1$ and
$\sin\theta_1= s_1$.

The {\em Jacobian matrix} for this transformation is given as
\begin{align*}
    J(\theta_1) =
    \matr{
    \frac{\partial x_1}{\partial r} &
    \frac{\partial x_1}{\partial \theta_1}\\[1ex]
    \frac{\partial x_2}{\partial r} &
    \frac{\partial x_2}{\partial \theta_1}\\
    } =
    \matr{
    c_1 & -r s_1\\
    s_1 & r c_1\\
    }
\end{align*}

Hypersphere volume is obtained by
integration over $r$ and $\theta_1$ 
(with $r > 0$, and $0 \le \theta_1 \le
2\pi$):
\begin{align*}
    \vol(S_2(r)) & = \int_r \int_{\theta_1}
    \Bigl|\det(J(\theta_1))\Bigr| \; dr \;
    d{\theta_1}\nonumber\\
    & = \int_0^r \int_0^{2\pi} r \; dr \; d{\theta_1} =
    \int_0^r r \; dr \int_0^{2\pi}  d{\theta_1}\nonumber\\
    & = \lB.\frac{r^2}{2}\rB|_0^r
    \cdot \Bigl.\theta_1\Bigr|_0^{2\pi} = \pi r^2
\end{align*}


\end{columns}
\end{frame}


\begin{frame}{Hypersphere Volume: Polar Coordinates in 3D}

  \begin{columns}

	\column{0.5\textwidth}
\centerline{
\scalebox{0.7}{
\definecolor{mygray}{gray}{0.95}
\psset{unit=0.5in,arrowscale=1.5}
\psset{viewpoint=30 -15 30 rtp2xyz,Decran=80,lightsrc=viewpoint}
\begin{pspicture}(-3,-3)(3,3.5)
%\psSolid[object=plan, definition=equation, action=draw*,
    %opacity=0.4, fillcolor=lightgray,incolor=lightgray,
    %args={[0 0 1 0]}, base=-1 1 -1 1]
\psSolid[object=sphere,hollow,linecolor=lightgray,
    opacity=0.75,r=1,incolor=white, %fillcolor=gray!25,
    ngrid=20 20]%
\axesIIID[axisnames={X_1,X_2,X_3}](-1,-1,-1)(1.75,1.25,1.25)
\psPoint[](0,0,0){O}
\psPoint[](0.5,0.5,0.707){A}
\psPoint[](0.15,0.15,0.4){F}
\psdot[dotscale=2,fillcolor=lightgray,dotstyle=Bo](A)
\uput[45](A){$(x_1,x_2,x_3)$}
\uput[0](F){$r$}
\psPoint[](0.5,0.5,0){B}
\psPoint[](1,0,0){C}
\psPoint[](0.5,0,0){D}
\psPoint[](0,0.5,0){E}
\psline[linewidth=3pt]{->}(O)(A)
\psline[linewidth=2pt,linestyle=dotted,dotsep=1pt]{->}(O)(B)
\psline[linestyle=dashed](B)(A)
\psline[linestyle=dashed](D)(B)
\psline[linestyle=dashed](E)(B)
\pstMarkAngle[MarkAngleRadius=0.7,LabelSep=0.9,
    arrows=->]{B}{O}{A}{$\btheta_1$}
\pstMarkAngle[MarkAngleRadius=0.5,LabelSep=0.7,
    arrows=->]{C}{O}{B}{$\btheta_2$}
\end{pspicture}
}}

\column{0.5\textwidth}
\small
$\bx=(x_1, x_2, x_2)$ in polar coordinates
\begin{align*}
    x_1 & = r \cos\theta_1 \cos\theta_2 = r c_1 c_2\\
    x_2 & = r \cos\theta_1 \sin\theta_2 = r c_1 s_2\\
    x_3 & = r \sin\theta_1 = r s_1
\end{align*}

The Jacobian matrix is given as
\begin{align*}
    J(\theta_1,\theta_2) =
    \matr{
    c_1 c_2 & -r s_1 c_2 & -r c_1 s_2\\
    c_1 s_2  & -r s_1 s_2 & r c_1 c_2\\
    s_1 & r c_1 & 0
    }
\end{align*}

The volume of the hypersphere for $d=3$ is obtained via a triple
integral with $r > 0$, $-\pi/2 \le \theta_1 \le \pi/2$,
and $0 \le \theta_2 \le 2\pi$
\begin{align*}
    \vol(S_3(r)) & = \int_r \int_{\theta_1} \int_{\theta_2}
    \Bigl|\det(J(\theta_1,\theta_2))\Bigr| \; dr \;
    d{\theta_1} \; d{\theta_2}\nonumber\\
    &= {4 \over 3} \pi r^3
\end{align*}


\end{columns}
\end{frame}


\begin{frame}{Hypersphere Volume in $d$ Dimensions}
  \small

The determinant of the 
$d$-dimensional Jacobian matrix is
\begin{align*}
    \det(J(\theta_1,\theta_2,\dots,\theta_{d-1})) =
    (-1)^d r^{d-1} c_1^{d-2} c_2^{d-3} \dots c_{d-2}
\end{align*}

\medskip
The volume of the hypersphere is given by the $d$-dimensional
integral with
$r > 0$, $-\pi/2 \le \theta_i \le \pi/2$ for all $i=1,
\dots, d-2$, and $0 \le \theta_{d-1} \le 2\pi$:
\begin{align*}
    \vol(S_d(r)) & =
    \int_r \int_{\theta_1} \int_{\theta_2} \dots
    \int_{\theta_{d-1}}
    \Bigl|\det(J(\theta_1,\theta_2,\dots,\theta_{d-1}))\Bigr| \; dr \;
    d{\theta_1} \; d{\theta_2} \dots d{\theta_{d-1}}\nonumber\\
    & = \int_0^r r^{d-1} dr
    \int_{-\pi/2}^{\pi/2} c_1^{d-2} d\theta_1 \;
    \dots
    \int_{-\pi/2}^{\pi/2} c_{d-2} d\theta_{d-2}
    \int_0^{2\pi} d\theta_{d-1}\\
	& = 
     {r^d \over d}
    \frac{\Gamma\lB(\frac{d-1}{2}\rB)\Gamma\lB({1\over 2}\rB)}
           {\Gamma\lB(\frac{d}{2}\rB)}
    \frac{\Gamma\lB(\frac{d-2}{2}\rB)\Gamma\lB({1\over 2}\rB)}
           {\Gamma\lB(\frac{d-1}{2}\rB)}
    \dots
    \frac{\Gamma\lB(1\rB)\Gamma\lB({1\over 2}\rB)}
           {\Gamma\lB(\frac{3}{2}\rB)}
    2\pi\nonumber\\
    & = \frac{\pi \Gamma\lB({\frac{1}{2}}\rB)^{d/2-1} r^d}
        {\frac{d}{2}\Gamma\lB(\frac{d}{2}\rB)}\nonumber\\
        & = \lB(\frac{\pi^{d/2}}{\Gamma\lB(\frac{d}{2}+1\rB)}
        \rB) r^d
\end{align*}
\end{frame}
