\lecture{dimreduction}{dimreduction}

\date{Chapter 7: Dimensionality Reduction}

\begin{frame}
\titlepage
\end{frame}


\begin{frame}{Dimensionality Reduction}
  The goal of dimensionality reduction is to find a lower dimensional
  representation of the data matrix $\bD$ to avoid the curse
  of dimensionality.

\medskip
Given $n \times d$ data matrix, 
each
point $\bx_i =
(x_{i1}, x_{i2}, \ldots, x_{id})^T$ is a vector in the
ambient
$d$-dimensional vector space spanned by the $d$ standard basis
vectors $\be_1, \be_2, \ldots, \be_d$.

\medskip
Given any other set of $d$
orthonormal vectors $\bu_1, \bu_2, \ldots, \bu_d$
we can
re-express each point $\bx$ as
\begin{align*}
\tcbhighmath{
    \bx = a_1 \bu_1 + a_2 \bu_2 + \cdots + a_d \bu_d
}
\end{align*}
where $\ba = (a_1, a_2, \ldots, a_d)^T$ represents the
coordinates of $\bx$ in the new basis. 
More compactly:
\begin{align*}
\tcbhighmath{
    \bx = \bU \ba
}
\end{align*}
where $\bU$ is the $d \times d$ orthogonal matrix, 
whose $i$th column comprises the $i$th basis vector $\bu_i$.
Thus $\bU^{-1} = \bU^T$, and we have
\begin{empheq}[box=\tcbhighmath]{align*}
\begin{split}
    \ba &= \bU^T \bx
\end{split}
\end{empheq}
\end{frame}
%


\begin{frame}{Optimal Basis: Projection in Lower Dimensional Space}

There are potentially inf\/{i}nite choices for the
 orthonormal
basis vectors. Our goal is to choose an 
{\em optimal} basis that preserves essential information about $\bD$.

\medskip
We are interested in f\/{i}nding the
optimal $r$-dimensional representation
of $\bD$, with $r \ll d$.
Projection of $\bx$ onto the f\/{i}rst $r$
basis vectors is given as
\begin{align*}
    \bx' & = a_1 \bu_1 + a_2 \bu_2 + \cdots + a_r \bu_r =
    \sum_{i=1}^r a_i \bu_i
	= \bU_r \ba_r
\end{align*}
where $\bU_r$ and $\ba_r$ comprises the $r$ basis vectors and
coordinates, respv. 
Also, restricting $\ba = \bU^T \bx$ to $r$ terms, we have
\begin{align*}	
\ba_r & = \bU_r^T \bx
\end{align*}


\medskip
The $r$-dimensional projection of $\bx$ is thus given as:
\begin{align*}
\tcbhighmath{
    \bx' = \bU_r\bU_r^T \bx = \bP_r \bx
}
\end{align*}
where $\bP_r = \bU_r \bU_r^T = \sum_{i=1}^r \bu_i \bu_i^T$ is the {\em
orthogonal projection
matrix} for the subspace spanned by the f\/{i}rst $r$ basis vectors.

\end{frame}


\begin{frame}{Optimal Basis: Error Vector}

Given the projected vector $\bx' = \bP_r \bx$, the corresponding
{\em error vector}, is the projection onto the remaining $d-r$ basis
vectors
\begin{align*}
\tcbhighmath{
    \bepsilon = \sum_{i=r+1}^d a_i \bu_i = \bx - \bx'
}
\end{align*}
The error vector $\bepsilon$ is orthogonal to $\bx'$.


\medskip
The goal of dimensionality reduction is to
seek an $r$-dimensional basis that gives the best
possible approximation $\bx_i'$
over all the points $\bx_i \in \bD$.
Alternatively, we seek to minimize the error $\bepsilon_i =
\bx_i - \bx_i'$ over all the points.
\end{frame}



\readdata{\dataSLWPL}{EDA/dimreduction/figs/iris-slwpl-c.dat}

\begin{frame}{Iris Data: Optimal One-dimensional Basis}

  \begin{center}
  \begin{tabular}{ccc}
\psset{unit=0.5in}
\psset{arrowscale=2}
\psset{Alpha=60,Beta=-30}
%\psset{Alpha=-190,Beta=-45}
\psset{nameX=$~$, nameY=$~$, nameZ=$~$}
%\subfloat[Original Basis]{
%\label{fig:eda:dimr:3dirisOrig}
\scalebox{0.55}{
\begin{pspicture}(-2,-4.5)(2,4.5)
\pstThreeDCoor[xMin=-2, xMax= 2, yMin=-2,
        yMax=2, zMin=-3, zMax=4, Dx=0.5, Dy=0.5, Dz=1,
        linewidth=2pt,linecolor=black]
\pstThreeDPut(2.3,0,0){$X_1$}
\pstThreeDPut(0,2.5,0){$X_2$}
\pstThreeDPut(0,0,4.3){$X_3$}
\psset{dotstyle=Bo,dotscale=1.75,fillcolor=lightgray}
\dataplotThreeD[plotstyle=dots,showpoints=true]{\dataSLWPL}
\pstThreeDBox[linecolor=gray](-2,-2,-3)(4,0,0)(0,4,0)(0,0,7)
\end{pspicture}
}%}
&
\hspace{1in}
&
\psset{unit=0.5in}
\psset{arrowscale=2}
%\psset{Alpha=-190,Beta=-45}
\psset{Alpha=60,Beta=-30}
\psset{nameX=$~$, nameY=$~$, nameZ=$~$}
\scalebox{0.55}{%
\begin{pspicture}(-2,-4.5)(2,4.5)
\pstThreeDCoor[xMin=-2, xMax= 2, yMin=-2,
        yMax=2, zMin=-3, zMax=4, Dx=0.5, Dy=0.5, Dz=1,
        linewidth=1pt, linecolor=black]
\pstThreeDPut(2.3,0,0){$X_1$}
\pstThreeDPut(0,2.5,0){$X_2$}
\pstThreeDPut(0,0,4.3){$X_3$}
\pstThreeDLine[linecolor=gray](0.057, -0.054, 0.441)(0.168, -0.038, 0.395)
\pstThreeDLine[linecolor=gray](1.057, 0.046, 1.141)(0.567, -0.129, 1.333)
\pstThreeDLine[linecolor=gray](0.757, -0.154, 0.841)(0.421, -0.096, 0.990)
\pstThreeDLine[linecolor=gray](-1.243, 0.146, -2.359)(-1.038, 0.236, -2.438)
\pstThreeDLine[linecolor=gray](0.157, -0.854, 0.241)(0.140, -0.032, 0.328)
\pstThreeDLine[linecolor=gray](-1.143, 0.146, -2.459)(-1.058, 0.240, -2.486)
\pstThreeDLine[linecolor=gray](0.657, -0.054, 2.041)(0.832, -0.189, 1.954)
\pstThreeDLine[linecolor=gray](-0.043, -0.354, 1.341)(0.485, -0.110, 1.140)
\pstThreeDLine[linecolor=gray](0.857, 0.046, 1.841)(0.787, -0.179, 1.849)
\pstThreeDLine[linecolor=gray](0.857, -0.554, 2.041)(0.879, -0.200, 2.066)
\pstThreeDLine[linecolor=gray](-0.743, 0.646, -2.259)(-0.943, 0.214, -2.215)
\pstThreeDLine[linecolor=gray](-0.743, 0.746, -2.259)(-0.947, 0.215, -2.224)
\pstThreeDLine[linecolor=gray](-0.143, -0.054, 0.441)(0.138, -0.031, 0.324)
\pstThreeDLine[linecolor=gray](0.257, -0.054, 1.141)(0.449, -0.102, 1.055)
\pstThreeDLine[linecolor=gray](-0.943, 0.046, -2.259)(-0.953, 0.217, -2.238)
\pstThreeDLine[linecolor=gray](-0.843, 0.346, -2.159)(-0.912, 0.207, -2.143)
\pstThreeDLine[linecolor=gray](-0.843, 0.346, -2.259)(-0.948, 0.215, -2.227)
\pstThreeDLine[linecolor=gray](-0.143, -0.254, 0.341)(0.109, -0.025, 0.256)
\pstThreeDLine[linecolor=gray](-0.843, 0.246, -2.359)(-0.980, 0.223, -2.303)
\pstThreeDLine[linecolor=gray](1.357, 0.146, 2.241)(1.003, -0.228, 2.356)
\pstThreeDLine[linecolor=gray](0.057, -0.054, 1.341)(0.490, -0.111, 1.151)
\pstThreeDLine[linecolor=gray](0.657, -0.054, 1.441)(0.617, -0.140, 1.450)
\pstThreeDLine[linecolor=gray](-0.143, 1.346, -2.259)(-0.876, 0.199, -2.058)
\pstThreeDLine[linecolor=gray](-0.343, -0.554, 0.241)(0.053, -0.012, 0.125)
\pstThreeDLine[linecolor=gray](-0.943, -0.554, 0.741)(0.141, -0.032, 0.330)
\pstThreeDLine[linecolor=gray](-0.843, 0.446, -2.159)(-0.916, 0.208, -2.151)
\pstThreeDLine[linecolor=gray](-0.343, -0.754, 0.241)(0.060, -0.014, 0.141)
\pstThreeDLine[linecolor=gray](-1.243, 0.046, -2.259)(-0.998, 0.227, -2.345)
\pstThreeDLine[linecolor=gray](1.357, -0.054, 2.041)(0.938, -0.213, 2.204)
\pstThreeDLine[linecolor=gray](0.957, 0.146, 2.141)(0.906, -0.206, 2.129)
\pstThreeDLine[linecolor=gray](-0.443, 0.846, -2.459)(-0.976, 0.222, -2.292)
\pstThreeDLine[linecolor=gray](-0.843, 0.146, -2.559)(-1.048, 0.238, -2.462)
\pstThreeDLine[linecolor=gray](-0.143, -0.554, 1.241)(0.441, -0.100, 1.036)
\pstThreeDLine[linecolor=gray](-0.043, -0.454, 0.241)(0.095, -0.022, 0.224)
\pstThreeDLine[linecolor=gray](-0.743, -0.554, -0.759)(-0.365, 0.083, -0.858)
\pstThreeDLine[linecolor=gray](-0.243, -0.554, 0.141)(0.033, -0.007, 0.077)
\pstThreeDLine[linecolor=gray](-0.043, -0.354, 0.341)(0.128, -0.029, 0.300)
\pstThreeDLine[linecolor=gray](-0.743, 0.746, -1.859)(-0.804, 0.183, -1.888)
\pstThreeDLine[linecolor=gray](0.457, -0.754, 0.641)(0.325, -0.074, 0.763)
\pstThreeDLine[linecolor=gray](0.457, -0.554, 1.241)(0.533, -0.121, 1.251)
\pstThreeDLine[linecolor=gray](-0.243, -0.054, 0.741)(0.230, -0.052, 0.540)
\pstThreeDLine[linecolor=gray](0.257, -0.054, 0.841)(0.342, -0.078, 0.803)
\pstThreeDLine[linecolor=gray](0.957, -0.054, 1.741)(0.770, -0.175, 1.809)
\pstThreeDLine[linecolor=gray](1.457, -0.154, 2.541)(1.136, -0.258, 2.668)
\pstThreeDLine[linecolor=gray](-0.243, -0.354, 0.441)(0.133, -0.030, 0.312)
\pstThreeDLine[linecolor=gray](-1.043, -0.054, -2.359)(-1.000, 0.227, -2.350)
\pstThreeDLine[linecolor=gray](1.257, -0.054, 2.141)(0.959, -0.218, 2.252)
\pstThreeDLine[linecolor=gray](-0.143, -0.454, -0.259)(-0.099, 0.022, -0.232)
\pstThreeDLine[linecolor=gray](-0.543, 0.646, -2.259)(-0.913, 0.207, -2.144)
\pstThreeDLine[linecolor=gray](-0.143, 0.746, -2.059)(-0.784, 0.178, -1.841)
\pstThreeDLine[linecolor=gray](-0.143, -0.154, 0.441)(0.141, -0.032, 0.332)
\pstThreeDLine[linecolor=gray](-0.243, -0.254, 1.141)(0.380, -0.086, 0.892)
\pstThreeDLine[linecolor=gray](-1.443, -0.054, -2.459)(-1.097, 0.249, -2.577)
\pstThreeDLine[linecolor=gray](0.457, 0.246, 2.241)(0.862, -0.196, 2.026)
\pstThreeDLine[linecolor=gray](-0.443, 0.346, -2.259)(-0.887, 0.202, -2.084)
\pstThreeDLine[linecolor=gray](0.457, 0.346, 1.841)(0.716, -0.163, 1.682)
\pstThreeDLine[linecolor=gray](1.057, 0.046, 1.641)(0.746, -0.170, 1.753)
\pstThreeDLine[linecolor=gray](1.857, -0.054, 2.341)(1.122, -0.255, 2.635)
\pstThreeDLine[linecolor=gray](0.257, -0.254, 0.941)(0.384, -0.087, 0.903)
\pstThreeDLine[linecolor=gray](-0.243, -0.154, -0.159)(-0.088, 0.020, -0.208)
\pstThreeDLine[linecolor=gray](0.257, -0.454, 1.841)(0.713, -0.162, 1.675)
\pstThreeDLine[linecolor=gray](0.557, -0.354, 1.541)(0.648, -0.147, 1.522)
\pstThreeDLine[linecolor=gray](-0.843, 0.446, -2.459)(-1.023, 0.232, -2.403)
\pstThreeDLine[linecolor=gray](-0.743, 0.246, -2.059)(-0.858, 0.195, -2.015)
\pstThreeDLine[linecolor=gray](-0.243, -0.054, 0.341)(0.087, -0.020, 0.204)
\pstThreeDLine[linecolor=gray](-0.443, -0.054, 0.741)(0.199, -0.045, 0.469)
\pstThreeDLine[linecolor=gray](-0.043, -0.254, 1.341)(0.482, -0.109, 1.132)
\pstThreeDLine[linecolor=gray](-0.943, 0.046, -2.259)(-0.953, 0.217, -2.238)
\pstThreeDLine[linecolor=gray](-1.243, 0.546, -2.759)(-1.195, 0.271, -2.806)
\pstThreeDLine[linecolor=gray](-0.643, 0.346, -2.359)(-0.953, 0.217, -2.239)
\pstThreeDLine[linecolor=gray](2.057, 0.746, 2.641)(1.232, -0.280, 2.893)
\pstThreeDLine[linecolor=gray](1.857, -0.454, 3.141)(1.422, -0.323, 3.339)
\pstThreeDLine[linecolor=gray](0.257, -0.254, 0.241)(0.134, -0.030, 0.315)
\pstThreeDLine[linecolor=gray](-0.343, 0.446, -2.459)(-0.947, 0.215, -2.224)
\pstThreeDLine[linecolor=gray](-1.243, 0.346, -2.359)(-1.045, 0.237, -2.454)
\pstThreeDLine[linecolor=gray](-1.143, 0.146, -2.159)(-0.951, 0.216, -2.234)
\pstThreeDLine[linecolor=gray](-1.443, -0.154, -2.359)(-1.058, 0.240, -2.485)
\pstThreeDLine[linecolor=gray](0.357, -0.254, 1.041)(0.435, -0.099, 1.023)
\pstThreeDLine[linecolor=gray](-1.043, -0.054, -2.359)(-1.000, 0.227, -2.350)
\pstThreeDLine[linecolor=gray](0.157, -0.154, 0.741)(0.294, -0.067, 0.691)
\pstThreeDLine[linecolor=gray](0.357, 0.346, 1.641)(0.629, -0.143, 1.478)
\pstThreeDLine[linecolor=gray](-0.843, -0.754, -0.459)(-0.266, 0.061, -0.626)
\pstThreeDLine[linecolor=gray](0.557, 0.146, 0.741)(0.345, -0.078, 0.810)
\pstThreeDLine[linecolor=gray](0.457, -0.154, 1.841)(0.733, -0.167, 1.722)
\pstThreeDLine[linecolor=gray](0.857, -0.054, 1.241)(0.576, -0.131, 1.353)
\pstThreeDLine[linecolor=gray](-0.843, -1.054, -0.259)(-0.184, 0.042, -0.433)
\pstThreeDLine[linecolor=gray](0.057, 0.146, 1.041)(0.376, -0.085, 0.883)
\pstThreeDLine[linecolor=gray](0.857, 0.246, 1.941)(0.816, -0.185, 1.917)
\pstThreeDLine[linecolor=gray](-0.443, 0.846, -2.059)(-0.833, 0.189, -1.956)
\pstThreeDLine[linecolor=gray](0.457, -0.354, 1.141)(0.490, -0.111, 1.151)
\pstThreeDLine[linecolor=gray](-1.043, 0.346, -2.159)(-0.943, 0.214, -2.214)
\pstThreeDLine[linecolor=gray](-1.443, 0.146, -2.459)(-1.104, 0.251, -2.593)
\pstThreeDLine[linecolor=gray](0.557, 0.146, 1.541)(0.631, -0.143, 1.482)
\pstThreeDLine[linecolor=gray](0.357, -0.854, 0.741)(0.349, -0.079, 0.820)
\pstThreeDLine[linecolor=gray](0.157, -0.854, 1.241)(0.497, -0.113, 1.168)
\pstThreeDLine[linecolor=gray](1.557, -0.254, 2.341)(1.083, -0.246, 2.544)
\pstThreeDLine[linecolor=gray](-0.943, -0.654, -0.459)(-0.285, 0.065, -0.669)
\pstThreeDLine[linecolor=gray](1.157, 0.146, 0.941)(0.508, -0.115, 1.192)
\pstThreeDLine[linecolor=gray](-0.343, -0.654, -0.059)(-0.051, 0.012, -0.119)
\pstThreeDLine[linecolor=gray](0.457, 0.246, 0.941)(0.398, -0.090, 0.934)
\pstThreeDLine[linecolor=gray](0.957, -0.254, 1.041)(0.527, -0.120, 1.237)
\pstThreeDLine[linecolor=gray](0.257, -0.154, 0.941)(0.381, -0.087, 0.895)
\pstThreeDLine[linecolor=gray](0.657, 0.146, 1.341)(0.575, -0.131, 1.350)
\pstThreeDLine[linecolor=gray](0.857, 0.246, 1.941)(0.816, -0.185, 1.917)
\pstThreeDLine[linecolor=gray](0.857, 0.046, 0.641)(0.358, -0.081, 0.841)
\pstThreeDLine[linecolor=gray](-1.043, 0.346, -1.859)(-0.835, 0.190, -1.962)
\pstThreeDLine[linecolor=gray](-0.943, -0.054, -2.359)(-0.985, 0.224, -2.314)
\pstThreeDLine[linecolor=gray](1.057, 0.146, 1.941)(0.850, -0.193, 1.997)
\pstThreeDLine[linecolor=gray](-1.343, -0.754, -2.459)(-1.058, 0.240, -2.484)
\pstThreeDLine[linecolor=gray](-1.543, -0.054, -2.659)(-1.184, 0.269, -2.781)
\pstThreeDLine[linecolor=gray](-0.643, -0.354, 0.141)(-0.035, 0.008, -0.083)
\pstThreeDLine[linecolor=gray](-0.843, 0.546, -2.359)(-0.991, 0.225, -2.327)
\pstThreeDLine[linecolor=gray](0.557, -0.154, 0.541)(0.284, -0.064, 0.666)
\pstThreeDLine[linecolor=gray](-0.643, 0.446, -2.259)(-0.921, 0.209, -2.163)
\pstThreeDLine[linecolor=gray](-0.043, -0.354, 0.141)(0.056, -0.013, 0.132)
\pstThreeDLine[linecolor=gray](-0.343, 1.146, -2.359)(-0.935, 0.213, -2.197)
\pstThreeDLine[linecolor=gray](1.757, -0.054, 2.841)(1.285, -0.292, 3.019)
\pstThreeDLine[linecolor=gray](0.457, -0.254, 1.341)(0.558, -0.127, 1.311)
\pstThreeDLine[linecolor=gray](0.557, 0.046, 1.741)(0.706, -0.160, 1.658)
\pstThreeDLine[linecolor=gray](0.457, -0.554, 1.141)(0.497, -0.113, 1.167)
\pstThreeDLine[linecolor=gray](-0.043, -0.354, 1.341)(0.485, -0.110, 1.140)
\pstThreeDLine[linecolor=gray](-0.843, -0.054, -2.159)(-0.898, 0.204, -2.110)
\pstThreeDLine[linecolor=gray](0.857, 0.046, 0.941)(0.465, -0.106, 1.093)
\pstThreeDLine[linecolor=gray](0.157, -0.354, 1.341)(0.516, -0.117, 1.211)
\pstThreeDLine[linecolor=gray](-0.743, 0.446, -2.359)(-0.972, 0.221, -2.283)
\pstThreeDLine[linecolor=gray](-1.043, 0.046, -2.159)(-0.932, 0.212, -2.190)
\pstThreeDLine[linecolor=gray](-0.143, -0.254, 0.741)(0.252, -0.057, 0.592)
\pstThreeDLine[linecolor=gray](-0.743, 0.746, -2.159)(-0.911, 0.207, -2.140)
\pstThreeDLine[linecolor=gray](0.757, -0.054, 0.641)(0.346, -0.079, 0.814)
\pstThreeDLine[linecolor=gray](0.557, -0.254, 1.841)(0.752, -0.171, 1.766)
\pstThreeDLine[linecolor=gray](-0.643, 1.046, -2.259)(-0.942, 0.214, -2.212)
\pstThreeDLine[linecolor=gray](0.557, -0.254, 1.841)(0.752, -0.171, 1.766)
\pstThreeDLine[linecolor=gray](1.857, -0.254, 2.941)(1.343, -0.305, 3.155)
\pstThreeDLine[linecolor=gray](-0.043, 0.946, -2.559)(-0.954, 0.217, -2.241)
\pstThreeDLine[linecolor=gray](-0.943, 0.046, -2.259)(-0.953, 0.217, -2.238)
\pstThreeDLine[linecolor=gray](-0.443, 0.646, -2.259)(-0.897, 0.204, -2.108)
\pstThreeDLine[linecolor=gray](-0.743, 0.446, -2.359)(-0.972, 0.221, -2.283)
\pstThreeDLine[linecolor=gray](0.157, 0.346, 0.741)(0.277, -0.063, 0.651)
\pstThreeDLine[linecolor=gray](0.657, -0.054, 1.741)(0.724, -0.165, 1.702)
\pstThreeDLine[linecolor=gray](-0.343, -0.654, 0.041)(-0.015, 0.003, -0.035)
\pstThreeDLine[linecolor=gray](1.357, 0.546, 2.341)(1.025, -0.233, 2.407)
\pstThreeDLine[linecolor=gray](1.057, 0.046, 1.341)(0.639, -0.145, 1.501)
\pstThreeDLine[linecolor=gray](0.357, -0.154, 0.541)(0.253, -0.058, 0.595)
\pstThreeDLine[linecolor=gray](0.657, -0.254, 0.841)(0.410, -0.093, 0.962)
\pstThreeDLine[linecolor=gray](0.157, -0.054, 1.041)(0.398, -0.090, 0.935)
\pstThreeDLine[linecolor=gray](-0.443, 0.346, -2.059)(-0.816, 0.185, -1.916)
\pstThreeDLine[linecolor=gray](-0.343, -0.454, 0.641)(0.193, -0.044, 0.453)
\pstThreeDLine[linecolor=gray](0.857, -0.054, 1.441)(0.648, -0.147, 1.521)
\pstThreeDLine[linecolor=gray](1.857, 0.746, 2.941)(1.309, -0.297, 3.074)
\pstThreeDLine[linecolor=gray](-0.743, 0.346, -2.259)(-0.933, 0.212, -2.191)

\psset{dotstyle=Bo,dotscale=1.75,fillcolor=lightgray}
\dataplotThreeD[plotstyle=dots,showpoints=true]{\dataSLWPL}
\pstThreeDLine[linewidth=2pt,arrows=->](1.43,-0.32,3.36)(-1.43,0.32,-3.36)
\pstThreeDPut(-1.6,0.35,-3.7){$\bu_1$}
%\psset{dotstyle=Bo,dotscale=1.75,fillcolor=white}
%\dataplotThreeD[plotstyle=dots,showpoints=true]{\dataIrisOneD}
\pstThreeDBox[linecolor=gray](-2,-2,-3)(4,0,0)(0,4,0)(0,0,7)
\end{pspicture}
}\\
Iris Data: 3D & & Optimal 1D Basis\\
\end{tabular}
\end{center}
\end{frame}



\readdata{\dataIrisDL}{EDA/dimreduction/figs/iris-2dL.dat}
\readdata{\dataIrisDR}{EDA/dimreduction/figs/iris-2dR.dat}
\begin{frame}{Iris Data: Optimal 2D Basis}
\psset{unit=0.5in}
\psset{arrowscale=2}
\psset{Alpha=60,Beta=-30}
  \begin{center}
  \begin{tabular}{ccc}
%\psset{Alpha=-190,Beta=-45}
\psset{nameX=$~$, nameY=$~$, nameZ=$~$}
%\subfloat[Original Basis]{
%\label{fig:eda:dimr:3dirisOrig}
\scalebox{0.6}{
\begin{pspicture}(-2,-4.5)(2,4.5)
\pstThreeDCoor[xMin=-2, xMax= 2, yMin=-2,
        yMax=2, zMin=-3, zMax=4, Dx=0.5, Dy=0.5, Dz=1,
        linewidth=2pt,linecolor=black]
\pstThreeDPut(2.3,0,0){$X_1$}
\pstThreeDPut(0,2.5,0){$X_2$}
\pstThreeDPut(0,0,4.3){$X_3$}
\psset{dotstyle=Bo,dotscale=1.75,fillcolor=lightgray}
\dataplotThreeD[plotstyle=dots,showpoints=true]{\dataSLWPL}
\pstThreeDBox[linecolor=gray](-2,-2,-3)(4,0,0)(0,4,0)(0,0,7)
\end{pspicture}
}%}
&
\hspace{1in}
&
\psset{unit=0.5in}
\psset{arrowscale=2,dotscale=1.75}
\psset{Alpha=60,Beta=-30}
\psset{nameX=$~$, nameY=$~$, nameZ=$~$}
%\subfloat[Optimal basis]{
%\label{fig:eda:dimr:2dirisA}
\scalebox{0.55}{%
\begin{pspicture}(-2,-4.5)(2,4.5)
%\pstThreeDSquare[fillstyle=solid, fillcolor=lightgray]
\pstThreeDSquare[fillstyle=vlines,
    hatchcolor=lightgray, hatchwidth=0.1\pslinewidth,
    hatchsep=3\pslinewidth]
(3.99,2.64,2.56)(-2.86,0.64,-6.72)(-5.12,-5.92, 1.6)
\psset{dotstyle=Bo,fillcolor=gray}
\dataplotThreeD[plotstyle=dots,showpoints=true]{\dataIrisDR}
%\pstThreeDSquare[
%fillstyle=solid, fillcolor=lightgray]
%(2.71,1.16,2.96)(-2.86,0.64,-6.72)(-2.56,-2.96, 0.8)
\pstThreeDCoor[xMin=-2, xMax= 2, yMin=-2,
        yMax=2, zMin=-3, zMax=4, Dx=0.5, Dy=0.5, Dz=1,
        linewidth=1pt,linecolor=black,arrowscale=2]
\pstThreeDPut(2.3,0,0){$X_1$}
\pstThreeDPut(0,2.5,0){$X_2$}
\pstThreeDPut(0,0,4.3){$X_3$}
\pstThreeDLine[linecolor=black](0.057, -0.054, 0.441)(0.109, -0.107, 0.414)
\pstThreeDLine[linecolor=gray](1.057, 0.046, 1.141)(0.875, 0.228, 1.236)
\pstThreeDLine[linecolor=gray](0.757, -0.154, 0.841)(0.550, 0.053, 0.950)
\pstThreeDLine[linecolor=black](-1.243, 0.146, -2.359)(-1.174, 0.077, -2.395)
\pstThreeDLine[linecolor=gray](0.157, -0.854, 0.241)(-0.233, -0.464, 0.445)
\pstThreeDLine[linecolor=gray](-1.143, 0.146, -2.459)(-1.141, 0.144, -2.460)
\pstThreeDLine[linecolor=black](0.657, -0.054, 2.041)(0.813, -0.211, 1.960)
\pstThreeDLine[linecolor=black](-0.043, -0.354, 1.341)(0.128, -0.525, 1.252)
\pstThreeDLine[linecolor=black](0.857, 0.046, 1.841)(0.923, -0.021, 1.807)
\pstThreeDLine[linecolor=gray](0.857, -0.554, 2.041)(0.705, -0.402, 2.120)
\pstThreeDLine[linecolor=black](-0.743, 0.646, -2.259)(-0.651, 0.554, -2.307)
\pstThreeDLine[linecolor=black](-0.743, 0.746, -2.259)(-0.607, 0.609, -2.330)
\pstThreeDLine[linecolor=black](-0.143, -0.054, 0.441)(-0.003, -0.195, 0.368)
\pstThreeDLine[linecolor=black](0.257, -0.054, 1.141)(0.382, -0.180, 1.076)
\pstThreeDLine[linecolor=gray](-0.943, 0.046, -2.259)(-1.027, 0.130, -2.215)
\pstThreeDLine[linecolor=black](-0.843, 0.346, -2.159)(-0.816, 0.319, -2.173)
\pstThreeDLine[linecolor=gray](-0.843, 0.346, -2.259)(-0.839, 0.342, -2.261)
\pstThreeDLine[linecolor=black](-0.143, -0.254, 0.341)(-0.114, -0.284, 0.326)
\pstThreeDLine[linecolor=gray](-0.843, 0.246, -2.359)(-0.906, 0.309, -2.326)
\pstThreeDLine[linecolor=gray](1.357, 0.146, 2.241)(1.340, 0.163, 2.250)
\pstThreeDLine[linecolor=black](0.057, -0.054, 1.341)(0.316, -0.314, 1.206)
\pstThreeDLine[linecolor=black](0.657, -0.054, 1.441)(0.675, -0.073, 1.432)
\pstThreeDLine[linecolor=black](-0.143, 1.346, -2.259)(-0.007, 1.209, -2.330)
\pstThreeDLine[linecolor=gray](-0.343, -0.554, 0.241)(-0.381, -0.516, 0.261)
\pstThreeDLine[linecolor=black](-0.943, -0.554, 0.741)(-0.603, -0.895, 0.563)
\pstThreeDLine[linecolor=black](-0.843, 0.446, -2.159)(-0.772, 0.375, -2.196)
\pstThreeDLine[linecolor=gray](-0.343, -0.754, 0.241)(-0.469, -0.628, 0.307)
\pstThreeDLine[linecolor=black](-1.243, 0.046, -2.259)(-1.195, -0.002, -2.284)
\pstThreeDLine[linecolor=gray](1.357, -0.054, 2.041)(1.206, 0.097, 2.120)
\pstThreeDLine[linecolor=black](0.957, 0.146, 2.141)(1.092, 0.010, 2.070)
\pstThreeDLine[linecolor=gray](-0.443, 0.846, -2.459)(-0.441, 0.843, -2.460)
\pstThreeDLine[linecolor=gray](-0.843, 0.146, -2.559)(-0.996, 0.299, -2.479)
\pstThreeDLine[linecolor=black](-0.143, -0.554, 1.241)(-0.039, -0.658, 1.187)
\pstThreeDLine[linecolor=gray](-0.043, -0.454, 0.241)(-0.169, -0.328, 0.307)
\pstThreeDLine[linecolor=gray](-0.743, -0.554, -0.759)(-0.835, -0.462, -0.711)
\pstThreeDLine[linecolor=gray](-0.243, -0.554, 0.141)(-0.348, -0.449, 0.196)
\pstThreeDLine[linecolor=gray](-0.043, -0.354, 0.341)(-0.102, -0.296, 0.372)
\pstThreeDLine[linecolor=black](-0.743, 0.746, -1.859)(-0.515, 0.517, -1.978)
\pstThreeDLine[linecolor=gray](0.457, -0.754, 0.641)(0.072, -0.368, 0.843)
\pstThreeDLine[linecolor=gray](0.457, -0.554, 1.241)(0.297, -0.394, 1.325)
\pstThreeDLine[linecolor=black](-0.243, -0.054, 0.741)(0.010, -0.308, 0.609)
\pstThreeDLine[linecolor=black](0.257, -0.054, 0.841)(0.313, -0.111, 0.812)
\pstThreeDLine[linecolor=gray](0.957, -0.054, 1.741)(0.912, -0.010, 1.764)
\pstThreeDLine[linecolor=gray](1.457, -0.154, 2.541)(1.332, -0.030, 2.606)
\pstThreeDLine[linecolor=black](-0.243, -0.354, 0.441)(-0.191, -0.407, 0.414)
\pstThreeDLine[linecolor=gray](-1.043, -0.054, -2.359)(-1.150, 0.053, -2.303)
\pstThreeDLine[linecolor=gray](1.257, -0.054, 2.141)(1.172, 0.030, 2.185)
\pstThreeDLine[linecolor=gray](-0.143, -0.454, -0.259)(-0.340, -0.257, -0.156)
\pstThreeDLine[linecolor=gray](-0.543, 0.646, -2.259)(-0.539, 0.642, -2.261)
\pstThreeDLine[linecolor=gray](-0.143, 0.746, -2.059)(-0.225, 0.827, -2.016)
\pstThreeDLine[linecolor=black](-0.143, -0.154, 0.441)(-0.047, -0.251, 0.391)
\pstThreeDLine[linecolor=black](-0.243, -0.254, 1.141)(0.014, -0.512, 1.007)
\pstThreeDLine[linecolor=black](-1.443, -0.054, -2.459)(-1.398, -0.100, -2.483)
\pstThreeDLine[linecolor=black](0.457, 0.246, 2.241)(0.879, -0.177, 2.021)
\pstThreeDLine[linecolor=gray](-0.443, 0.346, -2.259)(-0.615, 0.518, -2.169)
\pstThreeDLine[linecolor=black](0.457, 0.346, 1.841)(0.831, -0.029, 1.646)
\pstThreeDLine[linecolor=gray](1.057, 0.046, 1.641)(0.990, 0.113, 1.676)
\pstThreeDLine[linecolor=gray](1.857, -0.054, 2.341)(1.555, 0.248, 2.499)
\pstThreeDLine[linecolor=gray](0.257, -0.254, 0.941)(0.248, -0.246, 0.946)
\pstThreeDLine[linecolor=black](-0.243, -0.154, -0.159)(-0.241, -0.157, -0.160)
\pstThreeDLine[linecolor=black](0.257, -0.454, 1.841)(0.367, -0.564, 1.784)
\pstThreeDLine[linecolor=gray](0.557, -0.354, 1.541)(0.510, -0.307, 1.566)
\pstThreeDLine[linecolor=gray](-0.843, 0.446, -2.459)(-0.841, 0.444, -2.460)
\pstThreeDLine[linecolor=gray](-0.743, 0.246, -2.059)(-0.781, 0.284, -2.039)
\pstThreeDLine[linecolor=black](-0.243, -0.054, 0.341)(-0.082, -0.216, 0.257)
\pstThreeDLine[linecolor=black](-0.443, -0.054, 0.741)(-0.102, -0.396, 0.563)
\pstThreeDLine[linecolor=black](-0.043, -0.254, 1.341)(0.172, -0.470, 1.229)
\pstThreeDLine[linecolor=gray](-0.943, 0.046, -2.259)(-1.027, 0.130, -2.215)
\pstThreeDLine[linecolor=black](-1.243, 0.546, -2.759)(-1.090, 0.393, -2.839)
\pstThreeDLine[linecolor=gray](-0.643, 0.346, -2.359)(-0.750, 0.453, -2.303)
\pstThreeDLine[linecolor=black](2.057, 0.746, 2.641)(2.088, 0.715, 2.625)
\pstThreeDLine[linecolor=gray](1.857, -0.454, 3.141)(1.563, -0.159, 3.295)
\pstThreeDLine[linecolor=gray](0.257, -0.254, 0.241)(0.088, -0.085, 0.330)
\pstThreeDLine[linecolor=gray](-0.343, 0.446, -2.459)(-0.561, 0.664, -2.345)
\pstThreeDLine[linecolor=black](-1.243, 0.346, -2.359)(-1.086, 0.189, -2.441)
\pstThreeDLine[linecolor=black](-1.143, 0.146, -2.159)(-1.072, 0.075, -2.196)
\pstThreeDLine[linecolor=black](-1.443, -0.154, -2.359)(-1.419, -0.179, -2.372)
\pstThreeDLine[linecolor=gray](0.357, -0.254, 1.041)(0.327, -0.225, 1.057)
\pstThreeDLine[linecolor=gray](-1.043, -0.054, -2.359)(-1.150, 0.053, -2.303)
\pstThreeDLine[linecolor=black](0.157, -0.154, 0.741)(0.190, -0.188, 0.724)
\pstThreeDLine[linecolor=black](0.357, 0.346, 1.641)(0.729, -0.027, 1.447)
\pstThreeDLine[linecolor=gray](-0.843, -0.754, -0.459)(-0.910, -0.687, -0.424)
\pstThreeDLine[linecolor=gray](0.557, 0.146, 0.741)(0.547, 0.156, 0.747)
\pstThreeDLine[linecolor=black](0.457, -0.154, 1.841)(0.611, -0.309, 1.761)
\pstThreeDLine[linecolor=gray](0.857, -0.054, 1.241)(0.742, 0.061, 1.301)
\pstThreeDLine[linecolor=gray](-0.843, -1.054, -0.259)(-0.996, -0.901, -0.179)
\pstThreeDLine[linecolor=black](0.057, 0.146, 1.041)(0.335, -0.133, 0.896)
\pstThreeDLine[linecolor=black](0.857, 0.246, 1.941)(1.034, 0.068, 1.849)
\pstThreeDLine[linecolor=black](-0.443, 0.846, -2.059)(-0.349, 0.751, -2.108)
\pstThreeDLine[linecolor=gray](0.457, -0.354, 1.141)(0.362, -0.259, 1.191)
\pstThreeDLine[linecolor=black](-1.043, 0.346, -2.159)(-0.928, 0.231, -2.219)
\pstThreeDLine[linecolor=black](-1.443, 0.146, -2.459)(-1.310, 0.012, -2.529)
\pstThreeDLine[linecolor=black](0.557, 0.146, 1.541)(0.730, -0.028, 1.451)
\pstThreeDLine[linecolor=gray](0.357, -0.854, 0.741)(-0.006, -0.491, 0.931)
\pstThreeDLine[linecolor=gray](0.157, -0.854, 1.241)(-0.003, -0.694, 1.325)
\pstThreeDLine[linecolor=gray](1.557, -0.254, 2.341)(1.299, 0.005, 2.476)
\pstThreeDLine[linecolor=black](-0.943, -0.654, -0.459)(-0.922, -0.675, -0.470)
\pstThreeDLine[linecolor=gray](1.157, 0.146, 0.941)(0.929, 0.374, 1.060)
\pstThreeDLine[linecolor=gray](-0.343, -0.654, -0.059)(-0.494, -0.503, 0.020)
\pstThreeDLine[linecolor=black](0.457, 0.246, 0.941)(0.580, 0.122, 0.877)
\pstThreeDLine[linecolor=gray](0.957, -0.254, 1.041)(0.664, 0.039, 1.194)
\pstThreeDLine[linecolor=black](0.257, -0.154, 0.941)(0.292, -0.190, 0.923)
\pstThreeDLine[linecolor=black](0.657, 0.146, 1.341)(0.740, 0.062, 1.298)
\pstThreeDLine[linecolor=black](0.857, 0.246, 1.941)(1.034, 0.068, 1.849)
\pstThreeDLine[linecolor=gray](0.857, 0.046, 0.641)(0.648, 0.255, 0.750)
\pstThreeDLine[linecolor=black](-1.043, 0.346, -1.859)(-0.859, 0.162, -1.955)
\pstThreeDLine[linecolor=gray](-0.943, -0.054, -2.359)(-1.094, 0.097, -2.280)
\pstThreeDLine[linecolor=black](1.057, 0.146, 1.941)(1.102, 0.100, 1.917)
\pstThreeDLine[linecolor=gray](-1.343, -0.754, -2.459)(-1.649, -0.447, -2.299)
\pstThreeDLine[linecolor=black](-1.543, -0.054, -2.659)(-1.500, -0.098, -2.682)
\pstThreeDLine[linecolor=black](-0.643, -0.354, 0.141)(-0.484, -0.514, 0.058)
\pstThreeDLine[linecolor=black](-0.843, 0.546, -2.359)(-0.774, 0.477, -2.395)
\pstThreeDLine[linecolor=gray](0.557, -0.154, 0.541)(0.369, 0.034, 0.640)
\pstThreeDLine[linecolor=gray](-0.643, 0.446, -2.259)(-0.683, 0.486, -2.238)
\pstThreeDLine[linecolor=gray](-0.043, -0.354, 0.141)(-0.148, -0.250, 0.196)
\pstThreeDLine[linecolor=black](-0.343, 1.146, -2.359)(-0.230, 1.032, -2.418)
\pstThreeDLine[linecolor=gray](1.757, -0.054, 2.841)(1.614, 0.089, 2.916)
\pstThreeDLine[linecolor=black](0.457, -0.254, 1.341)(0.452, -0.250, 1.344)
\pstThreeDLine[linecolor=black](0.557, 0.046, 1.741)(0.732, -0.130, 1.650)
\pstThreeDLine[linecolor=gray](0.457, -0.554, 1.141)(0.274, -0.371, 1.237)
\pstThreeDLine[linecolor=black](-0.043, -0.354, 1.341)(0.128, -0.525, 1.252)
\pstThreeDLine[linecolor=gray](-0.843, -0.054, -2.159)(-0.992, 0.095, -2.081)
\pstThreeDLine[linecolor=gray](0.857, 0.046, 0.941)(0.717, 0.186, 1.014)
\pstThreeDLine[linecolor=black](0.157, -0.354, 1.341)(0.240, -0.437, 1.298)
\pstThreeDLine[linecolor=gray](-0.743, 0.446, -2.359)(-0.762, 0.465, -2.349)
\pstThreeDLine[linecolor=gray](-1.043, 0.046, -2.159)(-1.060, 0.063, -2.150)
\pstThreeDLine[linecolor=black](-0.143, -0.254, 0.741)(-0.022, -0.376, 0.678)
\pstThreeDLine[linecolor=black](-0.743, 0.746, -2.159)(-0.584, 0.586, -2.242)
\pstThreeDLine[linecolor=gray](0.757, -0.054, 0.641)(0.548, 0.155, 0.750)
\pstThreeDLine[linecolor=black](0.557, -0.254, 1.841)(0.623, -0.321, 1.807)
\pstThreeDLine[linecolor=black](-0.643, 1.046, -2.259)(-0.419, 0.821, -2.376)
\pstThreeDLine[linecolor=black](0.557, -0.254, 1.841)(0.623, -0.321, 1.807)
\pstThreeDLine[linecolor=gray](1.857, -0.254, 2.941)(1.605, -0.001, 3.073)
\pstThreeDLine[linecolor=gray](-0.043, 0.946, -2.559)(-0.195, 1.098, -2.479)
\pstThreeDLine[linecolor=gray](-0.943, 0.046, -2.259)(-1.027, 0.130, -2.215)
\pstThreeDLine[linecolor=gray](-0.443, 0.646, -2.259)(-0.483, 0.686, -2.238)
\pstThreeDLine[linecolor=gray](-0.743, 0.446, -2.359)(-0.762, 0.465, -2.349)
\pstThreeDLine[linecolor=black](0.157, 0.346, 0.741)(0.410, 0.092, 0.609)
\pstThreeDLine[linecolor=black](0.657, -0.054, 1.741)(0.744, -0.142, 1.696)
\pstThreeDLine[linecolor=gray](-0.343, -0.654, 0.041)(-0.471, -0.526, 0.108)
\pstThreeDLine[linecolor=black](1.357, 0.546, 2.341)(1.538, 0.364, 2.246)
\pstThreeDLine[linecolor=gray](1.057, 0.046, 1.341)(0.921, 0.182, 1.412)
\pstThreeDLine[linecolor=gray](0.357, -0.154, 0.541)(0.257, -0.054, 0.594)
\pstThreeDLine[linecolor=gray](0.657, -0.254, 0.841)(0.450, -0.047, 0.950)
\pstThreeDLine[linecolor=black](0.157, -0.054, 1.041)(0.303, -0.201, 0.965)
\pstThreeDLine[linecolor=gray](-0.443, 0.346, -2.059)(-0.569, 0.472, -1.993)
\pstThreeDLine[linecolor=black](-0.343, -0.454, 0.641)(-0.245, -0.552, 0.590)
\pstThreeDLine[linecolor=gray](0.857, -0.054, 1.441)(0.787, 0.015, 1.477)
\pstThreeDLine[linecolor=black](1.857, 0.746, 2.941)(2.045, 0.558, 2.843)
\pstThreeDLine[linecolor=gray](-0.743, 0.346, -2.259)(-0.783, 0.386, -2.238)

\pstThreeDLine[linewidth=2pt,arrows=->](1.43,-0.32,3.36)(-1.43,0.32,-3.36)
\pstThreeDLine[linewidth=2pt,arrows=->](2.56,2.96,-0.8)(-2.56,-2.96,0.8)
\pstThreeDPut(-1.6,0.35,-3.7){$\bu_1$}
\pstThreeDPut(-2.88,-3.33,0.9){$\bu_2$}
\psset{dotstyle=Bo,fillcolor=white}
\dataplotThreeD[plotstyle=dots,showpoints=true]{\dataIrisDL}
\pstThreeDBox[linecolor=gray](-2,-2,-3)(4,0,0)(0,4,0)(0,0,7)
%\pstThreeDBox[](-3.5,-2.5,-3)(5,0,0)(0,4,0)(0,0,7.5)
\end{pspicture}
}\\
Iris Data (3D) & & Optimal 2D Basis\\
\end{tabular}
\end{center}
\end{frame}

\ifdefined\wox \begin{frame} \titlepage \end{frame} \fi

\begin{frame}{Principal Component Analysis}

Principal Component Analysis (PCA) is a technique that
seeks a $r$-dimensional basis that best captures the
variance in the data.

\medskip
The direction with the largest projected variance is called
the f\/{i}rst principal component.

\medskip
The orthogonal direction that captures
the second largest projected variance is called the second principal component,
and so on. 

\medskip
The direction that maximizes the
variance is also the one that minimizes the mean squared
error.
\end{frame}





\begin{frame}{Principal Component: Direction of Most Variance}
We seek to find the unit vector $\bu$ that maximizes the projected
variance of the points. Let $\bD$ be centered, and let $\cov$ be its
covariance matrix.

\medskip
The projection of $\bx_i$ on $\bu$ is given as
\begin{align*}
  \bx_i' =
  \lB(\frac{\bu^T \bx_i}{\bu^T\bu}\rB) \bu = (\bu^T \bx_i) \bu =
  a_i \bu
\end{align*}

\medskip
Across all the points, the 
projected variance along $\bu$ is
\begin{align*}
\tcbhighmath{
  \sigma^2_\bu  
  = \frac{1}{n}\sum_{i=1}^n (a_i-\mu_\bu)^2 
  = \frac{1}{n} \sum_{i=1}^n \bu^T \lB(\bx_{i}\bx_i^T\rB) \bu
   =  \bu^T \lB(\frac{1}{n}\sum_{i=1}^n \bx_i\bx_i^T\rB) \bu
  =  \bu^T \cov \bu
}
\end{align*}


\medskip
We have to find the
optimal basis vector $\bu$ that maximizes the projected variance 
$\sigma^2_\bu = \bu^T \cov \bu$, subject to the 
constraint that $\bu^T\bu=1$. The maximization objective is given as
\begin{align*}
  \max_\bu J(\bu) = 
  \bu^T \cov \bu - \alpha (\bu^T\bu-1)
\end{align*}

\end{frame}


\begin{frame}{Principal Component: Direction of Most Variance}

  Given the objective $\max_\bu J(\bu) =  
  \bu^T \cov \bu - \alpha (\bu^T\bu-1)$, we solve it by
setting the derivative of $J(\bu)$
with respect to $\bu$ to the zero vector, to obtain
\begin{align*}
  \frac{\partial}{\partial \bu} \lB(\bu^T \cov \bu - \alpha
  (\bu^T\bu-1)\rB) & = \bzero\\
  \text{that is, }2 \cov \bu - 2 \alpha \bu  &= \bzero \\
  \text{which implies } 
\tcbhighmath{
\cov \bu  = \alpha \bu
}
\end{align*}
Thus $\alpha$ is an eigenvalue of the covariance matrix
$\cov$, with the associated eigenvector $\bu$.

\medskip
Taking the dot product with $\bu$ on both sides, we have
\begin{align*}
\sigma^2_\bu  = \bu^T\cov\bu  \bu^T \alpha \bu = 
\alpha \bu^T\bu =  \alpha
\end{align*}

\medskip
To maximize the projected variance $\sigma_\bu^2$,
we thus choose the largest eigenvalue $\lambda_1$ of $\cov$, and
the dominant eigenvector $\bu_1$ specif\/{i}es the
direction of most variance, also called the
{\em f\/{i}rst principal
component}.
\end{frame}


\begin{frame}{Iris Data: First Principal Component}
  \centerline{
\psset{unit=0.5in}
\psset{arrowscale=2}
%\psset{Alpha=-190,Beta=-45}
\psset{Alpha=60,Beta=-30}
\psset{nameX=$~$, nameY=$~$, nameZ=$~$}
\scalebox{0.65}{%
\begin{pspicture}(-2,-4.5)(2,4.5)
\pstThreeDCoor[xMin=-2, xMax= 2, yMin=-2,
        yMax=2, zMin=-3, zMax=4, Dx=0.5, Dy=0.5, Dz=1,
        linewidth=1pt, linecolor=black]
\pstThreeDPut(2.3,0,0){$X_1$}
\pstThreeDPut(0,2.5,0){$X_2$}
\pstThreeDPut(0,0,4.3){$X_3$}
\pstThreeDLine[linecolor=gray](0.057, -0.054, 0.441)(0.168, -0.038, 0.395)
\pstThreeDLine[linecolor=gray](1.057, 0.046, 1.141)(0.567, -0.129, 1.333)
\pstThreeDLine[linecolor=gray](0.757, -0.154, 0.841)(0.421, -0.096, 0.990)
\pstThreeDLine[linecolor=gray](-1.243, 0.146, -2.359)(-1.038, 0.236, -2.438)
\pstThreeDLine[linecolor=gray](0.157, -0.854, 0.241)(0.140, -0.032, 0.328)
\pstThreeDLine[linecolor=gray](-1.143, 0.146, -2.459)(-1.058, 0.240, -2.486)
\pstThreeDLine[linecolor=gray](0.657, -0.054, 2.041)(0.832, -0.189, 1.954)
\pstThreeDLine[linecolor=gray](-0.043, -0.354, 1.341)(0.485, -0.110, 1.140)
\pstThreeDLine[linecolor=gray](0.857, 0.046, 1.841)(0.787, -0.179, 1.849)
\pstThreeDLine[linecolor=gray](0.857, -0.554, 2.041)(0.879, -0.200, 2.066)
\pstThreeDLine[linecolor=gray](-0.743, 0.646, -2.259)(-0.943, 0.214, -2.215)
\pstThreeDLine[linecolor=gray](-0.743, 0.746, -2.259)(-0.947, 0.215, -2.224)
\pstThreeDLine[linecolor=gray](-0.143, -0.054, 0.441)(0.138, -0.031, 0.324)
\pstThreeDLine[linecolor=gray](0.257, -0.054, 1.141)(0.449, -0.102, 1.055)
\pstThreeDLine[linecolor=gray](-0.943, 0.046, -2.259)(-0.953, 0.217, -2.238)
\pstThreeDLine[linecolor=gray](-0.843, 0.346, -2.159)(-0.912, 0.207, -2.143)
\pstThreeDLine[linecolor=gray](-0.843, 0.346, -2.259)(-0.948, 0.215, -2.227)
\pstThreeDLine[linecolor=gray](-0.143, -0.254, 0.341)(0.109, -0.025, 0.256)
\pstThreeDLine[linecolor=gray](-0.843, 0.246, -2.359)(-0.980, 0.223, -2.303)
\pstThreeDLine[linecolor=gray](1.357, 0.146, 2.241)(1.003, -0.228, 2.356)
\pstThreeDLine[linecolor=gray](0.057, -0.054, 1.341)(0.490, -0.111, 1.151)
\pstThreeDLine[linecolor=gray](0.657, -0.054, 1.441)(0.617, -0.140, 1.450)
\pstThreeDLine[linecolor=gray](-0.143, 1.346, -2.259)(-0.876, 0.199, -2.058)
\pstThreeDLine[linecolor=gray](-0.343, -0.554, 0.241)(0.053, -0.012, 0.125)
\pstThreeDLine[linecolor=gray](-0.943, -0.554, 0.741)(0.141, -0.032, 0.330)
\pstThreeDLine[linecolor=gray](-0.843, 0.446, -2.159)(-0.916, 0.208, -2.151)
\pstThreeDLine[linecolor=gray](-0.343, -0.754, 0.241)(0.060, -0.014, 0.141)
\pstThreeDLine[linecolor=gray](-1.243, 0.046, -2.259)(-0.998, 0.227, -2.345)
\pstThreeDLine[linecolor=gray](1.357, -0.054, 2.041)(0.938, -0.213, 2.204)
\pstThreeDLine[linecolor=gray](0.957, 0.146, 2.141)(0.906, -0.206, 2.129)
\pstThreeDLine[linecolor=gray](-0.443, 0.846, -2.459)(-0.976, 0.222, -2.292)
\pstThreeDLine[linecolor=gray](-0.843, 0.146, -2.559)(-1.048, 0.238, -2.462)
\pstThreeDLine[linecolor=gray](-0.143, -0.554, 1.241)(0.441, -0.100, 1.036)
\pstThreeDLine[linecolor=gray](-0.043, -0.454, 0.241)(0.095, -0.022, 0.224)
\pstThreeDLine[linecolor=gray](-0.743, -0.554, -0.759)(-0.365, 0.083, -0.858)
\pstThreeDLine[linecolor=gray](-0.243, -0.554, 0.141)(0.033, -0.007, 0.077)
\pstThreeDLine[linecolor=gray](-0.043, -0.354, 0.341)(0.128, -0.029, 0.300)
\pstThreeDLine[linecolor=gray](-0.743, 0.746, -1.859)(-0.804, 0.183, -1.888)
\pstThreeDLine[linecolor=gray](0.457, -0.754, 0.641)(0.325, -0.074, 0.763)
\pstThreeDLine[linecolor=gray](0.457, -0.554, 1.241)(0.533, -0.121, 1.251)
\pstThreeDLine[linecolor=gray](-0.243, -0.054, 0.741)(0.230, -0.052, 0.540)
\pstThreeDLine[linecolor=gray](0.257, -0.054, 0.841)(0.342, -0.078, 0.803)
\pstThreeDLine[linecolor=gray](0.957, -0.054, 1.741)(0.770, -0.175, 1.809)
\pstThreeDLine[linecolor=gray](1.457, -0.154, 2.541)(1.136, -0.258, 2.668)
\pstThreeDLine[linecolor=gray](-0.243, -0.354, 0.441)(0.133, -0.030, 0.312)
\pstThreeDLine[linecolor=gray](-1.043, -0.054, -2.359)(-1.000, 0.227, -2.350)
\pstThreeDLine[linecolor=gray](1.257, -0.054, 2.141)(0.959, -0.218, 2.252)
\pstThreeDLine[linecolor=gray](-0.143, -0.454, -0.259)(-0.099, 0.022, -0.232)
\pstThreeDLine[linecolor=gray](-0.543, 0.646, -2.259)(-0.913, 0.207, -2.144)
\pstThreeDLine[linecolor=gray](-0.143, 0.746, -2.059)(-0.784, 0.178, -1.841)
\pstThreeDLine[linecolor=gray](-0.143, -0.154, 0.441)(0.141, -0.032, 0.332)
\pstThreeDLine[linecolor=gray](-0.243, -0.254, 1.141)(0.380, -0.086, 0.892)
\pstThreeDLine[linecolor=gray](-1.443, -0.054, -2.459)(-1.097, 0.249, -2.577)
\pstThreeDLine[linecolor=gray](0.457, 0.246, 2.241)(0.862, -0.196, 2.026)
\pstThreeDLine[linecolor=gray](-0.443, 0.346, -2.259)(-0.887, 0.202, -2.084)
\pstThreeDLine[linecolor=gray](0.457, 0.346, 1.841)(0.716, -0.163, 1.682)
\pstThreeDLine[linecolor=gray](1.057, 0.046, 1.641)(0.746, -0.170, 1.753)
\pstThreeDLine[linecolor=gray](1.857, -0.054, 2.341)(1.122, -0.255, 2.635)
\pstThreeDLine[linecolor=gray](0.257, -0.254, 0.941)(0.384, -0.087, 0.903)
\pstThreeDLine[linecolor=gray](-0.243, -0.154, -0.159)(-0.088, 0.020, -0.208)
\pstThreeDLine[linecolor=gray](0.257, -0.454, 1.841)(0.713, -0.162, 1.675)
\pstThreeDLine[linecolor=gray](0.557, -0.354, 1.541)(0.648, -0.147, 1.522)
\pstThreeDLine[linecolor=gray](-0.843, 0.446, -2.459)(-1.023, 0.232, -2.403)
\pstThreeDLine[linecolor=gray](-0.743, 0.246, -2.059)(-0.858, 0.195, -2.015)
\pstThreeDLine[linecolor=gray](-0.243, -0.054, 0.341)(0.087, -0.020, 0.204)
\pstThreeDLine[linecolor=gray](-0.443, -0.054, 0.741)(0.199, -0.045, 0.469)
\pstThreeDLine[linecolor=gray](-0.043, -0.254, 1.341)(0.482, -0.109, 1.132)
\pstThreeDLine[linecolor=gray](-0.943, 0.046, -2.259)(-0.953, 0.217, -2.238)
\pstThreeDLine[linecolor=gray](-1.243, 0.546, -2.759)(-1.195, 0.271, -2.806)
\pstThreeDLine[linecolor=gray](-0.643, 0.346, -2.359)(-0.953, 0.217, -2.239)
\pstThreeDLine[linecolor=gray](2.057, 0.746, 2.641)(1.232, -0.280, 2.893)
\pstThreeDLine[linecolor=gray](1.857, -0.454, 3.141)(1.422, -0.323, 3.339)
\pstThreeDLine[linecolor=gray](0.257, -0.254, 0.241)(0.134, -0.030, 0.315)
\pstThreeDLine[linecolor=gray](-0.343, 0.446, -2.459)(-0.947, 0.215, -2.224)
\pstThreeDLine[linecolor=gray](-1.243, 0.346, -2.359)(-1.045, 0.237, -2.454)
\pstThreeDLine[linecolor=gray](-1.143, 0.146, -2.159)(-0.951, 0.216, -2.234)
\pstThreeDLine[linecolor=gray](-1.443, -0.154, -2.359)(-1.058, 0.240, -2.485)
\pstThreeDLine[linecolor=gray](0.357, -0.254, 1.041)(0.435, -0.099, 1.023)
\pstThreeDLine[linecolor=gray](-1.043, -0.054, -2.359)(-1.000, 0.227, -2.350)
\pstThreeDLine[linecolor=gray](0.157, -0.154, 0.741)(0.294, -0.067, 0.691)
\pstThreeDLine[linecolor=gray](0.357, 0.346, 1.641)(0.629, -0.143, 1.478)
\pstThreeDLine[linecolor=gray](-0.843, -0.754, -0.459)(-0.266, 0.061, -0.626)
\pstThreeDLine[linecolor=gray](0.557, 0.146, 0.741)(0.345, -0.078, 0.810)
\pstThreeDLine[linecolor=gray](0.457, -0.154, 1.841)(0.733, -0.167, 1.722)
\pstThreeDLine[linecolor=gray](0.857, -0.054, 1.241)(0.576, -0.131, 1.353)
\pstThreeDLine[linecolor=gray](-0.843, -1.054, -0.259)(-0.184, 0.042, -0.433)
\pstThreeDLine[linecolor=gray](0.057, 0.146, 1.041)(0.376, -0.085, 0.883)
\pstThreeDLine[linecolor=gray](0.857, 0.246, 1.941)(0.816, -0.185, 1.917)
\pstThreeDLine[linecolor=gray](-0.443, 0.846, -2.059)(-0.833, 0.189, -1.956)
\pstThreeDLine[linecolor=gray](0.457, -0.354, 1.141)(0.490, -0.111, 1.151)
\pstThreeDLine[linecolor=gray](-1.043, 0.346, -2.159)(-0.943, 0.214, -2.214)
\pstThreeDLine[linecolor=gray](-1.443, 0.146, -2.459)(-1.104, 0.251, -2.593)
\pstThreeDLine[linecolor=gray](0.557, 0.146, 1.541)(0.631, -0.143, 1.482)
\pstThreeDLine[linecolor=gray](0.357, -0.854, 0.741)(0.349, -0.079, 0.820)
\pstThreeDLine[linecolor=gray](0.157, -0.854, 1.241)(0.497, -0.113, 1.168)
\pstThreeDLine[linecolor=gray](1.557, -0.254, 2.341)(1.083, -0.246, 2.544)
\pstThreeDLine[linecolor=gray](-0.943, -0.654, -0.459)(-0.285, 0.065, -0.669)
\pstThreeDLine[linecolor=gray](1.157, 0.146, 0.941)(0.508, -0.115, 1.192)
\pstThreeDLine[linecolor=gray](-0.343, -0.654, -0.059)(-0.051, 0.012, -0.119)
\pstThreeDLine[linecolor=gray](0.457, 0.246, 0.941)(0.398, -0.090, 0.934)
\pstThreeDLine[linecolor=gray](0.957, -0.254, 1.041)(0.527, -0.120, 1.237)
\pstThreeDLine[linecolor=gray](0.257, -0.154, 0.941)(0.381, -0.087, 0.895)
\pstThreeDLine[linecolor=gray](0.657, 0.146, 1.341)(0.575, -0.131, 1.350)
\pstThreeDLine[linecolor=gray](0.857, 0.246, 1.941)(0.816, -0.185, 1.917)
\pstThreeDLine[linecolor=gray](0.857, 0.046, 0.641)(0.358, -0.081, 0.841)
\pstThreeDLine[linecolor=gray](-1.043, 0.346, -1.859)(-0.835, 0.190, -1.962)
\pstThreeDLine[linecolor=gray](-0.943, -0.054, -2.359)(-0.985, 0.224, -2.314)
\pstThreeDLine[linecolor=gray](1.057, 0.146, 1.941)(0.850, -0.193, 1.997)
\pstThreeDLine[linecolor=gray](-1.343, -0.754, -2.459)(-1.058, 0.240, -2.484)
\pstThreeDLine[linecolor=gray](-1.543, -0.054, -2.659)(-1.184, 0.269, -2.781)
\pstThreeDLine[linecolor=gray](-0.643, -0.354, 0.141)(-0.035, 0.008, -0.083)
\pstThreeDLine[linecolor=gray](-0.843, 0.546, -2.359)(-0.991, 0.225, -2.327)
\pstThreeDLine[linecolor=gray](0.557, -0.154, 0.541)(0.284, -0.064, 0.666)
\pstThreeDLine[linecolor=gray](-0.643, 0.446, -2.259)(-0.921, 0.209, -2.163)
\pstThreeDLine[linecolor=gray](-0.043, -0.354, 0.141)(0.056, -0.013, 0.132)
\pstThreeDLine[linecolor=gray](-0.343, 1.146, -2.359)(-0.935, 0.213, -2.197)
\pstThreeDLine[linecolor=gray](1.757, -0.054, 2.841)(1.285, -0.292, 3.019)
\pstThreeDLine[linecolor=gray](0.457, -0.254, 1.341)(0.558, -0.127, 1.311)
\pstThreeDLine[linecolor=gray](0.557, 0.046, 1.741)(0.706, -0.160, 1.658)
\pstThreeDLine[linecolor=gray](0.457, -0.554, 1.141)(0.497, -0.113, 1.167)
\pstThreeDLine[linecolor=gray](-0.043, -0.354, 1.341)(0.485, -0.110, 1.140)
\pstThreeDLine[linecolor=gray](-0.843, -0.054, -2.159)(-0.898, 0.204, -2.110)
\pstThreeDLine[linecolor=gray](0.857, 0.046, 0.941)(0.465, -0.106, 1.093)
\pstThreeDLine[linecolor=gray](0.157, -0.354, 1.341)(0.516, -0.117, 1.211)
\pstThreeDLine[linecolor=gray](-0.743, 0.446, -2.359)(-0.972, 0.221, -2.283)
\pstThreeDLine[linecolor=gray](-1.043, 0.046, -2.159)(-0.932, 0.212, -2.190)
\pstThreeDLine[linecolor=gray](-0.143, -0.254, 0.741)(0.252, -0.057, 0.592)
\pstThreeDLine[linecolor=gray](-0.743, 0.746, -2.159)(-0.911, 0.207, -2.140)
\pstThreeDLine[linecolor=gray](0.757, -0.054, 0.641)(0.346, -0.079, 0.814)
\pstThreeDLine[linecolor=gray](0.557, -0.254, 1.841)(0.752, -0.171, 1.766)
\pstThreeDLine[linecolor=gray](-0.643, 1.046, -2.259)(-0.942, 0.214, -2.212)
\pstThreeDLine[linecolor=gray](0.557, -0.254, 1.841)(0.752, -0.171, 1.766)
\pstThreeDLine[linecolor=gray](1.857, -0.254, 2.941)(1.343, -0.305, 3.155)
\pstThreeDLine[linecolor=gray](-0.043, 0.946, -2.559)(-0.954, 0.217, -2.241)
\pstThreeDLine[linecolor=gray](-0.943, 0.046, -2.259)(-0.953, 0.217, -2.238)
\pstThreeDLine[linecolor=gray](-0.443, 0.646, -2.259)(-0.897, 0.204, -2.108)
\pstThreeDLine[linecolor=gray](-0.743, 0.446, -2.359)(-0.972, 0.221, -2.283)
\pstThreeDLine[linecolor=gray](0.157, 0.346, 0.741)(0.277, -0.063, 0.651)
\pstThreeDLine[linecolor=gray](0.657, -0.054, 1.741)(0.724, -0.165, 1.702)
\pstThreeDLine[linecolor=gray](-0.343, -0.654, 0.041)(-0.015, 0.003, -0.035)
\pstThreeDLine[linecolor=gray](1.357, 0.546, 2.341)(1.025, -0.233, 2.407)
\pstThreeDLine[linecolor=gray](1.057, 0.046, 1.341)(0.639, -0.145, 1.501)
\pstThreeDLine[linecolor=gray](0.357, -0.154, 0.541)(0.253, -0.058, 0.595)
\pstThreeDLine[linecolor=gray](0.657, -0.254, 0.841)(0.410, -0.093, 0.962)
\pstThreeDLine[linecolor=gray](0.157, -0.054, 1.041)(0.398, -0.090, 0.935)
\pstThreeDLine[linecolor=gray](-0.443, 0.346, -2.059)(-0.816, 0.185, -1.916)
\pstThreeDLine[linecolor=gray](-0.343, -0.454, 0.641)(0.193, -0.044, 0.453)
\pstThreeDLine[linecolor=gray](0.857, -0.054, 1.441)(0.648, -0.147, 1.521)
\pstThreeDLine[linecolor=gray](1.857, 0.746, 2.941)(1.309, -0.297, 3.074)
\pstThreeDLine[linecolor=gray](-0.743, 0.346, -2.259)(-0.933, 0.212, -2.191)

\psset{dotstyle=Bo,dotscale=1.75,fillcolor=lightgray}
\dataplotThreeD[plotstyle=dots,showpoints=true]{\dataSLWPL}
\pstThreeDLine[linewidth=2pt,arrows=->](1.43,-0.32,3.36)(-1.43,0.32,-3.36)
\pstThreeDPut(-1.6,0.35,-3.7){$\bu_1$}
%\psset{dotstyle=Bo,dotscale=1.75,fillcolor=white}
%\dataplotThreeD[plotstyle=dots,showpoints=true]{\dataIrisOneD}
\pstThreeDBox[linecolor=gray](-2,-2,-3)(4,0,0)(0,4,0)(0,0,7)
\end{pspicture}
}}
\end{frame}



\begin{frame}{Minimum Squared Error Approach}
The direction that
maximizes the projected variance is also the
one that minimizes the average squared error.
The mean squared error ({\it MSE}) optimization condition is 
\begin{empheq}[box=\tcbhighmath]{align*}
\begin{split}
    MSE(\bu) & =  {1 \over n} \sum_{i=1}^n \norm{\bepsilon_i}^2
     =  {1 \over n} \sum_{i=1}^n \| \bx_i - \bx'_i \|^2 
	 =\sum_{i=1}^n \frac{\| \bx_i\|^2}{n}-\bu^T \cov \bu
\end{split}
\end{empheq}
Since the first term is fixed for a dataset $\bD$, we see that the
direction $\bu_1$ that maximizes the
variance is also the one that minimizes the MSE.
Further,
\begin{empheq}[box=\tcbhighmath]{align*}
\begin{split}
    \sum_{i=1}^n \frac{\| \bx_i\|^2}{n}-\bu^T \cov \bu = var(\bD) = 
	tr(\cov) = \sum_{i=1}^d \sigma_i^2
\end{split}
\end{empheq}

\medskip
Thus, for the direction $\bu_1$ that minimizes MSE, we have
\begin{empheq}[box=\tcbhighmath]{align*}
\begin{split}
    MSE(\bu_1) = var(\bD) - \bu_1^T\cov \bu_1 = var(\bD) - \lambda_1 
\end{split}
\end{empheq}
\end{frame}


\begin{frame}{Best 2-dimensional Approximation}

The best 2D subspace that captures the most variance in $\bD$ comprises
the eigenvectors $\bu_1$ and $\bu_2$ corresponding to the largest and
second largest eigenvalues $\lambda_1$ and $\lambda_2$, respv.

\medskip
Let $\bU_2 = \matr{\bu_1 & \bu_2}$ 
be the matrix whose columns correspond to the two principal components.
Given the point $\bx_i \in \bD$ its projected coordinates are computed as follows:
\begin{align*}
    \ba_i = \bU_2^T \bx_i
\end{align*}
Let $\bA$ denote the projected 2D dataset.
The total projected variance for $\bA$ is given as
\begin{align*}
    var(\bA)  = \bu_1^T\cov\bu_1  + \bu_2^T\cov \bu_2 =
	\bu_1^T\lambda_1\bu_1 + \bu_2^T\lambda_2\bu_2 =
    \lambda_1 + \lambda_2
\end{align*}

The first two principal components also
minimize the mean square error objective, since
\begin{align*}
    MSE & =  {1 \over n} \sum_{i=1}^n
    \norm{\bx_i-\bx_i'}^2
	= var(\bD) - {1\over n} \sum_{i=1}^n \lB(\bx_i^T \bP_2 \bx_i\rB)
	= var(\bD) - var(\bA)
\end{align*}
\end{frame}


\readdata{\dataIrisDLno}{EDA/dimreduction/figs/iris-2dLnonopt.dat}
\readdata{\dataIrisDRno}{EDA/dimreduction/figs/iris-2dRnonopt.dat}
\begin{frame}{Optimal and Non-optimal 2D Approximations}

The optimal subspace
    maximizes the variance, and minimizes the squared error,
    whereas the nonoptimal subspace captures less variance, and
     has a high mean squared error value, as seen from the 
	 lengths of the error vectors (line  segments).

\psset{unit=0.5in}
\psset{arrowscale=2,dotscale=1.75}
\psset{Alpha=60,Beta=-30}
\psset{nameX=$~$, nameY=$~$, nameZ=$~$}
\centerline{
%\subfloat[Optimal basis]{
%\label{fig:eda:dimr:2dirisA}
\scalebox{0.55}{%
\begin{pspicture}(-2,-4.5)(2,4.5)
%\pstThreeDSquare[fillstyle=solid, fillcolor=lightgray]
\pstThreeDSquare[fillstyle=vlines,
    hatchcolor=lightgray, hatchwidth=0.1\pslinewidth,
    hatchsep=3\pslinewidth]
(3.99,2.64,2.56)(-2.86,0.64,-6.72)(-5.12,-5.92, 1.6)
\psset{dotstyle=Bo,fillcolor=gray}
\dataplotThreeD[plotstyle=dots,showpoints=true]{\dataIrisDR}
%\pstThreeDSquare[
%fillstyle=solid, fillcolor=lightgray]
%(2.71,1.16,2.96)(-2.86,0.64,-6.72)(-2.56,-2.96, 0.8)
\pstThreeDCoor[xMin=-2, xMax= 2, yMin=-2,
        yMax=2, zMin=-3, zMax=4, Dx=0.5, Dy=0.5, Dz=1,
        linewidth=1pt,linecolor=black,arrowscale=2]
\pstThreeDPut(2.3,0,0){$X_1$}
\pstThreeDPut(0,2.5,0){$X_2$}
\pstThreeDPut(0,0,4.3){$X_3$}
\pstThreeDLine[linecolor=black](0.057, -0.054, 0.441)(0.109, -0.107, 0.414)
\pstThreeDLine[linecolor=gray](1.057, 0.046, 1.141)(0.875, 0.228, 1.236)
\pstThreeDLine[linecolor=gray](0.757, -0.154, 0.841)(0.550, 0.053, 0.950)
\pstThreeDLine[linecolor=black](-1.243, 0.146, -2.359)(-1.174, 0.077, -2.395)
\pstThreeDLine[linecolor=gray](0.157, -0.854, 0.241)(-0.233, -0.464, 0.445)
\pstThreeDLine[linecolor=gray](-1.143, 0.146, -2.459)(-1.141, 0.144, -2.460)
\pstThreeDLine[linecolor=black](0.657, -0.054, 2.041)(0.813, -0.211, 1.960)
\pstThreeDLine[linecolor=black](-0.043, -0.354, 1.341)(0.128, -0.525, 1.252)
\pstThreeDLine[linecolor=black](0.857, 0.046, 1.841)(0.923, -0.021, 1.807)
\pstThreeDLine[linecolor=gray](0.857, -0.554, 2.041)(0.705, -0.402, 2.120)
\pstThreeDLine[linecolor=black](-0.743, 0.646, -2.259)(-0.651, 0.554, -2.307)
\pstThreeDLine[linecolor=black](-0.743, 0.746, -2.259)(-0.607, 0.609, -2.330)
\pstThreeDLine[linecolor=black](-0.143, -0.054, 0.441)(-0.003, -0.195, 0.368)
\pstThreeDLine[linecolor=black](0.257, -0.054, 1.141)(0.382, -0.180, 1.076)
\pstThreeDLine[linecolor=gray](-0.943, 0.046, -2.259)(-1.027, 0.130, -2.215)
\pstThreeDLine[linecolor=black](-0.843, 0.346, -2.159)(-0.816, 0.319, -2.173)
\pstThreeDLine[linecolor=gray](-0.843, 0.346, -2.259)(-0.839, 0.342, -2.261)
\pstThreeDLine[linecolor=black](-0.143, -0.254, 0.341)(-0.114, -0.284, 0.326)
\pstThreeDLine[linecolor=gray](-0.843, 0.246, -2.359)(-0.906, 0.309, -2.326)
\pstThreeDLine[linecolor=gray](1.357, 0.146, 2.241)(1.340, 0.163, 2.250)
\pstThreeDLine[linecolor=black](0.057, -0.054, 1.341)(0.316, -0.314, 1.206)
\pstThreeDLine[linecolor=black](0.657, -0.054, 1.441)(0.675, -0.073, 1.432)
\pstThreeDLine[linecolor=black](-0.143, 1.346, -2.259)(-0.007, 1.209, -2.330)
\pstThreeDLine[linecolor=gray](-0.343, -0.554, 0.241)(-0.381, -0.516, 0.261)
\pstThreeDLine[linecolor=black](-0.943, -0.554, 0.741)(-0.603, -0.895, 0.563)
\pstThreeDLine[linecolor=black](-0.843, 0.446, -2.159)(-0.772, 0.375, -2.196)
\pstThreeDLine[linecolor=gray](-0.343, -0.754, 0.241)(-0.469, -0.628, 0.307)
\pstThreeDLine[linecolor=black](-1.243, 0.046, -2.259)(-1.195, -0.002, -2.284)
\pstThreeDLine[linecolor=gray](1.357, -0.054, 2.041)(1.206, 0.097, 2.120)
\pstThreeDLine[linecolor=black](0.957, 0.146, 2.141)(1.092, 0.010, 2.070)
\pstThreeDLine[linecolor=gray](-0.443, 0.846, -2.459)(-0.441, 0.843, -2.460)
\pstThreeDLine[linecolor=gray](-0.843, 0.146, -2.559)(-0.996, 0.299, -2.479)
\pstThreeDLine[linecolor=black](-0.143, -0.554, 1.241)(-0.039, -0.658, 1.187)
\pstThreeDLine[linecolor=gray](-0.043, -0.454, 0.241)(-0.169, -0.328, 0.307)
\pstThreeDLine[linecolor=gray](-0.743, -0.554, -0.759)(-0.835, -0.462, -0.711)
\pstThreeDLine[linecolor=gray](-0.243, -0.554, 0.141)(-0.348, -0.449, 0.196)
\pstThreeDLine[linecolor=gray](-0.043, -0.354, 0.341)(-0.102, -0.296, 0.372)
\pstThreeDLine[linecolor=black](-0.743, 0.746, -1.859)(-0.515, 0.517, -1.978)
\pstThreeDLine[linecolor=gray](0.457, -0.754, 0.641)(0.072, -0.368, 0.843)
\pstThreeDLine[linecolor=gray](0.457, -0.554, 1.241)(0.297, -0.394, 1.325)
\pstThreeDLine[linecolor=black](-0.243, -0.054, 0.741)(0.010, -0.308, 0.609)
\pstThreeDLine[linecolor=black](0.257, -0.054, 0.841)(0.313, -0.111, 0.812)
\pstThreeDLine[linecolor=gray](0.957, -0.054, 1.741)(0.912, -0.010, 1.764)
\pstThreeDLine[linecolor=gray](1.457, -0.154, 2.541)(1.332, -0.030, 2.606)
\pstThreeDLine[linecolor=black](-0.243, -0.354, 0.441)(-0.191, -0.407, 0.414)
\pstThreeDLine[linecolor=gray](-1.043, -0.054, -2.359)(-1.150, 0.053, -2.303)
\pstThreeDLine[linecolor=gray](1.257, -0.054, 2.141)(1.172, 0.030, 2.185)
\pstThreeDLine[linecolor=gray](-0.143, -0.454, -0.259)(-0.340, -0.257, -0.156)
\pstThreeDLine[linecolor=gray](-0.543, 0.646, -2.259)(-0.539, 0.642, -2.261)
\pstThreeDLine[linecolor=gray](-0.143, 0.746, -2.059)(-0.225, 0.827, -2.016)
\pstThreeDLine[linecolor=black](-0.143, -0.154, 0.441)(-0.047, -0.251, 0.391)
\pstThreeDLine[linecolor=black](-0.243, -0.254, 1.141)(0.014, -0.512, 1.007)
\pstThreeDLine[linecolor=black](-1.443, -0.054, -2.459)(-1.398, -0.100, -2.483)
\pstThreeDLine[linecolor=black](0.457, 0.246, 2.241)(0.879, -0.177, 2.021)
\pstThreeDLine[linecolor=gray](-0.443, 0.346, -2.259)(-0.615, 0.518, -2.169)
\pstThreeDLine[linecolor=black](0.457, 0.346, 1.841)(0.831, -0.029, 1.646)
\pstThreeDLine[linecolor=gray](1.057, 0.046, 1.641)(0.990, 0.113, 1.676)
\pstThreeDLine[linecolor=gray](1.857, -0.054, 2.341)(1.555, 0.248, 2.499)
\pstThreeDLine[linecolor=gray](0.257, -0.254, 0.941)(0.248, -0.246, 0.946)
\pstThreeDLine[linecolor=black](-0.243, -0.154, -0.159)(-0.241, -0.157, -0.160)
\pstThreeDLine[linecolor=black](0.257, -0.454, 1.841)(0.367, -0.564, 1.784)
\pstThreeDLine[linecolor=gray](0.557, -0.354, 1.541)(0.510, -0.307, 1.566)
\pstThreeDLine[linecolor=gray](-0.843, 0.446, -2.459)(-0.841, 0.444, -2.460)
\pstThreeDLine[linecolor=gray](-0.743, 0.246, -2.059)(-0.781, 0.284, -2.039)
\pstThreeDLine[linecolor=black](-0.243, -0.054, 0.341)(-0.082, -0.216, 0.257)
\pstThreeDLine[linecolor=black](-0.443, -0.054, 0.741)(-0.102, -0.396, 0.563)
\pstThreeDLine[linecolor=black](-0.043, -0.254, 1.341)(0.172, -0.470, 1.229)
\pstThreeDLine[linecolor=gray](-0.943, 0.046, -2.259)(-1.027, 0.130, -2.215)
\pstThreeDLine[linecolor=black](-1.243, 0.546, -2.759)(-1.090, 0.393, -2.839)
\pstThreeDLine[linecolor=gray](-0.643, 0.346, -2.359)(-0.750, 0.453, -2.303)
\pstThreeDLine[linecolor=black](2.057, 0.746, 2.641)(2.088, 0.715, 2.625)
\pstThreeDLine[linecolor=gray](1.857, -0.454, 3.141)(1.563, -0.159, 3.295)
\pstThreeDLine[linecolor=gray](0.257, -0.254, 0.241)(0.088, -0.085, 0.330)
\pstThreeDLine[linecolor=gray](-0.343, 0.446, -2.459)(-0.561, 0.664, -2.345)
\pstThreeDLine[linecolor=black](-1.243, 0.346, -2.359)(-1.086, 0.189, -2.441)
\pstThreeDLine[linecolor=black](-1.143, 0.146, -2.159)(-1.072, 0.075, -2.196)
\pstThreeDLine[linecolor=black](-1.443, -0.154, -2.359)(-1.419, -0.179, -2.372)
\pstThreeDLine[linecolor=gray](0.357, -0.254, 1.041)(0.327, -0.225, 1.057)
\pstThreeDLine[linecolor=gray](-1.043, -0.054, -2.359)(-1.150, 0.053, -2.303)
\pstThreeDLine[linecolor=black](0.157, -0.154, 0.741)(0.190, -0.188, 0.724)
\pstThreeDLine[linecolor=black](0.357, 0.346, 1.641)(0.729, -0.027, 1.447)
\pstThreeDLine[linecolor=gray](-0.843, -0.754, -0.459)(-0.910, -0.687, -0.424)
\pstThreeDLine[linecolor=gray](0.557, 0.146, 0.741)(0.547, 0.156, 0.747)
\pstThreeDLine[linecolor=black](0.457, -0.154, 1.841)(0.611, -0.309, 1.761)
\pstThreeDLine[linecolor=gray](0.857, -0.054, 1.241)(0.742, 0.061, 1.301)
\pstThreeDLine[linecolor=gray](-0.843, -1.054, -0.259)(-0.996, -0.901, -0.179)
\pstThreeDLine[linecolor=black](0.057, 0.146, 1.041)(0.335, -0.133, 0.896)
\pstThreeDLine[linecolor=black](0.857, 0.246, 1.941)(1.034, 0.068, 1.849)
\pstThreeDLine[linecolor=black](-0.443, 0.846, -2.059)(-0.349, 0.751, -2.108)
\pstThreeDLine[linecolor=gray](0.457, -0.354, 1.141)(0.362, -0.259, 1.191)
\pstThreeDLine[linecolor=black](-1.043, 0.346, -2.159)(-0.928, 0.231, -2.219)
\pstThreeDLine[linecolor=black](-1.443, 0.146, -2.459)(-1.310, 0.012, -2.529)
\pstThreeDLine[linecolor=black](0.557, 0.146, 1.541)(0.730, -0.028, 1.451)
\pstThreeDLine[linecolor=gray](0.357, -0.854, 0.741)(-0.006, -0.491, 0.931)
\pstThreeDLine[linecolor=gray](0.157, -0.854, 1.241)(-0.003, -0.694, 1.325)
\pstThreeDLine[linecolor=gray](1.557, -0.254, 2.341)(1.299, 0.005, 2.476)
\pstThreeDLine[linecolor=black](-0.943, -0.654, -0.459)(-0.922, -0.675, -0.470)
\pstThreeDLine[linecolor=gray](1.157, 0.146, 0.941)(0.929, 0.374, 1.060)
\pstThreeDLine[linecolor=gray](-0.343, -0.654, -0.059)(-0.494, -0.503, 0.020)
\pstThreeDLine[linecolor=black](0.457, 0.246, 0.941)(0.580, 0.122, 0.877)
\pstThreeDLine[linecolor=gray](0.957, -0.254, 1.041)(0.664, 0.039, 1.194)
\pstThreeDLine[linecolor=black](0.257, -0.154, 0.941)(0.292, -0.190, 0.923)
\pstThreeDLine[linecolor=black](0.657, 0.146, 1.341)(0.740, 0.062, 1.298)
\pstThreeDLine[linecolor=black](0.857, 0.246, 1.941)(1.034, 0.068, 1.849)
\pstThreeDLine[linecolor=gray](0.857, 0.046, 0.641)(0.648, 0.255, 0.750)
\pstThreeDLine[linecolor=black](-1.043, 0.346, -1.859)(-0.859, 0.162, -1.955)
\pstThreeDLine[linecolor=gray](-0.943, -0.054, -2.359)(-1.094, 0.097, -2.280)
\pstThreeDLine[linecolor=black](1.057, 0.146, 1.941)(1.102, 0.100, 1.917)
\pstThreeDLine[linecolor=gray](-1.343, -0.754, -2.459)(-1.649, -0.447, -2.299)
\pstThreeDLine[linecolor=black](-1.543, -0.054, -2.659)(-1.500, -0.098, -2.682)
\pstThreeDLine[linecolor=black](-0.643, -0.354, 0.141)(-0.484, -0.514, 0.058)
\pstThreeDLine[linecolor=black](-0.843, 0.546, -2.359)(-0.774, 0.477, -2.395)
\pstThreeDLine[linecolor=gray](0.557, -0.154, 0.541)(0.369, 0.034, 0.640)
\pstThreeDLine[linecolor=gray](-0.643, 0.446, -2.259)(-0.683, 0.486, -2.238)
\pstThreeDLine[linecolor=gray](-0.043, -0.354, 0.141)(-0.148, -0.250, 0.196)
\pstThreeDLine[linecolor=black](-0.343, 1.146, -2.359)(-0.230, 1.032, -2.418)
\pstThreeDLine[linecolor=gray](1.757, -0.054, 2.841)(1.614, 0.089, 2.916)
\pstThreeDLine[linecolor=black](0.457, -0.254, 1.341)(0.452, -0.250, 1.344)
\pstThreeDLine[linecolor=black](0.557, 0.046, 1.741)(0.732, -0.130, 1.650)
\pstThreeDLine[linecolor=gray](0.457, -0.554, 1.141)(0.274, -0.371, 1.237)
\pstThreeDLine[linecolor=black](-0.043, -0.354, 1.341)(0.128, -0.525, 1.252)
\pstThreeDLine[linecolor=gray](-0.843, -0.054, -2.159)(-0.992, 0.095, -2.081)
\pstThreeDLine[linecolor=gray](0.857, 0.046, 0.941)(0.717, 0.186, 1.014)
\pstThreeDLine[linecolor=black](0.157, -0.354, 1.341)(0.240, -0.437, 1.298)
\pstThreeDLine[linecolor=gray](-0.743, 0.446, -2.359)(-0.762, 0.465, -2.349)
\pstThreeDLine[linecolor=gray](-1.043, 0.046, -2.159)(-1.060, 0.063, -2.150)
\pstThreeDLine[linecolor=black](-0.143, -0.254, 0.741)(-0.022, -0.376, 0.678)
\pstThreeDLine[linecolor=black](-0.743, 0.746, -2.159)(-0.584, 0.586, -2.242)
\pstThreeDLine[linecolor=gray](0.757, -0.054, 0.641)(0.548, 0.155, 0.750)
\pstThreeDLine[linecolor=black](0.557, -0.254, 1.841)(0.623, -0.321, 1.807)
\pstThreeDLine[linecolor=black](-0.643, 1.046, -2.259)(-0.419, 0.821, -2.376)
\pstThreeDLine[linecolor=black](0.557, -0.254, 1.841)(0.623, -0.321, 1.807)
\pstThreeDLine[linecolor=gray](1.857, -0.254, 2.941)(1.605, -0.001, 3.073)
\pstThreeDLine[linecolor=gray](-0.043, 0.946, -2.559)(-0.195, 1.098, -2.479)
\pstThreeDLine[linecolor=gray](-0.943, 0.046, -2.259)(-1.027, 0.130, -2.215)
\pstThreeDLine[linecolor=gray](-0.443, 0.646, -2.259)(-0.483, 0.686, -2.238)
\pstThreeDLine[linecolor=gray](-0.743, 0.446, -2.359)(-0.762, 0.465, -2.349)
\pstThreeDLine[linecolor=black](0.157, 0.346, 0.741)(0.410, 0.092, 0.609)
\pstThreeDLine[linecolor=black](0.657, -0.054, 1.741)(0.744, -0.142, 1.696)
\pstThreeDLine[linecolor=gray](-0.343, -0.654, 0.041)(-0.471, -0.526, 0.108)
\pstThreeDLine[linecolor=black](1.357, 0.546, 2.341)(1.538, 0.364, 2.246)
\pstThreeDLine[linecolor=gray](1.057, 0.046, 1.341)(0.921, 0.182, 1.412)
\pstThreeDLine[linecolor=gray](0.357, -0.154, 0.541)(0.257, -0.054, 0.594)
\pstThreeDLine[linecolor=gray](0.657, -0.254, 0.841)(0.450, -0.047, 0.950)
\pstThreeDLine[linecolor=black](0.157, -0.054, 1.041)(0.303, -0.201, 0.965)
\pstThreeDLine[linecolor=gray](-0.443, 0.346, -2.059)(-0.569, 0.472, -1.993)
\pstThreeDLine[linecolor=black](-0.343, -0.454, 0.641)(-0.245, -0.552, 0.590)
\pstThreeDLine[linecolor=gray](0.857, -0.054, 1.441)(0.787, 0.015, 1.477)
\pstThreeDLine[linecolor=black](1.857, 0.746, 2.941)(2.045, 0.558, 2.843)
\pstThreeDLine[linecolor=gray](-0.743, 0.346, -2.259)(-0.783, 0.386, -2.238)

\pstThreeDLine[linewidth=2pt,arrows=->](1.43,-0.32,3.36)(-1.43,0.32,-3.36)
\pstThreeDLine[linewidth=2pt,arrows=->](2.56,2.96,-0.8)(-2.56,-2.96,0.8)
\pstThreeDPut(-1.6,0.35,-3.7){$\bu_1$}
\pstThreeDPut(-2.88,-3.33,0.9){$\bu_2$}
\psset{dotstyle=Bo,fillcolor=white}
\dataplotThreeD[plotstyle=dots,showpoints=true]{\dataIrisDL}
\pstThreeDBox[linecolor=gray](-2,-2,-3)(4,0,0)(0,4,0)(0,0,7)
%\pstThreeDBox[](-3.5,-2.5,-3)(5,0,0)(0,4,0)(0,0,7.5)
\end{pspicture}
}%}
\hspace{1in}
%\subfloat[Nonoptimal basis]{
%\label{fig:eda:dimr:2dirisB}
\scalebox{0.55}{%
\begin{pspicture}(-2,-4.5)(2,4.5)
\pstThreeDLine[linecolor=black](0.057, -0.054, 0.441)(-0.112, -0.016, 0.046)
\pstThreeDLine[linecolor=black](1.057, 0.046, 1.141)(0.489, 0.175, -0.191)
\pstThreeDLine[linecolor=black](0.757, -0.154, 0.841)(0.335, -0.058, -0.148)
\pstThreeDLine[linecolor=gray](-1.243, 0.146, -2.359)(-0.206, -0.090, 0.079)
\pstThreeDLine[linecolor=black](0.157, -0.854, 0.241)(0.017, -0.822, -0.087)
\pstThreeDLine[linecolor=gray](-1.143, 0.146, -2.459)(-0.085, -0.094, 0.027)
\pstThreeDLine[linecolor=black](0.657, -0.054, 2.041)(-0.175, 0.135, 0.088)
\pstThreeDLine[linecolor=black](-0.043, -0.354, 1.341)(-0.529, -0.244, 0.201)
\pstThreeDLine[linecolor=black](0.857, 0.046, 1.841)(0.069, 0.225, -0.008)
\pstThreeDLine[linecolor=black](0.857, -0.554, 2.041)(-0.023, -0.354, -0.025)
\pstThreeDLine[linecolor=gray](-0.743, 0.646, -2.259)(0.200, 0.432, -0.043)
\pstThreeDLine[linecolor=gray](-0.743, 0.746, -2.259)(0.203, 0.531, -0.035)
\pstThreeDLine[linecolor=black](-0.143, -0.054, 0.441)(-0.281, -0.023, 0.118)
\pstThreeDLine[linecolor=black](0.257, -0.054, 1.141)(-0.192, 0.048, 0.087)
\pstThreeDLine[linecolor=gray](-0.943, 0.046, -2.259)(0.009, -0.171, -0.021)
\pstThreeDLine[linecolor=gray](-0.843, 0.346, -2.159)(0.069, 0.139, -0.016)
\pstThreeDLine[linecolor=gray](-0.843, 0.346, -2.259)(0.105, 0.131, -0.032)
\pstThreeDLine[linecolor=black](-0.143, -0.254, 0.341)(-0.252, -0.229, 0.085)
\pstThreeDLine[linecolor=gray](-0.843, 0.246, -2.359)(0.137, 0.023, -0.056)
\pstThreeDLine[linecolor=black](1.357, 0.146, 2.241)(0.354, 0.374, -0.114)
\pstThreeDLine[linecolor=black](0.057, -0.054, 1.341)(-0.433, 0.057, 0.190)
\pstThreeDLine[linecolor=black](0.657, -0.054, 1.441)(0.039, 0.086, -0.008)
\pstThreeDLine[linecolor=gray](-0.143, 1.346, -2.259)(0.733, 1.147, -0.201)
\pstThreeDLine[linecolor=black](-0.343, -0.554, 0.241)(-0.397, -0.542, 0.116)
\pstThreeDLine[linecolor=black](-0.943, -0.554, 0.741)(-1.084, -0.522, 0.411)
\pstThreeDLine[linecolor=gray](-0.843, 0.446, -2.159)(0.072, 0.238, -0.008)
\pstThreeDLine[linecolor=black](-0.343, -0.754, 0.241)(-0.403, -0.740, 0.100)
\pstThreeDLine[linecolor=gray](-1.243, 0.046, -2.259)(-0.245, -0.181, 0.087)
\pstThreeDLine[linecolor=black](1.357, -0.054, 2.041)(0.418, 0.159, -0.163)
\pstThreeDLine[linecolor=black](0.957, 0.146, 2.141)(0.050, 0.352, 0.013)
\pstThreeDLine[linecolor=gray](-0.443, 0.846, -2.459)(0.533, 0.624, -0.166)
\pstThreeDLine[linecolor=gray](-0.843, 0.146, -2.559)(0.205, -0.092, -0.096)
\pstThreeDLine[linecolor=black](-0.143, -0.554, 1.241)(-0.585, -0.454, 0.205)
\pstThreeDLine[linecolor=black](-0.043, -0.454, 0.241)(-0.139, -0.432, 0.017)
\pstThreeDLine[linecolor=gray](-0.743, -0.554, -0.759)(-0.378, -0.637, 0.099)
\pstThreeDLine[linecolor=black](-0.243, -0.554, 0.141)(-0.276, -0.547, 0.065)
\pstThreeDLine[linecolor=black](-0.043, -0.354, 0.341)(-0.171, -0.325, 0.041)
\pstThreeDLine[linecolor=gray](-0.743, 0.746, -1.859)(0.060, 0.563, 0.029)
\pstThreeDLine[linecolor=black](0.457, -0.754, 0.641)(0.132, -0.680, -0.122)
\pstThreeDLine[linecolor=black](0.457, -0.554, 1.241)(-0.076, -0.433, -0.010)
\pstThreeDLine[linecolor=black](-0.243, -0.054, 0.741)(-0.473, -0.002, 0.201)
\pstThreeDLine[linecolor=black](0.257, -0.054, 0.841)(-0.085, 0.024, 0.039)
\pstThreeDLine[linecolor=black](0.957, -0.054, 1.741)(0.187, 0.121, -0.068)
\pstThreeDLine[linecolor=black](1.457, -0.154, 2.541)(0.321, 0.104, -0.127)
\pstThreeDLine[linecolor=black](-0.243, -0.354, 0.441)(-0.376, -0.324, 0.129)
\pstThreeDLine[linecolor=gray](-1.043, -0.054, -2.359)(-0.043, -0.281, -0.009)
\pstThreeDLine[linecolor=black](1.257, -0.054, 2.141)(0.298, 0.164, -0.111)
\pstThreeDLine[linecolor=gray](-0.143, -0.454, -0.259)(-0.045, -0.476, -0.027)
\pstThreeDLine[linecolor=gray](-0.543, 0.646, -2.259)(0.369, 0.439, -0.115)
\pstThreeDLine[linecolor=gray](-0.143, 0.746, -2.059)(0.640, 0.568, -0.218)
\pstThreeDLine[linecolor=black](-0.143, -0.154, 0.441)(-0.285, -0.122, 0.109)
\pstThreeDLine[linecolor=black](-0.243, -0.254, 1.141)(-0.623, -0.168, 0.249)
\pstThreeDLine[linecolor=gray](-1.443, -0.054, -2.459)(-0.346, -0.303, 0.118)
\pstThreeDLine[linecolor=black](0.457, 0.246, 2.241)(-0.406, 0.442, 0.215)
\pstThreeDLine[linecolor=gray](-0.443, 0.346, -2.259)(0.444, 0.144, -0.175)
\pstThreeDLine[linecolor=black](0.457, 0.346, 1.841)(-0.259, 0.509, 0.160)
\pstThreeDLine[linecolor=black](1.057, 0.046, 1.641)(0.311, 0.216, -0.111)
\pstThreeDLine[linecolor=black](1.857, -0.054, 2.341)(0.735, 0.201, -0.293)
\pstThreeDLine[linecolor=black](0.257, -0.254, 0.941)(-0.128, -0.167, 0.038)
\pstThreeDLine[linecolor=gray](-0.243, -0.154, -0.159)(-0.155, -0.174, 0.049)
\pstThreeDLine[linecolor=black](0.257, -0.454, 1.841)(-0.456, -0.292, 0.166)
\pstThreeDLine[linecolor=black](0.557, -0.354, 1.541)(-0.091, -0.207, 0.019)
\pstThreeDLine[linecolor=gray](-0.843, 0.446, -2.459)(0.180, 0.214, -0.056)
\pstThreeDLine[linecolor=gray](-0.743, 0.246, -2.059)(0.114, 0.051, -0.044)
\pstThreeDLine[linecolor=black](-0.243, -0.054, 0.341)(-0.330, -0.034, 0.137)
\pstThreeDLine[linecolor=black](-0.443, -0.054, 0.741)(-0.643, -0.009, 0.273)
\pstThreeDLine[linecolor=black](-0.043, -0.254, 1.341)(-0.525, -0.145, 0.210)
\pstThreeDLine[linecolor=gray](-0.943, 0.046, -2.259)(0.009, -0.171, -0.021)
\pstThreeDLine[linecolor=gray](-1.243, 0.546, -2.759)(-0.049, 0.275, 0.047)
\pstThreeDLine[linecolor=gray](-0.643, 0.346, -2.359)(0.310, 0.129, -0.119)
\pstThreeDLine[linecolor=black](2.057, 0.746, 2.641)(0.825, 1.026, -0.252)
\pstThreeDLine[linecolor=black](1.857, -0.454, 3.141)(0.435, -0.131, -0.198)
\pstThreeDLine[linecolor=black](0.257, -0.254, 0.241)(0.123, -0.224, -0.074)
\pstThreeDLine[linecolor=gray](-0.343, 0.446, -2.459)(0.603, 0.231, -0.235)
\pstThreeDLine[linecolor=gray](-1.243, 0.346, -2.359)(-0.199, 0.109, 0.095)
\pstThreeDLine[linecolor=gray](-1.143, 0.146, -2.159)(-0.192, -0.070, 0.075)
\pstThreeDLine[linecolor=gray](-1.443, -0.154, -2.359)(-0.386, -0.394, 0.126)
\pstThreeDLine[linecolor=black](0.357, -0.254, 1.041)(-0.079, -0.155, 0.019)
\pstThreeDLine[linecolor=gray](-1.043, -0.054, -2.359)(-0.043, -0.281, -0.009)
\pstThreeDLine[linecolor=black](0.157, -0.154, 0.741)(-0.138, -0.087, 0.050)
\pstThreeDLine[linecolor=black](0.357, 0.346, 1.641)(-0.273, 0.489, 0.163)
\pstThreeDLine[linecolor=gray](-0.843, -0.754, -0.459)(-0.577, -0.815, 0.167)
\pstThreeDLine[linecolor=black](0.557, 0.146, 0.741)(0.212, 0.224, -0.069)
\pstThreeDLine[linecolor=black](0.457, -0.154, 1.841)(-0.277, 0.013, 0.119)
\pstThreeDLine[linecolor=black](0.857, -0.054, 1.241)(0.281, 0.077, -0.112)
\pstThreeDLine[linecolor=gray](-0.843, -1.054, -0.259)(-0.659, -1.096, 0.175)
\pstThreeDLine[linecolor=black](0.057, 0.146, 1.041)(-0.319, 0.231, 0.158)
\pstThreeDLine[linecolor=black](0.857, 0.246, 1.941)(0.041, 0.431, 0.024)
\pstThreeDLine[linecolor=gray](-0.443, 0.846, -2.059)(0.390, 0.657, -0.102)
\pstThreeDLine[linecolor=black](0.457, -0.354, 1.141)(-0.033, -0.243, -0.009)
\pstThreeDLine[linecolor=gray](-1.043, 0.346, -2.159)(-0.101, 0.132, 0.056)
\pstThreeDLine[linecolor=gray](-1.443, 0.146, -2.459)(-0.339, -0.105, 0.134)
\pstThreeDLine[linecolor=black](0.557, 0.146, 1.541)(-0.074, 0.289, 0.060)
\pstThreeDLine[linecolor=black](0.357, -0.854, 0.741)(0.008, -0.775, -0.078)
\pstThreeDLine[linecolor=black](0.157, -0.854, 1.241)(-0.341, -0.741, 0.073)
\pstThreeDLine[linecolor=black](1.557, -0.254, 2.341)(0.474, -0.008, -0.202)
\pstThreeDLine[linecolor=gray](-0.943, -0.654, -0.459)(-0.658, -0.719, 0.211)
\pstThreeDLine[linecolor=black](1.157, 0.146, 0.941)(0.649, 0.261, -0.251)
\pstThreeDLine[linecolor=gray](-0.343, -0.654, -0.059)(-0.293, -0.666, 0.060)
\pstThreeDLine[linecolor=black](0.457, 0.246, 0.941)(0.059, 0.336, 0.007)
\pstThreeDLine[linecolor=black](0.957, -0.254, 1.041)(0.430, -0.134, -0.196)
\pstThreeDLine[linecolor=black](0.257, -0.154, 0.941)(-0.124, -0.067, 0.046)
\pstThreeDLine[linecolor=black](0.657, 0.146, 1.341)(0.082, 0.277, -0.008)
\pstThreeDLine[linecolor=black](0.857, 0.246, 1.941)(0.041, 0.431, 0.024)
\pstThreeDLine[linecolor=black](0.857, 0.046, 0.641)(0.499, 0.127, -0.200)
\pstThreeDLine[linecolor=gray](-1.043, 0.346, -1.859)(-0.208, 0.156, 0.104)
\pstThreeDLine[linecolor=gray](-0.943, -0.054, -2.359)(0.042, -0.278, -0.045)
\pstThreeDLine[linecolor=black](1.057, 0.146, 1.941)(0.207, 0.339, -0.055)
\pstThreeDLine[linecolor=gray](-1.343, -0.754, -2.459)(-0.286, -0.994, 0.025)
\pstThreeDLine[linecolor=gray](-1.543, -0.054, -2.659)(-0.360, -0.323, 0.122)
\pstThreeDLine[linecolor=gray](-0.643, -0.354, 0.141)(-0.608, -0.362, 0.224)
\pstThreeDLine[linecolor=gray](-0.843, 0.546, -2.359)(0.147, 0.321, -0.032)
\pstThreeDLine[linecolor=black](0.557, -0.154, 0.541)(0.273, -0.090, -0.125)
\pstThreeDLine[linecolor=gray](-0.643, 0.446, -2.259)(0.278, 0.237, -0.095)
\pstThreeDLine[linecolor=black](-0.043, -0.354, 0.141)(-0.100, -0.341, 0.009)
\pstThreeDLine[linecolor=gray](-0.343, 1.146, -2.359)(0.592, 0.933, -0.162)
\pstThreeDLine[linecolor=black](1.757, -0.054, 2.841)(0.471, 0.238, -0.178)
\pstThreeDLine[linecolor=black](0.457, -0.254, 1.341)(-0.101, -0.127, 0.031)
\pstThreeDLine[linecolor=black](0.557, 0.046, 1.741)(-0.149, 0.206, 0.083)
\pstThreeDLine[linecolor=black](0.457, -0.554, 1.141)(-0.040, -0.441, -0.026)
\pstThreeDLine[linecolor=black](-0.043, -0.354, 1.341)(-0.529, -0.244, 0.201)
\pstThreeDLine[linecolor=gray](-0.843, -0.054, -2.159)(0.055, -0.258, -0.048)
\pstThreeDLine[linecolor=black](0.857, 0.046, 0.941)(0.391, 0.152, -0.152)
\pstThreeDLine[linecolor=black](0.157, -0.354, 1.341)(-0.359, -0.237, 0.130)
\pstThreeDLine[linecolor=gray](-0.743, 0.446, -2.359)(0.229, 0.225, -0.076)
\pstThreeDLine[linecolor=gray](-1.043, 0.046, -2.159)(-0.111, -0.166, 0.031)
\pstThreeDLine[linecolor=black](-0.143, -0.254, 0.741)(-0.395, -0.197, 0.149)
\pstThreeDLine[linecolor=gray](-0.743, 0.746, -2.159)(0.167, 0.539, -0.019)
\pstThreeDLine[linecolor=black](0.757, -0.054, 0.641)(0.410, 0.025, -0.172)
\pstThreeDLine[linecolor=black](0.557, -0.254, 1.841)(-0.195, -0.083, 0.075)
\pstThreeDLine[linecolor=gray](-0.643, 1.046, -2.259)(0.298, 0.832, -0.047)
\pstThreeDLine[linecolor=black](0.557, -0.254, 1.841)(-0.195, -0.083, 0.075)
\pstThreeDLine[linecolor=black](1.857, -0.254, 2.941)(0.514, 0.051, -0.214)
\pstThreeDLine[linecolor=gray](-0.043, 0.946, -2.559)(0.911, 0.729, -0.317)
\pstThreeDLine[linecolor=gray](-0.943, 0.046, -2.259)(0.009, -0.171, -0.021)
\pstThreeDLine[linecolor=gray](-0.443, 0.646, -2.259)(0.454, 0.442, -0.151)
\pstThreeDLine[linecolor=gray](-0.743, 0.446, -2.359)(0.229, 0.225, -0.076)
\pstThreeDLine[linecolor=black](0.157, 0.346, 0.741)(-0.120, 0.409, 0.091)
\pstThreeDLine[linecolor=black](0.657, -0.054, 1.741)(-0.068, 0.111, 0.040)
\pstThreeDLine[linecolor=gray](-0.343, -0.654, 0.041)(-0.328, -0.657, 0.076)
\pstThreeDLine[linecolor=black](1.357, 0.546, 2.341)(0.332, 0.779, -0.066)
\pstThreeDLine[linecolor=black](1.057, 0.046, 1.341)(0.418, 0.191, -0.159)
\pstThreeDLine[linecolor=black](0.357, -0.154, 0.541)(0.103, -0.096, -0.053)
\pstThreeDLine[linecolor=black](0.657, -0.254, 0.841)(0.247, -0.161, -0.121)
\pstThreeDLine[linecolor=black](0.157, -0.054, 1.041)(-0.241, 0.036, 0.106)
\pstThreeDLine[linecolor=gray](-0.443, 0.346, -2.059)(0.372, 0.161, -0.143)
\pstThreeDLine[linecolor=black](-0.343, -0.454, 0.641)(-0.536, -0.410, 0.189)
\pstThreeDLine[linecolor=black](0.857, -0.054, 1.441)(0.209, 0.093, -0.080)
\pstThreeDLine[linecolor=black](1.857, 0.746, 2.941)(0.548, 1.043, -0.132)
\pstThreeDLine[linecolor=gray](-0.743, 0.346, -2.259)(0.189, 0.134, -0.068)

\psset{dotstyle=Bo,fillcolor=gray}
\dataplotThreeD[plotstyle=dots,showpoints=true]{\dataIrisDLno}
%\pstThreeDSquare[fillstyle=solid, fillcolor=lightgray]
\pstThreeDSquare[fillstyle=vlines,
hatchcolor=lightgray,hatchwidth=0.1\pslinewidth,hatchsep=3\pslinewidth]
(3.55,1.97,-1.325)(-5.12,-5.92,1.6)(-1.98,1.98,1.05)
\pstThreeDCoor[xMin=-2, xMax= 2, yMin=-2,
        yMax=2, zMin=-3, zMax=4, Dx=0.5, Dy=0.5, Dz=1,
        linewidth=1pt,linecolor=black,arrowscale=2]
\pstThreeDPut(2.3,0,0){$X_1$}
\pstThreeDPut(0,2.5,0){$X_2$}
\pstThreeDPut(0,0,4.3){$X_3$}
\pstThreeDLine[linewidth=2pt,arrows=->](2.56,2.96,-0.8)(-2.56,-2.96,0.8)
\pstThreeDLine[linewidth=2pt,arrows=->](0.99,-0.99,-0.525)(-0.99,0.99,0.525)
\psset{dotstyle=Bo,fillcolor=white}
\dataplotThreeD[plotstyle=dots,showpoints=true]{\dataIrisDRno}
\pstThreeDBox[linecolor=gray](-2,-2,-3)(4,0,0)(0,4,0)(0,0,7)
%\pstThreeDBox[](-3.5,-2.5,-3)(5,0,0)(0,4,0)(0,0,7.5)
\end{pspicture}
}%}
}
\end{frame}



\begin{frame}{Best $\textbf{\textit{r}}$-dimensional Approximation}
  \small

To f\/{i}nd the best $r$-dimensional
approximation to $\bD$,
we compute the eigenvalues of $\cov$.
Because $\cov$ is positive
  semidef\/{i}nite, its eigenvalues are non-negative
\begin{align*}
  \lambda_{1} \geq \lambda_{2} \geq \cdots \lambda_{r}\geq
  \lambda_{r+1} \cdots \geq \lambda_{d} \ge 0
\end{align*}
We select the $r$ largest eigenvalues, and their
corresponding eigenvectors to form the best $r$-dimensional
approximation.

\medskip
{\bf Total Projected Variance:}
Let $\bU_r = \matr{\bu_1 & \cdots & \bu_r }$ be the $r$-dimensional
basis vector matrix, withe the 
projection matrix given as
$\bP_r = \bU_r\bU_r^T = \sum_{i=1}^r \bu_i \bu_i^T$.

\medskip
Let $\bA$ denote the dataset formed by the coordinates of the
projected points in the \hbox{$r$-dimensional} subspace.
The projected variance is given as
\begin{align*}
    var(\bA) = {1\over n} \sum_{i=1}^n \bx_i^T\bP_r\bx_i =
    \sum_{i=1}^r \bu_i^T \cov \bu_i = \sum_{i=1}^r \lambda_i
\end{align*}

\medskip
{\bf Mean Squared Error:}
The mean squared error in $r$~dimensions is
\begin{align*}
    MSE &= {1\over n} \sum_{i=1}^n \norm{\bx_i -
    \bx_i'}^2 = 
    var(\bD) - \sum_{i=1}^r \lambda_i = 
	\sum_{i=1}^d \lambda_i - \sum_{i=1}^r \lambda_i
\end{align*}
\end{frame}



\begin{frame}{Choosing the Dimensionality}

  One criteria for
choosing $r$ is to compute the fraction of the total variance
captured by the
f\/{i}rst $r$ principal components, computed as
\begin{align*}
\tcbhighmath{
  f(r) =
  \frac{\lambda_1+\lambda_2+\cdots+\lambda_r}
  {\lambda_1+\lambda_2+\cdots+\lambda_d}
  = \frac{\sum_{i=1}^r \lambda_i}{\sum_{i=1}^d \lambda_i}
  = \frac{\sum_{i=1}^r \lambda_i}{var(\bD)}
}
  \label{eq:eda:dimr:fracvar}
\end{align*}

Given a certain desired variance threshold, say $\alpha$,
starting from
the f\/{i}rst principal component, we keep on adding additional
components, and stop at the smallest value $r$,
for which $f(r) \geq \alpha$. In other words, we select the
fewest number of dimensions such that the subspace spanned by
those $r$ dimensions captures at least $\alpha$ fraction (say 0.9) 
of the
total variance.
\end{frame}


\newcommand{\PCA}{\textsc{PCA}}
\begin{frame}{Principal Component Analysis: Algorithm}
\begin{tightalgo}[H]{\textwidth-18pt}
\SetKwInOut{Algorithm}{\PCA\ ($\bD, \alpha$)}
\Algorithm{}
$\bmu = {1\over n}\sum_{i=1}^n \bx_i$ \tcp{compute mean}
$\bZ = \bD - \bone \cdot \bmu^T$ \tcp{center the data}
$\cov = {1\over n} \lB(\bZ^T \bZ\rB)$ \tcp{compute covariance matrix}
$(\lambda_1, \lambda_2, \ldots, \lambda_d) = \text{eigenvalues}(\cov)$ \tcp{compute eigenvalues}
$\bU = \matr{\bu_1 & \bu_2 & \cdots & \bu_d} = \text{eigenvectors}(\cov)$ \tcp{compute eigenvectors}
$f(r) = \frac{\sum_{i=1}^r \lambda_i}{\sum_{i=1}^d \lambda_i},\;\; 
\text{for all } r=1, 2, \ldots, d$ \tcp{fraction of total variance}
Choose smallest $r$ \ so that $f(r) \ge \alpha$ \tcp{choose dimensionality}
$\bU_r = \matr{\bu_1 & \bu_2 & \cdots & \bu_r}$ \tcp{reduced basis}
$\bA = \{\ba_i \mid \ba_i = \bU_r^T\bx_i, \text{for } i=1, \ldots, n\}$ \tcp{reduced dimensionality data}
\end{tightalgo}
\end{frame}

\begin{frame}{Iris Principal Components}
Covariance matrix:
    \begin{align*}
        \cov = \amatr{r}{
        0.681 &-0.039 &1.265\\
        -0.039& 0.187 &-0.320\\
        1.265 &-0.32 &3.092\\
        }
    \end{align*}
	
	\medskip
    The eigenvalues and eigenvectors of $\cov$
    \begin{alignat*}{3}
    \lambda_1 & = 3.662 &\qquad\lambda_2 &= 0.239 &\qquad\lambda_3 &= 0.059\\
        \bu_1 &= \amatr{r}{-0.390\\0.089\\-0.916} &\bu_2 &= \amatr{r}{-0.639\\-0.742\\0.200} &\bu_3 &= \amatr{r}{-0.663\\0.664\\0.346}
    \end{alignat*}

    The total variance is therefore $\lambda_1 + \lambda_2 +
    \lambda_3  = 3.662  +0.239 + 0.059 = 3.96$.

	The fraction of total variance for different values of $r$ is
    given as
    \begin{center}
    \vspace*{12pt plus2pt minus1pt}
    {\renewcommand{\arraystretch}{1.1}\begin{tabular}{|c||c|c|c|}
        \hline
        $r$ & 1 & 2 & 3\\
        \hline
        $f(r)$ & 0.925 & 0.985 & 1.0\\
        \hline
    \end{tabular}}
        \vspace*{12pt plus2pt minus1pt}
    \end{center}
	This $r=2$ PCs are need to capture $\alpha=0.95$ fraction of
	variance.

\end{frame}



\begin{frame}{Iris Data: Optimal 3D PC Basis}
\psset{unit=0.5in}
\psset{arrowscale=2}
\psset{Alpha=60,Beta=-30}
  \begin{center}
  \begin{tabular}{ccc}
%\psset{Alpha=-190,Beta=-45}
\psset{nameX=$~$, nameY=$~$, nameZ=$~$}
%\subfloat[Original Basis]{
%\label{fig:eda:dimr:3dirisOrig}
\scalebox{0.6}{
\begin{pspicture}(-2,-4.5)(2,4.5)
\pstThreeDCoor[xMin=-2, xMax= 2, yMin=-2,
        yMax=2, zMin=-3, zMax=4, Dx=0.5, Dy=0.5, Dz=1,
        linewidth=2pt,linecolor=black]
\pstThreeDPut(2.3,0,0){$X_1$}
\pstThreeDPut(0,2.5,0){$X_2$}
\pstThreeDPut(0,0,4.3){$X_3$}
\psset{dotstyle=Bo,dotscale=1.75,fillcolor=lightgray}
\dataplotThreeD[plotstyle=dots,showpoints=true]{\dataSLWPL}
\pstThreeDBox[linecolor=gray](-2,-2,-3)(4,0,0)(0,4,0)(0,0,7)
\end{pspicture}
}%}
&
\hspace{1in}
&
%\subfloat[Optimal Basis]{
%\label{fig:eda:dimr:3dirisOpt}
\scalebox{0.55}{
\begin{pspicture}(-2,-4.5)(2,4.5)
%\pstThreeDSquare[fillstyle=hlines, hatchangle=64,
%hatchcolor=gray,hatchwidth=0.1pt,hatchsep=3pt]
%(3.37,0.5,2.61)(-2.86,0.64,-6.72)(-2.56,-2.96, 0.8)
%\pstThreeDSquare[fillstyle=hlines, hatchangle=32,
%hatchcolor=gray,hatchwidth=0.1pt,hatchsep=3pt]
%(3.37,0.5,2.61)(-2.86,0.64,-6.72)(-1.32,1.32,0.7)
%\pstThreeDSquare[fillstyle=hlines, hatchangle=32,
%hatchcolor=gray,hatchwidth=0.1pt,hatchsep=3pt]
%(3.37,0.5,2.61)(-2.56,-2.96,0.8)(-1.32,1.32,0.7)
%\pstThreeDCoor[xMin=-2, xMax= 2, yMin=-2,
        %yMax=2, zMin=-3, zMax=4, Dx=0.5, Dy=0.5, Dz=1,
        %linewidth=1pt, linecolor=gray]
%\pstThreeDPut(2.3,0,0){$X_1$}
%\pstThreeDPut(0,2.5,0){$X_2$}
%\pstThreeDPut(0,0,4.3){$X_3$}
\psset{dotstyle=Bo,dotscale=1.75,fillcolor=lightgray}
\dataplotThreeD[plotstyle=dots,showpoints=true]{\dataSLWPL}
%\pstThreeDBox[](3.37,0.5,2.61)(-2.86,0.64,-6.72)(-2.56,-2.96,0.8)(-1.32,1.32,0.7)
\pstThreeDBox[](4.01,1.24,2.41)(-2.86,0.64,-6.72)(-3.84,-4.44,1.2)(-1.32,1.32,0.7)
\pstThreeDLine[linewidth=2pt,arrows=->](1.43,-0.32,3.36)(-1.43,0.32,-3.36)
%\pstThreeDLine[linewidth=2pt,arrows=->](1.28,1.48,-0.4)(-1.28,-1.48,0.4)
\pstThreeDLine[linewidth=2pt,arrows=->](0.66,-0.66,-0.35)(-0.66,0.66,0.35)
\pstThreeDPut(-1.6,0.35,-3.6){$\bu_1$}
%\pstThreeDPut(-1.4,-1.63,0.44){$\bu_2$}
\pstThreeDPut(-0.79,0.79,0.37){$\bu_3$}
%\pstThreeDLine[linewidth=2pt,arrows=->](2.56,2.96,-0.8)(-2.56,-2.96,0.8)
\pstThreeDLine[linewidth=2pt,arrows=->](1.92,2.22,-0.6)(-1.92,-2.22,0.6)
\pstThreeDPut(-2.05,-2.37,0.64){$\bu_2$}
\pstThreeDBox[linecolor=gray](-2,-2,-3)(4,0,0)(0,4,0)(0,0,7)
\end{pspicture}
}\\
Iris Data (3D) & & Optimal 3D Basis\\
\end{tabular}
\end{center}
\end{frame}



\readdata{\dataA}{EDA/dimreduction/figs/iris-2dproj-dimr.dat}
\begin{frame}{Iris Principal Components: Projected Data (2D)}

\centering
%\vspace{0.1in}
\scalebox{0.9}{
\psset{dotstyle=Bo,dotscale=1.5,fillcolor=lightgray,arrowscale=2,PointName=none}
\psset{xAxisLabel=$\bu_1$,yAxisLabel= $\bu_2$}
\psgraph[Dx=1,Dy=0.5,Ox=-4,Oy=-1.5]{->}(-4,-1.5)(4.1,1.6){4in}{2.5in}%
\dataplot[plotstyle=dots,showpoints=true]{\dataA}
\endpsgraph
}

\end{frame}


\begin{frame}{Geometry of PCA}
  \small
Geometrically, when $r=d$,
PCA corresponds to a orthogonal change of basis, 
so that the total variance
is captured by the sum of the variances along each of the
principal directions $\bu_1, \bu_2, \ldots, \bu_d$, and
further, all covariances are zero.

\smallskip
Let $\bU$ be the $d
\times d$ orthogonal matrix
$\bU = \matr{\bu_1 & \bu_2 & \cdots & \bu_d}$,
with $\bU^{-1} = \bU^T$.
Let $\bLambda = diag(\lambda_1, \cdots, \lambda_d)$ be
the diagonal matrix of eigenvalues.
Each principal component $\bu_i$ corresponds
to an eigenvector of the covariance matrix $\cov$
\begin{align*}
  \cov \bu_i = \lambda_i \bu_i \mbox{ for all } 1 \le i \le d
\end{align*}
which can be written compactly in matrix notation:
\begin{align*}
  \cov\bU =  \bU \bLambda \text{ which implies }
  \cov =  \bU \bLambda \bU^T
\end{align*}
Thus, $\bLambda$ represents the covariance matrix in the new PC basis.


\smallskip
In the new PC basis, the equation
\begin{align*}
    \bx^T \cov^{-1} \bx = 1
\end{align*}
def\/{i}nes a $d$-dimensional ellipsoid (or hyper-ellipse). The
eigenvectors $\bu_i$ of $\cov$, that is, the principal components,
are the directions for the principal axes of the ellipsoid. The
square roots of the eigenvalues, that is, $\sqrt{\lambda_i}$, give
the lengths of the semi-axes.

\end{frame}



\readdata{\dataSLWPP}{EDA/dimreduction/figs/iris-slwpl-p.dat}
\begin{frame}{Iris: Elliptic Contours in Standard Basis}
\psset{unit=0.5in}
\psset{arrowscale=2}
\centerline{
%\subfloat[Elliptic contours in standard basis]{
%\label{fig:eda:dimr:3dellipsoidSB}
\scalebox{0.6}{
\psset{Alpha=60,Beta=-30}
\psset{nameX=$~$, nameY=$~$, nameZ=$~$}
\begin{pspicture}(-4,-4.5)(4,4.5)
\psset{dotstyle=Bo,dotscale=1.75,fillcolor=lightgray}
\dataplotThreeD[plotstyle=dots,showpoints=true]{\dataSLWPL}
\pstThreeDBox[](4.01,1.24,2.41)(-2.86,0.64,-6.72)(-3.84,-4.44,1.2)(-1.32,1.32,0.7)
\pstThreeDLine[linewidth=2pt,arrows=->](1.43,-0.32,3.36)(-1.43,0.32,-3.36)
\pstThreeDLine[linewidth=2pt,arrows=->](0.66,-0.66,-0.35)(-0.66,0.66,0.35)
\pstThreeDPut(-1.6,0.35,-3.6){$\bu_1$}
\pstThreeDPut(-0.79,0.79,0.37){$\bu_3$}
\pstThreeDLine[linewidth=2pt,arrows=->](1.92,2.22,-0.6)(-1.92,-2.22,0.6)
\pstThreeDPut(-2.05,-2.37,0.64){$\bu_2$}
\pstThreeDBox[linecolor=gray](-2,-2,-3)(4,0,0)(0,4,0)(0,0,7)
\pstThreeDEllipse(0,0,0)(-1.43,0.32,-3.36)(-1.92,-2.22,0.6)
\pstThreeDEllipse[linestyle=dashed](0,0,0)(-1.43,0.32,-3.36)(-0.66,0.66,0.35)
\pstThreeDEllipse[linestyle=solid](0,0,0)(-1.92,-2.22,0.6)(-0.66,0.66,0.35)
\end{pspicture}
}}%}
\end{frame}

\begin{frame}[fragile]{Iris: Axis-Parallel Ellipsoid in PC Basis}
\centerline{
\scalebox{0.8}{
\centering
\psset{unit=1}
\psset{arrowscale=2}
\begin{pspicture}(-5,-3.3)(5,3)
\psset{viewpoint=30 60 30 rtp2xyz,Decran=50}
\psset{lightsrc=viewpoint,opacity=0.75,incolor=white}
%\psSolid[object=sphere, hollow, r=3.66, ngrid=20 20,
%transform={1 0.4 0.2 scaleOpoint3d}, linewidth=0.5pt]%
%\psSolid[object=parallelepiped,a=7.29,b=2.72,c=1.27,action=draw](0,0,0)
\psSolid[object=sphere, hollow, r=1.91, ngrid=20 20,
transform={1 0.4 0.2 scaleOpoint3d}, linewidth=0.5pt]%
\psSolid[object=parallelepiped,a=7.29,b=2.72,c=1.27,action=draw](0,0,0)
\psset{dotstyle=Bo,dotscale=1.75,fillcolor=lightgray}
\psPoint(-0.431, 0.037, 0.016){p0}
\psdot(p0)
\psPoint(-1.454, -0.192, -0.055){p1}
\psdot(p1)
\psPoint(-1.080, -0.080, -0.062){p2}
\psdot(p2)
\psPoint(2.660, 0.086, 0.021){p3}
\psdot(p3)
\psPoint(-0.358, 0.233, -0.117){p4}
\psdot(p4)
\psPoint(2.712, 0.052, 0.001){p5}
\psdot(p5)
\psPoint(-2.132, 0.012, 0.047){p6}
\psdot(p6)
\psPoint(-1.244, 0.224, 0.052){p7}
\psdot(p7)
\psPoint(-2.018, -0.085, 0.020){p8}
\psdot(p8)
\psPoint(-2.254, 0.109, -0.046){p9}
\psdot(p9)
\psPoint(2.417, -0.183, 0.028){p10}
\psdot(p10)
\psPoint(2.426, -0.212, 0.041){p11}
\psdot(p11)
\psPoint(-0.353, 0.088, 0.042){p12}
\psdot(p12)
\psPoint(-1.151, 0.042, 0.038){p13}
\psdot(p13)
\psPoint(2.442, 0.047, -0.025){p14}
\psdot(p14)
\psPoint(2.338, -0.060, 0.008){p15}
\psdot(p15)
\psPoint(2.430, -0.068, 0.001){p16}
\psdot(p16)
\psPoint(-0.279, 0.139, 0.009){p17}
\psdot(p17)
\psPoint(2.512, -0.046, -0.019){p18}
\psdot(p18)
\psPoint(-2.570, -0.211, -0.005){p19}
\psdot(p19)
\psPoint(-1.256, 0.109, 0.078){p20}
\psdot(p20)
\psPoint(-1.582, -0.036, 0.006){p21}
\psdot(p21)
\psPoint(2.245, -0.544, 0.041){p22}
\psdot(p22)
\psPoint(-0.136, 0.272, -0.011){p23}
\psdot(p23)
\psPoint(-0.360, 0.465, 0.103){p24}
\psdot(p24)
\psPoint(2.347, -0.090, 0.021){p25}
\psdot(p25)
\psPoint(-0.154, 0.331, -0.038){p26}
\psdot(p26)
\psPoint(2.559, 0.123, 0.014){p27}
\psdot(p27)
\psPoint(-2.405, -0.167, -0.046){p28}
\psdot(p28)
\psPoint(-2.323, -0.116, 0.041){p29}
\psdot(p29)
\psPoint(2.501, -0.335, 0.001){p30}
\psdot(p30)
\psPoint(2.687, -0.033, -0.046){p31}
\psdot(p31)
\psPoint(-1.131, 0.301, 0.031){p32}
\psdot(p32)
\psPoint(-0.245, 0.165, -0.038){p33}
\psdot(p33)
\psPoint(0.936, 0.294, -0.028){p34}
\psdot(p34)
\psPoint(-0.084, 0.238, -0.032){p35}
\psdot(p35)
\psPoint(-0.327, 0.144, -0.018){p36}
\psdot(p36)
\psPoint(2.060, -0.180, 0.069){p37}
\psdot(p37)
\psPoint(-0.833, 0.159, -0.116){p38}
\psdot(p38)
\psPoint(-1.365, 0.147, -0.048){p39}
\psdot(p39)
\psPoint(-0.589, 0.138, 0.076){p40}
\psdot(p40)
\psPoint(-0.876, 0.018, 0.017){p41}
\psdot(p41)
\psPoint(-1.974, -0.089, -0.013){p42}
\psdot(p42)
\psPoint(-2.911, -0.123, -0.037){p43}
\psdot(p43)
\psPoint(-0.341, 0.203, 0.016){p44}
\psdot(p44)
\psPoint(2.564, 0.094, -0.032){p45}
\psdot(p45)
\psPoint(-2.458, -0.134, -0.025){p46}
\psdot(p46)
\psPoint(0.253, 0.151, -0.059){p47}
\psdot(p47)
\psPoint(2.339, -0.234, 0.001){p48}
\psdot(p48)
\psPoint(2.009, -0.350, -0.025){p49}
\psdot(p49)
\psPoint(-0.362, 0.118, 0.029){p50}
\psdot(p50)
\psPoint(-0.974, 0.229, 0.078){p51}
\psdot(p51)
\psPoint(2.812, 0.188, 0.014){p52}
\psdot(p52)
\psPoint(-2.210, -0.010, 0.127){p53}
\psdot(p53)
\psPoint(2.274, -0.170, -0.052){p54}
\psdot(p54)
\psPoint(-1.835, -0.072, 0.113){p55}
\psdot(p55)
\psPoint(-1.912, -0.152, -0.020){p56}
\psdot(p56)
\psPoint(-2.875, -0.271, -0.091){p57}
\psdot(p57)
\psPoint(-0.985, 0.085, -0.003){p58}
\psdot(p58)
\psPoint(0.227, 0.095, 0.001){p59}
\psdot(p59)
\psPoint(-1.828, 0.217, 0.033){p60}
\psdot(p60)
\psPoint(-1.661, 0.086, -0.014){p61}
\psdot(p61)
\psPoint(2.622, -0.114, 0.001){p62}
\psdot(p62)
\psPoint(2.199, -0.048, -0.011){p63}
\psdot(p63)
\psPoint(-0.223, 0.106, 0.049){p64}
\psdot(p64)
\psPoint(-0.511, 0.189, 0.103){p65}
\psdot(p65)
\psPoint(-1.235, 0.194, 0.065){p66}
\psdot(p66)
\psPoint(2.442, 0.047, -0.025){p67}
\psdot(p67)
\psPoint(3.062, -0.065, 0.046){p68}
\psdot(p68)
\psPoint(2.443, -0.127, -0.032){p69}
\psdot(p69)
\psPoint(-3.157, -0.536, 0.009){p70}
\psdot(p70)
\psPoint(-3.644, -0.088, -0.089){p71}
\psdot(p71)
\psPoint(-0.344, 0.029, -0.051){p72}
\psdot(p72)
\psPoint(2.427, -0.242, -0.066){p73}
\psdot(p73)
\psPoint(2.677, 0.026, 0.047){p74}
\psdot(p74)
\psPoint(2.437, 0.076, 0.021){p75}
\psdot(p75)
\psPoint(2.711, 0.226, 0.007){p76}
\psdot(p76)
\psPoint(-1.116, 0.068, -0.009){p77}
\psdot(p77)
\psPoint(2.564, 0.094, -0.032){p78}
\psdot(p78)
\psPoint(-0.754, 0.065, 0.010){p79}
\psdot(p79)
\psPoint(-1.613, -0.062, 0.112){p80}
\psdot(p80)
\psPoint(0.683, 0.403, -0.020){p81}
\psdot(p81)
\psPoint(-0.884, -0.126, -0.003){p82}
\psdot(p82)
\psPoint(-1.879, 0.076, 0.047){p83}
\psdot(p83)
\psPoint(-1.477, -0.104, -0.035){p84}
\psdot(p84)
\psPoint(0.473, 0.508, -0.046){p85}
\psdot(p85)
\psPoint(-0.964, 0.026, 0.084){p86}
\psdot(p86)
\psPoint(-2.092, -0.137, 0.054){p87}
\psdot(p87)
\psPoint(2.135, -0.303, 0.028){p88}
\psdot(p88)
\psPoint(-1.256, 0.080, -0.028){p89}
\psdot(p89)
\psPoint(2.416, -0.009, 0.035){p90}
\psdot(p90)
\psPoint(2.829, 0.129, 0.040){p91}
\psdot(p91)
\psPoint(-1.617, -0.062, 0.052){p92}
\psdot(p92)
\psPoint(-0.894, 0.222, -0.109){p93}
\psdot(p93)
\psPoint(-1.274, 0.313, -0.048){p94}
\psdot(p94)
\psPoint(-2.776, -0.135, -0.078){p95}
\psdot(p95)
\psPoint(0.730, 0.399, 0.006){p96}
\psdot(p96)
\psPoint(-1.301, -0.264, -0.069){p97}
\psdot(p97)
\psPoint(0.130, 0.277, -0.045){p98}
\psdot(p98)
\psPoint(-1.019, -0.114, 0.037){p99}
\psdot(p99)
\psPoint(-1.350, -0.086, -0.088){p100}
\psdot(p100)
\psPoint(-0.976, 0.056, 0.011){p101}
\psdot(p101)
\psPoint(-1.473, -0.104, 0.025){p102}
\psdot(p102)
\psPoint(-2.092, -0.137, 0.054){p103}
\psdot(p103)
\psPoint(-0.918, -0.181, -0.063){p104}
\psdot(p104)
\psPoint(2.141, 0.015, 0.055){p105}
\psdot(p105)
\psPoint(2.525, 0.068, -0.046){p106}
\psdot(p106)
\psPoint(-2.178, -0.158, 0.014){p107}
\psdot(p107)
\psPoint(2.711, 0.370, -0.092){p108}
\psdot(p108)
\psPoint(3.034, 0.198, 0.013){p109}
\psdot(p109)
\psPoint(0.090, 0.281, 0.048){p110}
\psdot(p110)
\psPoint(2.539, -0.136, 0.021){p111}
\psdot(p111)
\psPoint(-0.727, -0.053, -0.057){p112}
\psdot(p112)
\psPoint(2.361, -0.149, -0.012){p113}
\psdot(p113)
\psPoint(-0.144, 0.128, -0.031){p114}
\psdot(p114)
\psPoint(2.397, -0.442, 0.034){p115}
\psdot(p115)
\psPoint(-3.294, -0.205, -0.043){p116}
\psdot(p116)
\psPoint(-1.430, 0.066, -0.001){p117}
\psdot(p117)
\psPoint(-1.809, -0.016, 0.053){p118}
\psdot(p118)
\psPoint(-1.273, 0.139, -0.055){p119}
\psdot(p119)
\psPoint(-1.244, 0.224, 0.052){p120}
\psdot(p120)
\psPoint(2.303, 0.059, -0.045){p121}
\psdot(p121)
\psPoint(-1.193, -0.157, -0.042){p122}
\psdot(p122)
\psPoint(-1.322, 0.173, 0.025){p123}
\psdot(p123)
\psPoint(2.491, -0.131, -0.006){p124}
\psdot(p124)
\psPoint(2.389, 0.080, -0.005){p125}
\psdot(p125)
\psPoint(-0.646, 0.171, 0.037){p126}
\psdot(p126)
\psPoint(2.335, -0.204, 0.048){p127}
\psdot(p127)
\psPoint(-0.888, -0.126, -0.063){p128}
\psdot(p128)
\psPoint(-1.927, 0.081, 0.020){p129}
\psdot(p129)
\psPoint(2.414, -0.327, 0.068){p130}
\psdot(p130)
\psPoint(-1.927, 0.081, 0.020){p131}
\psdot(p131)
\psPoint(-3.443, -0.164, -0.076){p132}
\psdot(p132)
\psPoint(2.446, -0.475, -0.046){p133}
\psdot(p133)
\psPoint(2.442, 0.047, -0.025){p134}
\psdot(p134)
\psPoint(2.300, -0.259, -0.012){p135}
\psdot(p135)
\psPoint(2.491, -0.131, -0.006){p136}
\psdot(p136)
\psPoint(-0.710, -0.083, 0.077){p137}
\psdot(p137)
\psPoint(-1.857, -0.012, 0.026){p138}
\psdot(p138)
\psPoint(0.038, 0.285, -0.038){p139}
\psdot(p139)
\psPoint(-2.627, -0.321, 0.055){p140}
\psdot(p140)
\psPoint(-1.637, -0.176, -0.041){p141}
\psdot(p141)
\psPoint(-0.649, -0.002, -0.030){p142}
\psdot(p142)
\psPoint(-1.050, -0.025, -0.062){p143}
\psdot(p143)
\psPoint(-1.020, 0.059, 0.044){p144}
\psdot(p144)
\psPoint(2.090, -0.154, -0.038){p145}
\psdot(p145)
\psPoint(-0.494, 0.274, 0.030){p146}
\psdot(p146)
\psPoint(-1.660, -0.088, -0.021){p147}
\psdot(p147)
\psPoint(-3.354, -0.461, 0.057){p148}
\psdot(p148)
\psPoint(2.391, -0.094, -0.012){p149}
\psdot(p149)

\axesIIID[axisnames={\bu_1,\bu_2,\bu_3},linewidth=2pt](-3.65,-1.36,-0.64)(3.75,1.8,1.2)
\end{pspicture}
}}%}
\end{frame}

\ifdefined\wox \begin{frame} \titlepage \end{frame} \fi

\begin{frame}{Kernel Principal Component Analysis}

Principal component analysis can be extended to f\/{i}nd nonlinear
``directions'' in the data using kernel methods.
Kernel PCA f\/{i}nds the directions of most
variance in the feature space instead of the input space.  
Using the {\em kernel trick}, all PCA operations 
can be carried out in terms
of the kernel function in input space, without having to transform the
data into feature space.


\bigskip
Let $\phi$ be a function that maps a point
$\bx$ in input space to its image $\phi(\bx_i)$ in feature space. 
Let the points in
feature space be centered and let $\cov_\phi$ be the covariance matrix. 
The first PC in feature space correspond to the dominant eigenvector
\begin{align*}
  \cov_\phi \bu_1 = \lambda_1 \bu_1
\end{align*}
  where
\begin{align*}
  \cov_\phi = {1\over n}\dsum_{i=1}^n \phi(\bx_i) \phi(\bx_i)^T
\end{align*}

\end{frame}

\begin{frame}{Kernel Principal Component Analysis}
It can be shown that $\bu_1 = \dsum_{i=1}^n c_{i} \phi(\bx_i)$.
That is, the PC direction in 
feature space is a linear combination
of the transformed points. 

\medskip
The coefficients are captured in the weight vector
$$\bc = \matr{c_1, c_2, \cdots, c_n}^T$$

\medskip
Substituting into the eigen-decomposition of $\cov_\phi$ and
simplifying, we get:
\begin{align*}
  \bK\bc = n \lambda_1 \bc = \eta_1 \bc
\end{align*}
Thus,
the weight vector $\bc$ is the eigenvector
corresponding to the
largest eigenvalue $\eta_1$ of the kernel matrix
$\bK$.
\end{frame}


\begin{frame}{Kernel Principal Component Analysis}
  \small
  The weight vector $\bc$ can be used to then find $\bu_1$ via
$\bu_1 = \dsum_{i=1}^n c_{i} \phi(\bx_i)$.

\medskip
The only constraint we impose is that $\bu_1$
should be
normalized to be a unit vector, which implies
  $\norm{\bc}^2 = {1 \over \eta_1}$.


\medskip
We cannot compute directly the
principal direction, but we can
project any point $\phi(\bx)$ onto the principal
direction $\bu_1$, as follows:
\begin{align*}
  \bu_1^T\phi(\bx) = \dsum_{i=1}^n c_i\phi(\bx_i)^T\phi(\bx) =
  \dsum_{i=1}^n c_i K(\bx_i,\bx)
\end{align*}
which requires only kernel operations. 

\medskip
We can obtain the additional principal
components by solving for the other eigenvalues and eigenvectors of
$$\bK\bc_j = n \lambda_j \bc_j = \eta_j \bc_j$$
If we sort the
eigenvalues of $\bK$ in decreasing order
$\eta_1 \ge \eta_2 \ge \cdots
\ge \eta_n \ge 0$, we can obtain the $j$th principal component as
the corresponding eigenvector $\bc_{j}$.
The variance
along the $j$th principal component is given as 
$\lambda_{j} = {\eta_{j} \over n}$. 
\end{frame}


\begin{frame}{Kernel PCA Algorithm}
\begin{tightalgo}[H]{\textwidth-18pt}
\newcommand{\KPCA}{\textsc{KernelPCA}}
\SetKwInOut{Algorithm}{\KPCA\ ($\bD, K, \alpha$)}
\Algorithm{}
$\bK = \bigl\{K(\bx_i, \bx_{j}) \bigr\}_{i,j=1,\ldots,n}$ \tcp{compute $n \times n$ kernel matrix}
$\bK  = (\bI - {1\over n}\bone_{n\times n}) \bK (\bI - {1\over
n}\bone_{n \times n})$ \tcp{center the kernel matrix}
$(\eta_1, \eta_2, \ldots, \eta_d) =
\text{eigenvalues}(\bK)$ \tcp{compute eigenvalues}
$\matr{\bc_1 & \bc_2 & \cdots & \bc_n} =
\text{eigenvectors}(\bK)$ \tcp{compute eigenvectors}
$\lambda_i = {\eta_i \over n} \text{ for all } i = 1, \ldots,
n$ \tcp{compute variance for each component}
$\bc_i = \sqrt{1 \over \eta_i} \cdot \bc_i \text{ for all } i =
1, \ldots, n$ \tcp{ensure that $\bu_i^T\bu_i=1$}
$f(r) = \frac{\sum_{i=1}^r \lambda_i}{\sum_{i=1}^d \lambda_i},
\;\; \text{for all } r=1, 2, \ldots, d$ \tcp{fraction of total
variance}
Choose smallest $r$ so that $f(r) \ge \alpha$ \tcp{choose
dimensionality}
$\bC_r = \matr{\bc_1 & \bc_2 & \cdots & \bc_r}$ \tcp{reduced
basis}
$\bA = \{\ba_i \mid \ba_i = \bC_r^T\bK_i, \text{for } i=1, \ldots,
n\}$ \tcp{reduced dimensionality data}
\end{tightalgo}
\end{frame}


\readdata{\dataI}{EDA/dimreduction/figs/iris-2d-nonlinear.dat}
\begin{frame}{Nonlinear Iris Data: PCA in Input Space}
\psset{stepFactor=0.4}
\psset{dotstyle=Bo,dotscale=1.5,fillcolor=lightgray,arrowscale=2,PointName=none}
\psset{xAxisLabel=$X_1$,yAxisLabel= $X_2$}
\psset{xAxisLabelPos={c,-0.4in},yAxisLabelPos={-0.4in,c}}
\centerline{
\scalebox{0.90}{\hspace{0.25in}
\psgraph[axesstyle=frame,Dx=0.5,Dy=0.5,Ox=-0.5,Oy=-1]{->}(-0.5,-1)(1.5,1.5){2in}{2.5in}%
\dataplot[plotstyle=dots,showpoints=true]{\dataI}
\psline[linewidth=2pt]{->}(-0.316,-1)(0.474,1.5)
\uput[r](0.4,1.55){$\bu_1$}
\psplotImp[algebraic](-0.5,-1)(1.5,1.5){%
        0.3015*x+0.9535*y}
\psplotImp[algebraic](-0.5,-1)(1.5,1.5){%
        -0.25+0.3015*x+0.9535*y}
\psplotImp[algebraic](-0.5,-1)(1.5,1.5){%
        0.25+0.3015*x+0.9535*y}
\psplotImp[algebraic](-0.5,-1)(1.5,1.5){%
        -0.5+0.3015*x+0.9535*y}
\psplotImp[algebraic](-0.5,-1)(1.5,1.5){%
        0.5+0.3015*x+0.9535*y}
\psplotImp[algebraic](-0.5,-1)(1.5,1.5){%
        -0.75+0.3015*x+0.9535*y}
\psplotImp[algebraic](-0.5,-1)(1.5,1.5){%
        0.75+0.3015*x+0.9535*y}
\psplotImp[algebraic](-0.5,-1)(1.5,1.5){%
        -1+0.3015*x+0.9535*y}
\psplotImp[algebraic](-0.5,-1)(1.5,1.5)
\hspace{0.75in}
%\subfloat[$\lambda_2=0.087$]{
%\label{fig:eda:dimr:2dnonlinearL2}
\psgraph[axesstyle=frame,Dx=0.5,Dy=0.5,Ox=-0.5,Oy=-1]{->}(-0.5,-1)(1.5,1.5){2in}{2.5in}%
\dataplot[plotstyle=dots,showpoints=true]{\dataI}
\psline[linewidth=2pt]{->}(-0.5,0.158)(1.5,-0.474)
\uput[r](1.5,-0.5){$\bu_2$}
\psplotImp[algebraic](-0.5,-1)(1.5,1.5){%
        -0.9535*x+0.3015*y}
\psplotImp[algebraic](-0.5,-1)(1.5,1.5){%
        -0.25-0.9535*x+0.3015*y}
\psplotImp[algebraic](-0.5,-1)(1.5,1.5){%
        0.25-0.9535*x+0.3015*y}
\psplotImp[algebraic](-0.5,-1)(1.5,1.5){%
        -0.5-0.9535*x+0.3015*y}
\psplotImp[algebraic](-0.5,-1)(1.5,1.5){%
        0.5-0.9535*x+0.3015*y}
\psplotImp[algebraic](-0.5,-1)(1.5,1.5){%
        -0.75-0.9535*x+0.3015*y}
\psplotImp[algebraic](-0.5,-1)(1.5,1.5){%
        0.75-0.9535*x+0.3015*y}
\psplotImp[algebraic](-0.5,-1)(1.5,1.5){%
        1-0.9535*x+0.3015*y}
\psplotImp[algebraic](-0.5,-1)(1.5,1.5)
\end{frame}


\readdata{\dataLIN}{EDA/dimreduction/figs/iris-nonlinear-PCA.dat}
\begin{frame}{Nonlinear Iris Data: Projection onto PCs}
\scalebox{0.9}{
\psset{dotstyle=Bo,dotscale=1.5,fillcolor=lightgray,arrowscale=2,PointName=none}
\psset{xAxisLabel=$\bu_1$,yAxisLabel= $\bu_2$}
%\psset{xAxisLabelPos={c,-0.4in},yAxisLabelPos={-0.4in,c}}
\centerline{
\psgraph[Dx=0.5,Dy=0.5,Ox=-0.75,Oy=-1.5]{->}(-0.75,-1.5)(1.75,0.5){3.5in}{2.5in}%
\dataplot[plotstyle=dots,showpoints=true]{\dataLIN}
\endpsgraph
}
}
\end{frame}


\def\mye{2.7183}
\begin{frame}[fragile]{Kernel PCA: 3 PCs (Contours of Constant Projection)} 
  \framesubtitle{Homogeneous Quadratic Kernel: $K(\bx_i, \bx_j) =
  (\bx_i^T\bx_j)^2$}
\centering 
\begin{figure}[!t]
%\vspace{0.1in} 
\psset{stepFactor=0.4}
\psset{dotstyle=Bo,dotscale=1.5,fillcolor=lightgray,arrowscale=2,PointName=none}
\psset{xAxisLabel=$X_1$,yAxisLabel= $X_2$}
\psset{xAxisLabelPos={c,-0.4in},yAxisLabelPos={-0.4in,c}}
\setcounter{subfigure}{0}
\captionsetup[subfloat]{captionskip=35pt} 
\centerline{
\hspace{0.25in} 
\subfloat[$\lambda_1=0.2067$]{
\scalebox{0.6}{
\psgraph[axesstyle=frame,Dx=0.5,Dy=0.5,Ox=-0.5,Oy=-1]{->}(-0.5,-1)(1.5,1.5){2in}{2.5in}%
\dataplot[plotstyle=dots,showpoints=true]{\dataI}
\psplotImp[algebraic](-0.5,-1)(1.5,1.5){%
        -0.01+1.0426*x*y+0.995*x^2+0.914*y^2}
\psplotImp[algebraic](-0.5,-1)(1.5,1.5){%
        -0.1+1.0426*x*y+0.995*x^2+0.914*y^2}
\psplotImp[algebraic](-0.5,-1)(1.5,1.5){%
        -0.25+1.0426*x*y+0.995*x^2+0.914*y^2}
\psplotImp[algebraic](-0.5,-1)(1.5,1.5){%
        -0.5+1.0426*x*y+0.995*x^2+0.914*y^2}
\psplotImp[algebraic](-0.5,-1)(1.5,1.5){%
        -1+1.0426*x*y+0.995*x^2+0.914*y^2}
\psplotImp[algebraic](-0.5,-1)(1.5,1.5){%
        -2+1.0426*x*y+0.995*x^2+0.914*y^2}
\psplotImp[algebraic](-0.5,-1)(1.5,1.5){%
        -4+1.0426*x*y+0.995*x^2+0.914*y^2}
\endpsgraph
} }
\hspace{0.25in} 
\subfloat[$\lambda_2=0.0596$]{
\scalebox{0.6}{
\psgraph[axesstyle=frame,Dx=0.5,Dy=0.5,Ox=-0.5,Oy=-1]{->}(-0.5,-1)(1.5,1.5){2in}{2.5in}%
\dataplot[plotstyle=dots,showpoints=true]{\dataI}
\psplotImp[algebraic](-0.5,-1)(1.5,1.5){%
        0.01+0.9077*x*y-0.6088*x^2-1.4077*y^2}
\psplotImp[algebraic](-0.5,-1)(1.5,1.5){%
        -0.1+0.9077*x*y-0.6088*x^2-1.4077*y^2}
\psplotImp[algebraic](-0.5,-1)(1.5,1.5){%
        0.1+0.9077*x*y-0.6088*x^2-1.4077*y^2}
\psplotImp[algebraic](-0.5,-1)(1.5,1.5){%
        -0.25+0.9077*x*y-0.6088*x^2-1.4077*y^2}
\psplotImp[algebraic](-0.5,-1)(1.5,1.5){%
        0.25+0.9077*x*y-0.6088*x^2-1.4077*y^2}
\psplotImp[algebraic](-0.5,-1)(1.5,1.5){%
        0.5+0.9077*x*y-0.6088*x^2-1.4077*y^2}
\psplotImp[algebraic](-0.5,-1)(1.5,1.5){%
        1+0.9077*x*y-0.6088*x^2-1.4077*y^2}
\psplotImp[algebraic](-0.5,-1)(1.5,1.5){%
        1.5+0.9077*x*y-0.6088*x^2-1.4077*y^2}
\psplotImp[algebraic](-0.5,-1)(1.5,1.5){%
        2+0.9077*x*y-0.6088*x^2-1.4077*y^2}
\endpsgraph
}} 
\hspace{0.25in} 
%\vspace{0.2in} \centerline{ \hspace{0.25in}
\subfloat[$\lambda_3=0.0184$]{
\scalebox{0.6}{
\psgraph[axesstyle=frame,Dx=0.5,Dy=0.5,Ox=-0.5,Oy=-1]{->}(-0.5,-1)(1.5,1.5){2in}{2.5in}%
\dataplot[plotstyle=dots,showpoints=true]{\dataI}
\psplotImp[algebraic](-0.5,-1)(1.5,1.5){%
        1-0.298*x*y+1.625*x^2-1.087*y^2}
\psplotImp[algebraic](-0.5,-1)(1.5,1.5){%
        0.75-0.298*x*y+1.625*x^2-1.087*y^2}
\psplotImp[algebraic](-0.5,-1)(1.5,1.5){%
        0.5-0.298*x*y+1.625*x^2-1.087*y^2}
\psplotImp[algebraic](-0.5,-1)(1.5,1.5){%
        0.3-0.298*x*y+1.625*x^2-1.087*y^2}
\psplotImp[algebraic](-0.5,-1)(1.5,1.5){%
        0.1-0.298*x*y+1.625*x^2-1.087*y^2}
\psplotImp[algebraic](-0.5,-1)(1.5,1.5){%
        -0.1-0.298*x*y+1.625*x^2-1.087*y^2}
\psplotImp[algebraic](-0.5,-1)(1.5,1.5){%
        -0.3-0.298*x*y+1.625*x^2-1.087*y^2}
\psplotImp[algebraic](-0.5,-1)(1.5,1.5){%
        -0.6-0.298*x*y+1.625*x^2-1.087*y^2}
\psplotImp[algebraic](-0.5,-1)(1.5,1.5){%
        -1-0.298*x*y+1.625*x^2-1.087*y^2}
\psplotImp[algebraic](-0.5,-1)(1.5,1.5){%
        -1.5-0.298*x*y+1.625*x^2-1.087*y^2}
\psplotImp[algebraic](-0.5,-1)(1.5,1.5){%
        -2-0.298*x*y+1.625*x^2-1.087*y^2}
%%\psplotImp[linewidth=2pt, algebraic](-0.5,-1)(1.5,1.5){%
%%0.298*\mye^(-1*x^2)+0.625*\mye^(-1*y^2)-0.5}
\endpsgraph
}}
}
\end{figure}
\end{frame}


\readdata{\dataKQH}{EDA/dimreduction/figs/iris-2d-kPC-quadraticH.dat}
\begin{frame}{Kernel PCA: Projected Points onto 2 PCs}
  \framesubtitle{Homogeneous Quadratic Kernel: $K(\bx_i, \bx_j) =
  (\bx_i^T\bx_j)^2$}
\hspace*{0.5in}
\scalebox{0.9}{
\centering
%\vspace{0.1in}
\psset{dotstyle=Bo,dotscale=1.5,fillcolor=lightgray,arrowscale=2,PointName=none}
\psset{xAxisLabel=$\bu_1$,yAxisLabel= $\bu_2$}
\psgraph[Dx=0.5,Dy=0.5,Ox=-0.5,Oy=-2]{->}(-0.5,-2)(4,0.5){4in}{2.5in}%
\dataplot[plotstyle=dots,showpoints=true]{\dataKQH}
\endpsgraph
}
\end{frame}


\ifdefined\wox \begin{frame} \titlepage \end{frame} \fi


\begin{frame}{Singular Value Decomposition}
  \small

  Principal components analysis is a special case of a more general
matrix decomposition method called {\em Singular Value
Decomposition (SVD)}.  
PCA yields the following decomposition of the covariance
matrix:
\begin{align*}
   \cov & = \bU \bLambda \bU^T
\end{align*}
where the covariance matrix has been factorized into
the orthogonal matrix $\bU$ containing its eigenvectors, and a
diagonal matrix $\bLambda$ containing its eigenvalues (sorted in
decreasing order).

\medskip
SVD generalizes the above factorization for any matrix. In particular
for an $n \times d$ data matrix $\bD$ with $n$ points and $d$ columns, SVD factorizes $\bD$ as follows:
\begin{align*}
  \bD &= \bL \bDelta \bR^T
\end{align*}

where $\bL$ is a orthogonal $n \times n$ matrix,
$\bR$ is an
orthogonal $d \times d$ matrix, and $\bDelta$ is an 
$n \times d$ ``diagonal'' matrix, 
defined as $\bDelta(i,i) = \delta_i$, and $0$
otherwise.
The columns of $\bL$ are called the {\em left singular
  and the columns of $\bR$ (or rows of $\bR^T$) are called the
\em right singular vectors}.
The entries $\delta_{i}$ are called the 
{\em singular values} of $\bD$, and they are all non-negative.
\end{frame}

\begin{frame}{Reduced SVD}
  \small
If the rank of $\bD$ is $r \le \min(n,d)$, then
there are only $r$ nonzero
singular values, ordered as follows:
$\delta_1 \ge
\delta_2 \ge \cdots \ge \delta_r > 0$.

\medskip
We
discard the left and right singular vectors that correspond to zero
singular values, to obtain the {\em reduced SVD} as
\begin{align*}
\bD = \bL_r \bDelta_r \bR_r^T
\end{align*}
where $\bL_r$ is the $n \times r$ matrix of the left
singular vectors,
$\bR_r$ is the $d \times r$ matrix of the right
singular vectors, and $\bDelta_r$ is the $r \times r$ diagonal
matrix containing the positive singular vectors.

\medskip
The reduced SVD leads directly to
the {\em spectral decomposition} of
$\bD$ given as
\begin{align*}
\bD   = & \sum_{i=1}^r \delta_i \bl_i \br^T_i
\end{align*}

The best rank $q$ approximation to the
original data $\bD$ is the matrix
$\bD_q = \sum_{i=1}^q \delta_i \bl_i \br^T_i$
that
minimizes the expression $\|\bD - \bD_q\|_F$,
where $\|\bA\|_F = \sqrt{\sum_{i=1}^n \sum_{j=1}^d \bA(i,j)^2}$
is called the {\em Frobenius Norm} of $\bA$.
\end{frame}



\begin{frame}{Connection Between SVD and PCA}
  \small 
Assume $\bD$ has been centered, and let $\bD = \bL \bDelta \bR^T$ via
SVD. 
Consider the {\em scatter matrix} for $\bD$, given as $\bD^T\bD$.
We have
\begin{align*}
    \bD^T\bD  = \lB(\bL \bDelta \bR^T\rB)^T
         \lB(\bL \bDelta \bR^T \rB)
         = \bR \bDelta^T \bL^T \bL \bDelta \bR^T
         = \bR (\bDelta^T \bDelta) \bR^T
         = \bR \bDelta^2_d \bR^T
\end{align*}
where $\bDelta^2_d$ is
the $d \times d$ diagonal matrix def\/{i}ned as
$\bDelta^2_d(i,i) = \delta_i^2$, for
$i=1,\ldots,d$.

\medskip
The covariance matrix of centered $\bD$ is given as
$\cov = {1\over n} \bD^T\bD$; we get
\begin{align*}
    \bD^T\bD & = n \cov\nonumber\\
        & = n \bU \bLambda \bU^T\nonumber\\
        & = \bU (n \bLambda) \bU^T
\end{align*}

\medskip
The right singular vectors $\bR$ are the same as
the eigenvectors of $\cov$.
The singular values of $\bD$ are
related to the eigenvalues of $\cov$ as
\begin{align*}
n \lambda_i = \delta_i^2\nonumber
\text{, which implies } \lambda_i = {\delta_i^2 \over n},
\text{ for } i=1, \ldots, d
\end{align*}

\medskip
Likewise the left
singular vectors in $\bL$ are the eigenvectors of the matrix $n
\times n$ matrix $\bD\bD^T$, and the corresponding eigenvalues
are given as $\delta^2_i$.
\end{frame}

