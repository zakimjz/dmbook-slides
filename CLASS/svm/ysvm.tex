\lecture{svm}{svm}

\date{Chapter 21: Support Vector Machines}

\begin{frame}
\titlepage
\end{frame}



\begin{frame}{Hyperplanes}

Let $\bD = \{ (\bx_i, y_i) \}_{i=1}^n$ be a classif\/{i}cation
dataset, with $n$ points in a $d$-dimensional space. We 
assume that there are only two class labels, that is, $y_i \in
\{+1, -1\}$, denoting the positive and negative classes.

\medskip
A hyperplane in $d$ dimensions is given as the set of all points $\bx
\in \setR^d$ that satisfy the equation
  $h(\bx) = 0$,
  where $h(\bx)$ is the {\em hyperplane function}:
\begin{align*}
h(\bx) & =  \bw^{T}\bx+b 
=  w_1x_1 + w_2x_2 + \cdots + w_dx_d + b
\end{align*}
Here, $\bw$ is a $d$ dimensional {\em weight vector} and $b$ is a
scalar, called the {\em bias}.

\medskip
For points that lie on the hyperplane, we have
\begin{align*}
  h(\bx) = \bw^T\bx+b = 0
\end{align*}

\medskip
The weight
vector $\bw$ specif\/{i}es the direction that is orthogonal or 
normal to the
hyperplane, which f\/{i}xes the orientation of the hyperplane, whereas
the bias $b$ f\/{i}xes the offset of the hyperplane in the $d$-dimensional
space, i.e., where
the hyperplane intersects each of the axes:
\begin{align*}
  w_ix_i = -b \;\;\;\mbox{ or }\;\;\; x_i = \frac{-b}{w_i}
\end{align*}
\end{frame}



\begin{frame}{Separating Hyperplane}
A hyperplane splits the $d$-dimensional data space
into two {\em half-spaces}. 

\medskip
A dataset is said to be {\em linearly
separable} if each half-space has points only from a single class.

\medskip
If the input
dataset is linearly separable, then we can f\/{i}nd a {\em separating}
hyperplane $h(\bx)=0$, such that for all points labeled $y_i=-1$, we have
$h(\bx_i) < 0$, and for all points labeled $y_i = +1$, we have
$h(\bx_i) > 0$. 

\medskip
The hyperplane function $h(\bx)$ thus serves as a 
linear classif\/{i}er or a linear discriminant,
which predicts the class $y$ for any given point $\bx$,  according to
the decision rule:
\begin{align*}
  y =
  \begin{cases}
    +1 & \mbox{if } h(\bx) > 0\\
    -1 & \mbox{if } h(\bx) < 0
  \end{cases}
\end{align*}
\end{frame}


\begin{frame}{Geometry of a Hyperplane: Distance}
\small
  Consider a point $\bx \in \setR^d$ that does not lie
on the hyperplane. Let $\bx_p$ be the orthogonal projection of $\bx$ on
the hyperplane, and let $\br = \bx - \bx_p$.
Then we can write $\bx$ as
\begin{align*}
    \bx  & = \bx_p + \br = 
	\bx_p + r {\bw \over \|\bw\|}
\end{align*}
where $r$ is the {\em directed distance} of the point $\bx$ from
$\bx_p$.  

To obtain an expression for $r$, consider the value $h(\bx)$, we have:
\begin{align*}
  h(\bx) & = h\lB(\bx_p + r {\bw \over \|\bw\|}\rB) = 
  \bw^T\lB(\bx_p + r {\bw \over \|\bw\|}\rB) + b = 
  r \| \bw \|
\end{align*}

The directed distance $r$ of point $\bx$ to the
hyperplane is thus:
\begin{align*}
  r = {h(\bx) \over \| \bw \|}
\end{align*}

To obtain distance, which must be non-negative, we 
multiply $r$ by the class label $y_i$ of the point $\bx_i$
because when $h(\bx_i) < 0$, the class is $-1$, and when $h(\bx_i)
> 0$ the class is $+1$:
\begin{align*}
  \dist_i = {y_i h(\bx_i) \over \| \bw \|}
\end{align*}


\end{frame}





\begin{frame}[fragile]{Geometry of a Hyperplane in 2D}
   %h(x) = 2y + 5x - 20 = 0
\begin{columns}
\vspace*{-4cm}
\column{0.5\textwidth}
  \centerline{
  \scalebox{0.6}{
    \psset{unit=0.5in,dotscale=2,arrowscale=2}
    \pspicture[](-1,-1)(6,6)
    \definecolor{mygray}{gray}{0.75}%
    \pspolygon[linecolor=mygray,fillcolor=mygray,fillstyle=solid](-1,-1)(4.4,-1)(1.6,6)(-1,6) %
    \psaxes[]{<->}(0,0)(-1,-1)(6,6)
    \psgrid[subgriddiv=1,gridcolor=gray,griddots=10,gridlabels=0](0,0)(-1,-1)(6,6)
    \psdots[dotstyle=o](4,3)(3.5,4.25)(5.5,3.5)(4,4)(4.5,1.75)(5,4)(5.5,2.5)(4.9,4.5)
    \psdots[dotstyle=triangle](2,2)(2.5,0.75)(0.5,1.5)(1.5,1.5)(1,2.5)(1.25,0.5)
    \pcline[linewidth=3pt]{-}(1.6,6)(4.4,-1)
    \naput[nrot=:U,npos=0.1,labelsep=2pt]{$h(\bx) = 0$}
    \psdot[](5,4)\uput[45](5,4){$\bx$}
    \psdot[](2.76,3.1)\uput[225](2.76,3.1){$\bx_p$}
    \uput[225](0,0){$\bzero$}
    \pcline[linewidth=3pt,arrowscale=1,linecolor=gray]{->}(2.25,4.375)(3.18,4.75)
    %\psline[linestyle=dashed]{-}(5,4)(2.76,3.1)
    \pcline[linewidth=2pt,arrowscale=1.5]{->}(2.76,3.1)(5,4)
    \nbput[nrot=:U, labelsep=2.5pt]{$\br = r \; {\bw \over \| \bw \|}$}
    \psline[linewidth=2pt,linestyle=dashed]{-}(0,0)(3.45,1.34)
    %\uput[0]{-65}(1.7,5.6){$h(\bx) = 0$}
    \uput[0](0,5.8){$h(\bx) < 0$}
    \uput[0](2.5, 5.8){$h(\bx) > 0$}
    %\uput[90](1.75,0.75){${b \over \|\bw\|}$}
    \uput[30](3,4.65){$\bw \over \| \bw \|$}
    \psdots[dotstyle=o](4,2.9)
    %\psbrace[rot=270,ref=b,linewidth=0.5pt](3.45,1.34)(0,0)
    \uput[30](1.25,0.85){${b \over \|\bw\|}$}
    \endpspicture
  }
  }
%\vspace*{1cm}
	\mbox{ }
\column{0.5\textwidth}
\begin{small}
\noindent
$\bp = (p_1, p_2) = (4,0)$, $\bq = (q_1, q_2) = (2,5)$ 
 $$-\frac{w_1}{w_2}  = {q_2 - p_2 \over q_1 - p_1} = {5-0 \over
 2-4} = -{5 \over 2}$$
Given $(4,0)$, the offset $b$ is:
\begin{align*}
  b  = -5 x_1 -2 x_2 = -5 \cdot 4 - 2 \cdot 0 = -20
\end{align*}
Given $\bw =  \matr{5\\2}$  and $b=-20$:
\begin{align*}
h(\bx) =  \bw^T \bx + b = \matr{5 & 2} \matr{x_1\\x_2} - 20 = 0
\end{align*}
\begin{align*}
\dist = y \; r = -1\; r = {-b \over \| \bw \|}
= {-(-20) \over \sqrt{29}} = 3.71
\end{align*}
\end{small}
\end{columns}
\end{frame}


\begin{frame}{Margin and Support Vectors}

The distance of a point $\bx$ from the
hyperplane $h(\bx)=0$ is thus given as
\begin{align*}
\tcbhighmath{
  \dist = y\; r = {y\; h(\bx) \over \| \bw \|}
}
\end{align*}

The {\em margin} is the minimum distance of a point from the 
separating
hyperplane:
\begin{align*}
\tcbhighmath{
  \dist^* = \min_{\bx_i} \; \lB\{ {y_i(\bw^T\bx_i + b) \over \| \bw \|} \rB\}
}
\end{align*}

All the points (or vectors) that achieve the minimum
distance are called {\em support vectors}
for the hyperplane. They satisfy the condition:
\begin{align*}
\dist^* & = {y^*(\bw^T\bx^* + b) \over \| \bw \|}
\end{align*}
where $y^*$ is the class label for $\bx^*$.
\end{frame}


\begin{frame}{Canonical Hyperplane}
Multiplying the hyperplane equation on both sides by some scalar $s$ yields an equivalent
hyperplane:
\begin{align*}
    s \; h(\bx) = s\; \bw^T\bx + s\; b = (s\bw)^T\bx + (sb) = 0
\end{align*}

\medskip
To obtain the unique or {\em canonical} hyperplane,
we choose the scalar $s={1 \over y^*(\bw^T\bx^* + b)}$ so
that the absolute distance of a support vector from the
hyperplane is $1$, i.e., the margin is  
\begin{align*}
  \dist^* = {y^* (\bw^T\bx^* + b) \over \norm{\bw}} = {1 \over \| \bw \|}
\end{align*}

\medskip
For the canonical hyperplane, for each support vector $\bx^*_i$
(with label $y_i^*$), we have $y^*_i h(\bx^*_i) = 1$, and for any
point that is not a support vector we have \hbox{$y_i h(\bx_i) > 1$}.
Over all points, we have
\begin{align*}
\tcbhighmath{
  y_i \; (\bw^T\bx_i + b) \ge 1, \mbox{ for all points } \bx_i \in \bD
}
\end{align*}
\end{frame}


\begin{frame}{Separating Hyperplane: Margin and Support Vectors}
  \framesubtitle{Shaded points are support vectors}
\begin{columns}
\column{0.5\textwidth}
\hspace*{1cm}
\scalebox{0.6}{
  \centerline{
    \psset{unit=0.5in,dotscale=2,arrowscale=2}
    \pspicture[](-1,-1)(6,6)
    \psaxes[]{<->}(0,0)(-1,-1)(6,6)
    \psgrid[subgriddiv=1,gridcolor=gray,griddots=10,gridlabels=0](0,0)(-1,-1)(6,6)
    \pcline[linewidth=3pt]{-}(1.6,6)(4.4,-1)
    \naput*[nrot=:U,npos=0.1,labelsep=2pt]{$h(\bx) = 0$}
    \psline[linewidth=1.5pt,linestyle=dashed]{-}(2.8,6)(5.6,-1)
    \psline[linewidth=1.5pt,linestyle=dashed]{-}(0.4,6)(3.2,-1)
    \psdots[dotstyle=Bo,fillcolor=gray](4.5,1.75)(4,3)(3.5,4.25)
    \psdots[dotstyle=Btriangle,fillcolor=gray](2,2)(2.5,0.75)
    \psline[]{<->}(2,2)(3.035,2.414)
    \psline[]{<->}(2.965,2.586)(4,3)
    %\psline[]{<->}(1.25,6)(4.75,-1)
    %
    \psdots[dotstyle=o](5.5,3.5)(4,4)(5,4)(5.5,2.5)(4.9,4.5)
    \psdots[dotstyle=triangle](0.5,1.5)(1.5,1.5)(1,2.5)(1.25,0.5)
    %\uput[0]{-65}(1.7,5.5){$h(\bx) = 0$}
    \uput[90](2.4,2.2){$1 \over \| \bw \|$}
    \uput[90](3.3,2.75){$1 \over \| \bw \|$}
    \endpspicture
  }
}
\column{0.5\textwidth}
\begin{small}
\begin{align*}
    h(\bx) = \matr{5\\2}^T \bx -20 = 0
\end{align*}
Given $\bx^* = (2,2)^T$, $y^*=-1$.
$$s = {1 \over y^*h(\bx^*)} =
{1 \over -1 \lB(\matr{5 \\ 2}^T \matr{2 \\ 2} - 20\rB)} = {1
\over 6}$$
$$\bw = {1\over 6} \matr{5\\ 2} = \matr{ {5/6} \\ {2/6}} \hspace*{1cm} b = {-20 \over 6}$$
\begin{align*}
h(\bx) = \matr{5/6\\ 2/6}^T \bx - 20/6 =
\matr{0.833\\0.333}^T\bx - 3.33
\end{align*}
\begin{align*}
  \dist^* = {y^* \;h(\bx^*) \over \| \bw \|} =
  {1 \over
  \sqrt{ \lB({5\over6}\rB)^2+\lB({2\over6}\rB)^2}}  = {6 \over \sqrt{29}}
  = 1.114
\end{align*}
\end{small}
\end{columns}
\end{frame}



\begin{frame}{SVM: Linear and Separable Case}

Assume that the points are linearly separable, that is, there exists a separating hyperplane that perfectly classif\/{i}es each point. 

\medskip
The goal of SVMs is to choose the canonical
hyperplane, $h^*$,
that yields the maximum margin among all possible separating
hyperplanes
\begin{align*}
    h^* = \arg \max_{\bw,b} \lB\{ {1 \over \| \bw \|} \rB\}
\end{align*}
We can obtain an equivalent minimization
formulation:
\begin{empheq}[box=\tcbhighmath]{align*}
\begin{split}
    \mbox{\bf Objective Function: } &  \min_{\bw,b}
    \lB\{ \frac{\| \bw \|^2}{2} \rB\} \\
  \mbox{\bf Linear Constraints: }& y_i \; (\bw^T\bx_i + b) \ge 1,\;\;
  \forall \bx_i \in \bD
\end{split}
\end{empheq}
\end{frame}


\begin{frame}{SVM: Linear and Separable Case}
  \small
We turn the constrained SVM optimization into an unconstrained one by
introducing a
Lagrange multiplier $\alpha_i$ for each constraint.
The new objective function, called the {\em Lagrangian}, then becomes
\begin{align*}
    \min \; L = {1 \over 2}\|\bw\|^2 - \dsum_{i=1}^n \alpha_i \lB(y_i
  (\bw^T\bx_i + b) - 1\rB)
\end{align*}
$L$ should be minimized w.r.t. $\bw$ and $b$,
and it should be maximized w.r.t. $\alpha_i$.

Taking the derivative of $L$ with respect to
$\bw$ and $b$, and setting those to zero, we
obtain
\begin{align*}
\frac{\partial}{\partial \bw}L & =
\bw-\dsum_{i=1}^n \alpha_{i}y_{i}\bx_{i} = \bzero \;\;\;\mbox { or }\;\;\;
\bw = \dsum_{i=1}^n \alpha_{i}y_{i}\bx_{i}\\
\frac{\partial}{\partial b} L & = \dsum_{i=1}^n \alpha_{i}y_{i}= 0
\end{align*}

\medskip
We can see that $\bw$ can be expressed as a linear combination of the
data points $\bx_i$, with the signed Lagrange multipliers,
$\alpha_iy_i$, serving as the coeff\/{i}cients. 

\medskip
Further, the sum of the signed
Lagrange multipliers, $\alpha_iy_i$, must be zero.
\end{frame}


\begin{frame}{SVM: Linear and Separable Case}
Incorporating $\bw = \dsum_{i=1}^n \alpha_{i}y_{i}\bx_{i}$ and 
$\dsum_{i=1}^n \alpha_{i}y_{i}= 0$ into the Lagrangian we obtain the new 
{\em dual
  Lagrangian} objective function, 
  which is specified purely in terms of the Lagrange multipliers:
\begin{empheq}[box=\tcbhighmath]{align*}
\begin{split}
    \mbox{\bf Objective Function: } &  \max_{\balpha}\;\;
     L_{dual} = \dsum_{i=1}^n\alpha_{i} - {1 \over 2} \dsum_{i=1}^n
  \dsum_{j=1}^n \alpha_i\alpha_{j} y_i y_{j} \bx_i^T\bx_{j}\\
  \mbox{\bf Linear Constraints: }& \alpha_i \ge 0, \;
  \forall i \in \bD, \text{ and } \sum_{i=1}^n \alpha_i y_i = 0
\end{split}
\end{empheq}
where $\balpha = (\alpha_1, \alpha_2,\ldots, \alpha_n)^T$ is the vector comprising the Lagrange multipliers. 

\medskip
$L_{dual}$ is a convex quadratic programming problem (note the
$\alpha_i\alpha_{j}$ terms), which admits a unique optimal solution.
\end{frame}



\begin{frame}{SVM: Linear and Separable Case}
Once we have obtained the $\alpha_i$ values for $i=1,\ldots, n$,
we can solve for the weight vector $\bw$ and the bias $b$. Each of the
Lagrange multipliers $\alpha_i$ satisfies the KKT conditions at the
optimal solution:
\begin{align*}
  \alpha_i \lB(y_i (\bw^T\bx_i + b) - 1\rB) = 0
\end{align*}
which gives rise to two cases:
\begin{enumerate}[(1)]
\item $\alpha_i = 0$, or
\item $y_i (\bw^T\bx_i + b) - 1 = 0$, which implies
  $y_i (\bw^T\bx_i + b) = 1$
\end{enumerate}

\medskip
This is a very important result because if $\alpha_i
> 0$, then $y_i (\bw^T\bx_i + b) = 1$, and thus the point $\bx_i$
must be a support vector. 

\medskip
On the other hand, if $y_i (\bw^T\bx_i +
b) > 1$, then $\alpha_i = 0$, that is, if a point is not a support
vector, then $\alpha_i = 0$.
\end{frame}


\begin{frame}{Linear and Separable Case: Weight Vector and Bias}
Once we know $\alpha_i$ for all points, we can compute the weight
vector $\bw$ by taking the
summation only for the support vectors:
\begin{align*}
\tcbhighmath{
    \bw = \dsum_{\mathclap{i, \alpha_i > 0}} \alpha_i y_i \bx_i
}
\end{align*}
Only the support vectors determine $\bw$, since 
$\alpha_i=0$ for other points.

To compute the bias $b$, we f\/{i}rst compute one solution $b_i$, per
support vector, as follows:
\begin{align*}
  y_i (\bw^T\bx_i + b) = 1, \text{ which implies }
  b_i = {1 \over y_i} - \bw^T\bx_i = y_i - \bw^T\bx_i
\end{align*}
The bias $b$ is taken as the average value:
\begin{align*}
\tcbhighmath{
   b = \displaystyle \mbox{avg}_{\alpha_i > 0} \{ b_i\}
}
\end{align*}
\end{frame}



\begin{frame}{SVM Classif\/{i}er}
Given the optimal hyperplane function $h(\bx) = \bw^T\bx + b$,
for any new point $\bz$, we predict its class as
\begin{align*}
\tcbhighmath{
  \hat{y}  =  \sign(h(\bz)) = \sign(\bw^T\bz + b)
}
\end{align*}
where the $\sign(\cdot)$ function returns $+1$ if its argument is positive, and
$-1$ if its argument is negative.
\end{frame}



\begin{frame}{Example Dataset: Separable Case}
\renewcommand{\arraystretch}{1.1}
\begin{columns}
\column{0.5\textwidth}
\hspace*{1cm}
\scalebox{0.6}{
  \centerline{
    \psset{unit=0.5in,dotscale=2,arrowscale=2}
    \pspicture[](-1,-1)(6,6)
    \psaxes[]{<->}(0,0)(-1,-1)(6,6)
    \psgrid[subgriddiv=1,gridcolor=gray,griddots=10,gridlabels=0](0,0)(-1,-1)(6,6)
    %\pcline[linewidth=3pt]{-}(1.6,6)(4.4,-1)
    \naput*[nrot=:U,npos=0.1,labelsep=2pt]{$h(\bx) = 0$}
    %\psline[linewidth=1.5pt,linestyle=dashed]{-}(2.8,6)(5.6,-1)
    %\psline[linewidth=1.5pt,linestyle=dashed]{-}(0.4,6)(3.2,-1)
    \psdots[dotstyle=o](4.5,1.75)(4,3)(3.5,4.25)
    \psdots[dotstyle=triangle](2,2)(2.5,0.75)
    %\psline[]{<->}(2,2)(3.035,2.414)
    %\psline[]{<->}(2.965,2.586)(4,3)
    %\psline[]{<->}(1.25,6)(4.75,-1)
    %
    \psdots[dotstyle=o](5.5,3.5)(4,4)(5,4)(5.5,2.5)(4.9,4.5)
    \psdots[dotstyle=triangle](0.5,1.5)(1.5,1.5)(1,2.5)(1.25,0.5)
    %\uput[0]{-65}(1.7,5.5){$h(\bx) = 0$}
    %\uput[90](2.4,2.2){$1 \over \| \bw \|$}
    %\uput[90](3.3,2.75){$1 \over \| \bw \|$}
    \endpspicture
  }
}
\column{0.5\textwidth}
\begin{center}
\begin{tabular}{|c|c|c|c|}
        \hline
        $\bx_i$ & $x_{i1}$ & $x_{i2}$ & $y_i$\\
        \hline
        $\bx_1$ & $3.5$ & $4.25$ & $+1$\\
        $\bx_2$ & $4$ & $3$ & $+1$\\
        $\bx_3$ & $4$ & $4$ & $+1$\\
        $\bx_4$ & $4.5$ & $1.75$ & $+1$\\
        $\bx_5$ & $4.9$ & $4.5$ & $+1$\\
        $\bx_6$ & $5$ & $4$ & $+1$\\
        $\bx_7$ & $5.5$ & $2.5$ & $+1$\\
        $\bx_8$ & $5.5$ & $3.5$ & $+1$\\
        $\bx_9$ & $0.5$ & $1.5$ & $-1$\\
        $\bx_{10}$ & $1$ & $2.5$ & $-1$\\
        $\bx_{11}$ & $1.25$ & $0.5$ & $-1$\\
        $\bx_{12}$ & $1.5$ & $1.5$ & $-1$\\
        $\bx_{13}$ & $2$ & $2$ & $-1$\\
        $\bx_{14}$ & $2.5$ & $0.75$ & $-1$\\
        \hline
    \end{tabular}%}{}
\end{center}
\end{columns}
\end{frame}

\begin{frame}{Optimal Separating Hyperplane}
  \begin{columns}
	\column{0.4\textwidth}
\begin{figure}[!t]
\scalebox{0.5}{
  \centerline{
    \psset{unit=0.5in,dotscale=2,arrowscale=2}
    \pspicture[](-1,-1)(6,6)
    \psaxes[]{<->}(0,0)(-1,-1)(6,6)
    \psgrid[subgriddiv=1,gridcolor=gray,griddots=10,gridlabels=0](0,0)(-1,-1)(6,6)
    \pcline[linewidth=3pt]{-}(1.6,6)(4.4,-1)
    \naput*[nrot=:U,npos=0.1,labelsep=2pt]{$h(\bx) = 0$}
    \psline[linewidth=1.5pt,linestyle=dashed]{-}(2.8,6)(5.6,-1)
    \psline[linewidth=1.5pt,linestyle=dashed]{-}(0.4,6)(3.2,-1)
    \psdots[dotstyle=Bo,fillcolor=gray](4.5,1.75)(4,3)(3.5,4.25)
    \psdots[dotstyle=Btriangle,fillcolor=gray](2,2)(2.5,0.75)
    \psline[]{<->}(2,2)(3.035,2.414)
    \psline[]{<->}(2.965,2.586)(4,3)
    %\psline[]{<->}(1.25,6)(4.75,-1)
    %
    \psdots[dotstyle=o](5.5,3.5)(4,4)(5,4)(5.5,2.5)(4.9,4.5)
    \psdots[dotstyle=triangle](0.5,1.5)(1.5,1.5)(1,2.5)(1.25,0.5)
    %\uput[0]{-65}(1.7,5.5){$h(\bx) = 0$}
    \uput[90](2.4,2.2){$1 \over \| \bw \|$}
    \uput[90](3.3,2.75){$1 \over \| \bw \|$}
    \endpspicture
  }
}
\end{figure}

\column{0.6\textwidth}
\small
    Solving the $L_{dual}$ quadratic program yields 
	\begin{center}
\scriptsize
	  {\tabcolsep12pt\renewcommand{\arraystretch}{1.1}%
	\begin{tabular}{|c|c|c|c|c|}
        \hline
        $\bx_i$ & $x_{i1}$ & $x_{i2}$ & $y_i$ & $\alpha_i$\\
        \hline
        $\bx_1$ & $3.5$ & $4.25$ & $+1$ & $0.0437$\\
        $\bx_2$ & $4$ & $3$ & $+1$ & $0.2162$\\
        $\bx_4$ & $4.5$ & $1.75$ & $+1$ & $0.1427$\\
        $\bx_{13}$ & $2$ & $2$ & $-1$ & $0.3589$\\
        $\bx_{14}$ & $2.5$ & $0.75$ & $-1$ & $0.0437$\\
        \hline
    \end{tabular}}
    \end{center}
The weight vector and bias are:
\begin{align*}
    \bw & =
    \dsum_{\mathclap{i, \alpha_i > 0}} \alpha_i y_i \bx_i
	= \matr{0.833\\0.334}\\
b & = \mbox{avg} \{b_i\} = -3.332
\end{align*}
The optimal hyperplane is given as follows:
\begin{align*}
    h(\bx) = \matr{0.833\\0.334}^T \bx - 3.332 = 0
\end{align*}
\end{columns}
\end{frame}

\ifdefined\wox \begin{frame} \titlepage \end{frame} \fi

\begin{frame}{Soft Margin SVM: Linear and Nonseparable Case}
The assumption that the dataset be perfectly linearly
separable is unrealistic. 
SVMs can handle non-separable
points by introducing {\em slack variables} $\xi_i$ 
as follows: 
\begin{align*}
\tcbhighmath{
y_{i} (\bw^{T}\bx_{i}+b) \geq 1-\xi_{i}
}
\end{align*}
where $\xi_i \ge 0$ is the slack variable for point $\bx_i$, which
indicates how much the point violates the separability condition,
that is, the point may no longer be at least $1/\norm{\bw}$ away
from the hyperplane. 

\medskip
The slack values indicate three types of
points. If $\xi_i=0$, then the corresponding point $\bx_i$ is at least
$1 \over \|\bw\|$ away from the hyperplane. 

\medskip
If $0 < \xi_i
< 1$, then the point is within the margin and still correctly
classif\/{i}ed, that is, it is on the correct side of the hyperplane.

\medskip
However, if $\xi_i \ge 1$ then the point is misclassif\/{i}ed and
appears on the wrong side of the hyperplane.
\end{frame}





\begin{frame}{Soft Margin Hyperplane}
  \framesubtitle{Shaded points are the support vectors}
\begin{figure}[!b]
  %\vspace{0.1in}
  %h(x) = 2y + 5x - 10 = 0
\scalebox{0.8}{
  \centerline{
    \psset{unit=0.5in,dotscale=2,arrowscale=2}
    \pspicture[](-1,-1)(6,6)
    \psaxes[]{<->}(0,0)(-1,-1)(6,6)
    \psgrid[subgriddiv=1,gridcolor=gray,griddots=10,gridlabels=0](0,0)(-1,-1)(6,6)
    \pcline[linewidth=3pt]{-}(1.6,6)(4.4,-1)
    \naput*[nrot=:U,npos=0.1,labelsep=1pt]{$h(\bx) = 0$}
    \psline[linewidth=1.5pt,linestyle=dashed]{-}(2.8,6)(5.6,-1)
    \psline[linewidth=1.5pt,linestyle=dashed]{-}(0.4,6)(3.2,-1)
    \psdots[dotstyle=Bo,fillcolor=gray](4,3)(3.5,4.25)(4.5,1.75)(4,2)(2,3)
    \psdots[dotstyle=Btriangle,fillcolor=gray](2,2)(2.5,0.75)(3,2)(5,3)
    %slacks
    \psset{PointSymbol=none,PointName=none}
    \pstGeonode[](4,3){na}
    \pstGeonode[](4.5,1.75){nb}
    \pstGeonode[](2,3){s1}
    \pstProjection[]{na}{nb}{s1}[sp1]
    \pstLineAB[]{s1}{sp1}
    \pstGeonode[](4,2){s2}
    \pstProjection[]{na}{nb}{s2}[sp2]
    \pstLineAB[]{s2}{sp2}
    \pstGeonode[](2,2){ma}
    \pstGeonode[](2.5,0.75){mb}
    \pstGeonode[](3,2){s3}
    \pstProjection[]{ma}{mb}{s3}[sp3]
    \pstLineAB[]{s3}{sp3}
    \pstGeonode[](5,3){s4}
    \pstProjection[]{ma}{mb}{s4}[sp4]
    \pstLineAB[]{s4}{sp4}
    %
    \psdots[dotstyle=o](5.5,3.5)(4,4)(5,4)(5.5,2.5)(4.9,4.5)
    \psdots[dotstyle=triangle](0.5,1.5)(1.5,1.5)(1,2.5)(1.25,0.5)
    %margin
    \pstGeonode[](2.4,4){p1}
    \vspace{0.2in}
    \pstProjection[]{na}{nb}{p1}[pp1]
    \pstLineAB[arrows=<->]{p1}{pp1}
    \pstProjection[]{ma}{mb}{p1}[pp2]
    \pstLineAB[arrows=<->]{p1}{pp2}
    %\psline[]{<->}(2,2.1)(3,2.5)
    %\psline[]{<->}(3,2.5)(4,2.9)
    \uput[90](1.7,3.8){${1 \over \|\bw\|}$}
    \uput[90](2.75,4.2){${1 \over \|\bw\|}$}
    \endpspicture
  }
}
\end{figure}
\end{frame}


\begin{frame}{SVM: Soft Margin or Linearly Non-separable Case}
In the nonseparable case, also called the {\em soft margin}
the SVM objective function is
\begin{empheq}[box=\tcbhighmath]{align*}
\begin{split}
  \mbox{\bf Objective Function: } & \min_{\bw,b,\xi_i} \lB\{ {\| \bw \|^2 \over 2} + C
  \dsum_{i=1}^n (\xi_i)^k \rB \}\\
  \mbox{\bf Linear Constraints: } & y_i \; (\bw^T\bx_i + b) \ge 1 - \xi_i,\;\;
  \forall \bx_i \in \bD\\
  & \xi_i \ge 0 \; \; \forall \bx_i \in \bD
\end{split}
\end{empheq}
where $C$
and $k$ are constants that incorporate the cost of
misclassif\/{i}cation.  

\medskip
The term $\sum_{i=1}^n (\xi_i)^k$ gives the
{\em loss}, that is, an estimate of the deviation from the
separable case. 

\medskip
The scalar $C$ is a {\em
regularization constant} that controls the trade-off between
maximizing the margin  or minimizing the loss.
For example, if $C \to 0$, then the loss component essentially
disappears, and the objective defaults to maximizing the margin.
On the other hand, if $C \to \infty$, then the margin ceases to
have much effect, and the objective function tries to minimize the
loss. 
\end{frame}


\begin{frame}{SVM: Soft Margin Loss Function}

The constant $k$ governs the form of the loss. When $k=1$, called {\em hinge loss}, the goal is
to minimize the sum of the slack variables, whereas when $k=2$,
called {\em quadratic loss}, the goal is to minimize the sum of
the squared slack variables.

\medskip{\bf Hinge Loss:}
Assuming $k=1$, the SVM dual Lagrangian is given as
\begin{align*}
\tcbhighmath{
     \max_{\balpha}\;\;
     L_{dual} = \dsum_{i=1}^n\alpha_{i} - {1 \over 2} \dsum_{i=1}^n
  \dsum_{j=1}^n \alpha_i\alpha_{j} y_i y_{j} \bx_i^T\bx_{j}
}
\end{align*}
The only difference from the separable case is that
$0 \le \alpha_i \le C$.

\medskip {\bf Quadratic Loss:} Assuming $k=2$,
the dual objective is:
\begin{align*}
\tcbhighmath{
    \max_{\balpha}\;\;
     L_{dual} = \dsum_{i=1}^n\alpha_{i} - {1 \over 2} \dsum_{i=1}^n
     \dsum_{j=1}^n \alpha_i\alpha_{j} y_i y_{j} \lB(\bx_i^T\bx_{j} +
     {1\over 2C}\delta_{ij} \rB)
}
\end{align*}
where $\delta$ is the {\em Kronecker delta} function, def\/{i}ned as
$\delta_{ij} = 1$ if and only if $i=j$.

\end{frame}


\begin{frame}{Example Dataset: Linearly Non-separable Case}
\begin{columns}
\column{0.5\textwidth}
\hspace*{1.5cm}
\scalebox{0.6}{
  \centerline{
    \psset{unit=0.5in,dotscale=2,arrowscale=2}
    \pspicture[](-1,-1)(6,6)
    \psaxes[]{<->}(0,0)(-1,-1)(6,6)
    \psgrid[subgriddiv=1,gridcolor=gray,griddots=10,gridlabels=0](0,0)(-1,-1)(6,6)
    %\pcline[linewidth=3pt]{-}(1.6,6)(4.4,-1)
    %\naput*[nrot=:U,npos=0.1,labelsep=1pt]{$h(\bx) = 0$}
    %\psline[linewidth=1.5pt,linestyle=dashed]{-}(2.8,6)(5.6,-1)
    %\psline[linewidth=1.5pt,linestyle=dashed]{-}(0.4,6)(3.2,-1)
    \psdots[dotstyle=o](4,3)(3.5,4.25)(4.5,1.75)(4,2)(2,3)
    \psdots[dotstyle=triangle](2,2)(2.5,0.75)(3,2)(5,3)
    %slacks
    \psset{PointSymbol=none,PointName=none}
    %\pstGeonode[](4,3){na}
    %\pstGeonode[](4.5,1.75){nb}
    %\pstGeonode[](2,3){s1}
    %\pstProjection[]{na}{nb}{s1}[sp1]
    %\pstLineAB[]{s1}{sp1}
    %\pstGeonode[](4,2){s2}
    %\pstProjection[]{na}{nb}{s2}[sp2]
    %\pstLineAB[]{s2}{sp2}
    %\pstGeonode[](2,2){ma}
    %\pstGeonode[](2.5,0.75){mb}
    %\pstGeonode[](3,2){s3}
    %\pstProjection[]{ma}{mb}{s3}[sp3]
    %\pstLineAB[]{s3}{sp3}
    %\pstGeonode[](5,3){s4}
    %\pstProjection[]{ma}{mb}{s4}[sp4]
    %\pstLineAB[]{s4}{sp4}
    %
    \psdots[dotstyle=o](5.5,3.5)(4,4)(5,4)(5.5,2.5)(4.9,4.5)
    \psdots[dotstyle=triangle](0.5,1.5)(1.5,1.5)(1,2.5)(1.25,0.5)
    %margin
    %\pstGeonode[](2.4,4){p1}
    \vspace{0.2in}
    %\pstProjection[]{na}{nb}{p1}[pp1]
    %\pstLineAB[arrows=<->]{p1}{pp1}
    %\pstProjection[]{ma}{mb}{p1}[pp2]
    %\pstLineAB[arrows=<->]{p1}{pp2}
    %\psline[]{<->}(2,2.1)(3,2.5)
    %\psline[]{<->}(3,2.5)(4,2.9)
    %\uput[90](1.7,3.8){${1 \over \|\bw\|}$}
    %\uput[90](2.75,4.2){${1 \over \|\bw\|}$}
    \endpspicture
  }
}
\column{0.5\textwidth}
  \scriptsize
\renewcommand{\arraystretch}{1.1}
\begin{center}
\begin{tabular}{|c|c|c|c|}
        \hline
        $\bx_i$ & $x_{i1}$ & $x_{i2}$ & $y_i$\\
        \hline
        $\bx_1$ & $3.5$ & $4.25$ & $+1$\\
        $\bx_2$ & $4$ & $3$ & $+1$\\
        $\bx_3$ & $4$ & $4$ & $+1$\\
        $\bx_4$ & $4.5$ & $1.75$ & $+1$\\
        $\bx_5$ & $4.9$ & $4.5$ & $+1$\\
        $\bx_6$ & $5$ & $4$ & $+1$\\
        $\bx_7$ & $5.5$ & $2.5$ & $+1$\\
        $\bx_8$ & $5.5$ & $3.5$ & $+1$\\
        $\bx_9$ & $0.5$ & $1.5$ & $-1$\\
        $\bx_{10}$ & $1$ & $2.5$ & $-1$\\
        $\bx_{11}$ & $1.25$ & $0.5$ & $-1$\\
        $\bx_{12}$ & $1.5$ & $1.5$ & $-1$\\
        $\bx_{13}$ & $2$ & $2$ & $-1$\\
        $\bx_{14}$ & $2.5$ & $0.75$ & $-1$\\
        \hline
        $\bx_{15}$ & 4 & 2 & $+1$\\
        $\bx_{16}$ & 2 & 3 & $+1$\\
        $\bx_{17}$ & 3 & 2 & $-1$\\
        $\bx_{18}$ & 5 & 3 & $-1$\\
		\hline
    \end{tabular}%}{}
\end{center}
\end{columns}
\end{frame}


\begin{frame}{Example Dataset: Linearly Non-separable Case}

    Let $k=1$ and $C=1$, then solving the $L_{dual}$ yields the
    following support vectors and Lagrangian values
    $\alpha_i$:
    \begin{center}%\vspace*{6pt}
    {\tabcolsep12pt\renewcommand{\arraystretch}{1.1}\begin{tabular}{|c|c|c|c|c|}
        \hline
        $\bx_i$ & $x_{i1}$ & $x_{i2}$ & $y_i$ & $\alpha_i$\\
        \hline
        $\bx_1$ & 3.5 & 4.25 & $+1$ & 0.0271\\
        $\bx_2$ & 4 & 3 & $+1$ & 0.2162\\
        $\bx_4$ & 4.5 & 1.75 & $+1$ & 0.9928\\
        $\bx_{13}$ & 2 & 2 & $-1$ & 0.9928\\
        $\bx_{14}$ & 2.5 & 0.75 & $-1$ & 0.2434\\
        $\bx_{15}$ & 4 & 2 & $+1$ & 1\\
        $\bx_{16}$ & 2 & 3 & $+1$ & 1\\
        $\bx_{17}$ & 3 & 2 & $-1$ & 1\\
        $\bx_{18}$ & 5 & 3 & $-1$ & 1\\
        \hline
   \end{tabular}}\vspace*{6pt}
    \end{center}
The optimal hyperplane is given as follows:
\begin{align*}
    h(\bx) = \cramped{\matr{0.834\\0.333}^T} \bx - 3.334 = 0
\end{align*}
\end{frame}

\begin{frame}{Example Dataset: Linearly Non-separable Case}

The slack $\xi_i = 0$ for all points that are not support
vectors, and also for those support vectors that are on the
margin. Slack is positive only for the remaining support
vectors and it can be computed as:
    $\xi_i = 1 - y_i(\bw^T\bx_i+b)$.

Thus, for all support vectors not on the margin, we have
\begin{center}\vspace*{10pt}\small
    {\tabcolsep12pt\renewcommand{\arraystretch}{1.1}\begin{tabular}{|c|c|c|c|}
        \hline
        $\bx_i$ & $\bw^T\bx_i$ & $\bw^T\bx_i + b$ & $\xi_i = 1
        - y_i(\bw^T\bx_i+b)$\\
        \hline
        $\bx_{15}$ & 4.001 & 0.667 & 0.333\\
        $\bx_{16}$ & 2.667 & $-0.667$ & 1.667\\
        $\bx_{17}$ & 3.167 & $-0.167$ & 0.833\\
        $\bx_{18}$ & 5.168 & 1.834 & 2.834\\
        \hline
    \end{tabular}}\vspace*{10pt}
\end{center}
The total slack is given as
$$\sum_i \xi_i = \xi_{15} + \xi_{16} + \xi_{17} + \xi_{18} =
0.333+ 1.667 +
0.833 + 2.834 = 5.667$$

The slack variable $\xi_i > 1$ for those points that
are misclassif\/{i}ed (i.e., are on the wrong side of the
hyperplane), namely $\bx_{16} = (3,3)^T$ and $\bx_{18} = (5,3)^T$.
The other two points are correctly classif\/{i}ed, but lie within the
margin, and thus satisfy $0 < \xi_i < 1$.
\end{frame}


\ifdefined\wox \begin{frame} \titlepage \end{frame} \fi


\begin{frame}{Kernel SVM: Nonlinear Case}

The linear SVM approach can be used for datasets with a nonlinear
decision boundary via the kernel trick.

\medskip
Conceptually, the idea is to map the
original $d$-dimensional points $\bx_i$ in the input space to
points $\phi(\bx_i)$ in a high-dimensional feature space via some
nonlinear transformation $\phi$. 

\medskip
Given the extra flexibility, it
is more likely that the points $\phi(\bx_i)$ might be linearly
separable in the feature space. 

\medskip
A linear
decision surface in feature space actually corresponds to a
nonlinear decision surface in the input space. 

\medskip
Further, the kernel
trick allows us to carry out all operations via the kernel
function computed in input space, rather than having to map the
points into feature space.
\end{frame}



\begin{frame}{Nonlinear SVM}
There is no linear classif\/{i}er that can discriminate between the
points. However, there exists a perfect quadratic classif\/{i}er that can
separate the two classes.

\begin{center}
 $\phi(\bx) = (\sqrt{2}x_1, \sqrt{2}x_2, x_1^2, x_2^2, \sqrt{2}x_1x_2)^T$
\end{center}

\begin{figure}[!t]
\scalebox{0.7}{
  \centerline{
    \psset{unit=0.5in,dotscale=2,arrowscale=2}
    \pspicture[](-0.5,-0.5)(8,6)
   \definecolor{mygray}{gray}{0.75}
    \psaxes[]{->}(0,0)(0,0)(8,6)
    \psellipse[linewidth=3pt,fillcolor=mygray,
    fillstyle=solid](4.05,2.91)(2.78,1.55)
    \psgrid[subgriddiv=1,griddots=10,gridcolor=gray,gridlabels=0](0,0)(0,0)(8,6)
    \psellipse[linestyle=dashed,linewidth=1.5pt](4.05,2.91)(1.957,1.093)
    \psellipse[linestyle=dashed,linewidth=1.5pt](4.05,2.91)(3.41,1.906)
    \psdots[dotstyle=Btriangle,fillcolor=gray](4,4)(6,3)
    \psdots[dotstyle=Bo,fillcolor=gray](1,2)(4,1)(6,4.5)(7,2)
    \psdots[dotstyle=triangle](4,3)(5,3.25)(5.5,3.25)(3.5,2.5)(3.5,3.5)(2.5,3)(2.75,2.75)(3,3.25)(4.25,2.75)(4.5,2.5)
    \psdots[dotstyle=o](4,5.5)(1.25,4.5)(2,1)(2.25,5)(3,0.75)(3,5)(3,6)(5,1)(5,5.5)(6,1.25)(6.5,1)(7,4.5)(2,4.6)
    \endpspicture
  }
}
\end{figure}
\end{frame}


\begin{frame}{Nonlinear SVMs: Kernel Trick}
To apply the kernel trick for nonlinear SVM classif\/{i}cation, we
have to show that all operations require only the kernel function:
\begin{align*}
K(\bx_i, \bx_{j}) = \phi(\bx_i)^T \phi(\bx_{j})
%\label{eq:class:svm:K}
\end{align*}

\medskip
Applying $\phi$ to each point, we
can obtain the new dataset in the
feature space
$\bD_\phi = \{ \phi(\bx_i), y_i \}_{i=1}^n$.

\medskip
The SVM objective function in
feature space is given as
\begin{empheq}[box=\tcbhighmath]{align*}
\begin{split}
        \mbox{\bf Objective Function: } & \min_{\bw,b,\xi_i}
        \lB\{ {\| \bw \|^2 \over 2} + C \sum_{i=1}^n (\xi_i)^k\rB\}\\
  \mbox{\bf Linear Constraints: } & y_i \; (\bw^T\phi(\bx_i) + b) \ge
  1 - \xi_i, \text{and } \xi_i \ge 0,\;\; \forall \bx_i \in \bD
\end{split}
\end{empheq}
where $\bw$ is the weight vector, $b$ is the bias, and $\xi_i$ are the slack
variables, all in feature space.
\end{frame}



\begin{frame}{Nonlinear SVMs: Kernel Trick}
For hinge loss, the dual Lagrangian in feature space is given as
\begin{align*}
    \max_{\balpha} \; L_{dual} & = \dsum_{i=1}^n\alpha_{i} - {1 \over 2} \dsum_{i=1}^n
  \dsum_{j=1}^n \alpha_i\alpha_{j} y_i y_{j} \phi(\bx_i)^T
  \phi(\bx_{j})\\
  & =  \dsum_{i=1}^n\alpha_{i} - {1 \over 2} \dsum_{i=1}^n
  \dsum_{j=1}^n \alpha_i\alpha_{j} y_i y_{j} K(\bx_i,\bx_{j})
\end{align*}
Subject to the constraints that $0 \le \alpha_i \le C$, and
$\sum_{i=1}^n \alpha_i y_i = 0$.

\medskip
The dual Lagrangian depends only on the
dot product between two vectors in feature space
$\phi(\bx_i)^T \phi(\bx_{j})= K(\bx_i,\bx_{j})$, and thus we can
solve the optimization problem using the kernel matrix $\bK =
\{K(\bx_i, \bx_{j}) \}_{i,j=1,\ldots,n}$.


\medskip 
For quadratic loss,
the dual Lagrangian
corresponds to the use of a new kernel
\begin{align*}
    K_q(\bx_i, \bx_{j}) = \bx_i^T\bx_{j} + {1\over 2C}\delta_{ij}
    = K(\bx_i, \bx_{j}) + {1\over 2C}\delta_{ij}
\end{align*}
\end{frame}



\begin{frame}{Nonlinear SVMs: Weight Vector and Bias}
We cannot directly obtain the weight vector without transforming the
points, since
\begin{align*}
    \bw  =  \dsum_{\mathclap{\alpha_i>0}} \alpha_i y_i \phi(\bx_i)
\end{align*}


However, we can compute the bias via kernel operations, since
\begin{align*}
    b_i  & =  y_i - \bw^T\phi(\bx_i) = 
     y_i - \dsum_{\alpha_{j} >0} \alpha_{j} y_{j}
    K(\bx_{j},\bx_i)
\end{align*}

Likewise, we can 
predict the class for a new point $\bz$ as follows:
\begin{align*}
  \hat{y}  & =  \sign(\bw^T\phi(\bz) + b)
         =  
\tcbhighmath{ 
\sign\lB(\sum_{\alpha_i> 0} \alpha_i y_i K(\bx_i,\bz) + b\rB)
}
\end{align*}

All SVM operations can be carried out in
terms of the kernel function $K(\bx_i, \bx_{j}) =
\phi(\bx_i)^T\phi(\bx_{j})$. 
Thus, any nonlinear kernel function can be used to
do nonlinear classif\/{i}cation in the input space.

\end{frame}



\begin{frame}{Nonlinear SVM: Inhomogeneous Quadratic Kernel}
\begin{figure}[!t]
\scalebox{0.7}{
  \centerline{
    \psset{unit=0.5in,dotscale=2,arrowscale=2}
    \pspicture[](-0.5,-0.5)(8,6)
   \definecolor{mygray}{gray}{0.75}
    \psaxes[]{->}(0,0)(0,0)(8,6)
    \psellipse[linewidth=3pt,fillcolor=mygray,
    fillstyle=solid](4.05,2.91)(2.78,1.55)
    \psgrid[subgriddiv=1,griddots=10,gridcolor=gray,gridlabels=0](0,0)(0,0)(8,6)
    \psellipse[linestyle=dashed,linewidth=1.5pt](4.05,2.91)(1.957,1.093)
    \psellipse[linestyle=dashed,linewidth=1.5pt](4.05,2.91)(3.41,1.906)
    \psdots[dotstyle=Btriangle,fillcolor=gray](4,4)(6,3)
    \psdots[dotstyle=Bo,fillcolor=gray](1,2)(4,1)(6,4.5)(7,2)
    \psdots[dotstyle=triangle](4,3)(5,3.25)(5.5,3.25)(3.5,2.5)(3.5,3.5)(2.5,3)(2.75,2.75)(3,3.25)(4.25,2.75)(4.5,2.5)
    \psdots[dotstyle=o](4,5.5)(1.25,4.5)(2,1)(2.25,5)(3,0.75)(3,5)(3,6)(5,1)(5,5.5)(6,1.25)(6.5,1)(7,4.5)(2,4.6)
    \endpspicture
  }
}
\end{figure}
The optimal quadratic hyperplane is obtained by setting $C=4$, and
using an inhomogeneous polynomial kernel
of degree $q=2$:
    \begin{align*}
        K(\bx_i, \bx_{j}) = \phi(\bx_i)^T\phi(\bx_{j})= (1 +
        \bx_i^T\bx_{j})^2
    \end{align*}
\end{frame}


\begin{frame}{Nonlinear SVM: Inhomogeneous Quadratic Kernel}

    $\phi$ maps $\bx_i$ into feature space as follows:
    \begin{align*}
    \phi\lB(\bx = (x_{1},x_{2})^T\rB) = \lB(1, \sqrt{2}x_{1},
    \sqrt{2}x_{2}, x_{1}^2, x_{2}^2, \sqrt{2}x_{1}x_{2}\rB)^T
    \end{align*}

 $\bx_1 = (1,2)^T$ is transformed into 
$$\phi(\bx_i) = \lB(1, \sqrt{2}\cdot1, \sqrt{2}\cdot2, 1^2, 2^2, \sqrt{2}\cdot1\cdot2\rB)^T = (1,1.41, 2.83, 1, 2, 2.83)^T$$


Solving $L_{dual}$, we found the following six support vectors:
    \begin{center}
    {\tabcolsep12pt\renewcommand{\arraystretch}{1.1}\begin{tabular}{|c||c|c||c|c|}
        \hline
        $\bx_i$ & $(x_{i1}, x_{i2})^T$ & $\phi(\bx_i)$ & $y_i$ & $\alpha_i$\\
        \hline
          $\bx_1$ & $(1,2)^T$ & $(1,1.41,2.83,1,4,2.83)^T$ & $+1$ & $0.6198$\\
          $\bx_2$ & $(4,1)^T$ & $(1,5.66,1.41,16,1,5.66)^T$ & $+1$ & $2.069$\\
          $\bx_3$ & $(6,4.5)^T$ & $(1,8.49,6.36,36,20.25,38.18)^T$ & $+1$ & $3.803$\\
          $\bx_4$ & $(7,2)^T$ & $(1,9.90,2.83,49,4,19.80)^T$ & $+1$ & $0.3182$\\
          $\bx_5$ & $(4,4)^T$ & $(1,5.66,5.66,16,16,15.91)^T$ & $-1$ & $2.9598$\\
          $\bx_6$ & $(6,3)^T$ & $(1,8.49,4.24,36,9,25.46)^T$ & $-1$ & $3.8502$\\
        \hline
    \end{tabular}}\vspace*{6pt}
    \end{center}
\end{frame}

\begin{frame}{Nonlinear SVM: Inhomogeneous Quadratic Kernel}
 We compute the weight vector for the hyperplane:
\begin{align*}
    \bw & = \dsum_{\alpha_i > 0} \alpha_i y_i \phi(\bx_i)
    = (0,-1.413,-3.298,0.256,0.82,-0.018)^T
\end{align*}
as well as the bias:
    $$b = -8.841$$

The decision boundary in input space corresponds to an ellipse,
centered at $(4.046,2.907)$, with axis lengths $2.78$ and $1.55$. 

\medskip

Notice that we explicitly transformed all the points into the feature space just for illustration purposes.

\medskip

The kernel trick allows us to achieve the same goal using only the kernel function.


\end{frame}

\ifdefined\wox \begin{frame} \titlepage \end{frame} \fi

\begin{frame}{SVM Training Algorithms}
Instead of dealing
explicitly with the bias $b$, we map each point $\bx_i \in
\setR^d$ to the point $\bx'_i \in \setR^{d+1}$ as follows:
\begin{align*}
    \bx'_i = (x_{i1},\ldots,x_{id},1)^T
\end{align*}
We also map the weight vector to $\setR^{d+1}$, with
$w_{d+1} = b$,
so that
\begin{align*}
\bw = (w_1, \ldots, w_d, b)^T 
\end{align*}

\medskip
The equation of the hyperplane
is then given as follows:
\begin{align*}
    h(\bx'): \bw^T\bx' = 
    w_1x_{i1} + \cdots + w_dx_{id} + b = 0
\end{align*}

\medskip
After the mapping, the constraint $\sum_{i=1}^n \alpha_i y_i =
0$ does not apply in the SVM dual formulations.
The new set of constraints is
given as
\begin{align*}
y_i \bw^T\bx \ge 1 - \xi_i
\end{align*}
\end{frame}



\begin{frame}{Dual Optimization: Gradient Ascent}
The dual optimization objective for
hinge loss  is given as
\begin{align*}
    \max_{\balpha} J(\balpha) =
    \dsum_{i=1}^n\alpha_{i} - {1 \over 2} \dsum_{i=1}^n
  \dsum_{j=1}^n \alpha_i\alpha_{j} y_i y_{j} K(\bx_i,\bx_{j})
\end{align*}
subject to the constraints $0 \le \alpha_i \le C$ for all
$i=1,\ldots,n$. Here $\balpha = (\alpha_1, \alpha_2, \cdots, \alpha_n)^T
\in \setR^n$.

\medskip
The gradient or the rate of change in the objective function at
$\balpha$ is given as the partial derivative of $J(\balpha)$ with
respect to $\balpha$, that is, with respect to each $\alpha_k$:
\begin{align*}
    \grad J(\balpha) =
   \matr{ \dfrac{\partial J(\balpha)}{\partial \alpha_1},
    \dfrac{\partial J(\balpha)}{\partial \alpha_2}, \ldots,
    \dfrac{\partial J(\balpha)}{\partial \alpha_n}
    }^T
\end{align*}
where the $k$th component of the gradient is obtained by
differentiating $J(\alpha_k)$ with respect to $\alpha_k$:
\begin{align*}
    {\partial J(\balpha) \over \partial \alpha_k} =
    {\partial J(\alpha_k) \over \partial \alpha_k} =
    1 - y_k
    \lB( \dsum_{i=1}^n \alpha_i y_i K(\bx_i,\bx_k) \rB)
\end{align*}
\end{frame}


\begin{frame}{Stochastic Gradient Ascent}

  Starting from an initial $\balpha$, the
gradient ascent approach successively updates by moving in the direction
of the gradient $\grad J(\balpha)$:
\begin{align*}
    \balpha_{t+1} = \balpha_t + \eta_t \grad J(\balpha_t)
    %\label{eq:class:svm:updateA}
\end{align*}
where $\balpha_t$ is the estimate at the $t$th step, and $\eta_t$ is the
step size. 

\medskip
The optimal step size is:
\begin{align*}
    \eta_k = {1 \over K(\bx_k, \bx_k)}
\end{align*}

\medskip
Instead of updating the entire $\balpha$ vector in each step, in
the stochastic gradient ascent approach, we update each component $\alpha_k$ independently and immediately
use the new value to update other components. 
The update rule for the $k$-th component
is given as
\begin{align*}
    \alpha_{k} & = \alpha_{k} + \eta_{k}
    \dfrac{\partial J(\balpha)}{\partial \alpha_k}
    = \alpha_{k} + \eta_{k}
    \lB(1 - y_k\dsum_{i=1}^n \alpha_i y_i K(\bx_i,\bx_k)
    \rB)
\end{align*}
\end{frame}



\begin{frame}[fragile]{Algorithm \textsc SVM-Dual}
%\begin{tightalgo}[!b]{\textwidth-18pt}
\begin{footnotesize}
\begin{tightalgo}[H]{\textwidth-18pt}
\SetKwInOut{Algorithm}{\textsc{SVM-Dual} ($\bD, K, C, \epsilon$)}
\Algorithm{} 
\lForEach{$\bx_i \in \bD$}{ $\bx_i \assign \matr{\bx_i\\ 1}$} %\tcp{map to $\setR^{d+1}$}  
\uIf{loss = {\tt hinge}}{ $\bK \assign \{K(\bx_i, \bx_{j})\}_{i,j=1,\ldots,n}$ \tcp{kernel matrix, hinge loss} } 
\ElseIf{loss = {\tt quadratic}}{ $\bK \assign \{K(\bx_i, \bx_{j})+{1\over 2C}\delta_{ij}\}_{i,j=1,\ldots,n}$ \tcp{kernel matrix, quadratic loss} } 
\lFor{$k = 1, \ldots, n$}{$\eta_k \assign {1\over K(\bx_k,\bx_k)}$} %\tcp{set step size} 
$t \assign 0$\; 
$\balpha_0 \assign (0, \ldots, 0)^T$\;
\Repeat{$\norm{\balpha_t-\balpha_{t-1}} \le \epsilon$} {%
$\balpha \assign \balpha_t$\;
\For{$k = 1 \text{ to } n$ }{%
\tcp{update $k$th component of $\balpha$}
$\alpha_k \assign \alpha_{k} + \eta_{k}
    \Bigl(1 - y_k\dsum_{i=1}^n \alpha_i y_i K(\bx_i,\bx_k)
    \Bigr)$\;
    \lIf{$\alpha_k < 0$}{$\alpha_k \assign 0$}
    \lIf{$\alpha_k > C$}{$\alpha_k \assign C$}
}%
$\balpha_{t+1} = \balpha$\;
$t \assign t+1$\;
}%
\end{tightalgo}
\end{footnotesize}
\end{frame}




\readdata{\dataSLW}{CLASS/svm/figs/iris-slwc.txt}
\begin{frame}{SVM Dual Algorithm: Iris Data -- Linear Kernel}
  \framesubtitle{$c_1$: {\tt Iris-setosa} (circles) and $c_2$: other
  types of Iris flowers (triangles)}
\begin{figure}[!t]
    %\vspace{0.2in}
    \scalebox{0.6}{
        \centering
        \psset{dotscale=1.5,fillcolor=lightgray,
                    arrowscale=2,PointName=none}
                    \psset{xAxisLabel=$X_1$, yAxisLabel= $X_2$}
        \psgraph[tickstyle=bottom,Dx=0.5,Dy=0.5,
        Ox=4,Oy=2]{->}(4.0,2.0)(8.5,4.5){4in}{3in}%
        \listplot[plotstyle=dots,dotstyle=Bo,showpoints=true,
            nEnd=50,plotNo=1,plotNoMax=2]{\dataSLW}
        \listplot[plotstyle=dots,dotstyle=Btriangle,showpoints=true,
            nStart=51,plotNo=1,plotNoMax=2]{\dataSLW}
        \begin{psclip}{%
            \psline[](4,2)(4,4.5)(8.5,4.5)(8.5,2)(4,2)}
        \psplotImp[linecolor=gray,algebraic](4,2)(8.5,4.5){%
                2.74*x-3.74*y-3.09}
        \psplotImp[linewidth=2pt,
            algebraic](4,0)(8.5,10.5){%
                8.56*x-7.14*y-23.12}
        \end{psclip}
        \uput[0](6.25,4.6){$h_{1000}$}
        \uput[0](7.1,4.6){$h_{10}$}
        \endpsgraph
     }
\end{figure}
\small
    Hyperplane $h_{10}$ uses $C=10$ and
    $h_{1000}$ uses $C=1000$:
    \begin{align*}
        h_{10}(\bx): & \quad 2.74 x_1 - 3.74 x_2 -3.09 = 0\\
        h_{1000}(\bx): &\quad 8.56 x_1 -7.14 x_2 -23.12 = 0
    \end{align*}
    $h_{10}$ has a larger margin, but
    also a larger slack; $h_{1000}$ 
	has a smaller margin, but it minimizes the slack.
\end{frame}




\readdata{\dataPC}{CLASS/svm/figs/iris-PC.txt}
\begin{frame}{SVM Dual Algorithm: Quadratic versus Linear Kernel}
  \framesubtitle{$c_1$: {\tt Iris-versicolor} (circles) and $c_2$: other
  types of Iris flowers (triangles)}
\begin{figure}[!t]
    %\vspace{0.2in}
    \scalebox{0.8}{
        \centering
        \psset{dotscale=1.5,fillcolor=lightgray,
                    arrowscale=2,PointName=none}
        \psset{xAxisLabel=$\bu_1$, yAxisLabel= $\bu_2$}
        \psgraph[tickstyle=bottom,Ox=-4,Oy=-1.5,Dx=1,Dy=0.5]{->}(-4.0,-1.5)(4.0,1.5){4in}{3in}%
        \listplot[plotstyle=dots,dotstyle=Bo,showpoints=true,
            nEnd=50,plotNo=1,plotNoMax=2]{\dataPC}
        \listplot[plotstyle=dots,dotstyle=Btriangle,showpoints=true,
            nStart=51,plotNo=1,plotNoMax=2]{\dataPC}
        \begin{psclip}{%
            \psline[](-4,-1.5)(-4,1.5)(4,1.5)(4,-1.5)(-4,-1.5)}
        \psplotImp[linewidth=2pt,algebraic](-4,-2)(4,3.5){%
        1.86*x^2+1.87*x*y+0.14*x+0.85*y^2-1.22*y-3.25   }
        \psplotImp[linecolor=gray,algebraic](-4,-2)(4,3.5){%
        0.161*x+1.9*y+0.8025}
    \end{psclip}
    \uput[0](4,-0.8){$h_l$}
    \uput[0](-2.5,1.6){$h_q$}
        \endpsgraph
}
\end{figure}
\end{frame}



\begin{frame}{Primal Solution: Newton Optimization}

Consider the primal optimization function for soft margin SVMs.
With $\bw, \bx_i
\in \setR^{d+1}$, we have to minimize the
objective function:
\begin{align*}
    \min_{\bw} J(\bw) = \frac{1}{2} \| \bw \|^2 + C \dsum_{i=1}^n
    (\xi_i)^k
\end{align*}
subject to the linear constraints:
\begin{align*}
    y_i \; (\bw^T\bx_i) \ge 1 - \xi_i \text{ and } \xi_i \ge 0
    \text{ for all } i=1,\ldots,n
\end{align*}
Rearranging the above, we obtain an expression
for $\xi_i$
\begin{align*}
    \xi_i & \ge 1 - y_i \; (\bw^T\bx_i) \text{ and } \xi_i \ge
    0, \text{ which implies that} \notag\\
    \xi_i & = \max\lB\{0, 1-y_i \; (\bw^T\bx_i)\rB\} 
\end{align*}

\end{frame}


\begin{frame}{Primal Solution: Newton Optimization, Quadratic Loss}

  The objective function can be rewritten as
\begin{align*}
    J(\bw) & = \frac{1}{2} \| \bw \|^2 +
    C \dsum_{i=1}^n  \max\lB\{0, 1-y_i \; (\bw^T\bx_i)\rB\}^k\notag\\
    & = \frac{1}{2} \| \bw \|^2 +
    C \dsum_{\mathclap{y_i(\bw^T\bx_i)<1}} \lB(1 - y_i (\bw^T\bx_i)\rB)^k
\end{align*}

  For quadratic loss, we have $k=2$ and 
  the gradient or the rate of change of the objective function at
$\bw$ is given as the partial derivative of $J(\bw)$ with respect
to $\bw$:
\begin{align*}
    \grad_\bw = {\partial J(\bw) \over \partial \bw}
 = \bw - 2C \bv + 2C \bS \bw
\end{align*}
where the vector $\bv$ and the matrix $\bS$ are given as
\begin{align*}
    \bv & =\dsum_{\mathclap{y_i(\bw^T\bx_i)<1}} y_i \bx_i &
\bS &= \dsum_{\mathclap{y_i(\bw^T\bx_i)<1}} \bx_i \bx_i^T
\end{align*}
\end{frame}



\begin{frame}{Primal Solution: Newton Optimization, Quadratic Loss}

The {\em Hessian matrix} is
def\/{i}ned as the matrix of second-order
partial derivatives of $J(\bw)$ with respect to $\bw$,
which is given as
\begin{align*}
    \bH_\bw = {\partial \grad_\bw \over \partial \bw} = \bI +
    2C\bS
\end{align*}

\medskip
Because we want to minimize the objective function $J(\bw)$, we
should move in the direction opposite to the gradient. The Newton
optimization update rule for $\bw$ is given as
\begin{align*}
    \bw_{t+1} & = \bw_t - \eta_t \bH_{\bw_t}^{-1} \grad_{\bw_t}
\end{align*}
where $\eta_t > 0$ is a scalar value denoting the step size at
iteration $t$. 

\end{frame}



\begin{frame}[fragile]{Primal SVM Algorithm}
\begin{tightalgo}[H]{\textwidth-18pt}
\SetKwInOut{Algorithm}{\textsc{SVM-Primal} ($\bD, C, \epsilon$)}
\Algorithm{} 
\ForEach{$\bx_i \in \bD$}{ 
$\bx_i \assign \matr{\bx_i\\ 1}$ \tcp{map to $\setR^{d+1}$} } 
$t \assign 0$\; 
$\bw_0 \assign (0, \ldots, 0)^T$ \tcp{initialize $\bw_t \in \setR^{d+1}$} 
\Repeat{
$\norm{\bw_{t} - \bw_{t-1}} \le \epsilon$}
{ $\bv \assign \dsum_{\mathclap{y_i(\bw_t^T\bx_i)<1}} y_i \bx_i$\;
\nllabel{alg:class:svm:LpNewton:GHbeg} $\bS \assign
\dsum_{\mathclap{y_i(\bw_t^T\bx_i)<1}} \bx_i \bx_i^T$\; 
$\grad \assign (\bI+2C\bS) \bw_t - 2C\bv$ \tcp{gradient}\; 
$\bH \assign \bI + 2C\bS$ \tcp{Hessian}\;
\nllabel{alg:class:svm:LpNewton:GHend} $\bw_{t+1} \assign \bw_t - \eta_t \bH^{-1}\grad$ \tcp{Newton update rule}%[Eq.\nosp\eqref{eq:class:svm:newtonstep}]} \nllabel{alg:class:svm:LpNewton:update}\; $t \assign t+1$ 
}
\end{tightalgo}
\end{frame}



\begin{frame}{SVMs: Dual and Primal Solutions}
  \framesubtitle{$c_1$: {\tt Iris-setosa} (circles) and $c_2$: other
  types of Iris flowers (triangles)}
\begin{figure}[!t]
    %\vspace{0.2in}
    \scalebox{0.8}{
        \centering
        \psset{dotscale=1.5,fillcolor=lightgray,
                    arrowscale=2,PointName=none}
                    \psset{xAxisLabel=$X_1$, yAxisLabel= $X_2$}
        \psgraph[tickstyle=bottom,Dx=0.5,Dy=0.5,
        Ox=4,Oy=2]{->}(4.0,2.0)(8.5,4.5){4in}{3in}%
        \listplot[plotstyle=dots,dotstyle=Bo,showpoints=true,
            nEnd=50,plotNo=1,plotNoMax=2]{\dataSLW}
        \listplot[plotstyle=dots,dotstyle=Btriangle,showpoints=true,
            nStart=51,plotNo=1,plotNoMax=2]{\dataSLW}
        \begin{psclip}{%
            \psline[](4,2)(4,4.5)(8.5,4.5)(8.5,2)(4,2)}
        \psplotImp[linecolor=gray,
            algebraic](4,0)(8.5,10.5){%
                7.468*x -6.337*y -19.887}
        \psplotImp[linewidth=2pt,
            algebraic](4,0)(8.5,10.5){%
            7.47371498*x  -6.34019883*y -19.9082496}
        \end{psclip}
        \uput[0](6.2,4.6){$h_{d},h_p$}
        \endpsgraph
}
\end{figure}
\end{frame}


\begin{frame}[fragile]{SVM Primal Kernel Algorithm: Newton Optimization}
  The linear soft margin primal algorithm, with quadratic loss, 
  can easily be extended to work
  on any kernel matrix $\bK$:

\begin{tightalgo}[H]{\textwidth-18pt}
  \small
\SetKwInOut{Algorithm}{\textsc{SVM-Primal-Kernel} ($\bD, K, C, \epsilon$)}
\Algorithm{}
\ForEach{$\bx_i \in \bD$}{
$\bx_i \assign \matr{\bx_i\\ 1}$ \tcp{map to $\setR^{d+1}$}
}
$\bK \assign \{K(\bx_i, \bx_{j})\}_{i,j=1,\ldots,n}$ \tcp{compute kernel matrix}
$t \assign 0$\;
$\bbeta_0 \assign (0, \ldots, 0)^T$ \tcp{initialize $\bbeta_t \in \setR^{n}$}
\Repeat{$\norm{\bbeta_{t} - \bbeta_{t-1}} \le \epsilon$}
{
$\bv \assign \dsum_{\mathclap{y_i(\bK_i^T\bbeta_t)<1}} y_i \bK_i$\;
\nllabel{alg:class:svm:LpKNewton:GHbeg}
$\bS \assign \dsum_{\mathclap{y_i(\bK_i^T\bbeta_t)<1}} \bK_i \bK_i^T$\;
$\grad \assign (\bK+2C\bS) \bbeta_t - 2C\bv$ \tcp{gradient}
$\bH \assign \bK + 2C\bS$ \tcp{Hessian}
\nllabel{alg:class:svm:LpKNewton:GHend}
$\bbeta_{t+1} \assign \bbeta_t - \eta_t \bH^{-1}\grad$
\tcp{Newton update rule}
\nllabel{alg:class:svm:LpKNewton:update}
$t \assign t+1$\;
}
\end{tightalgo}
\end{frame}


\begin{frame}{SVM Quadratic Kernel: Dual and Primal Solutions}
  \framesubtitle{$c_1$: {\tt Iris-versicolor} (circles) and $c_2$: other
  types of Iris flowers (triangles)}
\begin{figure}[!t]
    %\vspace{0.2in}
    \scalebox{0.8}{
        \centering
        \psset{dotscale=1.5,fillcolor=lightgray,
                    arrowscale=2,PointName=none}
        \psset{xAxisLabel=$\bu_1$, yAxisLabel= $\bu_2$}
        \psgraph[tickstyle=bottom,Ox=-4,Oy=-1.5,Dx=1,Dy=0.5]{->}(-4.0,-1.5)(4.0,1.5){4in}{3in}%
        \listplot[plotstyle=dots,dotstyle=Bo,showpoints=true,
            nEnd=50,plotNo=1,plotNoMax=2]{\dataPC}
        \listplot[plotstyle=dots,dotstyle=Btriangle,showpoints=true,
            nStart=51,plotNo=1,plotNoMax=2]{\dataPC}
        \begin{psclip}{%
            \psline[](-4,-1.5)(-4,1.5)(4,1.5)(4,-1.5)(-4,-1.5)}
        \psplotImp[linewidth=2pt,algebraic](-4,-2)(4,3.5){%
        1.4*x^2+1.34*x*y-0.051*x+0.664*y^2-0.96*y-2.66}
        \psplotImp[linewidth=2pt,linecolor=gray,algebraic](-4,-2)(4,3.5){%
        0.87*x^2 + 0.64*x*y - 0.5*x + 0.43*y^2 - 1.04*y - 2.398}
    \end{psclip}
    \uput[0](0.6,1.6){$h_d$}
    \uput[0](1.4,1.6){$h_p$}
        \endpsgraph
}
\end{figure}
\end{frame}
