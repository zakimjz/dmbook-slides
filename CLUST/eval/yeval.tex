\lecture{eval}{eval}

\date{Chapter 17: Clustering Validation}
\newcommand{\NMI}{\mathit{NMI}}
\newcommand{\VI}{\mathit{VI}}

\begin{frame}
\titlepage
\end{frame}


\begin{frame}{Clustering Validation and Evaluation}
Cluster validation and assessment encompasses three main tasks:
{\em clustering
evaluation} seeks to assess the goodness or quality of the clustering,
{\em clustering stability} seeks to understand the sensitivity of the clustering result to various algorithmic parameters,
for example, the number of clusters,
and {\em clustering tendency} assesses
the suitability of applying clustering in the f\/{i}rst place, that is, whether
the data has any inherent grouping structure.

\bigskip
Validity measures can be divided into three main types:

\begin{description}

\item [External:] External validation measures employ criteria that are
  not inherent to the dataset, e.g., class labels. 

\item [Internal:] Internal validation measures employ criteria that are
  derived from the data itself, e.g., intracluster
  and intercluster distances.

\item [Relative:] Relative validation measures aim to directly compare
  different clusterings, usually those
  obtained via different parameter settings for the same algorithm.
\end{description}
\end{frame}



\begin{frame}{External Measures}
External measures assume that the correct or
ground-truth clustering is known {\it a priori}, which 
is used to evaluate a
given clustering.  

\medskip
Let $\bD = \{\bx_i\}_{i=1}^n$ be a dataset consisting of $n$ points in a
$d$-dimensional space, partitioned into $k$ clusters.  Let $y_i \in
\left\{ 1, 2,\ldots , k \right\}$ denote the ground-truth cluster
membership or label information for each point.  

\medskip
The ground-truth
clustering is given as 
$\cT = \left\{ T_1, T_2, \ldots, T_k \right\}$,
where the cluster $T_{j}$ consists of all the points with label $j$, 
i.e.,
$T_{j} = \left\{ \bx_i \in \bD | y_i = j \right\}$.  
We refer to $\cT$ as the ground-truth {\em
partitioning}, and to each $T_i$ as a {\em partition}.



\medskip
Let $\cC=\{C_1,
\ldots, C_r\}$ denote a clustering of the same dataset into $r$
clusters, obtained via some clustering algorithm, and let $\hy_i \in
\left\{ 1, 2, \ldots, r \right\}$ denote the cluster label for $\bx_i$.
\end{frame}

\begin{frame}{External Measures}
External evaluation measures try
capture the extent to which points from the same partition appear in the
same cluster, and the extent to which points from different partitions
are grouped in different clusters. 

\medskip
All of the external measures rely on the $r\times k$
{\em contingency table} $\bN$ that is
induced by a clustering $\cC$ and the ground-truth partitioning $\cT$, def\/{i}ned as follows
\begin{align*}
  \bN(i,j) = n_{ij}  = \left| C_i \cap T_{j} \right|
\end{align*}
The count $n_{ij}$ denotes the number of points that are
common to cluster $C_i$ and ground-truth partition $T_{j}$.

\medskip
Let $n_{i} = |C_i|$ denote the number of points in cluster
$C_i$, and let $m_{j} = |T_{j}|$ denote the number of points in partition
$T_{j}$.  

\medskip
The contingency table can be computed from $\cT$ and $\cC$ in
$O(n)$ time by examining the partition and cluster labels, $y_i$ and
$\hy_i$, for each point $\bx_i \in \bD$ and incrementing the
corresponding count $n_{y_i\hy_i}$.
\end{frame}


\begin{frame}{Matching Based Measures: Purity}
Purity quantif\/{i}es the extent to which a
cluster $C_i$ contains entities from only one partition: 
\begin{align*}
  \mathit{purity}_i = \frac{1}{n_{i}}\max_{j=1}^k\; \{n_{ij}\}
\end{align*}

\bigskip
The purity of clustering $\cC$ is def\/{i}ned as the weighted sum of the
clusterwise purity values:
\begin{align*}
\tcbhighmath{
  \mathit{purity} = \sum_{i=1}^r \frac{n_{i}}{n} \mathit{purity}_i =
  \frac{1}{n}\sum_{i=1}^r \max_{j=1}^k \{n_{ij}\}
}
\end{align*}
where the ratio $\tfrac{n_{i}}{n}$ denotes the fraction of points in
cluster $C_i$.  
\end{frame}


\begin{frame}{Matching Based Measures: Maximum Matching}
The maximum matching
measure selects the mapping between clusters and partitions, such
that the sum of the number of common points ($n_{ij}$) is
maximized, provided that only one cluster can match with a given
partition.

\medskip
Let $G$ be a bipartite graph over the vertex set $V = \cC \cup \cT$, and
let the edge set be $E = \{ (C_i, T_{j}) \}$ with 
edge weights $w(C_i,T_{j}) = n_{ij}$.
A {\em matching} $M$ in $G$ is a subset of $E$, such that the edges in
$M$ are pairwise nonadjacent, that is, they do not have a common vertex.

\medskip
The {\em maximum weight matching} in $G$ is given as:
\begin{align*}
  \mathit{match} = \arg \max_M \lB\{ \frac{w(M)}{n} \rB\}
\end{align*}
where $w(M)$ is the sum of the sum of
all the edge weights in matching $M$, given as $w(M) = \sum_{e \in M} w(e)$
\end{frame}




\begin{frame}{Matching Based Measures: F-measure}
  \small
Given cluster $C_i$, let $j_i$ denote the
partition that contains the maximum number of points from $C_i$,
that is, $j_i = \max_{j=1}^k \{ n_{ij} \}$.  

\medskip
The {\em precision} of a
cluster $C_i$ is the same as its purity:
\begin{align*}
\tcbhighmath{
  \mathit{prec}_i = \frac{1}{n_{i}}\max_{j=1}^k \left\{ n_{ij} \right\} =
  \frac{n_{ij_i}}{n_i}
}
\end{align*}

\medskip
The {\em recall} of cluster $C_i$ is def\/{i}ned as
\begin{align*}
\tcbhighmath{
  \mathit{recall}_i = \frac{n_{ij_i}}{\card{T_{j_i}}} =\frac{n_{ij_i}}{m_{j_i}}
}
 \end{align*}
 where $m_{j_i} = \card{T_{j_i}}$.

\medskip
\end{frame}

\begin{frame}{Matching Based Measures: F-measure}
The F-measure is the harmonic mean of the precision and recall values for
each $C_i$
\begin{align*}
\tcbhighmath{
  F_i = \frac{2}{\frac{1}{\mathit{prec}_i} + \frac{1}{\mathit{recall}_i}} =
  \frac{2 \cdot \mathit{prec}_i \cdot \mathit{recall}_i}{\mathit{prec}_i + \mathit{recall}_i} =
  \frac{2 \; n_{ij_i}}{n_{i} + m_{j_i}}
}
\end{align*}

\medskip
The F-measure for the clustering $\cC$ is the mean of clusterwise
F-meaure values:
\begin{align*}
  F = \frac{1}{r} \sum_{i=1}^r F_i
\end{align*}
\end{frame}




\begin{frame}[fragile]{K-means: Iris Principal Components Data}
\framesubtitle{Good Case}
\setcounter{subfigure}{0}
\begin{figure}
    \centering
    \captionsetup[subfloat]{captionskip=20pt}
    \def\pshlabel#1{ {\footnotesize $#1$}}
    \def\psvlabel#1{ {\footnotesize $#1$}}
    \psset{xAxisLabel=$\bu_1$,yAxisLabel= $\bu_2$}
    \psset{xunit=0.5in,yunit=0.65in,dotscale=1.5,arrowscale=2,PointName=none}
    \centerline{
	\scalebox{0.6}{
    %\subfloat[K-means: good]{\label{fig:clust:eval:kexgood}
        %\pspicture[](-5,-2)(3.5,1.5)
        %\psaxes[tickstyle=bottom,Dx=1,Ox=-4,Dy=0.5,Oy=-1.5]{->}(-4,-1.5)(3.5,1.5)
        \psgraph[tickstyle=bottom,Dx=1,Ox=-4,Dy=0.5,Oy=-1.5]{->}%
        (-4,-1.5)(3.5,1.5){3.5in}{2in}
        \psset{dotstyle=Bsquare,fillcolor=lightgray}
        \input{CLUST/eval/irisPCgood-C1}
        \psset{fillcolor=white}
        \input{CLUST/eval/irisPCgood-W1}
        \psset{dotstyle=Bo,fillcolor=lightgray}
        \input{CLUST/eval/irisPCgood-C2}
        \psset{fillcolor=white}
        \input{CLUST/eval/irisPCgood-W2}
        \psset{dotstyle=Btriangle,fillcolor=lightgray}
        \input{CLUST/eval/irisPCgood-C3}
        \psset{fillcolor=white}
        \psdot[](-1.46,0.50)
\psdot[](-1.56,0.27)
\psdot[](-1.28,0.69)

        \psset{fillcolor=black}
        \pstGeonode[PointSymbol=Bo, dotscale=2](2.64,0.19){A}
        \pstGeonode[PointSymbol=Btriangle,
        dotscale=2](-2.35,0.27){B}
        \pstGeonode[PointSymbol=Bsquare,
        dotscale=2](-0.66,-0.33){C}
        \psclip{\psframe[](-4,-1.5)(3.5,1.5)}%
        {
        \psset{linestyle=none, PointSymbol=none}
        \pstMediatorAB{A}{B}{K}{KP}
        \pstMediatorAB{C}{A}{J}{JP}
        \pstMediatorAB{B}{C}{I}{IP}
        \pstInterLL[PointSymbol=none]{I}{IP}{J}{JP}{O}
        \psset{linewidth=1pt,linestyle=dashed}
        \pstGeonode[PointSymbol=none](-4,-1.5){a}(-4,1.5){b}(3.5,1.5){c}(3.5,-1.5){d}
        \pstInterLL[PointSymbol=none]{O}{I}{a}{b}{oi}
        \pstLineAB{O}{oi}
        \pstInterLL[PointSymbol=none]{O}{J}{a}{d}{oj}
        \pstLineAB{O}{oj}
        \pstInterLL[PointSymbol=none]{O}{K}{b}{c}{ok}
        \pstLineAB{O}{ok}
        }
        \endpsclip
        %\endpspicture
        \endpsgraph
    }}
  \end{figure}
    
Contingency table:
\small
\begin{align*}
  \begin{array}{c|ccc|c}
  & 
  \texttt{iris-setosa} & \texttt{iris-versicolor}
  & \texttt{iris-virginica} &\\
  &  T_1 & T_2 & T_3 & n_i\\
  \hline
  C_1 \text{(squares)} & 0 & 47 & 14 & 61\\
  C_2 \text{(circles)} & 50 &  0 &  0 & 50\\
  C_3 \text{(triangles)} & 0 &  3 & 36 &  39\\
\hline
m_{j} & 50 & 50 & 50 & n=100
\end{array}
\end{align*}
\normalsize
$\mathit{purity} = 0.887$, $\mathit{match} = 0.887$, $F = 0.885$.
\end{frame}

\begin{frame}[fragile]{K-means: Iris Principal Components Data}
\framesubtitle{Bad Case}
\begin{figure}
    \captionsetup[subfloat]{captionskip=20pt}
    \def\pshlabel#1{ {\footnotesize $#1$}}
    \def\psvlabel#1{ {\footnotesize $#1$}}
    \psset{xAxisLabel=$\bu_1$,yAxisLabel= $\bu_2$}
    \psset{xunit=0.5in,yunit=0.65in,dotscale=1.5,arrowscale=2,PointName=none}
    \centerline{
	\scalebox{0.6}{
        %\pspicture[](-5,-2)(3.5,1.5)
        %\psaxes[tickstyle=bottom,Dx=1,Ox=-4,Dy=0.5,Oy=-1.5]{->}(-4,-1.5)(3.5,1.5)
        \psgraph[tickstyle=bottom,Dx=1,Ox=-4,Dy=0.5,Oy=-1.5]{->}%
        (-4,-1.5)(3.5,1.5){3.5in}{2in}
        \psset{dotstyle=Bsquare,fillcolor=lightgray}
        \psdot[](2.54,0.44)
\psdot[](2.59,0.52)
\psdot[](2.47,0.14)
\psdot[](2.63,0.17)
\psdot[](2.70,0.12)
\psdot[](2.38,1.34)
\psdot[](2.41,0.20)
\psdot[](2.62,0.82)
\psdot[](2.87,0.08)
\psdot[](2.21,0.44)
\psdot[](2.54,0.59)
\psdot[](2.20,0.88)
\psdot[](2.41,0.42)
\psdot[](2.77,0.27)
\psdot[](2.30,0.11)
\psdot[](3.22,0.14)
\psdot[](2.64,0.32)
\psdot[](2.63,0.61)
\psdot[](2.28,0.75)
\psdot[](2.73,0.33)
\psdot[](2.56,0.37)
\psdot[](2.60,1.10)
\psdot[](2.68,0.33)
\psdot[](2.54,0.51)
\psdot[](2.65,0.82)
\psdot[](2.64,1.19)
\psdot[](2.51,0.65)
\psdot[](2.65,0.32)
\psdot[](2.31,0.40)
\psdot[](2.59,0.24)

        \psset{fillcolor=white}
        \input{CLUST/eval/irisPCbad-W1}
        \psset{dotstyle=Bo,fillcolor=lightgray}
        \psdot[](2.84,-0.22)
\psdot[](2.89,-0.14)
\psdot[](2.67,-0.11)
\psdot[](2.75,-0.31)
\psdot[](2.79,-0.23)
\psdot[](2.98,-0.48)
\psdot[](2.67,-0.11)
\psdot[](2.82,-0.08)
\psdot[](2.63,-0.19)
\psdot[](2.89,-0.57)
\psdot[](2.72,-0.24)
\psdot[](2.61,0.02)
\psdot[](3.00,-0.33)
\psdot[](2.36,-0.03)
\psdot[](2.72,-0.17)
\psdot[](2.85,-0.93)
\psdot[](3.23,-0.50)
\psdot[](2.51,-0.14)
\psdot[](2.59,-0.20)
\psdot[](2.67,-0.11)

        \psset{fillcolor=white}
        \psdot[](0.91,-0.75)
\psdot[](0.71,-1.01)
\psdot[](0.51,-1.26)
\psdot[](0.75,-1.00)

        \psset{dotstyle=Btriangle,fillcolor=lightgray}
        \input{CLUST/eval/irisPCbad-C3}
        \psset{fillcolor=white}
        \input{CLUST/eval/irisPCbad-W3}
        \psset{fillcolor=black}
        \pstGeonode[PointSymbol=Bsquare, dotscale=2](2.562,0.486){A}
        \pstGeonode[PointSymbol=Bo,
        dotscale=2](2.419,-0.379){B}
        \pstGeonode[PointSymbol=Btriangle,
        dotscale=2](-1.405,-0.057){C}
        \psclip{\psframe[](-4,-1.5)(3.5,1.5)}%
        {
        \psset{linestyle=none, PointSymbol=none}
        \pstMediatorAB{A}{B}{K}{KP}
        \pstMediatorAB{C}{A}{J}{JP}
        \pstMediatorAB{B}{C}{I}{IP}
        \pstInterLL[PointSymbol=none]{I}{IP}{J}{JP}{O}
        \psset{linewidth=1pt,linestyle=dashed}
        \pstGeonode[PointSymbol=none](-4,-1.5){a}(-4,1.5){b}(3.5,1.5){c}(3.5,-1.5){d}
        \pstInterLL[PointSymbol=none]{O}{I}{a}{b}{oi}
        \pstLineAB{O}{oi}
        \pstInterLL[PointSymbol=none]{O}{J}{b}{c}{oj}
        \pstLineAB{O}{oj}
        \pstInterLL[PointSymbol=none]{O}{K}{a}{d}{ok}
        \pstLineAB{O}{ok}
        }
        \endpsclip
        %\endpspicture
        \endpsgraph
    }}
\end{figure}

Contingency table:
  \small
\begin{align*}
  \begin{array}{c|ccc|c}
  & \texttt{iris-setosa} & \texttt{iris-versicolor}
  & \texttt{iris-virginica} & \\
  & T_1 & T_2 & T_3  & n_i\\
  \hline
C_1 (squares) & 30 & 0 & 0 & 30\\
C_2 (circles) & 20 &  4 &  0 & 24\\
C_3 (triangles) & 0 &  46 & 50 & 96\\
\hline
m_{j} &50 & 50 & 50 &n=150\\
  \end{array}
\end{align*}
\normalsize
$\mathit{purity} = 0.667$, $\mathit{match} = 0.560$, $F=0.658$
\end{frame}



\begin{frame}{Entropy-based Measures: Conditional Entropy}
The entropy of a clustering $\cC$  and partitioning $\cT$ is given as
\begin{align*}
  H(\cC) & = - \sum_{i=1}^r p_{C_i} \log p_{C_i} &
  H(\cT) & = - \sum_{j=1}^k p_{T_{j}} \log p_{T_{j}}
\end{align*}
where $p_{C_i} = \tfrac{n_i}{n}$ and $p_{T_{j}} = \tfrac{m_{j}}{n}$
are the probabilities of cluster $C_i$ and partition $T_{j}$.

\bigskip
The cluster-specif\/{i}c entropy of $\cT$, that is, the conditional entropy of
$\cT$ with respect to cluster $C_i$ is def\/{i}ned as
\begin{align*}
  H(\cT|C_i) = - \sum_{j=1}^k
  \lB( \frac{n_{ij}}{n_i} \rB)
  \log \lB( \frac{n_{ij}}{n_i} \rB)
\end{align*}
\end{frame}

\begin{frame}{Entropy-based Measures: Conditional Entropy}
The conditional entropy of $\cT$ given clustering $\cC$ is def\/{i}ned
as the weighted sum:
\begin{align*}
  H(\cT|\cC) & = \sum_{i=1}^r \frac{n_{i}}{n} H(\cT|C_i)
  = 
\tcbhighmath{
-\sum_{i=1}^r \sum_{j=1}^k p_{ij} \log \lB(
  \frac{p_{ij}}{p_{C_i}}\rB)
}
\\
  & = H(\cC,\cT) - H(\cC)
\end{align*}
where $p_{ij}= \tfrac{n_{ij}}{n}$ is the probability that a point in
cluster $i$ also belongs to partition and 
where $H(\cC,\cT) = - \sum_{i=1}^r \sum_{j=1}^k p_{ij} \log p_{ij}$ is
the joint entropy of $\cC$ and $\cT$.  

\bigskip
$H(\cT|\cC) = 0$ if and only if
$\cT$ is completely determined by $\cC$, corresponding to the ideal
clustering.
If $\cC$ and $\cT$ are independent of
each other, then $H(\cT|\cC) = H(\cT)$.
\end{frame}


\begin{frame}{Entropy-based Measures: Normalized Mutual Information} 
The {\em mutual
information} tries to quantify the amount of shared information
between the clustering $\cC$ and partitioning $\cT$, and it is
def\/{i}ned as
\begin{align*}
\tcbhighmath{
  I(\cC,\cT) = \sum_{i=1}^r \sum_{j=1}^k p_{ij} \log
  \lB(\frac{p_{ij}}{p_{C_i} \cdot p_{T_{j}}} \rB)
}
\end{align*}
When $\cC$ and $\cT$
are independent then $p_{ij} = p_{C_i} \cdot p_{T_{j}}$, and thus
$I(\cC,\cT) = 0$. However, there is no upper bound on the mutual
information.


\bigskip
The {\em normalized mutual information} (NMI) is def\/{i}ned as
the geometric mean:
\begin{align*}
\tcbhighmath{
  \NMI(\cC,\cT) = \sqrt{\frac{I(\cC,\cT)}{H(\cC)} \cdot \frac{I(\cC,\cT)}{H(\cT)}} =
  \frac{I(\cC,\cT)}{\sqrt{H(\cC)\cdot H(\cT)}}
}
\end{align*}
The NMI value lies in the range $[0,1]$. Values close to $1$ indicate a
good clustering.
\end{frame}




\begin{frame}{Entropy-based Measures: Variation of Information}
This criterion is based on the mutual information
between the clustering $\cC$ and the ground-truth partitioning
$\cT$, and their entropy; it is def\/{i}ned as
\begin{align*}
  \VI(\cC,\cT) &= (H(\cT)-I(\cC,\cT)) + (H(\cC)-I(\cC,\cT))\nonumber\\
    & = H(\cT)+H(\cC)-2I(\cC,\cT)
\end{align*}
Variation of information (VI) is zero only when $\cC$ and $\cT$ are identical. Thus, the
lower the VI value the better the clustering $\cC$.

\bigskip
VI can also be expressed as:
\begin{align*}
  \VI(\cC,\cT) & = H(\cT|\cC) + H(\cC|\cT)\\
\end{align*}
\vspace*{-1.0cm}
\begin{empheq}[box=\tcbhighmath]{align*}
\begin{split}
  \VI(\cC,\cT) & = 2H(\cT,\cC)-H(\cT)-H(\cC)
\end{split}
\end{empheq}
\end{frame}


\begin{frame}[fragile]{K-means: Iris Principal Components Data}
\framesubtitle{Good Case}

\setcounter{subfigure}{0}
\begin{figure}
    \centering
    \captionsetup[subfloat]{captionskip=20pt}
    \def\pshlabel#1{ {\footnotesize $#1$}}
    \def\psvlabel#1{ {\footnotesize $#1$}}
    \psset{xAxisLabel=$\bu_1$,yAxisLabel= $\bu_2$}
    \psset{xunit=0.5in,yunit=0.65in,dotscale=1.5,arrowscale=2,PointName=none}
    \centerline{
    \subfloat[K-means: good]{
	\scalebox{0.6}{
        %\pspicture[](-5,-2)(3.5,1.5)
        %\psaxes[tickstyle=bottom,Dx=1,Ox=-4,Dy=0.5,Oy=-1.5]{->}(-4,-1.5)(3.5,1.5)
        \psgraph[tickstyle=bottom,Dx=1,Ox=-4,Dy=0.5,Oy=-1.5]{->}%
        (-4,-1.5)(3.5,1.5){3.5in}{2in}
        \psset{dotstyle=Bsquare,fillcolor=lightgray}
        \input{CLUST/eval/irisPCgood-C1}
        \psset{fillcolor=white}
        \input{CLUST/eval/irisPCgood-W1}
        \psset{dotstyle=Bo,fillcolor=lightgray}
        \input{CLUST/eval/irisPCgood-C2}
        \psset{fillcolor=white}
        \input{CLUST/eval/irisPCgood-W2}
        \psset{dotstyle=Btriangle,fillcolor=lightgray}
        \input{CLUST/eval/irisPCgood-C3}
        \psset{fillcolor=white}
        \psdot[](-1.46,0.50)
\psdot[](-1.56,0.27)
\psdot[](-1.28,0.69)

        \psset{fillcolor=black}
        \pstGeonode[PointSymbol=Bo, dotscale=2](2.64,0.19){A}
        \pstGeonode[PointSymbol=Btriangle,
        dotscale=2](-2.35,0.27){B}
        \pstGeonode[PointSymbol=Bsquare,
        dotscale=2](-0.66,-0.33){C}
        \psclip{\psframe[](-4,-1.5)(3.5,1.5)}%
        {
        \psset{linestyle=none, PointSymbol=none}
        \pstMediatorAB{A}{B}{K}{KP}
        \pstMediatorAB{C}{A}{J}{JP}
        \pstMediatorAB{B}{C}{I}{IP}
        \pstInterLL[PointSymbol=none]{I}{IP}{J}{JP}{O}
        \psset{linewidth=1pt,linestyle=dashed}
        \pstGeonode[PointSymbol=none](-4,-1.5){a}(-4,1.5){b}(3.5,1.5){c}(3.5,-1.5){d}
        \pstInterLL[PointSymbol=none]{O}{I}{a}{b}{oi}
        \pstLineAB{O}{oi}
        \pstInterLL[PointSymbol=none]{O}{J}{a}{d}{oj}
        \pstLineAB{O}{oj}
        \pstInterLL[PointSymbol=none]{O}{K}{b}{c}{ok}
        \pstLineAB{O}{ok}
        }
        \endpsclip
        %\endpspicture
        \endpsgraph
		}} \hspace{0.25in}
		\subfloat[K-means: bad]{
	  \scalebox{0.6}{
        %\pspicture[](-5,-2)(3.5,1.5)
        %\psaxes[tickstyle=bottom,Dx=1,Ox=-4,Dy=0.5,Oy=-1.5]{->}(-4,-1.5)(3.5,1.5)
        \psgraph[tickstyle=bottom,Dx=1,Ox=-4,Dy=0.5,Oy=-1.5]{->}%
        (-4,-1.5)(3.5,1.5){3.5in}{2in}
        \psset{dotstyle=Bsquare,fillcolor=lightgray}
        \psdot[](2.54,0.44)
\psdot[](2.59,0.52)
\psdot[](2.47,0.14)
\psdot[](2.63,0.17)
\psdot[](2.70,0.12)
\psdot[](2.38,1.34)
\psdot[](2.41,0.20)
\psdot[](2.62,0.82)
\psdot[](2.87,0.08)
\psdot[](2.21,0.44)
\psdot[](2.54,0.59)
\psdot[](2.20,0.88)
\psdot[](2.41,0.42)
\psdot[](2.77,0.27)
\psdot[](2.30,0.11)
\psdot[](3.22,0.14)
\psdot[](2.64,0.32)
\psdot[](2.63,0.61)
\psdot[](2.28,0.75)
\psdot[](2.73,0.33)
\psdot[](2.56,0.37)
\psdot[](2.60,1.10)
\psdot[](2.68,0.33)
\psdot[](2.54,0.51)
\psdot[](2.65,0.82)
\psdot[](2.64,1.19)
\psdot[](2.51,0.65)
\psdot[](2.65,0.32)
\psdot[](2.31,0.40)
\psdot[](2.59,0.24)

        \psset{fillcolor=white}
        \input{CLUST/eval/irisPCbad-W1}
        \psset{dotstyle=Bo,fillcolor=lightgray}
        \psdot[](2.84,-0.22)
\psdot[](2.89,-0.14)
\psdot[](2.67,-0.11)
\psdot[](2.75,-0.31)
\psdot[](2.79,-0.23)
\psdot[](2.98,-0.48)
\psdot[](2.67,-0.11)
\psdot[](2.82,-0.08)
\psdot[](2.63,-0.19)
\psdot[](2.89,-0.57)
\psdot[](2.72,-0.24)
\psdot[](2.61,0.02)
\psdot[](3.00,-0.33)
\psdot[](2.36,-0.03)
\psdot[](2.72,-0.17)
\psdot[](2.85,-0.93)
\psdot[](3.23,-0.50)
\psdot[](2.51,-0.14)
\psdot[](2.59,-0.20)
\psdot[](2.67,-0.11)

        \psset{fillcolor=white}
        \psdot[](0.91,-0.75)
\psdot[](0.71,-1.01)
\psdot[](0.51,-1.26)
\psdot[](0.75,-1.00)

        \psset{dotstyle=Btriangle,fillcolor=lightgray}
        \input{CLUST/eval/irisPCbad-C3}
        \psset{fillcolor=white}
        \input{CLUST/eval/irisPCbad-W3}
        \psset{fillcolor=black}
        \pstGeonode[PointSymbol=Bsquare, dotscale=2](2.562,0.486){A}
        \pstGeonode[PointSymbol=Bo,
        dotscale=2](2.419,-0.379){B}
        \pstGeonode[PointSymbol=Btriangle,
        dotscale=2](-1.405,-0.057){C}
        \psclip{\psframe[](-4,-1.5)(3.5,1.5)}%
        {
        \psset{linestyle=none, PointSymbol=none}
        \pstMediatorAB{A}{B}{K}{KP}
        \pstMediatorAB{C}{A}{J}{JP}
        \pstMediatorAB{B}{C}{I}{IP}
        \pstInterLL[PointSymbol=none]{I}{IP}{J}{JP}{O}
        \psset{linewidth=1pt,linestyle=dashed}
        \pstGeonode[PointSymbol=none](-4,-1.5){a}(-4,1.5){b}(3.5,1.5){c}(3.5,-1.5){d}
        \pstInterLL[PointSymbol=none]{O}{I}{a}{b}{oi}
        \pstLineAB{O}{oi}
        \pstInterLL[PointSymbol=none]{O}{J}{b}{c}{oj}
        \pstLineAB{O}{oj}
        \pstInterLL[PointSymbol=none]{O}{K}{a}{d}{ok}
        \pstLineAB{O}{ok}
        }
        \endpsclip
        %\endpspicture
        \endpsgraph
    }}
	}
\end{figure}
\begin{center}
\begin{tabular}{|l|ccc|ccc|}
  \hline
  & $\mathit{purity}$ & $\mathit{match}$ & $F$ &
  $H(\cT|\cC)$ & $\NMI$  & $\VI$\\
  \hline
  (a) \text{ Good } & 0.887 & 0.887 & 0.885 & 0.418 & 0.742   & 0.812\\
  (b) \text{ Bad } & 0.667 & 0.560 & 0.658 & 0.743 &  0.587  & 1.200\\
  \hline
\end{tabular}
\end{center}
\end{frame}



\begin{frame}{Pairwise Measures}
Given clustering $\cC$ and ground-truth partitioning $\cT$, let
$\bx_i, \bx_{j} \in \bD$ be any two points, with $i\ne j$. 
Let $y_i$ denote
the true partition label and let $\hy_i$ denote the cluster label for
point $\bx_i$.

\medskip
If both $\bx_i$ and $\bx_{j}$ belong to the same cluster, that is, $\hy_i =
\hy_{j}$, we call it
a {\em positive} event, and if they do not belong to the same cluster,
that is, $\hy_i \ne \hy_{j}$, we call that a {\em negative} event.
Depending on whether there is agreement between the cluster labels and
partition labels, there are four possibilities to consider:

\begin{description}

\item[\textit{True Positives:}] 
  $\bx_i$ and $\bx_{j}$ belong to the same
  partition in $\cT$, and they are also in the same cluster in $\cC$. The number of true
  positive pairs is given as
  \begin{align*}
\tcbhighmath{
  \mathit{TP} = \bigl|\{(\bx_i, \bx_{j}):\; y_i = y_{j}
  \text{ and } \hy_i = \hy_{j} \}\bigr|
}
  \end{align*}

\item[\textit{False Negatives:}] 
  $\bx_i$ and $\bx_{j}$ belong to the same
  partition in $\cT$, but they do not belong to the  same cluster in $\cC$.
The
  number of all false negative pairs is given as
  \begin{align*}
\tcbhighmath{
  \mathit{FN} = \bigl|\{(\bx_i, \bx_{j}):\; y_i = y_{j}
  \text{ and } \hy_i \ne \hy_{j} \}\bigr|
}
  \end{align*}
\end{description}
\end{frame}

\begin{frame}{Pairwise Measures}
  \begin{description}
\item[\textit{False Positives:}]
  $\bx_i$ and $\bx_{j}$ do not belong to the
  same partition in $\cT$, but they do belong to the same cluster in $\cC$.
  The
  number of false positive pairs is given as
  \begin{align*}
\tcbhighmath{
  \mathit{FP} = \bigl|\{(\bx_i, \bx_{j}):\; y_i \ne y_{j}
  \text{ and } \hy_i = \hy_{j} \}\bigr|
}
  \end{align*}

\item[\textit{True Negatives:}] $\bx_i$ and $\bx_{j}$ neither belong to
  the same partition in $\cT$, nor do they belong to the same cluster in
  $\cC$.  The number of such true negative pairs is given as
    \begin{align*}
\tcbhighmath{
  \mathit{TN} = \bigl|\{(\bx_i, \bx_{j}):\; y_i \ne y_{j}
  \text{ and } \hy_i \ne \hy_{j} \}\bigr|
}
  \end{align*}
\end{description}

Because there are $N = {n \choose 2} = \frac{n(n-1)}{2}$ pairs of points, we
have the following identity:
\begin{align*}
  N = \mathit{TP} + \mathit{FN} + \mathit{FP} + \mathit{TN}
\end{align*}
\end{frame}

\begin{frame}{Pairwise Measures: TP, TN, FP, FN}
They can be computed eff\/{i}ciently using the
contingency table $\bN = \left\{ n_{ij} \right\}$.
The number of true positives is given as
\begin{align*}
  \mathit{TP} 
  & = \frac{1}{2} \biggl(\Bigl(\sum_{i=1}^r \sum_{j=1}^k n_{ij}^2\Bigr)
  - n \biggr)
\end{align*}
The false negatives can be computed as
\begin{align*}
  \mathit{FN} 
  & = \frac{1}{2}\biggl(
  \sum_{j=1}^k m_{j}^2 - \sum_{i=1}^r \sum_{j=1}^k n_{ij}^2 \biggr)
\end{align*}
The number of false positives are:
\begin{align*}
\mathit{FP} & =
  \frac{1}{2}\biggl(
  \sum_{i=1}^r n_i^2 - \sum_{i=1}^r \sum_{j=1}^k n_{ij}^2 \biggr)
\end{align*}

F{i}nally, the number of true negatives can be obtained via
\begin{align*}
  \mathit{TN} = N - (\mathit{TP} + \mathit{FN} + \mathit{FP}) = \frac{1}{2} \biggl(
  n^2 - \sum_{i=1}^r n_i^2 - \sum_{j=1}^k m_{j}^2 +
  \sum_{i=1}^r \sum_{j=1}^k n_{ij}^2
  \biggr)
\end{align*}
\end{frame}



\begin{frame}{Pairwise Measures: Jaccard Coeff\/{i}cient, Rand Statistic}
\small
{\bf Jaccard Coeff\/{i}cient:} measures the fraction of true positive point pairs,
but after ignoring the true negative:
\begin{align*}
\tcbhighmath{
\mathit{Jaccard} = \frac{\mathit{TP}}{\mathit{TP} + \mathit{FN} + \mathit{FP}}
}
\end{align*}

\medskip
{\bf Rand Statistic:} 
measures the fraction of true positives and true negatives over
all point pairs:
\begin{align*}
\tcbhighmath{
\mathit{Rand} = \frac{\mathit{TP} + \mathit{TN}}{N}
}
\end{align*}
\end{frame}

\begin{frame}{Pairwise Measures: FM Measure}

\medskip {\bf Fowlkes-Mallows Measure:} 
Def\/{i}ne the overall
{\em pairwise
precision} and
{\em pairwise recall} values for a clustering $\cC$, as follows:
\begin{align*}
  \mathit{prec} & = \mathit{TP}/\mathit{TP}+\mathit{FP} &
  \mathit{recall} & = \mathit{TP}/\mathit{TP}+\mathit{FN}
\end{align*}
The Fowlkes--Mallows (FM) measure is def\/{i}ned as the
geometric mean of the pairwise precision and recall
\begin{align*}
\tcbhighmath{
\mathit{FM} = \sqrt{\mathit{prec}\cdot \mathit{recall}} = \frac{\mathit{TP}}{\sqrt{ (\mathit{TP}+\mathit{FN})(\mathit{TP}+\mathit{FP}) }}
}
\end{align*}
\end{frame}


\begin{frame}[fragile]{K-means: Iris Principal Components Data}
\framesubtitle{Good Case}

\setcounter{subfigure}{0}
\begin{columns}
  \column{0.5\textwidth}
\begin{figure}
    \centering
    \captionsetup[subfloat]{captionskip=20pt}
    \def\pshlabel#1{ {\footnotesize $#1$}}
    \def\psvlabel#1{ {\footnotesize $#1$}}
    \psset{xAxisLabel=$\bu_1$,yAxisLabel= $\bu_2$}
    \psset{xunit=0.5in,yunit=0.65in,dotscale=1.5,arrowscale=2,PointName=none}
    \centerline{
	\scalebox{0.5}{
    %\subfloat[K-means: good]{\label{fig:clust:eval:kexgood}
        %\pspicture[](-5,-2)(3.5,1.5)
        %\psaxes[tickstyle=bottom,Dx=1,Ox=-4,Dy=0.5,Oy=-1.5]{->}(-4,-1.5)(3.5,1.5)
        \psgraph[tickstyle=bottom,Dx=1,Ox=-4,Dy=0.5,Oy=-1.5]{->}%
        (-4,-1.5)(3.5,1.5){3.5in}{2in}
        \psset{dotstyle=Bsquare,fillcolor=lightgray}
        \input{CLUST/eval/irisPCgood-C1}
        \psset{fillcolor=white}
        \input{CLUST/eval/irisPCgood-W1}
        \psset{dotstyle=Bo,fillcolor=lightgray}
        \input{CLUST/eval/irisPCgood-C2}
        \psset{fillcolor=white}
        \input{CLUST/eval/irisPCgood-W2}
        \psset{dotstyle=Btriangle,fillcolor=lightgray}
        \input{CLUST/eval/irisPCgood-C3}
        \psset{fillcolor=white}
        \psdot[](-1.46,0.50)
\psdot[](-1.56,0.27)
\psdot[](-1.28,0.69)

        \psset{fillcolor=black}
        \pstGeonode[PointSymbol=Bo, dotscale=2](2.64,0.19){A}
        \pstGeonode[PointSymbol=Btriangle,
        dotscale=2](-2.35,0.27){B}
        \pstGeonode[PointSymbol=Bsquare,
        dotscale=2](-0.66,-0.33){C}
        \psclip{\psframe[](-4,-1.5)(3.5,1.5)}%
        {
        \psset{linestyle=none, PointSymbol=none}
        \pstMediatorAB{A}{B}{K}{KP}
        \pstMediatorAB{C}{A}{J}{JP}
        \pstMediatorAB{B}{C}{I}{IP}
        \pstInterLL[PointSymbol=none]{I}{IP}{J}{JP}{O}
        \psset{linewidth=1pt,linestyle=dashed}
        \pstGeonode[PointSymbol=none](-4,-1.5){a}(-4,1.5){b}(3.5,1.5){c}(3.5,-1.5){d}
        \pstInterLL[PointSymbol=none]{O}{I}{a}{b}{oi}
        \pstLineAB{O}{oi}
        \pstInterLL[PointSymbol=none]{O}{J}{a}{d}{oj}
        \pstLineAB{O}{oj}
        \pstInterLL[PointSymbol=none]{O}{K}{b}{c}{ok}
        \pstLineAB{O}{ok}
        }
        \endpsclip
        %\endpspicture
        \endpsgraph
    }}
  \end{figure}
    
  \column{0.5\textwidth}
Contingency table:
\small
\begin{align*}
  \matr{ & \vline& \textbf{setosa} & \textbf{versicolor} &
  \textbf{virginica}\\[-3pt]
  & \vline & T_1 & T_2 & T_3\\
  \hline
C_1 &\vline& 0 & 47 & 14\\
C_2 &\vline& 50 &  0 &  0\\
C_3 &\vline& 0 &  3 & 36\\
}
\end{align*}
\end{columns}
The number of true
positives is:
\begin{align*}
  \mathit{TP} & = {47 \choose 2} + {14 \choose 2} + {50 \choose 2} + {3 \choose 2}
  + {36 \choose 2} = 3030
\end{align*}
Likewise, we have
$\mathit{FN} = 645$, 
$\mathit{FP} = 766$,
$\mathit{TN} =6734$, and $N = {150 \choose 2} = 11175$.

We therefore have:
$\mathit{Jaccard} = 0.682$, $\mathit{Rand}=0.887$, 
$\mathit{FM} = 0.811$.

For the ``bad'' clustering, we have:
$\mathit{Jaccard} = 0.477$, $\mathit{Rand}=0.717$, 
$\mathit{FM} = 0.657$.

\end{frame}


\begin{frame}{Correlation Measures: Hubert statistic}
\small
  Let $\bX$ and $\bY$ be two symmetric $n \times n$ matrices,
and let $N = {n \choose 2}$.
Let $\bx, \by \in \setR^{N}$ denote the
vectors obtained by
linearizing the upper triangular elements (excluding the main diagonal)
of $\bX$ and $\bY$.

\medskip
Let $\mu_X$ denote the element-wise mean of $\bx$,
given as
\begin{align*}
  \mu_X & = \frac{1}{N}\sum_{i=1}^{n-1}\sum_{j=i+1}^n
  \bX(i,j) = \frac{1}{N} \bx^T\bx
\end{align*}
and let $\bz_x$ denote the centered $\bx$ vector, def\/{i}ned as
$ \bz_x = \bx - \bone \cdot \mu_X$

\medskip
The Hubert statistic is def\/{i}ned as 
\begin{align*}
\tcbhighmath{
  \Gamma =
  \frac{1}{N} \sum_{i=1}^{n-1} \sum_{j=i+1}^n
\bX(i,j)\cdot \bY(i,j) = \frac{1}{N} \bx^T\by
}
\end{align*}

\medskip
The normalized Hubert statistic is def\/{i}ned as the
element-wise correlation
\begin{align*}
\tcbhighmath{
  \Gamma_n =  \frac{\bz_x^T \bz_y}{\norm{\bz_x} \cdot \norm{\bz_y}} =
  \cos \theta
}
\end{align*}
\end{frame}



\begin{frame}{Correlation-based Measure: Discretized Hubert Statistic}
Let $\bT$ and $\bC$ be the $n \times n$ matrices def\/{i}ned as
\begin{align*}
  \bT(i,j) & =
  \begin{cases}
    1 & \text{if } y_i = y_{j}, i \ne j\\
    0 & \text{otherwise}
  \end{cases} &
  \bC(i,j) & =
  \begin{cases}
    1 & \text{if } \hy_i = \hy_{j}, i \ne j\\
    0 & \text{otherwise}
 \end{cases}
\end{align*}
Let $\bt, \bc \in \setR^N$ denote the $N$-dimensional vectors
comprising the upper triangular elements (excluding the diagonal)
of $\bT$ and $\bC$.
Let $\bz_t$ and $\bz_c$ denote the centered $\bt$ and $\bc$
vectors.

\medskip
The discretized Hubert statistic is computed
by setting $\bx = \bt$ and $\by = \bc$:
\begin{align*}
  \Gamma = \frac{1}{N} \bt^T\bc = \frac{\mathit{TP}}{N}
\end{align*}

\medskip
The
normalized version of the discretized Hubert statistic is simply
the correlation between $\bt$ and $\bc$
\begin{align*}
  \Gamma_n &
  = \frac{\bz_t^T\bz_c}
  {\norm{\bz_t}\cdot \norm{\bz_c}} 
=\frac{\tfrac{\mathit{TP}}{N} - \mu_T \mu_C}
  {\sqrt{\mu_T\mu_C (1-\mu_T)(1-\mu_C)}}
\end{align*}
where $\mu_T = \tfrac{\mathit{TP}+\mathit{FN}}{N}$ and $\mu_C =
\tfrac{\mathit{TP}+\mathit{FP}}{N}$.
\end{frame}


\begin{frame}{Internal Measures}
Internal evaluation measures do not have recourse to the ground-truth
partitioning. To evaluate the
quality of the clustering, internal measures
therefore have to utilize notions of intracluster similarity
or compactness, contrasted with notions of intercluster separation,
with usually a trade-off in maximizing these two aims.

\medskip
The internal measures are based on the $n\times n$ {\em
distance matrix}, also called the {\em proximity matrix}, of all
pairwise distances among the $n$ points:
\begin{align*}
  \bW = \Bigl\{ \dist(\bx_i, \bx_{j}) \Bigr\}_{i,j=1}^n 
\end{align*}
where $\dist(\bx_i, \bx_{j}) = \norm{\bx_i - \bx_{j}}_2$
is the Euclidean
distance between $\bx_i, \bx_{j} \in \bD$.


\medskip
The proximity matrix $\bW$ is 
the adjacency matrix of the weighted complete graph $G$ over
the $n$ points, that is, with nodes $V = \{\bx_i \mid \bx_i \in \bD\}$, edges
$E=\{(\bx_i, \bx_{j})\mid \bx_i, \bx_{j} \in \bD\}$, and edge weights
$w_{ij} = \bW(i, j)$ for all $\bx_i, \bx_{j}\in \bD$.
\end{frame}

\begin{frame}{Internal Measures}
The clustering $\cC$ can be considered as a
$k$-way cut in $G$.
Given any subsets $S,R \subset V$, def\/{i}ne $W(S,R)$ as the sum of the
weights on all edges with one vertex in $S$ and the other in $R$, given
as
\begin{align*}
  W(S, R) = \sum_{\bx_i \in S} \sum_{\bx_{j} \in R} w_{ij}
\end{align*}
We denote by $\ol{S} = V - S$ the complementary set
of vertices.

\medskip
The sum of all the intracluster and intercluster 
weights are given as
\begin{align*}
  W_{in} & = \frac{1}{2} \sum_{i=1}^k W(C_i,C_i) &
  W_{out} & = \frac{1}{2} \sum_{i=1}^k W(C_i, \ol{C_i})
  = \sum_{i=1}^{k-1} \sum_{j > i} W(C_i, C_{j})
\end{align*}

The number of distinct intracluster and intracluster edges is given as
\begin{align*}
  N_{in} &= \sum_{i=1}^k {n_i \choose 2} & 
  N_{out} &= \sum_{i=1}^{k-1} \sum_{j=i+1}^k n_i \cdot n_{j} 
\end{align*}
\end{frame}



\begin{frame}[fragile]{Clusterings as Graphs: Iris (Good Case)}
\setcounter{subfigure}{0}
\begin{figure}
    \centering
    \captionsetup[subfloat]{captionskip=20pt}
    \def\pshlabel#1{ {\footnotesize $#1$}}
    \def\psvlabel#1{ {\footnotesize $#1$}}
    \psset{xAxisLabel=$\bu_1$,yAxisLabel= $\bu_2$}
    \psset{xunit=0.5in,yunit=0.65in,dotscale=1.5,arrowscale=2,PointName=none}
    \centerline{
	\scalebox{0.55}{
    %\subfloat[K-means: good]{\label{fig:clust:eval:kexgood}
        %\pspicture[](-5,-2)(3.5,1.5)
        %\psaxes[tickstyle=bottom,Dx=1,Ox=-4,Dy=0.5,Oy=-1.5]{->}(-4,-1.5)(3.5,1.5)
        \psgraph[tickstyle=bottom,Dx=1,Ox=-4,Dy=0.5,Oy=-1.5]{->}%
        (-4,-1.5)(3.5,1.5){3.5in}{2in}
        \psset{dotstyle=Bsquare,fillcolor=lightgray}
        \input{CLUST/eval/irisPCgood-C1}
        \psset{fillcolor=white}
        \input{CLUST/eval/irisPCgood-W1}
        \psset{dotstyle=Bo,fillcolor=lightgray}
        \input{CLUST/eval/irisPCgood-C2}
        \psset{fillcolor=white}
        \input{CLUST/eval/irisPCgood-W2}
        \psset{dotstyle=Btriangle,fillcolor=lightgray}
        \input{CLUST/eval/irisPCgood-C3}
        \psset{fillcolor=white}
        \psdot[](-1.46,0.50)
\psdot[](-1.56,0.27)
\psdot[](-1.28,0.69)

        \psset{fillcolor=black}
        \pstGeonode[PointSymbol=Bo, dotscale=2](2.64,0.19){A}
        \pstGeonode[PointSymbol=Btriangle,
        dotscale=2](-2.35,0.27){B}
        \pstGeonode[PointSymbol=Bsquare,
        dotscale=2](-0.66,-0.33){C}
        \psclip{\psframe[](-4,-1.5)(3.5,1.5)}%
        {
        \psset{linestyle=none, PointSymbol=none}
        \pstMediatorAB{A}{B}{K}{KP}
        \pstMediatorAB{C}{A}{J}{JP}
        \pstMediatorAB{B}{C}{I}{IP}
        \pstInterLL[PointSymbol=none]{I}{IP}{J}{JP}{O}
        \psset{linewidth=1pt,linestyle=dashed}
        \pstGeonode[PointSymbol=none](-4,-1.5){a}(-4,1.5){b}(3.5,1.5){c}(3.5,-1.5){d}
        \pstInterLL[PointSymbol=none]{O}{I}{a}{b}{oi}
        \pstLineAB{O}{oi}
        \pstInterLL[PointSymbol=none]{O}{J}{a}{d}{oj}
        \pstLineAB{O}{oj}
        \pstInterLL[PointSymbol=none]{O}{K}{b}{c}{ok}
        \pstLineAB{O}{ok}
        }
        \endpsclip
        %\endpspicture
        \endpsgraph
    }
	\hspace{0.2in}
	\scalebox{0.55}{
        %\pspicture[](-5,-2)(3.5,1.5)
        %\psaxes[tickstyle=bottom,Dx=1,Ox=-4,Dy=0.5,Oy=-1.5]{->}
          %(-4,-1.5)(3.5,1.5)
        \psgraph[tickstyle=bottom,Dx=1,Ox=-4,Dy=0.5,Oy=-1.5]{->}%
        (-4,-1.5)(3.5,1.5){3.5in}{2in}
        \pnode(-0.511, -0.102){n0}
\pnode(-1.464, 0.504){n1}
\pnode(-1.043, 0.229){n2}
\pnode(2.840, -0.221){n3}
\pnode(-0.262, -0.548){n4}
\pnode(2.890, -0.137){n5}
\pnode(-2.350, -0.042){n6}
\pnode(-1.414, -0.575){n7}
\pnode(-2.314, 0.183){n8}
\pnode(-2.320, -0.246){n9}
\pnode(2.543, 0.440){n10}
\pnode(2.587, 0.520){n11}
\pnode(-0.331, -0.211){n12}
\pnode(-1.291, -0.116){n13}
\pnode(2.674, -0.107){n14}
\pnode(2.469, 0.138){n15}
\pnode(2.626, 0.170){n16}
\pnode(-0.298, -0.347){n17}
\pnode(2.704, 0.115){n18}
\pnode(-2.614, 0.558){n19}
\pnode(-1.390, -0.283){n20}
\pnode(-1.764, 0.079){n21}
\pnode(2.384, 1.345){n22}
\pnode(-0.165, -0.680){n23}
\pnode(-0.519, -1.191){n24}
\pnode(2.406, 0.196){n25}
\pnode(-0.181, -0.826){n26}
\pnode(2.746, -0.311){n27}
\pnode(-2.388, 0.463){n28}
\pnode(-2.563, 0.276){n29}
\pnode(2.623, 0.818){n30}
\pnode(2.867, 0.077){n31}
\pnode(-1.345, -0.776){n32}
\pnode(-0.229, -0.402){n33}
\pnode(0.908, -0.752){n34}
\pnode(-0.043, -0.581){n35}
\pnode(-0.235, -0.332){n36}
\pnode(2.209, 0.443){n37}
\pnode(-0.813, -0.371){n38}
\pnode(-1.526, -0.375){n39}
\pnode(-0.660, -0.352){n40}
\pnode(-0.890, -0.034){n41}
\pnode(-2.165, 0.215){n42}
\pnode(-2.932, 0.352){n43}
\pnode(-0.356, -0.503){n44}
\pnode(2.787, -0.228){n45}
\pnode(-2.616, 0.342){n46}
\pnode(0.307, -0.365){n47}
\pnode(2.543, 0.586){n48}
\pnode(2.199, 0.879){n49}
\pnode(-0.375, -0.292){n50}
\pnode(-1.198, -0.606){n51}
\pnode(2.982, -0.480){n52}
\pnode(-2.532, -0.012){n53}
\pnode(2.410, 0.418){n54}
\pnode(-2.145, 0.139){n55}
\pnode(-2.108, 0.371){n56}
\pnode(-3.077, 0.686){n57}
\pnode(-0.921, -0.182){n58}
\pnode(0.175, -0.252){n59}
\pnode(-1.780, -0.501){n60}
\pnode(-1.802, -0.216){n61}
\pnode(2.770, 0.271){n62}
\pnode(2.303, 0.106){n63}
\pnode(-0.245, -0.267){n64}
\pnode(-0.587, -0.483){n65}
\pnode(-1.585, -0.539){n66}
\pnode(2.674, -0.107){n67}
\pnode(3.216, 0.142){n68}
\pnode(2.640, 0.319){n69}
\pnode(-3.232, 1.371){n70}
\pnode(-3.795, 0.253){n71}
\pnode(-0.357, -0.067){n72}
\pnode(2.625, 0.607){n73}
\pnode(2.821, -0.082){n74}
\pnode(2.633, -0.190){n75}
\pnode(2.888, -0.571){n76}
\pnode(-1.258, -0.179){n77}
\pnode(2.716, -0.243){n78}
\pnode(-0.812, -0.162){n79}
\pnode(-1.902, 0.116){n80}
\pnode(0.707, -1.008){n81}
\pnode(-0.932, 0.319){n82}
\pnode(-1.971, -0.181){n83}
\pnode(-1.557, 0.267){n84}
\pnode(0.511, -1.262){n85}
\pnode(-1.116, -0.084){n86}
\pnode(-2.419, 0.304){n87}
\pnode(2.280, 0.748){n88}
\pnode(-1.388, -0.204){n89}
\pnode(2.613, 0.022){n90}
\pnode(2.998, -0.334){n91}
\pnode(-1.905, 0.119){n92}
\pnode(-0.942, -0.542){n93}
\pnode(-1.298, -0.761){n94}
\pnode(-2.841, 0.373){n95}
\pnode(0.751, -1.001){n96}
\pnode(-1.285, 0.685){n97}
\pnode(0.192, -0.677){n98}
\pnode(-1.095, 0.284){n99}
\pnode(-1.331, 0.245){n100}
\pnode(-0.984, -0.124){n101}
\pnode(-1.662, 0.242){n102}
\pnode(-2.276, 0.333){n103}
\pnode(-0.928, 0.468){n104}
\pnode(2.356, -0.031){n105}
\pnode(2.715, -0.170){n106}
\pnode(-2.428, 0.377){n107}
\pnode(2.852, -0.933){n108}
\pnode(3.225, -0.503){n109}
\pnode(0.010, -0.721){n110}
\pnode(2.729, 0.334){n111}
\pnode(-0.714, 0.150){n112}
\pnode(2.562, 0.375){n113}
\pnode(-0.135, -0.312){n114}
\pnode(2.597, 1.100){n115}
\pnode(-3.397, 0.547){n116}
\pnode(-1.443, -0.144){n117}
\pnode(-1.905, 0.048){n118}
\pnode(-1.296, -0.328){n119}
\pnode(-1.414, -0.575){n120}
\pnode(2.508, -0.139){n121}
\pnode(-1.220, 0.408){n122}
\pnode(-1.379, -0.421){n123}
\pnode(2.684, 0.327){n124}
\pnode(2.588, -0.197){n125}
\pnode(-0.640, -0.417){n126}
\pnode(2.538, 0.510){n127}
\pnode(-0.900, 0.330){n128}
\pnode(-2.123, -0.211){n129}
\pnode(2.648, 0.820){n130}
\pnode(-2.159, -0.218){n131}
\pnode(-3.499, 0.457){n132}
\pnode(2.644, 1.186){n133}
\pnode(2.674, -0.107){n134}
\pnode(2.507, 0.652){n135}
\pnode(2.648, 0.319){n136}
\pnode(-0.807, 0.195){n137}
\pnode(-1.949, 0.041){n138}
\pnode(0.070, -0.703){n139}
\pnode(-2.918, 0.780){n140}
\pnode(-1.922, 0.409){n141}
\pnode(-0.642, 0.019){n142}
\pnode(-1.087, 0.075){n143}
\pnode(-1.169, -0.165){n144}
\pnode(2.311, 0.398){n145}
\pnode(-0.463, -0.670){n146}
\pnode(-1.944, 0.187){n147}
\pnode(-3.489, 1.172){n148}
\pnode(2.590, 0.236){n149}
\psset{linewidth=0.01pt,linecolor=lightgray}
\ncline[]{n0}{n2}
\ncline[]{n0}{n4}
\ncline[]{n0}{n7}
\ncline[]{n0}{n12}
\ncline[]{n0}{n13}
\ncline[]{n0}{n17}
\ncline[]{n0}{n20}
\ncline[]{n0}{n23}
\ncline[]{n0}{n24}
\ncline[]{n0}{n26}
\ncline[]{n0}{n32}
\ncline[]{n0}{n33}
\ncline[]{n0}{n34}
\ncline[]{n0}{n35}
\ncline[]{n0}{n36}
\ncline[]{n0}{n38}
\ncline[]{n0}{n39}
\ncline[]{n0}{n40}
\ncline[]{n0}{n41}
\ncline[]{n0}{n44}
\ncline[]{n0}{n47}
\ncline[]{n0}{n50}
\ncline[]{n0}{n51}
\ncline[]{n0}{n58}
\ncline[]{n0}{n59}
\ncline[]{n0}{n64}
\ncline[]{n0}{n65}
\ncline[]{n0}{n66}
\ncline[]{n0}{n72}
\ncline[]{n0}{n77}
\ncline[]{n0}{n79}
\ncline[]{n0}{n81}
\ncline[]{n0}{n82}
\ncline[]{n0}{n85}
\ncline[]{n0}{n86}
\ncline[]{n0}{n89}
\ncline[]{n0}{n93}
\ncline[]{n0}{n94}
\ncline[]{n0}{n96}
\ncline[]{n0}{n98}
\ncline[]{n0}{n99}
\ncline[]{n0}{n100}
\ncline[]{n0}{n101}
\ncline[]{n0}{n104}
\ncline[]{n0}{n110}
\ncline[]{n0}{n112}
\ncline[]{n0}{n114}
\ncline[]{n0}{n117}
\ncline[]{n0}{n119}
\ncline[]{n0}{n120}
\ncline[]{n0}{n122}
\ncline[]{n0}{n123}
\ncline[]{n0}{n126}
\ncline[]{n0}{n128}
\ncline[]{n0}{n137}
\ncline[]{n0}{n139}
\ncline[]{n0}{n142}
\ncline[]{n0}{n143}
\ncline[]{n0}{n144}
\ncline[]{n0}{n146}
\ncline[]{n1}{n6}
\ncline[]{n1}{n8}
\ncline[]{n1}{n9}
\ncline[]{n1}{n19}
\ncline[]{n1}{n21}
\ncline[]{n1}{n28}
\ncline[]{n1}{n29}
\ncline[]{n1}{n42}
\ncline[]{n1}{n43}
\ncline[]{n1}{n46}
\ncline[]{n1}{n53}
\ncline[]{n1}{n55}
\ncline[]{n1}{n56}
\ncline[]{n1}{n57}
\ncline[]{n1}{n60}
\ncline[]{n1}{n61}
\ncline[]{n1}{n70}
\ncline[]{n1}{n71}
\ncline[]{n1}{n80}
\ncline[]{n1}{n83}
\ncline[]{n1}{n84}
\ncline[]{n1}{n87}
\ncline[]{n1}{n92}
\ncline[]{n1}{n95}
\ncline[]{n1}{n97}
\ncline[]{n1}{n102}
\ncline[]{n1}{n103}
\ncline[]{n1}{n107}
\ncline[]{n1}{n116}
\ncline[]{n1}{n118}
\ncline[]{n1}{n129}
\ncline[]{n1}{n131}
\ncline[]{n1}{n132}
\ncline[]{n1}{n138}
\ncline[]{n1}{n140}
\ncline[]{n1}{n141}
\ncline[]{n1}{n147}
\ncline[]{n1}{n148}
\ncline[]{n2}{n4}
\ncline[]{n2}{n7}
\ncline[]{n2}{n12}
\ncline[]{n2}{n13}
\ncline[]{n2}{n17}
\ncline[]{n2}{n20}
\ncline[]{n2}{n23}
\ncline[]{n2}{n24}
\ncline[]{n2}{n26}
\ncline[]{n2}{n32}
\ncline[]{n2}{n33}
\ncline[]{n2}{n34}
\ncline[]{n2}{n35}
\ncline[]{n2}{n36}
\ncline[]{n2}{n38}
\ncline[]{n2}{n39}
\ncline[]{n2}{n40}
\ncline[]{n2}{n41}
\ncline[]{n2}{n44}
\ncline[]{n2}{n47}
\ncline[]{n2}{n50}
\ncline[]{n2}{n51}
\ncline[]{n2}{n58}
\ncline[]{n2}{n59}
\ncline[]{n2}{n64}
\ncline[]{n2}{n65}
\ncline[]{n2}{n66}
\ncline[]{n2}{n72}
\ncline[]{n2}{n77}
\ncline[]{n2}{n79}
\ncline[]{n2}{n81}
\ncline[]{n2}{n82}
\ncline[]{n2}{n85}
\ncline[]{n2}{n86}
\ncline[]{n2}{n89}
\ncline[]{n2}{n93}
\ncline[]{n2}{n94}
\ncline[]{n2}{n96}
\ncline[]{n2}{n98}
\ncline[]{n2}{n99}
\ncline[]{n2}{n100}
\ncline[]{n2}{n101}
\ncline[]{n2}{n104}
\ncline[]{n2}{n110}
\ncline[]{n2}{n112}
\ncline[]{n2}{n114}
\ncline[]{n2}{n117}
\ncline[]{n2}{n119}
\ncline[]{n2}{n120}
\ncline[]{n2}{n122}
\ncline[]{n2}{n123}
\ncline[]{n2}{n126}
\ncline[]{n2}{n128}
\ncline[]{n2}{n137}
\ncline[]{n2}{n139}
\ncline[]{n2}{n142}
\ncline[]{n2}{n143}
\ncline[]{n2}{n144}
\ncline[]{n2}{n146}
\ncline[]{n3}{n5}
\ncline[]{n3}{n10}
\ncline[]{n3}{n11}
\ncline[]{n3}{n14}
\ncline[]{n3}{n15}
\ncline[]{n3}{n16}
\ncline[]{n3}{n18}
\ncline[]{n3}{n22}
\ncline[]{n3}{n25}
\ncline[]{n3}{n27}
\ncline[]{n3}{n30}
\ncline[]{n3}{n31}
\ncline[]{n3}{n37}
\ncline[]{n3}{n45}
\ncline[]{n3}{n48}
\ncline[]{n3}{n49}
\ncline[]{n3}{n52}
\ncline[]{n3}{n54}
\ncline[]{n3}{n62}
\ncline[]{n3}{n63}
\ncline[]{n3}{n67}
\ncline[]{n3}{n68}
\ncline[]{n3}{n69}
\ncline[]{n3}{n73}
\ncline[]{n3}{n74}
\ncline[]{n3}{n75}
\ncline[]{n3}{n76}
\ncline[]{n3}{n78}
\ncline[]{n3}{n88}
\ncline[]{n3}{n90}
\ncline[]{n3}{n91}
\ncline[]{n3}{n105}
\ncline[]{n3}{n106}
\ncline[]{n3}{n108}
\ncline[]{n3}{n109}
\ncline[]{n3}{n111}
\ncline[]{n3}{n113}
\ncline[]{n3}{n115}
\ncline[]{n3}{n121}
\ncline[]{n3}{n124}
\ncline[]{n3}{n125}
\ncline[]{n3}{n127}
\ncline[]{n3}{n130}
\ncline[]{n3}{n133}
\ncline[]{n3}{n134}
\ncline[]{n3}{n135}
\ncline[]{n3}{n136}
\ncline[]{n3}{n145}
\ncline[]{n3}{n149}
\ncline[]{n4}{n7}
\ncline[]{n4}{n12}
\ncline[]{n4}{n13}
\ncline[]{n4}{n17}
\ncline[]{n4}{n20}
\ncline[]{n4}{n23}
\ncline[]{n4}{n24}
\ncline[]{n4}{n26}
\ncline[]{n4}{n32}
\ncline[]{n4}{n33}
\ncline[]{n4}{n34}
\ncline[]{n4}{n35}
\ncline[]{n4}{n36}
\ncline[]{n4}{n38}
\ncline[]{n4}{n39}
\ncline[]{n4}{n40}
\ncline[]{n4}{n41}
\ncline[]{n4}{n44}
\ncline[]{n4}{n47}
\ncline[]{n4}{n50}
\ncline[]{n4}{n51}
\ncline[]{n4}{n58}
\ncline[]{n4}{n59}
\ncline[]{n4}{n64}
\ncline[]{n4}{n65}
\ncline[]{n4}{n66}
\ncline[]{n4}{n72}
\ncline[]{n4}{n77}
\ncline[]{n4}{n79}
\ncline[]{n4}{n81}
\ncline[]{n4}{n82}
\ncline[]{n4}{n85}
\ncline[]{n4}{n86}
\ncline[]{n4}{n89}
\ncline[]{n4}{n93}
\ncline[]{n4}{n94}
\ncline[]{n4}{n96}
\ncline[]{n4}{n98}
\ncline[]{n4}{n99}
\ncline[]{n4}{n100}
\ncline[]{n4}{n101}
\ncline[]{n4}{n104}
\ncline[]{n4}{n110}
\ncline[]{n4}{n112}
\ncline[]{n4}{n114}
\ncline[]{n4}{n117}
\ncline[]{n4}{n119}
\ncline[]{n4}{n120}
\ncline[]{n4}{n122}
\ncline[]{n4}{n123}
\ncline[]{n4}{n126}
\ncline[]{n4}{n128}
\ncline[]{n4}{n137}
\ncline[]{n4}{n139}
\ncline[]{n4}{n142}
\ncline[]{n4}{n143}
\ncline[]{n4}{n144}
\ncline[]{n4}{n146}
\ncline[]{n5}{n10}
\ncline[]{n5}{n11}
\ncline[]{n5}{n14}
\ncline[]{n5}{n15}
\ncline[]{n5}{n16}
\ncline[]{n5}{n18}
\ncline[]{n5}{n22}
\ncline[]{n5}{n25}
\ncline[]{n5}{n27}
\ncline[]{n5}{n30}
\ncline[]{n5}{n31}
\ncline[]{n5}{n37}
\ncline[]{n5}{n45}
\ncline[]{n5}{n48}
\ncline[]{n5}{n49}
\ncline[]{n5}{n52}
\ncline[]{n5}{n54}
\ncline[]{n5}{n62}
\ncline[]{n5}{n63}
\ncline[]{n5}{n67}
\ncline[]{n5}{n68}
\ncline[]{n5}{n69}
\ncline[]{n5}{n73}
\ncline[]{n5}{n74}
\ncline[]{n5}{n75}
\ncline[]{n5}{n76}
\ncline[]{n5}{n78}
\ncline[]{n5}{n88}
\ncline[]{n5}{n90}
\ncline[]{n5}{n91}
\ncline[]{n5}{n105}
\ncline[]{n5}{n106}
\ncline[]{n5}{n108}
\ncline[]{n5}{n109}
\ncline[]{n5}{n111}
\ncline[]{n5}{n113}
\ncline[]{n5}{n115}
\ncline[]{n5}{n121}
\ncline[]{n5}{n124}
\ncline[]{n5}{n125}
\ncline[]{n5}{n127}
\ncline[]{n5}{n130}
\ncline[]{n5}{n133}
\ncline[]{n5}{n134}
\ncline[]{n5}{n135}
\ncline[]{n5}{n136}
\ncline[]{n5}{n145}
\ncline[]{n5}{n149}
\ncline[]{n6}{n8}
\ncline[]{n6}{n9}
\ncline[]{n6}{n19}
\ncline[]{n6}{n21}
\ncline[]{n6}{n28}
\ncline[]{n6}{n29}
\ncline[]{n6}{n42}
\ncline[]{n6}{n43}
\ncline[]{n6}{n46}
\ncline[]{n6}{n53}
\ncline[]{n6}{n55}
\ncline[]{n6}{n56}
\ncline[]{n6}{n57}
\ncline[]{n6}{n60}
\ncline[]{n6}{n61}
\ncline[]{n6}{n70}
\ncline[]{n6}{n71}
\ncline[]{n6}{n80}
\ncline[]{n6}{n83}
\ncline[]{n6}{n84}
\ncline[]{n6}{n87}
\ncline[]{n6}{n92}
\ncline[]{n6}{n95}
\ncline[]{n6}{n97}
\ncline[]{n6}{n102}
\ncline[]{n6}{n103}
\ncline[]{n6}{n107}
\ncline[]{n6}{n116}
\ncline[]{n6}{n118}
\ncline[]{n6}{n129}
\ncline[]{n6}{n131}
\ncline[]{n6}{n132}
\ncline[]{n6}{n138}
\ncline[]{n6}{n140}
\ncline[]{n6}{n141}
\ncline[]{n6}{n147}
\ncline[]{n6}{n148}
\ncline[]{n7}{n12}
\ncline[]{n7}{n13}
\ncline[]{n7}{n17}
\ncline[]{n7}{n20}
\ncline[]{n7}{n23}
\ncline[]{n7}{n24}
\ncline[]{n7}{n26}
\ncline[]{n7}{n32}
\ncline[]{n7}{n33}
\ncline[]{n7}{n34}
\ncline[]{n7}{n35}
\ncline[]{n7}{n36}
\ncline[]{n7}{n38}
\ncline[]{n7}{n39}
\ncline[]{n7}{n40}
\ncline[]{n7}{n41}
\ncline[]{n7}{n44}
\ncline[]{n7}{n47}
\ncline[]{n7}{n50}
\ncline[]{n7}{n51}
\ncline[]{n7}{n58}
\ncline[]{n7}{n59}
\ncline[]{n7}{n64}
\ncline[]{n7}{n65}
\ncline[]{n7}{n66}
\ncline[]{n7}{n72}
\ncline[]{n7}{n77}
\ncline[]{n7}{n79}
\ncline[]{n7}{n81}
\ncline[]{n7}{n82}
\ncline[]{n7}{n85}
\ncline[]{n7}{n86}
\ncline[]{n7}{n89}
\ncline[]{n7}{n93}
\ncline[]{n7}{n94}
\ncline[]{n7}{n96}
\ncline[]{n7}{n98}
\ncline[]{n7}{n99}
\ncline[]{n7}{n100}
\ncline[]{n7}{n101}
\ncline[]{n7}{n104}
\ncline[]{n7}{n110}
\ncline[]{n7}{n112}
\ncline[]{n7}{n114}
\ncline[]{n7}{n117}
\ncline[]{n7}{n119}
\ncline[]{n7}{n120}
\ncline[]{n7}{n122}
\ncline[]{n7}{n123}
\ncline[]{n7}{n126}
\ncline[]{n7}{n128}
\ncline[]{n7}{n137}
\ncline[]{n7}{n139}
\ncline[]{n7}{n142}
\ncline[]{n7}{n143}
\ncline[]{n7}{n144}
\ncline[]{n7}{n146}
\ncline[]{n8}{n9}
\ncline[]{n8}{n19}
\ncline[]{n8}{n21}
\ncline[]{n8}{n28}
\ncline[]{n8}{n29}
\ncline[]{n8}{n42}
\ncline[]{n8}{n43}
\ncline[]{n8}{n46}
\ncline[]{n8}{n53}
\ncline[]{n8}{n55}
\ncline[]{n8}{n56}
\ncline[]{n8}{n57}
\ncline[]{n8}{n60}
\ncline[]{n8}{n61}
\ncline[]{n8}{n70}
\ncline[]{n8}{n71}
\ncline[]{n8}{n80}
\ncline[]{n8}{n83}
\ncline[]{n8}{n84}
\ncline[]{n8}{n87}
\ncline[]{n8}{n92}
\ncline[]{n8}{n95}
\ncline[]{n8}{n97}
\ncline[]{n8}{n102}
\ncline[]{n8}{n103}
\ncline[]{n8}{n107}
\ncline[]{n8}{n116}
\ncline[]{n8}{n118}
\ncline[]{n8}{n129}
\ncline[]{n8}{n131}
\ncline[]{n8}{n132}
\ncline[]{n8}{n138}
\ncline[]{n8}{n140}
\ncline[]{n8}{n141}
\ncline[]{n8}{n147}
\ncline[]{n8}{n148}
\ncline[]{n9}{n19}
\ncline[]{n9}{n21}
\ncline[]{n9}{n28}
\ncline[]{n9}{n29}
\ncline[]{n9}{n42}
\ncline[]{n9}{n43}
\ncline[]{n9}{n46}
\ncline[]{n9}{n53}
\ncline[]{n9}{n55}
\ncline[]{n9}{n56}
\ncline[]{n9}{n57}
\ncline[]{n9}{n60}
\ncline[]{n9}{n61}
\ncline[]{n9}{n70}
\ncline[]{n9}{n71}
\ncline[]{n9}{n80}
\ncline[]{n9}{n83}
\ncline[]{n9}{n84}
\ncline[]{n9}{n87}
\ncline[]{n9}{n92}
\ncline[]{n9}{n95}
\ncline[]{n9}{n97}
\ncline[]{n9}{n102}
\ncline[]{n9}{n103}
\ncline[]{n9}{n107}
\ncline[]{n9}{n116}
\ncline[]{n9}{n118}
\ncline[]{n9}{n129}
\ncline[]{n9}{n131}
\ncline[]{n9}{n132}
\ncline[]{n9}{n138}
\ncline[]{n9}{n140}
\ncline[]{n9}{n141}
\ncline[]{n9}{n147}
\ncline[]{n9}{n148}
\ncline[]{n10}{n11}
\ncline[]{n10}{n14}
\ncline[]{n10}{n15}
\ncline[]{n10}{n16}
\ncline[]{n10}{n18}
\ncline[]{n10}{n22}
\ncline[]{n10}{n25}
\ncline[]{n10}{n27}
\ncline[]{n10}{n30}
\ncline[]{n10}{n31}
\ncline[]{n10}{n37}
\ncline[]{n10}{n45}
\ncline[]{n10}{n48}
\ncline[]{n10}{n49}
\ncline[]{n10}{n52}
\ncline[]{n10}{n54}
\ncline[]{n10}{n62}
\ncline[]{n10}{n63}
\ncline[]{n10}{n67}
\ncline[]{n10}{n68}
\ncline[]{n10}{n69}
\ncline[]{n10}{n73}
\ncline[]{n10}{n74}
\ncline[]{n10}{n75}
\ncline[]{n10}{n76}
\ncline[]{n10}{n78}
\ncline[]{n10}{n88}
\ncline[]{n10}{n90}
\ncline[]{n10}{n91}
\ncline[]{n10}{n105}
\ncline[]{n10}{n106}
\ncline[]{n10}{n108}
\ncline[]{n10}{n109}
\ncline[]{n10}{n111}
\ncline[]{n10}{n113}
\ncline[]{n10}{n115}
\ncline[]{n10}{n121}
\ncline[]{n10}{n124}
\ncline[]{n10}{n125}
\ncline[]{n10}{n127}
\ncline[]{n10}{n130}
\ncline[]{n10}{n133}
\ncline[]{n10}{n134}
\ncline[]{n10}{n135}
\ncline[]{n10}{n136}
\ncline[]{n10}{n145}
\ncline[]{n10}{n149}
\ncline[]{n11}{n14}
\ncline[]{n11}{n15}
\ncline[]{n11}{n16}
\ncline[]{n11}{n18}
\ncline[]{n11}{n22}
\ncline[]{n11}{n25}
\ncline[]{n11}{n27}
\ncline[]{n11}{n30}
\ncline[]{n11}{n31}
\ncline[]{n11}{n37}
\ncline[]{n11}{n45}
\ncline[]{n11}{n48}
\ncline[]{n11}{n49}
\ncline[]{n11}{n52}
\ncline[]{n11}{n54}
\ncline[]{n11}{n62}
\ncline[]{n11}{n63}
\ncline[]{n11}{n67}
\ncline[]{n11}{n68}
\ncline[]{n11}{n69}
\ncline[]{n11}{n73}
\ncline[]{n11}{n74}
\ncline[]{n11}{n75}
\ncline[]{n11}{n76}
\ncline[]{n11}{n78}
\ncline[]{n11}{n88}
\ncline[]{n11}{n90}
\ncline[]{n11}{n91}
\ncline[]{n11}{n105}
\ncline[]{n11}{n106}
\ncline[]{n11}{n108}
\ncline[]{n11}{n109}
\ncline[]{n11}{n111}
\ncline[]{n11}{n113}
\ncline[]{n11}{n115}
\ncline[]{n11}{n121}
\ncline[]{n11}{n124}
\ncline[]{n11}{n125}
\ncline[]{n11}{n127}
\ncline[]{n11}{n130}
\ncline[]{n11}{n133}
\ncline[]{n11}{n134}
\ncline[]{n11}{n135}
\ncline[]{n11}{n136}
\ncline[]{n11}{n145}
\ncline[]{n11}{n149}
\ncline[]{n12}{n13}
\ncline[]{n12}{n17}
\ncline[]{n12}{n20}
\ncline[]{n12}{n23}
\ncline[]{n12}{n24}
\ncline[]{n12}{n26}
\ncline[]{n12}{n32}
\ncline[]{n12}{n33}
\ncline[]{n12}{n34}
\ncline[]{n12}{n35}
\ncline[]{n12}{n36}
\ncline[]{n12}{n38}
\ncline[]{n12}{n39}
\ncline[]{n12}{n40}
\ncline[]{n12}{n41}
\ncline[]{n12}{n44}
\ncline[]{n12}{n47}
\ncline[]{n12}{n50}
\ncline[]{n12}{n51}
\ncline[]{n12}{n58}
\ncline[]{n12}{n59}
\ncline[]{n12}{n64}
\ncline[]{n12}{n65}
\ncline[]{n12}{n66}
\ncline[]{n12}{n72}
\ncline[]{n12}{n77}
\ncline[]{n12}{n79}
\ncline[]{n12}{n81}
\ncline[]{n12}{n82}
\ncline[]{n12}{n85}
\ncline[]{n12}{n86}
\ncline[]{n12}{n89}
\ncline[]{n12}{n93}
\ncline[]{n12}{n94}
\ncline[]{n12}{n96}
\ncline[]{n12}{n98}
\ncline[]{n12}{n99}
\ncline[]{n12}{n100}
\ncline[]{n12}{n101}
\ncline[]{n12}{n104}
\ncline[]{n12}{n110}
\ncline[]{n12}{n112}
\ncline[]{n12}{n114}
\ncline[]{n12}{n117}
\ncline[]{n12}{n119}
\ncline[]{n12}{n120}
\ncline[]{n12}{n122}
\ncline[]{n12}{n123}
\ncline[]{n12}{n126}
\ncline[]{n12}{n128}
\ncline[]{n12}{n137}
\ncline[]{n12}{n139}
\ncline[]{n12}{n142}
\ncline[]{n12}{n143}
\ncline[]{n12}{n144}
\ncline[]{n12}{n146}
\ncline[]{n13}{n17}
\ncline[]{n13}{n20}
\ncline[]{n13}{n23}
\ncline[]{n13}{n24}
\ncline[]{n13}{n26}
\ncline[]{n13}{n32}
\ncline[]{n13}{n33}
\ncline[]{n13}{n34}
\ncline[]{n13}{n35}
\ncline[]{n13}{n36}
\ncline[]{n13}{n38}
\ncline[]{n13}{n39}
\ncline[]{n13}{n40}
\ncline[]{n13}{n41}
\ncline[]{n13}{n44}
\ncline[]{n13}{n47}
\ncline[]{n13}{n50}
\ncline[]{n13}{n51}
\ncline[]{n13}{n58}
\ncline[]{n13}{n59}
\ncline[]{n13}{n64}
\ncline[]{n13}{n65}
\ncline[]{n13}{n66}
\ncline[]{n13}{n72}
\ncline[]{n13}{n77}
\ncline[]{n13}{n79}
\ncline[]{n13}{n81}
\ncline[]{n13}{n82}
\ncline[]{n13}{n85}
\ncline[]{n13}{n86}
\ncline[]{n13}{n89}
\ncline[]{n13}{n93}
\ncline[]{n13}{n94}
\ncline[]{n13}{n96}
\ncline[]{n13}{n98}
\ncline[]{n13}{n99}
\ncline[]{n13}{n100}
\ncline[]{n13}{n101}
\ncline[]{n13}{n104}
\ncline[]{n13}{n110}
\ncline[]{n13}{n112}
\ncline[]{n13}{n114}
\ncline[]{n13}{n117}
\ncline[]{n13}{n119}
\ncline[]{n13}{n120}
\ncline[]{n13}{n122}
\ncline[]{n13}{n123}
\ncline[]{n13}{n126}
\ncline[]{n13}{n128}
\ncline[]{n13}{n137}
\ncline[]{n13}{n139}
\ncline[]{n13}{n142}
\ncline[]{n13}{n143}
\ncline[]{n13}{n144}
\ncline[]{n13}{n146}
\ncline[]{n14}{n15}
\ncline[]{n14}{n16}
\ncline[]{n14}{n18}
\ncline[]{n14}{n22}
\ncline[]{n14}{n25}
\ncline[]{n14}{n27}
\ncline[]{n14}{n30}
\ncline[]{n14}{n31}
\ncline[]{n14}{n37}
\ncline[]{n14}{n45}
\ncline[]{n14}{n48}
\ncline[]{n14}{n49}
\ncline[]{n14}{n52}
\ncline[]{n14}{n54}
\ncline[]{n14}{n62}
\ncline[]{n14}{n63}
\ncline[]{n14}{n67}
\ncline[]{n14}{n68}
\ncline[]{n14}{n69}
\ncline[]{n14}{n73}
\ncline[]{n14}{n74}
\ncline[]{n14}{n75}
\ncline[]{n14}{n76}
\ncline[]{n14}{n78}
\ncline[]{n14}{n88}
\ncline[]{n14}{n90}
\ncline[]{n14}{n91}
\ncline[]{n14}{n105}
\ncline[]{n14}{n106}
\ncline[]{n14}{n108}
\ncline[]{n14}{n109}
\ncline[]{n14}{n111}
\ncline[]{n14}{n113}
\ncline[]{n14}{n115}
\ncline[]{n14}{n121}
\ncline[]{n14}{n124}
\ncline[]{n14}{n125}
\ncline[]{n14}{n127}
\ncline[]{n14}{n130}
\ncline[]{n14}{n133}
\ncline[]{n14}{n134}
\ncline[]{n14}{n135}
\ncline[]{n14}{n136}
\ncline[]{n14}{n145}
\ncline[]{n14}{n149}
\ncline[]{n15}{n16}
\ncline[]{n15}{n18}
\ncline[]{n15}{n22}
\ncline[]{n15}{n25}
\ncline[]{n15}{n27}
\ncline[]{n15}{n30}
\ncline[]{n15}{n31}
\ncline[]{n15}{n37}
\ncline[]{n15}{n45}
\ncline[]{n15}{n48}
\ncline[]{n15}{n49}
\ncline[]{n15}{n52}
\ncline[]{n15}{n54}
\ncline[]{n15}{n62}
\ncline[]{n15}{n63}
\ncline[]{n15}{n67}
\ncline[]{n15}{n68}
\ncline[]{n15}{n69}
\ncline[]{n15}{n73}
\ncline[]{n15}{n74}
\ncline[]{n15}{n75}
\ncline[]{n15}{n76}
\ncline[]{n15}{n78}
\ncline[]{n15}{n88}
\ncline[]{n15}{n90}
\ncline[]{n15}{n91}
\ncline[]{n15}{n105}
\ncline[]{n15}{n106}
\ncline[]{n15}{n108}
\ncline[]{n15}{n109}
\ncline[]{n15}{n111}
\ncline[]{n15}{n113}
\ncline[]{n15}{n115}
\ncline[]{n15}{n121}
\ncline[]{n15}{n124}
\ncline[]{n15}{n125}
\ncline[]{n15}{n127}
\ncline[]{n15}{n130}
\ncline[]{n15}{n133}
\ncline[]{n15}{n134}
\ncline[]{n15}{n135}
\ncline[]{n15}{n136}
\ncline[]{n15}{n145}
\ncline[]{n15}{n149}
\ncline[]{n16}{n18}
\ncline[]{n16}{n22}
\ncline[]{n16}{n25}
\ncline[]{n16}{n27}
\ncline[]{n16}{n30}
\ncline[]{n16}{n31}
\ncline[]{n16}{n37}
\ncline[]{n16}{n45}
\ncline[]{n16}{n48}
\ncline[]{n16}{n49}
\ncline[]{n16}{n52}
\ncline[]{n16}{n54}
\ncline[]{n16}{n62}
\ncline[]{n16}{n63}
\ncline[]{n16}{n67}
\ncline[]{n16}{n68}
\ncline[]{n16}{n69}
\ncline[]{n16}{n73}
\ncline[]{n16}{n74}
\ncline[]{n16}{n75}
\ncline[]{n16}{n76}
\ncline[]{n16}{n78}
\ncline[]{n16}{n88}
\ncline[]{n16}{n90}
\ncline[]{n16}{n91}
\ncline[]{n16}{n105}
\ncline[]{n16}{n106}
\ncline[]{n16}{n108}
\ncline[]{n16}{n109}
\ncline[]{n16}{n111}
\ncline[]{n16}{n113}
\ncline[]{n16}{n115}
\ncline[]{n16}{n121}
\ncline[]{n16}{n124}
\ncline[]{n16}{n125}
\ncline[]{n16}{n127}
\ncline[]{n16}{n130}
\ncline[]{n16}{n133}
\ncline[]{n16}{n134}
\ncline[]{n16}{n135}
\ncline[]{n16}{n136}
\ncline[]{n16}{n145}
\ncline[]{n16}{n149}
\ncline[]{n17}{n20}
\ncline[]{n17}{n23}
\ncline[]{n17}{n24}
\ncline[]{n17}{n26}
\ncline[]{n17}{n32}
\ncline[]{n17}{n33}
\ncline[]{n17}{n34}
\ncline[]{n17}{n35}
\ncline[]{n17}{n36}
\ncline[]{n17}{n38}
\ncline[]{n17}{n39}
\ncline[]{n17}{n40}
\ncline[]{n17}{n41}
\ncline[]{n17}{n44}
\ncline[]{n17}{n47}
\ncline[]{n17}{n50}
\ncline[]{n17}{n51}
\ncline[]{n17}{n58}
\ncline[]{n17}{n59}
\ncline[]{n17}{n64}
\ncline[]{n17}{n65}
\ncline[]{n17}{n66}
\ncline[]{n17}{n72}
\ncline[]{n17}{n77}
\ncline[]{n17}{n79}
\ncline[]{n17}{n81}
\ncline[]{n17}{n82}
\ncline[]{n17}{n85}
\ncline[]{n17}{n86}
\ncline[]{n17}{n89}
\ncline[]{n17}{n93}
\ncline[]{n17}{n94}
\ncline[]{n17}{n96}
\ncline[]{n17}{n98}
\ncline[]{n17}{n99}
\ncline[]{n17}{n100}
\ncline[]{n17}{n101}
\ncline[]{n17}{n104}
\ncline[]{n17}{n110}
\ncline[]{n17}{n112}
\ncline[]{n17}{n114}
\ncline[]{n17}{n117}
\ncline[]{n17}{n119}
\ncline[]{n17}{n120}
\ncline[]{n17}{n122}
\ncline[]{n17}{n123}
\ncline[]{n17}{n126}
\ncline[]{n17}{n128}
\ncline[]{n17}{n137}
\ncline[]{n17}{n139}
\ncline[]{n17}{n142}
\ncline[]{n17}{n143}
\ncline[]{n17}{n144}
\ncline[]{n17}{n146}
\ncline[]{n18}{n22}
\ncline[]{n18}{n25}
\ncline[]{n18}{n27}
\ncline[]{n18}{n30}
\ncline[]{n18}{n31}
\ncline[]{n18}{n37}
\ncline[]{n18}{n45}
\ncline[]{n18}{n48}
\ncline[]{n18}{n49}
\ncline[]{n18}{n52}
\ncline[]{n18}{n54}
\ncline[]{n18}{n62}
\ncline[]{n18}{n63}
\ncline[]{n18}{n67}
\ncline[]{n18}{n68}
\ncline[]{n18}{n69}
\ncline[]{n18}{n73}
\ncline[]{n18}{n74}
\ncline[]{n18}{n75}
\ncline[]{n18}{n76}
\ncline[]{n18}{n78}
\ncline[]{n18}{n88}
\ncline[]{n18}{n90}
\ncline[]{n18}{n91}
\ncline[]{n18}{n105}
\ncline[]{n18}{n106}
\ncline[]{n18}{n108}
\ncline[]{n18}{n109}
\ncline[]{n18}{n111}
\ncline[]{n18}{n113}
\ncline[]{n18}{n115}
\ncline[]{n18}{n121}
\ncline[]{n18}{n124}
\ncline[]{n18}{n125}
\ncline[]{n18}{n127}
\ncline[]{n18}{n130}
\ncline[]{n18}{n133}
\ncline[]{n18}{n134}
\ncline[]{n18}{n135}
\ncline[]{n18}{n136}
\ncline[]{n18}{n145}
\ncline[]{n18}{n149}
\ncline[]{n19}{n21}
\ncline[]{n19}{n28}
\ncline[]{n19}{n29}
\ncline[]{n19}{n42}
\ncline[]{n19}{n43}
\ncline[]{n19}{n46}
\ncline[]{n19}{n53}
\ncline[]{n19}{n55}
\ncline[]{n19}{n56}
\ncline[]{n19}{n57}
\ncline[]{n19}{n60}
\ncline[]{n19}{n61}
\ncline[]{n19}{n70}
\ncline[]{n19}{n71}
\ncline[]{n19}{n80}
\ncline[]{n19}{n83}
\ncline[]{n19}{n84}
\ncline[]{n19}{n87}
\ncline[]{n19}{n92}
\ncline[]{n19}{n95}
\ncline[]{n19}{n97}
\ncline[]{n19}{n102}
\ncline[]{n19}{n103}
\ncline[]{n19}{n107}
\ncline[]{n19}{n116}
\ncline[]{n19}{n118}
\ncline[]{n19}{n129}
\ncline[]{n19}{n131}
\ncline[]{n19}{n132}
\ncline[]{n19}{n138}
\ncline[]{n19}{n140}
\ncline[]{n19}{n141}
\ncline[]{n19}{n147}
\ncline[]{n19}{n148}
\ncline[]{n20}{n23}
\ncline[]{n20}{n24}
\ncline[]{n20}{n26}
\ncline[]{n20}{n32}
\ncline[]{n20}{n33}
\ncline[]{n20}{n34}
\ncline[]{n20}{n35}
\ncline[]{n20}{n36}
\ncline[]{n20}{n38}
\ncline[]{n20}{n39}
\ncline[]{n20}{n40}
\ncline[]{n20}{n41}
\ncline[]{n20}{n44}
\ncline[]{n20}{n47}
\ncline[]{n20}{n50}
\ncline[]{n20}{n51}
\ncline[]{n20}{n58}
\ncline[]{n20}{n59}
\ncline[]{n20}{n64}
\ncline[]{n20}{n65}
\ncline[]{n20}{n66}
\ncline[]{n20}{n72}
\ncline[]{n20}{n77}
\ncline[]{n20}{n79}
\ncline[]{n20}{n81}
\ncline[]{n20}{n82}
\ncline[]{n20}{n85}
\ncline[]{n20}{n86}
\ncline[]{n20}{n89}
\ncline[]{n20}{n93}
\ncline[]{n20}{n94}
\ncline[]{n20}{n96}
\ncline[]{n20}{n98}
\ncline[]{n20}{n99}
\ncline[]{n20}{n100}
\ncline[]{n20}{n101}
\ncline[]{n20}{n104}
\ncline[]{n20}{n110}
\ncline[]{n20}{n112}
\ncline[]{n20}{n114}
\ncline[]{n20}{n117}
\ncline[]{n20}{n119}
\ncline[]{n20}{n120}
\ncline[]{n20}{n122}
\ncline[]{n20}{n123}
\ncline[]{n20}{n126}
\ncline[]{n20}{n128}
\ncline[]{n20}{n137}
\ncline[]{n20}{n139}
\ncline[]{n20}{n142}
\ncline[]{n20}{n143}
\ncline[]{n20}{n144}
\ncline[]{n20}{n146}
\ncline[]{n21}{n28}
\ncline[]{n21}{n29}
\ncline[]{n21}{n42}
\ncline[]{n21}{n43}
\ncline[]{n21}{n46}
\ncline[]{n21}{n53}
\ncline[]{n21}{n55}
\ncline[]{n21}{n56}
\ncline[]{n21}{n57}
\ncline[]{n21}{n60}
\ncline[]{n21}{n61}
\ncline[]{n21}{n70}
\ncline[]{n21}{n71}
\ncline[]{n21}{n80}
\ncline[]{n21}{n83}
\ncline[]{n21}{n84}
\ncline[]{n21}{n87}
\ncline[]{n21}{n92}
\ncline[]{n21}{n95}
\ncline[]{n21}{n97}
\ncline[]{n21}{n102}
\ncline[]{n21}{n103}
\ncline[]{n21}{n107}
\ncline[]{n21}{n116}
\ncline[]{n21}{n118}
\ncline[]{n21}{n129}
\ncline[]{n21}{n131}
\ncline[]{n21}{n132}
\ncline[]{n21}{n138}
\ncline[]{n21}{n140}
\ncline[]{n21}{n141}
\ncline[]{n21}{n147}
\ncline[]{n21}{n148}
\ncline[]{n22}{n25}
\ncline[]{n22}{n27}
\ncline[]{n22}{n30}
\ncline[]{n22}{n31}
\ncline[]{n22}{n37}
\ncline[]{n22}{n45}
\ncline[]{n22}{n48}
\ncline[]{n22}{n49}
\ncline[]{n22}{n52}
\ncline[]{n22}{n54}
\ncline[]{n22}{n62}
\ncline[]{n22}{n63}
\ncline[]{n22}{n67}
\ncline[]{n22}{n68}
\ncline[]{n22}{n69}
\ncline[]{n22}{n73}
\ncline[]{n22}{n74}
\ncline[]{n22}{n75}
\ncline[]{n22}{n76}
\ncline[]{n22}{n78}
\ncline[]{n22}{n88}
\ncline[]{n22}{n90}
\ncline[]{n22}{n91}
\ncline[]{n22}{n105}
\ncline[]{n22}{n106}
\ncline[]{n22}{n108}
\ncline[]{n22}{n109}
\ncline[]{n22}{n111}
\ncline[]{n22}{n113}
\ncline[]{n22}{n115}
\ncline[]{n22}{n121}
\ncline[]{n22}{n124}
\ncline[]{n22}{n125}
\ncline[]{n22}{n127}
\ncline[]{n22}{n130}
\ncline[]{n22}{n133}
\ncline[]{n22}{n134}
\ncline[]{n22}{n135}
\ncline[]{n22}{n136}
\ncline[]{n22}{n145}
\ncline[]{n22}{n149}
\ncline[]{n23}{n24}
\ncline[]{n23}{n26}
\ncline[]{n23}{n32}
\ncline[]{n23}{n33}
\ncline[]{n23}{n34}
\ncline[]{n23}{n35}
\ncline[]{n23}{n36}
\ncline[]{n23}{n38}
\ncline[]{n23}{n39}
\ncline[]{n23}{n40}
\ncline[]{n23}{n41}
\ncline[]{n23}{n44}
\ncline[]{n23}{n47}
\ncline[]{n23}{n50}
\ncline[]{n23}{n51}
\ncline[]{n23}{n58}
\ncline[]{n23}{n59}
\ncline[]{n23}{n64}
\ncline[]{n23}{n65}
\ncline[]{n23}{n66}
\ncline[]{n23}{n72}
\ncline[]{n23}{n77}
\ncline[]{n23}{n79}
\ncline[]{n23}{n81}
\ncline[]{n23}{n82}
\ncline[]{n23}{n85}
\ncline[]{n23}{n86}
\ncline[]{n23}{n89}
\ncline[]{n23}{n93}
\ncline[]{n23}{n94}
\ncline[]{n23}{n96}
\ncline[]{n23}{n98}
\ncline[]{n23}{n99}
\ncline[]{n23}{n100}
\ncline[]{n23}{n101}
\ncline[]{n23}{n104}
\ncline[]{n23}{n110}
\ncline[]{n23}{n112}
\ncline[]{n23}{n114}
\ncline[]{n23}{n117}
\ncline[]{n23}{n119}
\ncline[]{n23}{n120}
\ncline[]{n23}{n122}
\ncline[]{n23}{n123}
\ncline[]{n23}{n126}
\ncline[]{n23}{n128}
\ncline[]{n23}{n137}
\ncline[]{n23}{n139}
\ncline[]{n23}{n142}
\ncline[]{n23}{n143}
\ncline[]{n23}{n144}
\ncline[]{n23}{n146}
\ncline[]{n24}{n26}
\ncline[]{n24}{n32}
\ncline[]{n24}{n33}
\ncline[]{n24}{n34}
\ncline[]{n24}{n35}
\ncline[]{n24}{n36}
\ncline[]{n24}{n38}
\ncline[]{n24}{n39}
\ncline[]{n24}{n40}
\ncline[]{n24}{n41}
\ncline[]{n24}{n44}
\ncline[]{n24}{n47}
\ncline[]{n24}{n50}
\ncline[]{n24}{n51}
\ncline[]{n24}{n58}
\ncline[]{n24}{n59}
\ncline[]{n24}{n64}
\ncline[]{n24}{n65}
\ncline[]{n24}{n66}
\ncline[]{n24}{n72}
\ncline[]{n24}{n77}
\ncline[]{n24}{n79}
\ncline[]{n24}{n81}
\ncline[]{n24}{n82}
\ncline[]{n24}{n85}
\ncline[]{n24}{n86}
\ncline[]{n24}{n89}
\ncline[]{n24}{n93}
\ncline[]{n24}{n94}
\ncline[]{n24}{n96}
\ncline[]{n24}{n98}
\ncline[]{n24}{n99}
\ncline[]{n24}{n100}
\ncline[]{n24}{n101}
\ncline[]{n24}{n104}
\ncline[]{n24}{n110}
\ncline[]{n24}{n112}
\ncline[]{n24}{n114}
\ncline[]{n24}{n117}
\ncline[]{n24}{n119}
\ncline[]{n24}{n120}
\ncline[]{n24}{n122}
\ncline[]{n24}{n123}
\ncline[]{n24}{n126}
\ncline[]{n24}{n128}
\ncline[]{n24}{n137}
\ncline[]{n24}{n139}
\ncline[]{n24}{n142}
\ncline[]{n24}{n143}
\ncline[]{n24}{n144}
\ncline[]{n24}{n146}
\ncline[]{n25}{n27}
\ncline[]{n25}{n30}
\ncline[]{n25}{n31}
\ncline[]{n25}{n37}
\ncline[]{n25}{n45}
\ncline[]{n25}{n48}
\ncline[]{n25}{n49}
\ncline[]{n25}{n52}
\ncline[]{n25}{n54}
\ncline[]{n25}{n62}
\ncline[]{n25}{n63}
\ncline[]{n25}{n67}
\ncline[]{n25}{n68}
\ncline[]{n25}{n69}
\ncline[]{n25}{n73}
\ncline[]{n25}{n74}
\ncline[]{n25}{n75}
\ncline[]{n25}{n76}
\ncline[]{n25}{n78}
\ncline[]{n25}{n88}
\ncline[]{n25}{n90}
\ncline[]{n25}{n91}
\ncline[]{n25}{n105}
\ncline[]{n25}{n106}
\ncline[]{n25}{n108}
\ncline[]{n25}{n109}
\ncline[]{n25}{n111}
\ncline[]{n25}{n113}
\ncline[]{n25}{n115}
\ncline[]{n25}{n121}
\ncline[]{n25}{n124}
\ncline[]{n25}{n125}
\ncline[]{n25}{n127}
\ncline[]{n25}{n130}
\ncline[]{n25}{n133}
\ncline[]{n25}{n134}
\ncline[]{n25}{n135}
\ncline[]{n25}{n136}
\ncline[]{n25}{n145}
\ncline[]{n25}{n149}
\ncline[]{n26}{n32}
\ncline[]{n26}{n33}
\ncline[]{n26}{n34}
\ncline[]{n26}{n35}
\ncline[]{n26}{n36}
\ncline[]{n26}{n38}
\ncline[]{n26}{n39}
\ncline[]{n26}{n40}
\ncline[]{n26}{n41}
\ncline[]{n26}{n44}
\ncline[]{n26}{n47}
\ncline[]{n26}{n50}
\ncline[]{n26}{n51}
\ncline[]{n26}{n58}
\ncline[]{n26}{n59}
\ncline[]{n26}{n64}
\ncline[]{n26}{n65}
\ncline[]{n26}{n66}
\ncline[]{n26}{n72}
\ncline[]{n26}{n77}
\ncline[]{n26}{n79}
\ncline[]{n26}{n81}
\ncline[]{n26}{n82}
\ncline[]{n26}{n85}
\ncline[]{n26}{n86}
\ncline[]{n26}{n89}
\ncline[]{n26}{n93}
\ncline[]{n26}{n94}
\ncline[]{n26}{n96}
\ncline[]{n26}{n98}
\ncline[]{n26}{n99}
\ncline[]{n26}{n100}
\ncline[]{n26}{n101}
\ncline[]{n26}{n104}
\ncline[]{n26}{n110}
\ncline[]{n26}{n112}
\ncline[]{n26}{n114}
\ncline[]{n26}{n117}
\ncline[]{n26}{n119}
\ncline[]{n26}{n120}
\ncline[]{n26}{n122}
\ncline[]{n26}{n123}
\ncline[]{n26}{n126}
\ncline[]{n26}{n128}
\ncline[]{n26}{n137}
\ncline[]{n26}{n139}
\ncline[]{n26}{n142}
\ncline[]{n26}{n143}
\ncline[]{n26}{n144}
\ncline[]{n26}{n146}
\ncline[]{n27}{n30}
\ncline[]{n27}{n31}
\ncline[]{n27}{n37}
\ncline[]{n27}{n45}
\ncline[]{n27}{n48}
\ncline[]{n27}{n49}
\ncline[]{n27}{n52}
\ncline[]{n27}{n54}
\ncline[]{n27}{n62}
\ncline[]{n27}{n63}
\ncline[]{n27}{n67}
\ncline[]{n27}{n68}
\ncline[]{n27}{n69}
\ncline[]{n27}{n73}
\ncline[]{n27}{n74}
\ncline[]{n27}{n75}
\ncline[]{n27}{n76}
\ncline[]{n27}{n78}
\ncline[]{n27}{n88}
\ncline[]{n27}{n90}
\ncline[]{n27}{n91}
\ncline[]{n27}{n105}
\ncline[]{n27}{n106}
\ncline[]{n27}{n108}
\ncline[]{n27}{n109}
\ncline[]{n27}{n111}
\ncline[]{n27}{n113}
\ncline[]{n27}{n115}
\ncline[]{n27}{n121}
\ncline[]{n27}{n124}
\ncline[]{n27}{n125}
\ncline[]{n27}{n127}
\ncline[]{n27}{n130}
\ncline[]{n27}{n133}
\ncline[]{n27}{n134}
\ncline[]{n27}{n135}
\ncline[]{n27}{n136}
\ncline[]{n27}{n145}
\ncline[]{n27}{n149}
\ncline[]{n28}{n29}
\ncline[]{n28}{n42}
\ncline[]{n28}{n43}
\ncline[]{n28}{n46}
\ncline[]{n28}{n53}
\ncline[]{n28}{n55}
\ncline[]{n28}{n56}
\ncline[]{n28}{n57}
\ncline[]{n28}{n60}
\ncline[]{n28}{n61}
\ncline[]{n28}{n70}
\ncline[]{n28}{n71}
\ncline[]{n28}{n80}
\ncline[]{n28}{n83}
\ncline[]{n28}{n84}
\ncline[]{n28}{n87}
\ncline[]{n28}{n92}
\ncline[]{n28}{n95}
\ncline[]{n28}{n97}
\ncline[]{n28}{n102}
\ncline[]{n28}{n103}
\ncline[]{n28}{n107}
\ncline[]{n28}{n116}
\ncline[]{n28}{n118}
\ncline[]{n28}{n129}
\ncline[]{n28}{n131}
\ncline[]{n28}{n132}
\ncline[]{n28}{n138}
\ncline[]{n28}{n140}
\ncline[]{n28}{n141}
\ncline[]{n28}{n147}
\ncline[]{n28}{n148}
\ncline[]{n29}{n42}
\ncline[]{n29}{n43}
\ncline[]{n29}{n46}
\ncline[]{n29}{n53}
\ncline[]{n29}{n55}
\ncline[]{n29}{n56}
\ncline[]{n29}{n57}
\ncline[]{n29}{n60}
\ncline[]{n29}{n61}
\ncline[]{n29}{n70}
\ncline[]{n29}{n71}
\ncline[]{n29}{n80}
\ncline[]{n29}{n83}
\ncline[]{n29}{n84}
\ncline[]{n29}{n87}
\ncline[]{n29}{n92}
\ncline[]{n29}{n95}
\ncline[]{n29}{n97}
\ncline[]{n29}{n102}
\ncline[]{n29}{n103}
\ncline[]{n29}{n107}
\ncline[]{n29}{n116}
\ncline[]{n29}{n118}
\ncline[]{n29}{n129}
\ncline[]{n29}{n131}
\ncline[]{n29}{n132}
\ncline[]{n29}{n138}
\ncline[]{n29}{n140}
\ncline[]{n29}{n141}
\ncline[]{n29}{n147}
\ncline[]{n29}{n148}
\ncline[]{n30}{n31}
\ncline[]{n30}{n37}
\ncline[]{n30}{n45}
\ncline[]{n30}{n48}
\ncline[]{n30}{n49}
\ncline[]{n30}{n52}
\ncline[]{n30}{n54}
\ncline[]{n30}{n62}
\ncline[]{n30}{n63}
\ncline[]{n30}{n67}
\ncline[]{n30}{n68}
\ncline[]{n30}{n69}
\ncline[]{n30}{n73}
\ncline[]{n30}{n74}
\ncline[]{n30}{n75}
\ncline[]{n30}{n76}
\ncline[]{n30}{n78}
\ncline[]{n30}{n88}
\ncline[]{n30}{n90}
\ncline[]{n30}{n91}
\ncline[]{n30}{n105}
\ncline[]{n30}{n106}
\ncline[]{n30}{n108}
\ncline[]{n30}{n109}
\ncline[]{n30}{n111}
\ncline[]{n30}{n113}
\ncline[]{n30}{n115}
\ncline[]{n30}{n121}
\ncline[]{n30}{n124}
\ncline[]{n30}{n125}
\ncline[]{n30}{n127}
\ncline[]{n30}{n130}
\ncline[]{n30}{n133}
\ncline[]{n30}{n134}
\ncline[]{n30}{n135}
\ncline[]{n30}{n136}
\ncline[]{n30}{n145}
\ncline[]{n30}{n149}
\ncline[]{n31}{n37}
\ncline[]{n31}{n45}
\ncline[]{n31}{n48}
\ncline[]{n31}{n49}
\ncline[]{n31}{n52}
\ncline[]{n31}{n54}
\ncline[]{n31}{n62}
\ncline[]{n31}{n63}
\ncline[]{n31}{n67}
\ncline[]{n31}{n68}
\ncline[]{n31}{n69}
\ncline[]{n31}{n73}
\ncline[]{n31}{n74}
\ncline[]{n31}{n75}
\ncline[]{n31}{n76}
\ncline[]{n31}{n78}
\ncline[]{n31}{n88}
\ncline[]{n31}{n90}
\ncline[]{n31}{n91}
\ncline[]{n31}{n105}
\ncline[]{n31}{n106}
\ncline[]{n31}{n108}
\ncline[]{n31}{n109}
\ncline[]{n31}{n111}
\ncline[]{n31}{n113}
\ncline[]{n31}{n115}
\ncline[]{n31}{n121}
\ncline[]{n31}{n124}
\ncline[]{n31}{n125}
\ncline[]{n31}{n127}
\ncline[]{n31}{n130}
\ncline[]{n31}{n133}
\ncline[]{n31}{n134}
\ncline[]{n31}{n135}
\ncline[]{n31}{n136}
\ncline[]{n31}{n145}
\ncline[]{n31}{n149}
\ncline[]{n32}{n33}
\ncline[]{n32}{n34}
\ncline[]{n32}{n35}
\ncline[]{n32}{n36}
\ncline[]{n32}{n38}
\ncline[]{n32}{n39}
\ncline[]{n32}{n40}
\ncline[]{n32}{n41}
\ncline[]{n32}{n44}
\ncline[]{n32}{n47}
\ncline[]{n32}{n50}
\ncline[]{n32}{n51}
\ncline[]{n32}{n58}
\ncline[]{n32}{n59}
\ncline[]{n32}{n64}
\ncline[]{n32}{n65}
\ncline[]{n32}{n66}
\ncline[]{n32}{n72}
\ncline[]{n32}{n77}
\ncline[]{n32}{n79}
\ncline[]{n32}{n81}
\ncline[]{n32}{n82}
\ncline[]{n32}{n85}
\ncline[]{n32}{n86}
\ncline[]{n32}{n89}
\ncline[]{n32}{n93}
\ncline[]{n32}{n94}
\ncline[]{n32}{n96}
\ncline[]{n32}{n98}
\ncline[]{n32}{n99}
\ncline[]{n32}{n100}
\ncline[]{n32}{n101}
\ncline[]{n32}{n104}
\ncline[]{n32}{n110}
\ncline[]{n32}{n112}
\ncline[]{n32}{n114}
\ncline[]{n32}{n117}
\ncline[]{n32}{n119}
\ncline[]{n32}{n120}
\ncline[]{n32}{n122}
\ncline[]{n32}{n123}
\ncline[]{n32}{n126}
\ncline[]{n32}{n128}
\ncline[]{n32}{n137}
\ncline[]{n32}{n139}
\ncline[]{n32}{n142}
\ncline[]{n32}{n143}
\ncline[]{n32}{n144}
\ncline[]{n32}{n146}
\ncline[]{n33}{n34}
\ncline[]{n33}{n35}
\ncline[]{n33}{n36}
\ncline[]{n33}{n38}
\ncline[]{n33}{n39}
\ncline[]{n33}{n40}
\ncline[]{n33}{n41}
\ncline[]{n33}{n44}
\ncline[]{n33}{n47}
\ncline[]{n33}{n50}
\ncline[]{n33}{n51}
\ncline[]{n33}{n58}
\ncline[]{n33}{n59}
\ncline[]{n33}{n64}
\ncline[]{n33}{n65}
\ncline[]{n33}{n66}
\ncline[]{n33}{n72}
\ncline[]{n33}{n77}
\ncline[]{n33}{n79}
\ncline[]{n33}{n81}
\ncline[]{n33}{n82}
\ncline[]{n33}{n85}
\ncline[]{n33}{n86}
\ncline[]{n33}{n89}
\ncline[]{n33}{n93}
\ncline[]{n33}{n94}
\ncline[]{n33}{n96}
\ncline[]{n33}{n98}
\ncline[]{n33}{n99}
\ncline[]{n33}{n100}
\ncline[]{n33}{n101}
\ncline[]{n33}{n104}
\ncline[]{n33}{n110}
\ncline[]{n33}{n112}
\ncline[]{n33}{n114}
\ncline[]{n33}{n117}
\ncline[]{n33}{n119}
\ncline[]{n33}{n120}
\ncline[]{n33}{n122}
\ncline[]{n33}{n123}
\ncline[]{n33}{n126}
\ncline[]{n33}{n128}
\ncline[]{n33}{n137}
\ncline[]{n33}{n139}
\ncline[]{n33}{n142}
\ncline[]{n33}{n143}
\ncline[]{n33}{n144}
\ncline[]{n33}{n146}
\ncline[]{n34}{n35}
\ncline[]{n34}{n36}
\ncline[]{n34}{n38}
\ncline[]{n34}{n39}
\ncline[]{n34}{n40}
\ncline[]{n34}{n41}
\ncline[]{n34}{n44}
\ncline[]{n34}{n47}
\ncline[]{n34}{n50}
\ncline[]{n34}{n51}
\ncline[]{n34}{n58}
\ncline[]{n34}{n59}
\ncline[]{n34}{n64}
\ncline[]{n34}{n65}
\ncline[]{n34}{n66}
\ncline[]{n34}{n72}
\ncline[]{n34}{n77}
\ncline[]{n34}{n79}
\ncline[]{n34}{n81}
\ncline[]{n34}{n82}
\ncline[]{n34}{n85}
\ncline[]{n34}{n86}
\ncline[]{n34}{n89}
\ncline[]{n34}{n93}
\ncline[]{n34}{n94}
\ncline[]{n34}{n96}
\ncline[]{n34}{n98}
\ncline[]{n34}{n99}
\ncline[]{n34}{n100}
\ncline[]{n34}{n101}
\ncline[]{n34}{n104}
\ncline[]{n34}{n110}
\ncline[]{n34}{n112}
\ncline[]{n34}{n114}
\ncline[]{n34}{n117}
\ncline[]{n34}{n119}
\ncline[]{n34}{n120}
\ncline[]{n34}{n122}
\ncline[]{n34}{n123}
\ncline[]{n34}{n126}
\ncline[]{n34}{n128}
\ncline[]{n34}{n137}
\ncline[]{n34}{n139}
\ncline[]{n34}{n142}
\ncline[]{n34}{n143}
\ncline[]{n34}{n144}
\ncline[]{n34}{n146}
\ncline[]{n35}{n36}
\ncline[]{n35}{n38}
\ncline[]{n35}{n39}
\ncline[]{n35}{n40}
\ncline[]{n35}{n41}
\ncline[]{n35}{n44}
\ncline[]{n35}{n47}
\ncline[]{n35}{n50}
\ncline[]{n35}{n51}
\ncline[]{n35}{n58}
\ncline[]{n35}{n59}
\ncline[]{n35}{n64}
\ncline[]{n35}{n65}
\ncline[]{n35}{n66}
\ncline[]{n35}{n72}
\ncline[]{n35}{n77}
\ncline[]{n35}{n79}
\ncline[]{n35}{n81}
\ncline[]{n35}{n82}
\ncline[]{n35}{n85}
\ncline[]{n35}{n86}
\ncline[]{n35}{n89}
\ncline[]{n35}{n93}
\ncline[]{n35}{n94}
\ncline[]{n35}{n96}
\ncline[]{n35}{n98}
\ncline[]{n35}{n99}
\ncline[]{n35}{n100}
\ncline[]{n35}{n101}
\ncline[]{n35}{n104}
\ncline[]{n35}{n110}
\ncline[]{n35}{n112}
\ncline[]{n35}{n114}
\ncline[]{n35}{n117}
\ncline[]{n35}{n119}
\ncline[]{n35}{n120}
\ncline[]{n35}{n122}
\ncline[]{n35}{n123}
\ncline[]{n35}{n126}
\ncline[]{n35}{n128}
\ncline[]{n35}{n137}
\ncline[]{n35}{n139}
\ncline[]{n35}{n142}
\ncline[]{n35}{n143}
\ncline[]{n35}{n144}
\ncline[]{n35}{n146}
\ncline[]{n36}{n38}
\ncline[]{n36}{n39}
\ncline[]{n36}{n40}
\ncline[]{n36}{n41}
\ncline[]{n36}{n44}
\ncline[]{n36}{n47}
\ncline[]{n36}{n50}
\ncline[]{n36}{n51}
\ncline[]{n36}{n58}
\ncline[]{n36}{n59}
\ncline[]{n36}{n64}
\ncline[]{n36}{n65}
\ncline[]{n36}{n66}
\ncline[]{n36}{n72}
\ncline[]{n36}{n77}
\ncline[]{n36}{n79}
\ncline[]{n36}{n81}
\ncline[]{n36}{n82}
\ncline[]{n36}{n85}
\ncline[]{n36}{n86}
\ncline[]{n36}{n89}
\ncline[]{n36}{n93}
\ncline[]{n36}{n94}
\ncline[]{n36}{n96}
\ncline[]{n36}{n98}
\ncline[]{n36}{n99}
\ncline[]{n36}{n100}
\ncline[]{n36}{n101}
\ncline[]{n36}{n104}
\ncline[]{n36}{n110}
\ncline[]{n36}{n112}
\ncline[]{n36}{n114}
\ncline[]{n36}{n117}
\ncline[]{n36}{n119}
\ncline[]{n36}{n120}
\ncline[]{n36}{n122}
\ncline[]{n36}{n123}
\ncline[]{n36}{n126}
\ncline[]{n36}{n128}
\ncline[]{n36}{n137}
\ncline[]{n36}{n139}
\ncline[]{n36}{n142}
\ncline[]{n36}{n143}
\ncline[]{n36}{n144}
\ncline[]{n36}{n146}
\ncline[]{n37}{n45}
\ncline[]{n37}{n48}
\ncline[]{n37}{n49}
\ncline[]{n37}{n52}
\ncline[]{n37}{n54}
\ncline[]{n37}{n62}
\ncline[]{n37}{n63}
\ncline[]{n37}{n67}
\ncline[]{n37}{n68}
\ncline[]{n37}{n69}
\ncline[]{n37}{n73}
\ncline[]{n37}{n74}
\ncline[]{n37}{n75}
\ncline[]{n37}{n76}
\ncline[]{n37}{n78}
\ncline[]{n37}{n88}
\ncline[]{n37}{n90}
\ncline[]{n37}{n91}
\ncline[]{n37}{n105}
\ncline[]{n37}{n106}
\ncline[]{n37}{n108}
\ncline[]{n37}{n109}
\ncline[]{n37}{n111}
\ncline[]{n37}{n113}
\ncline[]{n37}{n115}
\ncline[]{n37}{n121}
\ncline[]{n37}{n124}
\ncline[]{n37}{n125}
\ncline[]{n37}{n127}
\ncline[]{n37}{n130}
\ncline[]{n37}{n133}
\ncline[]{n37}{n134}
\ncline[]{n37}{n135}
\ncline[]{n37}{n136}
\ncline[]{n37}{n145}
\ncline[]{n37}{n149}
\ncline[]{n38}{n39}
\ncline[]{n38}{n40}
\ncline[]{n38}{n41}
\ncline[]{n38}{n44}
\ncline[]{n38}{n47}
\ncline[]{n38}{n50}
\ncline[]{n38}{n51}
\ncline[]{n38}{n58}
\ncline[]{n38}{n59}
\ncline[]{n38}{n64}
\ncline[]{n38}{n65}
\ncline[]{n38}{n66}
\ncline[]{n38}{n72}
\ncline[]{n38}{n77}
\ncline[]{n38}{n79}
\ncline[]{n38}{n81}
\ncline[]{n38}{n82}
\ncline[]{n38}{n85}
\ncline[]{n38}{n86}
\ncline[]{n38}{n89}
\ncline[]{n38}{n93}
\ncline[]{n38}{n94}
\ncline[]{n38}{n96}
\ncline[]{n38}{n98}
\ncline[]{n38}{n99}
\ncline[]{n38}{n100}
\ncline[]{n38}{n101}
\ncline[]{n38}{n104}
\ncline[]{n38}{n110}
\ncline[]{n38}{n112}
\ncline[]{n38}{n114}
\ncline[]{n38}{n117}
\ncline[]{n38}{n119}
\ncline[]{n38}{n120}
\ncline[]{n38}{n122}
\ncline[]{n38}{n123}
\ncline[]{n38}{n126}
\ncline[]{n38}{n128}
\ncline[]{n38}{n137}
\ncline[]{n38}{n139}
\ncline[]{n38}{n142}
\ncline[]{n38}{n143}
\ncline[]{n38}{n144}
\ncline[]{n38}{n146}
\ncline[]{n39}{n40}
\ncline[]{n39}{n41}
\ncline[]{n39}{n44}
\ncline[]{n39}{n47}
\ncline[]{n39}{n50}
\ncline[]{n39}{n51}
\ncline[]{n39}{n58}
\ncline[]{n39}{n59}
\ncline[]{n39}{n64}
\ncline[]{n39}{n65}
\ncline[]{n39}{n66}
\ncline[]{n39}{n72}
\ncline[]{n39}{n77}
\ncline[]{n39}{n79}
\ncline[]{n39}{n81}
\ncline[]{n39}{n82}
\ncline[]{n39}{n85}
\ncline[]{n39}{n86}
\ncline[]{n39}{n89}
\ncline[]{n39}{n93}
\ncline[]{n39}{n94}
\ncline[]{n39}{n96}
\ncline[]{n39}{n98}
\ncline[]{n39}{n99}
\ncline[]{n39}{n100}
\ncline[]{n39}{n101}
\ncline[]{n39}{n104}
\ncline[]{n39}{n110}
\ncline[]{n39}{n112}
\ncline[]{n39}{n114}
\ncline[]{n39}{n117}
\ncline[]{n39}{n119}
\ncline[]{n39}{n120}
\ncline[]{n39}{n122}
\ncline[]{n39}{n123}
\ncline[]{n39}{n126}
\ncline[]{n39}{n128}
\ncline[]{n39}{n137}
\ncline[]{n39}{n139}
\ncline[]{n39}{n142}
\ncline[]{n39}{n143}
\ncline[]{n39}{n144}
\ncline[]{n39}{n146}
\ncline[]{n40}{n41}
\ncline[]{n40}{n44}
\ncline[]{n40}{n47}
\ncline[]{n40}{n50}
\ncline[]{n40}{n51}
\ncline[]{n40}{n58}
\ncline[]{n40}{n59}
\ncline[]{n40}{n64}
\ncline[]{n40}{n65}
\ncline[]{n40}{n66}
\ncline[]{n40}{n72}
\ncline[]{n40}{n77}
\ncline[]{n40}{n79}
\ncline[]{n40}{n81}
\ncline[]{n40}{n82}
\ncline[]{n40}{n85}
\ncline[]{n40}{n86}
\ncline[]{n40}{n89}
\ncline[]{n40}{n93}
\ncline[]{n40}{n94}
\ncline[]{n40}{n96}
\ncline[]{n40}{n98}
\ncline[]{n40}{n99}
\ncline[]{n40}{n100}
\ncline[]{n40}{n101}
\ncline[]{n40}{n104}
\ncline[]{n40}{n110}
\ncline[]{n40}{n112}
\ncline[]{n40}{n114}
\ncline[]{n40}{n117}
\ncline[]{n40}{n119}
\ncline[]{n40}{n120}
\ncline[]{n40}{n122}
\ncline[]{n40}{n123}
\ncline[]{n40}{n126}
\ncline[]{n40}{n128}
\ncline[]{n40}{n137}
\ncline[]{n40}{n139}
\ncline[]{n40}{n142}
\ncline[]{n40}{n143}
\ncline[]{n40}{n144}
\ncline[]{n40}{n146}
\ncline[]{n41}{n44}
\ncline[]{n41}{n47}
\ncline[]{n41}{n50}
\ncline[]{n41}{n51}
\ncline[]{n41}{n58}
\ncline[]{n41}{n59}
\ncline[]{n41}{n64}
\ncline[]{n41}{n65}
\ncline[]{n41}{n66}
\ncline[]{n41}{n72}
\ncline[]{n41}{n77}
\ncline[]{n41}{n79}
\ncline[]{n41}{n81}
\ncline[]{n41}{n82}
\ncline[]{n41}{n85}
\ncline[]{n41}{n86}
\ncline[]{n41}{n89}
\ncline[]{n41}{n93}
\ncline[]{n41}{n94}
\ncline[]{n41}{n96}
\ncline[]{n41}{n98}
\ncline[]{n41}{n99}
\ncline[]{n41}{n100}
\ncline[]{n41}{n101}
\ncline[]{n41}{n104}
\ncline[]{n41}{n110}
\ncline[]{n41}{n112}
\ncline[]{n41}{n114}
\ncline[]{n41}{n117}
\ncline[]{n41}{n119}
\ncline[]{n41}{n120}
\ncline[]{n41}{n122}
\ncline[]{n41}{n123}
\ncline[]{n41}{n126}
\ncline[]{n41}{n128}
\ncline[]{n41}{n137}
\ncline[]{n41}{n139}
\ncline[]{n41}{n142}
\ncline[]{n41}{n143}
\ncline[]{n41}{n144}
\ncline[]{n41}{n146}
\ncline[]{n42}{n43}
\ncline[]{n42}{n46}
\ncline[]{n42}{n53}
\ncline[]{n42}{n55}
\ncline[]{n42}{n56}
\ncline[]{n42}{n57}
\ncline[]{n42}{n60}
\ncline[]{n42}{n61}
\ncline[]{n42}{n70}
\ncline[]{n42}{n71}
\ncline[]{n42}{n80}
\ncline[]{n42}{n83}
\ncline[]{n42}{n84}
\ncline[]{n42}{n87}
\ncline[]{n42}{n92}
\ncline[]{n42}{n95}
\ncline[]{n42}{n97}
\ncline[]{n42}{n102}
\ncline[]{n42}{n103}
\ncline[]{n42}{n107}
\ncline[]{n42}{n116}
\ncline[]{n42}{n118}
\ncline[]{n42}{n129}
\ncline[]{n42}{n131}
\ncline[]{n42}{n132}
\ncline[]{n42}{n138}
\ncline[]{n42}{n140}
\ncline[]{n42}{n141}
\ncline[]{n42}{n147}
\ncline[]{n42}{n148}
\ncline[]{n43}{n46}
\ncline[]{n43}{n53}
\ncline[]{n43}{n55}
\ncline[]{n43}{n56}
\ncline[]{n43}{n57}
\ncline[]{n43}{n60}
\ncline[]{n43}{n61}
\ncline[]{n43}{n70}
\ncline[]{n43}{n71}
\ncline[]{n43}{n80}
\ncline[]{n43}{n83}
\ncline[]{n43}{n84}
\ncline[]{n43}{n87}
\ncline[]{n43}{n92}
\ncline[]{n43}{n95}
\ncline[]{n43}{n97}
\ncline[]{n43}{n102}
\ncline[]{n43}{n103}
\ncline[]{n43}{n107}
\ncline[]{n43}{n116}
\ncline[]{n43}{n118}
\ncline[]{n43}{n129}
\ncline[]{n43}{n131}
\ncline[]{n43}{n132}
\ncline[]{n43}{n138}
\ncline[]{n43}{n140}
\ncline[]{n43}{n141}
\ncline[]{n43}{n147}
\ncline[]{n43}{n148}
\ncline[]{n44}{n47}
\ncline[]{n44}{n50}
\ncline[]{n44}{n51}
\ncline[]{n44}{n58}
\ncline[]{n44}{n59}
\ncline[]{n44}{n64}
\ncline[]{n44}{n65}
\ncline[]{n44}{n66}
\ncline[]{n44}{n72}
\ncline[]{n44}{n77}
\ncline[]{n44}{n79}
\ncline[]{n44}{n81}
\ncline[]{n44}{n82}
\ncline[]{n44}{n85}
\ncline[]{n44}{n86}
\ncline[]{n44}{n89}
\ncline[]{n44}{n93}
\ncline[]{n44}{n94}
\ncline[]{n44}{n96}
\ncline[]{n44}{n98}
\ncline[]{n44}{n99}
\ncline[]{n44}{n100}
\ncline[]{n44}{n101}
\ncline[]{n44}{n104}
\ncline[]{n44}{n110}
\ncline[]{n44}{n112}
\ncline[]{n44}{n114}
\ncline[]{n44}{n117}
\ncline[]{n44}{n119}
\ncline[]{n44}{n120}
\ncline[]{n44}{n122}
\ncline[]{n44}{n123}
\ncline[]{n44}{n126}
\ncline[]{n44}{n128}
\ncline[]{n44}{n137}
\ncline[]{n44}{n139}
\ncline[]{n44}{n142}
\ncline[]{n44}{n143}
\ncline[]{n44}{n144}
\ncline[]{n44}{n146}
\ncline[]{n45}{n48}
\ncline[]{n45}{n49}
\ncline[]{n45}{n52}
\ncline[]{n45}{n54}
\ncline[]{n45}{n62}
\ncline[]{n45}{n63}
\ncline[]{n45}{n67}
\ncline[]{n45}{n68}
\ncline[]{n45}{n69}
\ncline[]{n45}{n73}
\ncline[]{n45}{n74}
\ncline[]{n45}{n75}
\ncline[]{n45}{n76}
\ncline[]{n45}{n78}
\ncline[]{n45}{n88}
\ncline[]{n45}{n90}
\ncline[]{n45}{n91}
\ncline[]{n45}{n105}
\ncline[]{n45}{n106}
\ncline[]{n45}{n108}
\ncline[]{n45}{n109}
\ncline[]{n45}{n111}
\ncline[]{n45}{n113}
\ncline[]{n45}{n115}
\ncline[]{n45}{n121}
\ncline[]{n45}{n124}
\ncline[]{n45}{n125}
\ncline[]{n45}{n127}
\ncline[]{n45}{n130}
\ncline[]{n45}{n133}
\ncline[]{n45}{n134}
\ncline[]{n45}{n135}
\ncline[]{n45}{n136}
\ncline[]{n45}{n145}
\ncline[]{n45}{n149}
\ncline[]{n46}{n53}
\ncline[]{n46}{n55}
\ncline[]{n46}{n56}
\ncline[]{n46}{n57}
\ncline[]{n46}{n60}
\ncline[]{n46}{n61}
\ncline[]{n46}{n70}
\ncline[]{n46}{n71}
\ncline[]{n46}{n80}
\ncline[]{n46}{n83}
\ncline[]{n46}{n84}
\ncline[]{n46}{n87}
\ncline[]{n46}{n92}
\ncline[]{n46}{n95}
\ncline[]{n46}{n97}
\ncline[]{n46}{n102}
\ncline[]{n46}{n103}
\ncline[]{n46}{n107}
\ncline[]{n46}{n116}
\ncline[]{n46}{n118}
\ncline[]{n46}{n129}
\ncline[]{n46}{n131}
\ncline[]{n46}{n132}
\ncline[]{n46}{n138}
\ncline[]{n46}{n140}
\ncline[]{n46}{n141}
\ncline[]{n46}{n147}
\ncline[]{n46}{n148}
\ncline[]{n47}{n50}
\ncline[]{n47}{n51}
\ncline[]{n47}{n58}
\ncline[]{n47}{n59}
\ncline[]{n47}{n64}
\ncline[]{n47}{n65}
\ncline[]{n47}{n66}
\ncline[]{n47}{n72}
\ncline[]{n47}{n77}
\ncline[]{n47}{n79}
\ncline[]{n47}{n81}
\ncline[]{n47}{n82}
\ncline[]{n47}{n85}
\ncline[]{n47}{n86}
\ncline[]{n47}{n89}
\ncline[]{n47}{n93}
\ncline[]{n47}{n94}
\ncline[]{n47}{n96}
\ncline[]{n47}{n98}
\ncline[]{n47}{n99}
\ncline[]{n47}{n100}
\ncline[]{n47}{n101}
\ncline[]{n47}{n104}
\ncline[]{n47}{n110}
\ncline[]{n47}{n112}
\ncline[]{n47}{n114}
\ncline[]{n47}{n117}
\ncline[]{n47}{n119}
\ncline[]{n47}{n120}
\ncline[]{n47}{n122}
\ncline[]{n47}{n123}
\ncline[]{n47}{n126}
\ncline[]{n47}{n128}
\ncline[]{n47}{n137}
\ncline[]{n47}{n139}
\ncline[]{n47}{n142}
\ncline[]{n47}{n143}
\ncline[]{n47}{n144}
\ncline[]{n47}{n146}
\ncline[]{n48}{n49}
\ncline[]{n48}{n52}
\ncline[]{n48}{n54}
\ncline[]{n48}{n62}
\ncline[]{n48}{n63}
\ncline[]{n48}{n67}
\ncline[]{n48}{n68}
\ncline[]{n48}{n69}
\ncline[]{n48}{n73}
\ncline[]{n48}{n74}
\ncline[]{n48}{n75}
\ncline[]{n48}{n76}
\ncline[]{n48}{n78}
\ncline[]{n48}{n88}
\ncline[]{n48}{n90}
\ncline[]{n48}{n91}
\ncline[]{n48}{n105}
\ncline[]{n48}{n106}
\ncline[]{n48}{n108}
\ncline[]{n48}{n109}
\ncline[]{n48}{n111}
\ncline[]{n48}{n113}
\ncline[]{n48}{n115}
\ncline[]{n48}{n121}
\ncline[]{n48}{n124}
\ncline[]{n48}{n125}
\ncline[]{n48}{n127}
\ncline[]{n48}{n130}
\ncline[]{n48}{n133}
\ncline[]{n48}{n134}
\ncline[]{n48}{n135}
\ncline[]{n48}{n136}
\ncline[]{n48}{n145}
\ncline[]{n48}{n149}
\ncline[]{n49}{n52}
\ncline[]{n49}{n54}
\ncline[]{n49}{n62}
\ncline[]{n49}{n63}
\ncline[]{n49}{n67}
\ncline[]{n49}{n68}
\ncline[]{n49}{n69}
\ncline[]{n49}{n73}
\ncline[]{n49}{n74}
\ncline[]{n49}{n75}
\ncline[]{n49}{n76}
\ncline[]{n49}{n78}
\ncline[]{n49}{n88}
\ncline[]{n49}{n90}
\ncline[]{n49}{n91}
\ncline[]{n49}{n105}
\ncline[]{n49}{n106}
\ncline[]{n49}{n108}
\ncline[]{n49}{n109}
\ncline[]{n49}{n111}
\ncline[]{n49}{n113}
\ncline[]{n49}{n115}
\ncline[]{n49}{n121}
\ncline[]{n49}{n124}
\ncline[]{n49}{n125}
\ncline[]{n49}{n127}
\ncline[]{n49}{n130}
\ncline[]{n49}{n133}
\ncline[]{n49}{n134}
\ncline[]{n49}{n135}
\ncline[]{n49}{n136}
\ncline[]{n49}{n145}
\ncline[]{n49}{n149}
\ncline[]{n50}{n51}
\ncline[]{n50}{n58}
\ncline[]{n50}{n59}
\ncline[]{n50}{n64}
\ncline[]{n50}{n65}
\ncline[]{n50}{n66}
\ncline[]{n50}{n72}
\ncline[]{n50}{n77}
\ncline[]{n50}{n79}
\ncline[]{n50}{n81}
\ncline[]{n50}{n82}
\ncline[]{n50}{n85}
\ncline[]{n50}{n86}
\ncline[]{n50}{n89}
\ncline[]{n50}{n93}
\ncline[]{n50}{n94}
\ncline[]{n50}{n96}
\ncline[]{n50}{n98}
\ncline[]{n50}{n99}
\ncline[]{n50}{n100}
\ncline[]{n50}{n101}
\ncline[]{n50}{n104}
\ncline[]{n50}{n110}
\ncline[]{n50}{n112}
\ncline[]{n50}{n114}
\ncline[]{n50}{n117}
\ncline[]{n50}{n119}
\ncline[]{n50}{n120}
\ncline[]{n50}{n122}
\ncline[]{n50}{n123}
\ncline[]{n50}{n126}
\ncline[]{n50}{n128}
\ncline[]{n50}{n137}
\ncline[]{n50}{n139}
\ncline[]{n50}{n142}
\ncline[]{n50}{n143}
\ncline[]{n50}{n144}
\ncline[]{n50}{n146}
\ncline[]{n51}{n58}
\ncline[]{n51}{n59}
\ncline[]{n51}{n64}
\ncline[]{n51}{n65}
\ncline[]{n51}{n66}
\ncline[]{n51}{n72}
\ncline[]{n51}{n77}
\ncline[]{n51}{n79}
\ncline[]{n51}{n81}
\ncline[]{n51}{n82}
\ncline[]{n51}{n85}
\ncline[]{n51}{n86}
\ncline[]{n51}{n89}
\ncline[]{n51}{n93}
\ncline[]{n51}{n94}
\ncline[]{n51}{n96}
\ncline[]{n51}{n98}
\ncline[]{n51}{n99}
\ncline[]{n51}{n100}
\ncline[]{n51}{n101}
\ncline[]{n51}{n104}
\ncline[]{n51}{n110}
\ncline[]{n51}{n112}
\ncline[]{n51}{n114}
\ncline[]{n51}{n117}
\ncline[]{n51}{n119}
\ncline[]{n51}{n120}
\ncline[]{n51}{n122}
\ncline[]{n51}{n123}
\ncline[]{n51}{n126}
\ncline[]{n51}{n128}
\ncline[]{n51}{n137}
\ncline[]{n51}{n139}
\ncline[]{n51}{n142}
\ncline[]{n51}{n143}
\ncline[]{n51}{n144}
\ncline[]{n51}{n146}
\ncline[]{n52}{n54}
\ncline[]{n52}{n62}
\ncline[]{n52}{n63}
\ncline[]{n52}{n67}
\ncline[]{n52}{n68}
\ncline[]{n52}{n69}
\ncline[]{n52}{n73}
\ncline[]{n52}{n74}
\ncline[]{n52}{n75}
\ncline[]{n52}{n76}
\ncline[]{n52}{n78}
\ncline[]{n52}{n88}
\ncline[]{n52}{n90}
\ncline[]{n52}{n91}
\ncline[]{n52}{n105}
\ncline[]{n52}{n106}
\ncline[]{n52}{n108}
\ncline[]{n52}{n109}
\ncline[]{n52}{n111}
\ncline[]{n52}{n113}
\ncline[]{n52}{n115}
\ncline[]{n52}{n121}
\ncline[]{n52}{n124}
\ncline[]{n52}{n125}
\ncline[]{n52}{n127}
\ncline[]{n52}{n130}
\ncline[]{n52}{n133}
\ncline[]{n52}{n134}
\ncline[]{n52}{n135}
\ncline[]{n52}{n136}
\ncline[]{n52}{n145}
\ncline[]{n52}{n149}
\ncline[]{n53}{n55}
\ncline[]{n53}{n56}
\ncline[]{n53}{n57}
\ncline[]{n53}{n60}
\ncline[]{n53}{n61}
\ncline[]{n53}{n70}
\ncline[]{n53}{n71}
\ncline[]{n53}{n80}
\ncline[]{n53}{n83}
\ncline[]{n53}{n84}
\ncline[]{n53}{n87}
\ncline[]{n53}{n92}
\ncline[]{n53}{n95}
\ncline[]{n53}{n97}
\ncline[]{n53}{n102}
\ncline[]{n53}{n103}
\ncline[]{n53}{n107}
\ncline[]{n53}{n116}
\ncline[]{n53}{n118}
\ncline[]{n53}{n129}
\ncline[]{n53}{n131}
\ncline[]{n53}{n132}
\ncline[]{n53}{n138}
\ncline[]{n53}{n140}
\ncline[]{n53}{n141}
\ncline[]{n53}{n147}
\ncline[]{n53}{n148}
\ncline[]{n54}{n62}
\ncline[]{n54}{n63}
\ncline[]{n54}{n67}
\ncline[]{n54}{n68}
\ncline[]{n54}{n69}
\ncline[]{n54}{n73}
\ncline[]{n54}{n74}
\ncline[]{n54}{n75}
\ncline[]{n54}{n76}
\ncline[]{n54}{n78}
\ncline[]{n54}{n88}
\ncline[]{n54}{n90}
\ncline[]{n54}{n91}
\ncline[]{n54}{n105}
\ncline[]{n54}{n106}
\ncline[]{n54}{n108}
\ncline[]{n54}{n109}
\ncline[]{n54}{n111}
\ncline[]{n54}{n113}
\ncline[]{n54}{n115}
\ncline[]{n54}{n121}
\ncline[]{n54}{n124}
\ncline[]{n54}{n125}
\ncline[]{n54}{n127}
\ncline[]{n54}{n130}
\ncline[]{n54}{n133}
\ncline[]{n54}{n134}
\ncline[]{n54}{n135}
\ncline[]{n54}{n136}
\ncline[]{n54}{n145}
\ncline[]{n54}{n149}
\ncline[]{n55}{n56}
\ncline[]{n55}{n57}
\ncline[]{n55}{n60}
\ncline[]{n55}{n61}
\ncline[]{n55}{n70}
\ncline[]{n55}{n71}
\ncline[]{n55}{n80}
\ncline[]{n55}{n83}
\ncline[]{n55}{n84}
\ncline[]{n55}{n87}
\ncline[]{n55}{n92}
\ncline[]{n55}{n95}
\ncline[]{n55}{n97}
\ncline[]{n55}{n102}
\ncline[]{n55}{n103}
\ncline[]{n55}{n107}
\ncline[]{n55}{n116}
\ncline[]{n55}{n118}
\ncline[]{n55}{n129}
\ncline[]{n55}{n131}
\ncline[]{n55}{n132}
\ncline[]{n55}{n138}
\ncline[]{n55}{n140}
\ncline[]{n55}{n141}
\ncline[]{n55}{n147}
\ncline[]{n55}{n148}
\ncline[]{n56}{n57}
\ncline[]{n56}{n60}
\ncline[]{n56}{n61}
\ncline[]{n56}{n70}
\ncline[]{n56}{n71}
\ncline[]{n56}{n80}
\ncline[]{n56}{n83}
\ncline[]{n56}{n84}
\ncline[]{n56}{n87}
\ncline[]{n56}{n92}
\ncline[]{n56}{n95}
\ncline[]{n56}{n97}
\ncline[]{n56}{n102}
\ncline[]{n56}{n103}
\ncline[]{n56}{n107}
\ncline[]{n56}{n116}
\ncline[]{n56}{n118}
\ncline[]{n56}{n129}
\ncline[]{n56}{n131}
\ncline[]{n56}{n132}
\ncline[]{n56}{n138}
\ncline[]{n56}{n140}
\ncline[]{n56}{n141}
\ncline[]{n56}{n147}
\ncline[]{n56}{n148}
\ncline[]{n57}{n60}
\ncline[]{n57}{n61}
\ncline[]{n57}{n70}
\ncline[]{n57}{n71}
\ncline[]{n57}{n80}
\ncline[]{n57}{n83}
\ncline[]{n57}{n84}
\ncline[]{n57}{n87}
\ncline[]{n57}{n92}
\ncline[]{n57}{n95}
\ncline[]{n57}{n97}
\ncline[]{n57}{n102}
\ncline[]{n57}{n103}
\ncline[]{n57}{n107}
\ncline[]{n57}{n116}
\ncline[]{n57}{n118}
\ncline[]{n57}{n129}
\ncline[]{n57}{n131}
\ncline[]{n57}{n132}
\ncline[]{n57}{n138}
\ncline[]{n57}{n140}
\ncline[]{n57}{n141}
\ncline[]{n57}{n147}
\ncline[]{n57}{n148}
\ncline[]{n58}{n59}
\ncline[]{n58}{n64}
\ncline[]{n58}{n65}
\ncline[]{n58}{n66}
\ncline[]{n58}{n72}
\ncline[]{n58}{n77}
\ncline[]{n58}{n79}
\ncline[]{n58}{n81}
\ncline[]{n58}{n82}
\ncline[]{n58}{n85}
\ncline[]{n58}{n86}
\ncline[]{n58}{n89}
\ncline[]{n58}{n93}
\ncline[]{n58}{n94}
\ncline[]{n58}{n96}
\ncline[]{n58}{n98}
\ncline[]{n58}{n99}
\ncline[]{n58}{n100}
\ncline[]{n58}{n101}
\ncline[]{n58}{n104}
\ncline[]{n58}{n110}
\ncline[]{n58}{n112}
\ncline[]{n58}{n114}
\ncline[]{n58}{n117}
\ncline[]{n58}{n119}
\ncline[]{n58}{n120}
\ncline[]{n58}{n122}
\ncline[]{n58}{n123}
\ncline[]{n58}{n126}
\ncline[]{n58}{n128}
\ncline[]{n58}{n137}
\ncline[]{n58}{n139}
\ncline[]{n58}{n142}
\ncline[]{n58}{n143}
\ncline[]{n58}{n144}
\ncline[]{n58}{n146}
\ncline[]{n59}{n64}
\ncline[]{n59}{n65}
\ncline[]{n59}{n66}
\ncline[]{n59}{n72}
\ncline[]{n59}{n77}
\ncline[]{n59}{n79}
\ncline[]{n59}{n81}
\ncline[]{n59}{n82}
\ncline[]{n59}{n85}
\ncline[]{n59}{n86}
\ncline[]{n59}{n89}
\ncline[]{n59}{n93}
\ncline[]{n59}{n94}
\ncline[]{n59}{n96}
\ncline[]{n59}{n98}
\ncline[]{n59}{n99}
\ncline[]{n59}{n100}
\ncline[]{n59}{n101}
\ncline[]{n59}{n104}
\ncline[]{n59}{n110}
\ncline[]{n59}{n112}
\ncline[]{n59}{n114}
\ncline[]{n59}{n117}
\ncline[]{n59}{n119}
\ncline[]{n59}{n120}
\ncline[]{n59}{n122}
\ncline[]{n59}{n123}
\ncline[]{n59}{n126}
\ncline[]{n59}{n128}
\ncline[]{n59}{n137}
\ncline[]{n59}{n139}
\ncline[]{n59}{n142}
\ncline[]{n59}{n143}
\ncline[]{n59}{n144}
\ncline[]{n59}{n146}
\ncline[]{n60}{n61}
\ncline[]{n60}{n70}
\ncline[]{n60}{n71}
\ncline[]{n60}{n80}
\ncline[]{n60}{n83}
\ncline[]{n60}{n84}
\ncline[]{n60}{n87}
\ncline[]{n60}{n92}
\ncline[]{n60}{n95}
\ncline[]{n60}{n97}
\ncline[]{n60}{n102}
\ncline[]{n60}{n103}
\ncline[]{n60}{n107}
\ncline[]{n60}{n116}
\ncline[]{n60}{n118}
\ncline[]{n60}{n129}
\ncline[]{n60}{n131}
\ncline[]{n60}{n132}
\ncline[]{n60}{n138}
\ncline[]{n60}{n140}
\ncline[]{n60}{n141}
\ncline[]{n60}{n147}
\ncline[]{n60}{n148}
\ncline[]{n61}{n70}
\ncline[]{n61}{n71}
\ncline[]{n61}{n80}
\ncline[]{n61}{n83}
\ncline[]{n61}{n84}
\ncline[]{n61}{n87}
\ncline[]{n61}{n92}
\ncline[]{n61}{n95}
\ncline[]{n61}{n97}
\ncline[]{n61}{n102}
\ncline[]{n61}{n103}
\ncline[]{n61}{n107}
\ncline[]{n61}{n116}
\ncline[]{n61}{n118}
\ncline[]{n61}{n129}
\ncline[]{n61}{n131}
\ncline[]{n61}{n132}
\ncline[]{n61}{n138}
\ncline[]{n61}{n140}
\ncline[]{n61}{n141}
\ncline[]{n61}{n147}
\ncline[]{n61}{n148}
\ncline[]{n62}{n63}
\ncline[]{n62}{n67}
\ncline[]{n62}{n68}
\ncline[]{n62}{n69}
\ncline[]{n62}{n73}
\ncline[]{n62}{n74}
\ncline[]{n62}{n75}
\ncline[]{n62}{n76}
\ncline[]{n62}{n78}
\ncline[]{n62}{n88}
\ncline[]{n62}{n90}
\ncline[]{n62}{n91}
\ncline[]{n62}{n105}
\ncline[]{n62}{n106}
\ncline[]{n62}{n108}
\ncline[]{n62}{n109}
\ncline[]{n62}{n111}
\ncline[]{n62}{n113}
\ncline[]{n62}{n115}
\ncline[]{n62}{n121}
\ncline[]{n62}{n124}
\ncline[]{n62}{n125}
\ncline[]{n62}{n127}
\ncline[]{n62}{n130}
\ncline[]{n62}{n133}
\ncline[]{n62}{n134}
\ncline[]{n62}{n135}
\ncline[]{n62}{n136}
\ncline[]{n62}{n145}
\ncline[]{n62}{n149}
\ncline[]{n63}{n67}
\ncline[]{n63}{n68}
\ncline[]{n63}{n69}
\ncline[]{n63}{n73}
\ncline[]{n63}{n74}
\ncline[]{n63}{n75}
\ncline[]{n63}{n76}
\ncline[]{n63}{n78}
\ncline[]{n63}{n88}
\ncline[]{n63}{n90}
\ncline[]{n63}{n91}
\ncline[]{n63}{n105}
\ncline[]{n63}{n106}
\ncline[]{n63}{n108}
\ncline[]{n63}{n109}
\ncline[]{n63}{n111}
\ncline[]{n63}{n113}
\ncline[]{n63}{n115}
\ncline[]{n63}{n121}
\ncline[]{n63}{n124}
\ncline[]{n63}{n125}
\ncline[]{n63}{n127}
\ncline[]{n63}{n130}
\ncline[]{n63}{n133}
\ncline[]{n63}{n134}
\ncline[]{n63}{n135}
\ncline[]{n63}{n136}
\ncline[]{n63}{n145}
\ncline[]{n63}{n149}
\ncline[]{n64}{n65}
\ncline[]{n64}{n66}
\ncline[]{n64}{n72}
\ncline[]{n64}{n77}
\ncline[]{n64}{n79}
\ncline[]{n64}{n81}
\ncline[]{n64}{n82}
\ncline[]{n64}{n85}
\ncline[]{n64}{n86}
\ncline[]{n64}{n89}
\ncline[]{n64}{n93}
\ncline[]{n64}{n94}
\ncline[]{n64}{n96}
\ncline[]{n64}{n98}
\ncline[]{n64}{n99}
\ncline[]{n64}{n100}
\ncline[]{n64}{n101}
\ncline[]{n64}{n104}
\ncline[]{n64}{n110}
\ncline[]{n64}{n112}
\ncline[]{n64}{n114}
\ncline[]{n64}{n117}
\ncline[]{n64}{n119}
\ncline[]{n64}{n120}
\ncline[]{n64}{n122}
\ncline[]{n64}{n123}
\ncline[]{n64}{n126}
\ncline[]{n64}{n128}
\ncline[]{n64}{n137}
\ncline[]{n64}{n139}
\ncline[]{n64}{n142}
\ncline[]{n64}{n143}
\ncline[]{n64}{n144}
\ncline[]{n64}{n146}
\ncline[]{n65}{n66}
\ncline[]{n65}{n72}
\ncline[]{n65}{n77}
\ncline[]{n65}{n79}
\ncline[]{n65}{n81}
\ncline[]{n65}{n82}
\ncline[]{n65}{n85}
\ncline[]{n65}{n86}
\ncline[]{n65}{n89}
\ncline[]{n65}{n93}
\ncline[]{n65}{n94}
\ncline[]{n65}{n96}
\ncline[]{n65}{n98}
\ncline[]{n65}{n99}
\ncline[]{n65}{n100}
\ncline[]{n65}{n101}
\ncline[]{n65}{n104}
\ncline[]{n65}{n110}
\ncline[]{n65}{n112}
\ncline[]{n65}{n114}
\ncline[]{n65}{n117}
\ncline[]{n65}{n119}
\ncline[]{n65}{n120}
\ncline[]{n65}{n122}
\ncline[]{n65}{n123}
\ncline[]{n65}{n126}
\ncline[]{n65}{n128}
\ncline[]{n65}{n137}
\ncline[]{n65}{n139}
\ncline[]{n65}{n142}
\ncline[]{n65}{n143}
\ncline[]{n65}{n144}
\ncline[]{n65}{n146}
\ncline[]{n66}{n72}
\ncline[]{n66}{n77}
\ncline[]{n66}{n79}
\ncline[]{n66}{n81}
\ncline[]{n66}{n82}
\ncline[]{n66}{n85}
\ncline[]{n66}{n86}
\ncline[]{n66}{n89}
\ncline[]{n66}{n93}
\ncline[]{n66}{n94}
\ncline[]{n66}{n96}
\ncline[]{n66}{n98}
\ncline[]{n66}{n99}
\ncline[]{n66}{n100}
\ncline[]{n66}{n101}
\ncline[]{n66}{n104}
\ncline[]{n66}{n110}
\ncline[]{n66}{n112}
\ncline[]{n66}{n114}
\ncline[]{n66}{n117}
\ncline[]{n66}{n119}
\ncline[]{n66}{n120}
\ncline[]{n66}{n122}
\ncline[]{n66}{n123}
\ncline[]{n66}{n126}
\ncline[]{n66}{n128}
\ncline[]{n66}{n137}
\ncline[]{n66}{n139}
\ncline[]{n66}{n142}
\ncline[]{n66}{n143}
\ncline[]{n66}{n144}
\ncline[]{n66}{n146}
\ncline[]{n67}{n68}
\ncline[]{n67}{n69}
\ncline[]{n67}{n73}
\ncline[]{n67}{n74}
\ncline[]{n67}{n75}
\ncline[]{n67}{n76}
\ncline[]{n67}{n78}
\ncline[]{n67}{n88}
\ncline[]{n67}{n90}
\ncline[]{n67}{n91}
\ncline[]{n67}{n105}
\ncline[]{n67}{n106}
\ncline[]{n67}{n108}
\ncline[]{n67}{n109}
\ncline[]{n67}{n111}
\ncline[]{n67}{n113}
\ncline[]{n67}{n115}
\ncline[]{n67}{n121}
\ncline[]{n67}{n124}
\ncline[]{n67}{n125}
\ncline[]{n67}{n127}
\ncline[]{n67}{n130}
\ncline[]{n67}{n133}
\ncline[]{n67}{n134}
\ncline[]{n67}{n135}
\ncline[]{n67}{n136}
\ncline[]{n67}{n145}
\ncline[]{n67}{n149}
\ncline[]{n68}{n69}
\ncline[]{n68}{n73}
\ncline[]{n68}{n74}
\ncline[]{n68}{n75}
\ncline[]{n68}{n76}
\ncline[]{n68}{n78}
\ncline[]{n68}{n88}
\ncline[]{n68}{n90}
\ncline[]{n68}{n91}
\ncline[]{n68}{n105}
\ncline[]{n68}{n106}
\ncline[]{n68}{n108}
\ncline[]{n68}{n109}
\ncline[]{n68}{n111}
\ncline[]{n68}{n113}
\ncline[]{n68}{n115}
\ncline[]{n68}{n121}
\ncline[]{n68}{n124}
\ncline[]{n68}{n125}
\ncline[]{n68}{n127}
\ncline[]{n68}{n130}
\ncline[]{n68}{n133}
\ncline[]{n68}{n134}
\ncline[]{n68}{n135}
\ncline[]{n68}{n136}
\ncline[]{n68}{n145}
\ncline[]{n68}{n149}
\ncline[]{n69}{n73}
\ncline[]{n69}{n74}
\ncline[]{n69}{n75}
\ncline[]{n69}{n76}
\ncline[]{n69}{n78}
\ncline[]{n69}{n88}
\ncline[]{n69}{n90}
\ncline[]{n69}{n91}
\ncline[]{n69}{n105}
\ncline[]{n69}{n106}
\ncline[]{n69}{n108}
\ncline[]{n69}{n109}
\ncline[]{n69}{n111}
\ncline[]{n69}{n113}
\ncline[]{n69}{n115}
\ncline[]{n69}{n121}
\ncline[]{n69}{n124}
\ncline[]{n69}{n125}
\ncline[]{n69}{n127}
\ncline[]{n69}{n130}
\ncline[]{n69}{n133}
\ncline[]{n69}{n134}
\ncline[]{n69}{n135}
\ncline[]{n69}{n136}
\ncline[]{n69}{n145}
\ncline[]{n69}{n149}
\ncline[]{n70}{n71}
\ncline[]{n70}{n80}
\ncline[]{n70}{n83}
\ncline[]{n70}{n84}
\ncline[]{n70}{n87}
\ncline[]{n70}{n92}
\ncline[]{n70}{n95}
\ncline[]{n70}{n97}
\ncline[]{n70}{n102}
\ncline[]{n70}{n103}
\ncline[]{n70}{n107}
\ncline[]{n70}{n116}
\ncline[]{n70}{n118}
\ncline[]{n70}{n129}
\ncline[]{n70}{n131}
\ncline[]{n70}{n132}
\ncline[]{n70}{n138}
\ncline[]{n70}{n140}
\ncline[]{n70}{n141}
\ncline[]{n70}{n147}
\ncline[]{n70}{n148}
\ncline[]{n71}{n80}
\ncline[]{n71}{n83}
\ncline[]{n71}{n84}
\ncline[]{n71}{n87}
\ncline[]{n71}{n92}
\ncline[]{n71}{n95}
\ncline[]{n71}{n97}
\ncline[]{n71}{n102}
\ncline[]{n71}{n103}
\ncline[]{n71}{n107}
\ncline[]{n71}{n116}
\ncline[]{n71}{n118}
\ncline[]{n71}{n129}
\ncline[]{n71}{n131}
\ncline[]{n71}{n132}
\ncline[]{n71}{n138}
\ncline[]{n71}{n140}
\ncline[]{n71}{n141}
\ncline[]{n71}{n147}
\ncline[]{n71}{n148}
\ncline[]{n72}{n77}
\ncline[]{n72}{n79}
\ncline[]{n72}{n81}
\ncline[]{n72}{n82}
\ncline[]{n72}{n85}
\ncline[]{n72}{n86}
\ncline[]{n72}{n89}
\ncline[]{n72}{n93}
\ncline[]{n72}{n94}
\ncline[]{n72}{n96}
\ncline[]{n72}{n98}
\ncline[]{n72}{n99}
\ncline[]{n72}{n100}
\ncline[]{n72}{n101}
\ncline[]{n72}{n104}
\ncline[]{n72}{n110}
\ncline[]{n72}{n112}
\ncline[]{n72}{n114}
\ncline[]{n72}{n117}
\ncline[]{n72}{n119}
\ncline[]{n72}{n120}
\ncline[]{n72}{n122}
\ncline[]{n72}{n123}
\ncline[]{n72}{n126}
\ncline[]{n72}{n128}
\ncline[]{n72}{n137}
\ncline[]{n72}{n139}
\ncline[]{n72}{n142}
\ncline[]{n72}{n143}
\ncline[]{n72}{n144}
\ncline[]{n72}{n146}
\ncline[]{n73}{n74}
\ncline[]{n73}{n75}
\ncline[]{n73}{n76}
\ncline[]{n73}{n78}
\ncline[]{n73}{n88}
\ncline[]{n73}{n90}
\ncline[]{n73}{n91}
\ncline[]{n73}{n105}
\ncline[]{n73}{n106}
\ncline[]{n73}{n108}
\ncline[]{n73}{n109}
\ncline[]{n73}{n111}
\ncline[]{n73}{n113}
\ncline[]{n73}{n115}
\ncline[]{n73}{n121}
\ncline[]{n73}{n124}
\ncline[]{n73}{n125}
\ncline[]{n73}{n127}
\ncline[]{n73}{n130}
\ncline[]{n73}{n133}
\ncline[]{n73}{n134}
\ncline[]{n73}{n135}
\ncline[]{n73}{n136}
\ncline[]{n73}{n145}
\ncline[]{n73}{n149}
\ncline[]{n74}{n75}
\ncline[]{n74}{n76}
\ncline[]{n74}{n78}
\ncline[]{n74}{n88}
\ncline[]{n74}{n90}
\ncline[]{n74}{n91}
\ncline[]{n74}{n105}
\ncline[]{n74}{n106}
\ncline[]{n74}{n108}
\ncline[]{n74}{n109}
\ncline[]{n74}{n111}
\ncline[]{n74}{n113}
\ncline[]{n74}{n115}
\ncline[]{n74}{n121}
\ncline[]{n74}{n124}
\ncline[]{n74}{n125}
\ncline[]{n74}{n127}
\ncline[]{n74}{n130}
\ncline[]{n74}{n133}
\ncline[]{n74}{n134}
\ncline[]{n74}{n135}
\ncline[]{n74}{n136}
\ncline[]{n74}{n145}
\ncline[]{n74}{n149}
\ncline[]{n75}{n76}
\ncline[]{n75}{n78}
\ncline[]{n75}{n88}
\ncline[]{n75}{n90}
\ncline[]{n75}{n91}
\ncline[]{n75}{n105}
\ncline[]{n75}{n106}
\ncline[]{n75}{n108}
\ncline[]{n75}{n109}
\ncline[]{n75}{n111}
\ncline[]{n75}{n113}
\ncline[]{n75}{n115}
\ncline[]{n75}{n121}
\ncline[]{n75}{n124}
\ncline[]{n75}{n125}
\ncline[]{n75}{n127}
\ncline[]{n75}{n130}
\ncline[]{n75}{n133}
\ncline[]{n75}{n134}
\ncline[]{n75}{n135}
\ncline[]{n75}{n136}
\ncline[]{n75}{n145}
\ncline[]{n75}{n149}
\ncline[]{n76}{n78}
\ncline[]{n76}{n88}
\ncline[]{n76}{n90}
\ncline[]{n76}{n91}
\ncline[]{n76}{n105}
\ncline[]{n76}{n106}
\ncline[]{n76}{n108}
\ncline[]{n76}{n109}
\ncline[]{n76}{n111}
\ncline[]{n76}{n113}
\ncline[]{n76}{n115}
\ncline[]{n76}{n121}
\ncline[]{n76}{n124}
\ncline[]{n76}{n125}
\ncline[]{n76}{n127}
\ncline[]{n76}{n130}
\ncline[]{n76}{n133}
\ncline[]{n76}{n134}
\ncline[]{n76}{n135}
\ncline[]{n76}{n136}
\ncline[]{n76}{n145}
\ncline[]{n76}{n149}
\ncline[]{n77}{n79}
\ncline[]{n77}{n81}
\ncline[]{n77}{n82}
\ncline[]{n77}{n85}
\ncline[]{n77}{n86}
\ncline[]{n77}{n89}
\ncline[]{n77}{n93}
\ncline[]{n77}{n94}
\ncline[]{n77}{n96}
\ncline[]{n77}{n98}
\ncline[]{n77}{n99}
\ncline[]{n77}{n100}
\ncline[]{n77}{n101}
\ncline[]{n77}{n104}
\ncline[]{n77}{n110}
\ncline[]{n77}{n112}
\ncline[]{n77}{n114}
\ncline[]{n77}{n117}
\ncline[]{n77}{n119}
\ncline[]{n77}{n120}
\ncline[]{n77}{n122}
\ncline[]{n77}{n123}
\ncline[]{n77}{n126}
\ncline[]{n77}{n128}
\ncline[]{n77}{n137}
\ncline[]{n77}{n139}
\ncline[]{n77}{n142}
\ncline[]{n77}{n143}
\ncline[]{n77}{n144}
\ncline[]{n77}{n146}
\ncline[]{n78}{n88}
\ncline[]{n78}{n90}
\ncline[]{n78}{n91}
\ncline[]{n78}{n105}
\ncline[]{n78}{n106}
\ncline[]{n78}{n108}
\ncline[]{n78}{n109}
\ncline[]{n78}{n111}
\ncline[]{n78}{n113}
\ncline[]{n78}{n115}
\ncline[]{n78}{n121}
\ncline[]{n78}{n124}
\ncline[]{n78}{n125}
\ncline[]{n78}{n127}
\ncline[]{n78}{n130}
\ncline[]{n78}{n133}
\ncline[]{n78}{n134}
\ncline[]{n78}{n135}
\ncline[]{n78}{n136}
\ncline[]{n78}{n145}
\ncline[]{n78}{n149}
\ncline[]{n79}{n81}
\ncline[]{n79}{n82}
\ncline[]{n79}{n85}
\ncline[]{n79}{n86}
\ncline[]{n79}{n89}
\ncline[]{n79}{n93}
\ncline[]{n79}{n94}
\ncline[]{n79}{n96}
\ncline[]{n79}{n98}
\ncline[]{n79}{n99}
\ncline[]{n79}{n100}
\ncline[]{n79}{n101}
\ncline[]{n79}{n104}
\ncline[]{n79}{n110}
\ncline[]{n79}{n112}
\ncline[]{n79}{n114}
\ncline[]{n79}{n117}
\ncline[]{n79}{n119}
\ncline[]{n79}{n120}
\ncline[]{n79}{n122}
\ncline[]{n79}{n123}
\ncline[]{n79}{n126}
\ncline[]{n79}{n128}
\ncline[]{n79}{n137}
\ncline[]{n79}{n139}
\ncline[]{n79}{n142}
\ncline[]{n79}{n143}
\ncline[]{n79}{n144}
\ncline[]{n79}{n146}
\ncline[]{n80}{n83}
\ncline[]{n80}{n84}
\ncline[]{n80}{n87}
\ncline[]{n80}{n92}
\ncline[]{n80}{n95}
\ncline[]{n80}{n97}
\ncline[]{n80}{n102}
\ncline[]{n80}{n103}
\ncline[]{n80}{n107}
\ncline[]{n80}{n116}
\ncline[]{n80}{n118}
\ncline[]{n80}{n129}
\ncline[]{n80}{n131}
\ncline[]{n80}{n132}
\ncline[]{n80}{n138}
\ncline[]{n80}{n140}
\ncline[]{n80}{n141}
\ncline[]{n80}{n147}
\ncline[]{n80}{n148}
\ncline[]{n81}{n82}
\ncline[]{n81}{n85}
\ncline[]{n81}{n86}
\ncline[]{n81}{n89}
\ncline[]{n81}{n93}
\ncline[]{n81}{n94}
\ncline[]{n81}{n96}
\ncline[]{n81}{n98}
\ncline[]{n81}{n99}
\ncline[]{n81}{n100}
\ncline[]{n81}{n101}
\ncline[]{n81}{n104}
\ncline[]{n81}{n110}
\ncline[]{n81}{n112}
\ncline[]{n81}{n114}
\ncline[]{n81}{n117}
\ncline[]{n81}{n119}
\ncline[]{n81}{n120}
\ncline[]{n81}{n122}
\ncline[]{n81}{n123}
\ncline[]{n81}{n126}
\ncline[]{n81}{n128}
\ncline[]{n81}{n137}
\ncline[]{n81}{n139}
\ncline[]{n81}{n142}
\ncline[]{n81}{n143}
\ncline[]{n81}{n144}
\ncline[]{n81}{n146}
\ncline[]{n82}{n85}
\ncline[]{n82}{n86}
\ncline[]{n82}{n89}
\ncline[]{n82}{n93}
\ncline[]{n82}{n94}
\ncline[]{n82}{n96}
\ncline[]{n82}{n98}
\ncline[]{n82}{n99}
\ncline[]{n82}{n100}
\ncline[]{n82}{n101}
\ncline[]{n82}{n104}
\ncline[]{n82}{n110}
\ncline[]{n82}{n112}
\ncline[]{n82}{n114}
\ncline[]{n82}{n117}
\ncline[]{n82}{n119}
\ncline[]{n82}{n120}
\ncline[]{n82}{n122}
\ncline[]{n82}{n123}
\ncline[]{n82}{n126}
\ncline[]{n82}{n128}
\ncline[]{n82}{n137}
\ncline[]{n82}{n139}
\ncline[]{n82}{n142}
\ncline[]{n82}{n143}
\ncline[]{n82}{n144}
\ncline[]{n82}{n146}
\ncline[]{n83}{n84}
\ncline[]{n83}{n87}
\ncline[]{n83}{n92}
\ncline[]{n83}{n95}
\ncline[]{n83}{n97}
\ncline[]{n83}{n102}
\ncline[]{n83}{n103}
\ncline[]{n83}{n107}
\ncline[]{n83}{n116}
\ncline[]{n83}{n118}
\ncline[]{n83}{n129}
\ncline[]{n83}{n131}
\ncline[]{n83}{n132}
\ncline[]{n83}{n138}
\ncline[]{n83}{n140}
\ncline[]{n83}{n141}
\ncline[]{n83}{n147}
\ncline[]{n83}{n148}
\ncline[]{n84}{n87}
\ncline[]{n84}{n92}
\ncline[]{n84}{n95}
\ncline[]{n84}{n97}
\ncline[]{n84}{n102}
\ncline[]{n84}{n103}
\ncline[]{n84}{n107}
\ncline[]{n84}{n116}
\ncline[]{n84}{n118}
\ncline[]{n84}{n129}
\ncline[]{n84}{n131}
\ncline[]{n84}{n132}
\ncline[]{n84}{n138}
\ncline[]{n84}{n140}
\ncline[]{n84}{n141}
\ncline[]{n84}{n147}
\ncline[]{n84}{n148}
\ncline[]{n85}{n86}
\ncline[]{n85}{n89}
\ncline[]{n85}{n93}
\ncline[]{n85}{n94}
\ncline[]{n85}{n96}
\ncline[]{n85}{n98}
\ncline[]{n85}{n99}
\ncline[]{n85}{n100}
\ncline[]{n85}{n101}
\ncline[]{n85}{n104}
\ncline[]{n85}{n110}
\ncline[]{n85}{n112}
\ncline[]{n85}{n114}
\ncline[]{n85}{n117}
\ncline[]{n85}{n119}
\ncline[]{n85}{n120}
\ncline[]{n85}{n122}
\ncline[]{n85}{n123}
\ncline[]{n85}{n126}
\ncline[]{n85}{n128}
\ncline[]{n85}{n137}
\ncline[]{n85}{n139}
\ncline[]{n85}{n142}
\ncline[]{n85}{n143}
\ncline[]{n85}{n144}
\ncline[]{n85}{n146}
\ncline[]{n86}{n89}
\ncline[]{n86}{n93}
\ncline[]{n86}{n94}
\ncline[]{n86}{n96}
\ncline[]{n86}{n98}
\ncline[]{n86}{n99}
\ncline[]{n86}{n100}
\ncline[]{n86}{n101}
\ncline[]{n86}{n104}
\ncline[]{n86}{n110}
\ncline[]{n86}{n112}
\ncline[]{n86}{n114}
\ncline[]{n86}{n117}
\ncline[]{n86}{n119}
\ncline[]{n86}{n120}
\ncline[]{n86}{n122}
\ncline[]{n86}{n123}
\ncline[]{n86}{n126}
\ncline[]{n86}{n128}
\ncline[]{n86}{n137}
\ncline[]{n86}{n139}
\ncline[]{n86}{n142}
\ncline[]{n86}{n143}
\ncline[]{n86}{n144}
\ncline[]{n86}{n146}
\ncline[]{n87}{n92}
\ncline[]{n87}{n95}
\ncline[]{n87}{n97}
\ncline[]{n87}{n102}
\ncline[]{n87}{n103}
\ncline[]{n87}{n107}
\ncline[]{n87}{n116}
\ncline[]{n87}{n118}
\ncline[]{n87}{n129}
\ncline[]{n87}{n131}
\ncline[]{n87}{n132}
\ncline[]{n87}{n138}
\ncline[]{n87}{n140}
\ncline[]{n87}{n141}
\ncline[]{n87}{n147}
\ncline[]{n87}{n148}
\ncline[]{n88}{n90}
\ncline[]{n88}{n91}
\ncline[]{n88}{n105}
\ncline[]{n88}{n106}
\ncline[]{n88}{n108}
\ncline[]{n88}{n109}
\ncline[]{n88}{n111}
\ncline[]{n88}{n113}
\ncline[]{n88}{n115}
\ncline[]{n88}{n121}
\ncline[]{n88}{n124}
\ncline[]{n88}{n125}
\ncline[]{n88}{n127}
\ncline[]{n88}{n130}
\ncline[]{n88}{n133}
\ncline[]{n88}{n134}
\ncline[]{n88}{n135}
\ncline[]{n88}{n136}
\ncline[]{n88}{n145}
\ncline[]{n88}{n149}
\ncline[]{n89}{n93}
\ncline[]{n89}{n94}
\ncline[]{n89}{n96}
\ncline[]{n89}{n98}
\ncline[]{n89}{n99}
\ncline[]{n89}{n100}
\ncline[]{n89}{n101}
\ncline[]{n89}{n104}
\ncline[]{n89}{n110}
\ncline[]{n89}{n112}
\ncline[]{n89}{n114}
\ncline[]{n89}{n117}
\ncline[]{n89}{n119}
\ncline[]{n89}{n120}
\ncline[]{n89}{n122}
\ncline[]{n89}{n123}
\ncline[]{n89}{n126}
\ncline[]{n89}{n128}
\ncline[]{n89}{n137}
\ncline[]{n89}{n139}
\ncline[]{n89}{n142}
\ncline[]{n89}{n143}
\ncline[]{n89}{n144}
\ncline[]{n89}{n146}
\ncline[]{n90}{n91}
\ncline[]{n90}{n105}
\ncline[]{n90}{n106}
\ncline[]{n90}{n108}
\ncline[]{n90}{n109}
\ncline[]{n90}{n111}
\ncline[]{n90}{n113}
\ncline[]{n90}{n115}
\ncline[]{n90}{n121}
\ncline[]{n90}{n124}
\ncline[]{n90}{n125}
\ncline[]{n90}{n127}
\ncline[]{n90}{n130}
\ncline[]{n90}{n133}
\ncline[]{n90}{n134}
\ncline[]{n90}{n135}
\ncline[]{n90}{n136}
\ncline[]{n90}{n145}
\ncline[]{n90}{n149}
\ncline[]{n91}{n105}
\ncline[]{n91}{n106}
\ncline[]{n91}{n108}
\ncline[]{n91}{n109}
\ncline[]{n91}{n111}
\ncline[]{n91}{n113}
\ncline[]{n91}{n115}
\ncline[]{n91}{n121}
\ncline[]{n91}{n124}
\ncline[]{n91}{n125}
\ncline[]{n91}{n127}
\ncline[]{n91}{n130}
\ncline[]{n91}{n133}
\ncline[]{n91}{n134}
\ncline[]{n91}{n135}
\ncline[]{n91}{n136}
\ncline[]{n91}{n145}
\ncline[]{n91}{n149}
\ncline[]{n92}{n95}
\ncline[]{n92}{n97}
\ncline[]{n92}{n102}
\ncline[]{n92}{n103}
\ncline[]{n92}{n107}
\ncline[]{n92}{n116}
\ncline[]{n92}{n118}
\ncline[]{n92}{n129}
\ncline[]{n92}{n131}
\ncline[]{n92}{n132}
\ncline[]{n92}{n138}
\ncline[]{n92}{n140}
\ncline[]{n92}{n141}
\ncline[]{n92}{n147}
\ncline[]{n92}{n148}
\ncline[]{n93}{n94}
\ncline[]{n93}{n96}
\ncline[]{n93}{n98}
\ncline[]{n93}{n99}
\ncline[]{n93}{n100}
\ncline[]{n93}{n101}
\ncline[]{n93}{n104}
\ncline[]{n93}{n110}
\ncline[]{n93}{n112}
\ncline[]{n93}{n114}
\ncline[]{n93}{n117}
\ncline[]{n93}{n119}
\ncline[]{n93}{n120}
\ncline[]{n93}{n122}
\ncline[]{n93}{n123}
\ncline[]{n93}{n126}
\ncline[]{n93}{n128}
\ncline[]{n93}{n137}
\ncline[]{n93}{n139}
\ncline[]{n93}{n142}
\ncline[]{n93}{n143}
\ncline[]{n93}{n144}
\ncline[]{n93}{n146}
\ncline[]{n94}{n96}
\ncline[]{n94}{n98}
\ncline[]{n94}{n99}
\ncline[]{n94}{n100}
\ncline[]{n94}{n101}
\ncline[]{n94}{n104}
\ncline[]{n94}{n110}
\ncline[]{n94}{n112}
\ncline[]{n94}{n114}
\ncline[]{n94}{n117}
\ncline[]{n94}{n119}
\ncline[]{n94}{n120}
\ncline[]{n94}{n122}
\ncline[]{n94}{n123}
\ncline[]{n94}{n126}
\ncline[]{n94}{n128}
\ncline[]{n94}{n137}
\ncline[]{n94}{n139}
\ncline[]{n94}{n142}
\ncline[]{n94}{n143}
\ncline[]{n94}{n144}
\ncline[]{n94}{n146}
\ncline[]{n95}{n97}
\ncline[]{n95}{n102}
\ncline[]{n95}{n103}
\ncline[]{n95}{n107}
\ncline[]{n95}{n116}
\ncline[]{n95}{n118}
\ncline[]{n95}{n129}
\ncline[]{n95}{n131}
\ncline[]{n95}{n132}
\ncline[]{n95}{n138}
\ncline[]{n95}{n140}
\ncline[]{n95}{n141}
\ncline[]{n95}{n147}
\ncline[]{n95}{n148}
\ncline[]{n96}{n98}
\ncline[]{n96}{n99}
\ncline[]{n96}{n100}
\ncline[]{n96}{n101}
\ncline[]{n96}{n104}
\ncline[]{n96}{n110}
\ncline[]{n96}{n112}
\ncline[]{n96}{n114}
\ncline[]{n96}{n117}
\ncline[]{n96}{n119}
\ncline[]{n96}{n120}
\ncline[]{n96}{n122}
\ncline[]{n96}{n123}
\ncline[]{n96}{n126}
\ncline[]{n96}{n128}
\ncline[]{n96}{n137}
\ncline[]{n96}{n139}
\ncline[]{n96}{n142}
\ncline[]{n96}{n143}
\ncline[]{n96}{n144}
\ncline[]{n96}{n146}
\ncline[]{n97}{n102}
\ncline[]{n97}{n103}
\ncline[]{n97}{n107}
\ncline[]{n97}{n116}
\ncline[]{n97}{n118}
\ncline[]{n97}{n129}
\ncline[]{n97}{n131}
\ncline[]{n97}{n132}
\ncline[]{n97}{n138}
\ncline[]{n97}{n140}
\ncline[]{n97}{n141}
\ncline[]{n97}{n147}
\ncline[]{n97}{n148}
\ncline[]{n98}{n99}
\ncline[]{n98}{n100}
\ncline[]{n98}{n101}
\ncline[]{n98}{n104}
\ncline[]{n98}{n110}
\ncline[]{n98}{n112}
\ncline[]{n98}{n114}
\ncline[]{n98}{n117}
\ncline[]{n98}{n119}
\ncline[]{n98}{n120}
\ncline[]{n98}{n122}
\ncline[]{n98}{n123}
\ncline[]{n98}{n126}
\ncline[]{n98}{n128}
\ncline[]{n98}{n137}
\ncline[]{n98}{n139}
\ncline[]{n98}{n142}
\ncline[]{n98}{n143}
\ncline[]{n98}{n144}
\ncline[]{n98}{n146}
\ncline[]{n99}{n100}
\ncline[]{n99}{n101}
\ncline[]{n99}{n104}
\ncline[]{n99}{n110}
\ncline[]{n99}{n112}
\ncline[]{n99}{n114}
\ncline[]{n99}{n117}
\ncline[]{n99}{n119}
\ncline[]{n99}{n120}
\ncline[]{n99}{n122}
\ncline[]{n99}{n123}
\ncline[]{n99}{n126}
\ncline[]{n99}{n128}
\ncline[]{n99}{n137}
\ncline[]{n99}{n139}
\ncline[]{n99}{n142}
\ncline[]{n99}{n143}
\ncline[]{n99}{n144}
\ncline[]{n99}{n146}
\ncline[]{n100}{n101}
\ncline[]{n100}{n104}
\ncline[]{n100}{n110}
\ncline[]{n100}{n112}
\ncline[]{n100}{n114}
\ncline[]{n100}{n117}
\ncline[]{n100}{n119}
\ncline[]{n100}{n120}
\ncline[]{n100}{n122}
\ncline[]{n100}{n123}
\ncline[]{n100}{n126}
\ncline[]{n100}{n128}
\ncline[]{n100}{n137}
\ncline[]{n100}{n139}
\ncline[]{n100}{n142}
\ncline[]{n100}{n143}
\ncline[]{n100}{n144}
\ncline[]{n100}{n146}
\ncline[]{n101}{n104}
\ncline[]{n101}{n110}
\ncline[]{n101}{n112}
\ncline[]{n101}{n114}
\ncline[]{n101}{n117}
\ncline[]{n101}{n119}
\ncline[]{n101}{n120}
\ncline[]{n101}{n122}
\ncline[]{n101}{n123}
\ncline[]{n101}{n126}
\ncline[]{n101}{n128}
\ncline[]{n101}{n137}
\ncline[]{n101}{n139}
\ncline[]{n101}{n142}
\ncline[]{n101}{n143}
\ncline[]{n101}{n144}
\ncline[]{n101}{n146}
\ncline[]{n102}{n103}
\ncline[]{n102}{n107}
\ncline[]{n102}{n116}
\ncline[]{n102}{n118}
\ncline[]{n102}{n129}
\ncline[]{n102}{n131}
\ncline[]{n102}{n132}
\ncline[]{n102}{n138}
\ncline[]{n102}{n140}
\ncline[]{n102}{n141}
\ncline[]{n102}{n147}
\ncline[]{n102}{n148}
\ncline[]{n103}{n107}
\ncline[]{n103}{n116}
\ncline[]{n103}{n118}
\ncline[]{n103}{n129}
\ncline[]{n103}{n131}
\ncline[]{n103}{n132}
\ncline[]{n103}{n138}
\ncline[]{n103}{n140}
\ncline[]{n103}{n141}
\ncline[]{n103}{n147}
\ncline[]{n103}{n148}
\ncline[]{n104}{n110}
\ncline[]{n104}{n112}
\ncline[]{n104}{n114}
\ncline[]{n104}{n117}
\ncline[]{n104}{n119}
\ncline[]{n104}{n120}
\ncline[]{n104}{n122}
\ncline[]{n104}{n123}
\ncline[]{n104}{n126}
\ncline[]{n104}{n128}
\ncline[]{n104}{n137}
\ncline[]{n104}{n139}
\ncline[]{n104}{n142}
\ncline[]{n104}{n143}
\ncline[]{n104}{n144}
\ncline[]{n104}{n146}
\ncline[]{n105}{n106}
\ncline[]{n105}{n108}
\ncline[]{n105}{n109}
\ncline[]{n105}{n111}
\ncline[]{n105}{n113}
\ncline[]{n105}{n115}
\ncline[]{n105}{n121}
\ncline[]{n105}{n124}
\ncline[]{n105}{n125}
\ncline[]{n105}{n127}
\ncline[]{n105}{n130}
\ncline[]{n105}{n133}
\ncline[]{n105}{n134}
\ncline[]{n105}{n135}
\ncline[]{n105}{n136}
\ncline[]{n105}{n145}
\ncline[]{n105}{n149}
\ncline[]{n106}{n108}
\ncline[]{n106}{n109}
\ncline[]{n106}{n111}
\ncline[]{n106}{n113}
\ncline[]{n106}{n115}
\ncline[]{n106}{n121}
\ncline[]{n106}{n124}
\ncline[]{n106}{n125}
\ncline[]{n106}{n127}
\ncline[]{n106}{n130}
\ncline[]{n106}{n133}
\ncline[]{n106}{n134}
\ncline[]{n106}{n135}
\ncline[]{n106}{n136}
\ncline[]{n106}{n145}
\ncline[]{n106}{n149}
\ncline[]{n107}{n116}
\ncline[]{n107}{n118}
\ncline[]{n107}{n129}
\ncline[]{n107}{n131}
\ncline[]{n107}{n132}
\ncline[]{n107}{n138}
\ncline[]{n107}{n140}
\ncline[]{n107}{n141}
\ncline[]{n107}{n147}
\ncline[]{n107}{n148}
\ncline[]{n108}{n109}
\ncline[]{n108}{n111}
\ncline[]{n108}{n113}
\ncline[]{n108}{n115}
\ncline[]{n108}{n121}
\ncline[]{n108}{n124}
\ncline[]{n108}{n125}
\ncline[]{n108}{n127}
\ncline[]{n108}{n130}
\ncline[]{n108}{n133}
\ncline[]{n108}{n134}
\ncline[]{n108}{n135}
\ncline[]{n108}{n136}
\ncline[]{n108}{n145}
\ncline[]{n108}{n149}
\ncline[]{n109}{n111}
\ncline[]{n109}{n113}
\ncline[]{n109}{n115}
\ncline[]{n109}{n121}
\ncline[]{n109}{n124}
\ncline[]{n109}{n125}
\ncline[]{n109}{n127}
\ncline[]{n109}{n130}
\ncline[]{n109}{n133}
\ncline[]{n109}{n134}
\ncline[]{n109}{n135}
\ncline[]{n109}{n136}
\ncline[]{n109}{n145}
\ncline[]{n109}{n149}
\ncline[]{n110}{n112}
\ncline[]{n110}{n114}
\ncline[]{n110}{n117}
\ncline[]{n110}{n119}
\ncline[]{n110}{n120}
\ncline[]{n110}{n122}
\ncline[]{n110}{n123}
\ncline[]{n110}{n126}
\ncline[]{n110}{n128}
\ncline[]{n110}{n137}
\ncline[]{n110}{n139}
\ncline[]{n110}{n142}
\ncline[]{n110}{n143}
\ncline[]{n110}{n144}
\ncline[]{n110}{n146}
\ncline[]{n111}{n113}
\ncline[]{n111}{n115}
\ncline[]{n111}{n121}
\ncline[]{n111}{n124}
\ncline[]{n111}{n125}
\ncline[]{n111}{n127}
\ncline[]{n111}{n130}
\ncline[]{n111}{n133}
\ncline[]{n111}{n134}
\ncline[]{n111}{n135}
\ncline[]{n111}{n136}
\ncline[]{n111}{n145}
\ncline[]{n111}{n149}
\ncline[]{n112}{n114}
\ncline[]{n112}{n117}
\ncline[]{n112}{n119}
\ncline[]{n112}{n120}
\ncline[]{n112}{n122}
\ncline[]{n112}{n123}
\ncline[]{n112}{n126}
\ncline[]{n112}{n128}
\ncline[]{n112}{n137}
\ncline[]{n112}{n139}
\ncline[]{n112}{n142}
\ncline[]{n112}{n143}
\ncline[]{n112}{n144}
\ncline[]{n112}{n146}
\ncline[]{n113}{n115}
\ncline[]{n113}{n121}
\ncline[]{n113}{n124}
\ncline[]{n113}{n125}
\ncline[]{n113}{n127}
\ncline[]{n113}{n130}
\ncline[]{n113}{n133}
\ncline[]{n113}{n134}
\ncline[]{n113}{n135}
\ncline[]{n113}{n136}
\ncline[]{n113}{n145}
\ncline[]{n113}{n149}
\ncline[]{n114}{n117}
\ncline[]{n114}{n119}
\ncline[]{n114}{n120}
\ncline[]{n114}{n122}
\ncline[]{n114}{n123}
\ncline[]{n114}{n126}
\ncline[]{n114}{n128}
\ncline[]{n114}{n137}
\ncline[]{n114}{n139}
\ncline[]{n114}{n142}
\ncline[]{n114}{n143}
\ncline[]{n114}{n144}
\ncline[]{n114}{n146}
\ncline[]{n115}{n121}
\ncline[]{n115}{n124}
\ncline[]{n115}{n125}
\ncline[]{n115}{n127}
\ncline[]{n115}{n130}
\ncline[]{n115}{n133}
\ncline[]{n115}{n134}
\ncline[]{n115}{n135}
\ncline[]{n115}{n136}
\ncline[]{n115}{n145}
\ncline[]{n115}{n149}
\ncline[]{n116}{n118}
\ncline[]{n116}{n129}
\ncline[]{n116}{n131}
\ncline[]{n116}{n132}
\ncline[]{n116}{n138}
\ncline[]{n116}{n140}
\ncline[]{n116}{n141}
\ncline[]{n116}{n147}
\ncline[]{n116}{n148}
\ncline[]{n117}{n119}
\ncline[]{n117}{n120}
\ncline[]{n117}{n122}
\ncline[]{n117}{n123}
\ncline[]{n117}{n126}
\ncline[]{n117}{n128}
\ncline[]{n117}{n137}
\ncline[]{n117}{n139}
\ncline[]{n117}{n142}
\ncline[]{n117}{n143}
\ncline[]{n117}{n144}
\ncline[]{n117}{n146}
\ncline[]{n118}{n129}
\ncline[]{n118}{n131}
\ncline[]{n118}{n132}
\ncline[]{n118}{n138}
\ncline[]{n118}{n140}
\ncline[]{n118}{n141}
\ncline[]{n118}{n147}
\ncline[]{n118}{n148}
\ncline[]{n119}{n120}
\ncline[]{n119}{n122}
\ncline[]{n119}{n123}
\ncline[]{n119}{n126}
\ncline[]{n119}{n128}
\ncline[]{n119}{n137}
\ncline[]{n119}{n139}
\ncline[]{n119}{n142}
\ncline[]{n119}{n143}
\ncline[]{n119}{n144}
\ncline[]{n119}{n146}
\ncline[]{n120}{n122}
\ncline[]{n120}{n123}
\ncline[]{n120}{n126}
\ncline[]{n120}{n128}
\ncline[]{n120}{n137}
\ncline[]{n120}{n139}
\ncline[]{n120}{n142}
\ncline[]{n120}{n143}
\ncline[]{n120}{n144}
\ncline[]{n120}{n146}
\ncline[]{n121}{n124}
\ncline[]{n121}{n125}
\ncline[]{n121}{n127}
\ncline[]{n121}{n130}
\ncline[]{n121}{n133}
\ncline[]{n121}{n134}
\ncline[]{n121}{n135}
\ncline[]{n121}{n136}
\ncline[]{n121}{n145}
\ncline[]{n121}{n149}
\ncline[]{n122}{n123}
\ncline[]{n122}{n126}
\ncline[]{n122}{n128}
\ncline[]{n122}{n137}
\ncline[]{n122}{n139}
\ncline[]{n122}{n142}
\ncline[]{n122}{n143}
\ncline[]{n122}{n144}
\ncline[]{n122}{n146}
\ncline[]{n123}{n126}
\ncline[]{n123}{n128}
\ncline[]{n123}{n137}
\ncline[]{n123}{n139}
\ncline[]{n123}{n142}
\ncline[]{n123}{n143}
\ncline[]{n123}{n144}
\ncline[]{n123}{n146}
\ncline[]{n124}{n125}
\ncline[]{n124}{n127}
\ncline[]{n124}{n130}
\ncline[]{n124}{n133}
\ncline[]{n124}{n134}
\ncline[]{n124}{n135}
\ncline[]{n124}{n136}
\ncline[]{n124}{n145}
\ncline[]{n124}{n149}
\ncline[]{n125}{n127}
\ncline[]{n125}{n130}
\ncline[]{n125}{n133}
\ncline[]{n125}{n134}
\ncline[]{n125}{n135}
\ncline[]{n125}{n136}
\ncline[]{n125}{n145}
\ncline[]{n125}{n149}
\ncline[]{n126}{n128}
\ncline[]{n126}{n137}
\ncline[]{n126}{n139}
\ncline[]{n126}{n142}
\ncline[]{n126}{n143}
\ncline[]{n126}{n144}
\ncline[]{n126}{n146}
\ncline[]{n127}{n130}
\ncline[]{n127}{n133}
\ncline[]{n127}{n134}
\ncline[]{n127}{n135}
\ncline[]{n127}{n136}
\ncline[]{n127}{n145}
\ncline[]{n127}{n149}
\ncline[]{n128}{n137}
\ncline[]{n128}{n139}
\ncline[]{n128}{n142}
\ncline[]{n128}{n143}
\ncline[]{n128}{n144}
\ncline[]{n128}{n146}
\ncline[]{n129}{n131}
\ncline[]{n129}{n132}
\ncline[]{n129}{n138}
\ncline[]{n129}{n140}
\ncline[]{n129}{n141}
\ncline[]{n129}{n147}
\ncline[]{n129}{n148}
\ncline[]{n130}{n133}
\ncline[]{n130}{n134}
\ncline[]{n130}{n135}
\ncline[]{n130}{n136}
\ncline[]{n130}{n145}
\ncline[]{n130}{n149}
\ncline[]{n131}{n132}
\ncline[]{n131}{n138}
\ncline[]{n131}{n140}
\ncline[]{n131}{n141}
\ncline[]{n131}{n147}
\ncline[]{n131}{n148}
\ncline[]{n132}{n138}
\ncline[]{n132}{n140}
\ncline[]{n132}{n141}
\ncline[]{n132}{n147}
\ncline[]{n132}{n148}
\ncline[]{n133}{n134}
\ncline[]{n133}{n135}
\ncline[]{n133}{n136}
\ncline[]{n133}{n145}
\ncline[]{n133}{n149}
\ncline[]{n134}{n135}
\ncline[]{n134}{n136}
\ncline[]{n134}{n145}
\ncline[]{n134}{n149}
\ncline[]{n135}{n136}
\ncline[]{n135}{n145}
\ncline[]{n135}{n149}
\ncline[]{n136}{n145}
\ncline[]{n136}{n149}
\ncline[]{n137}{n139}
\ncline[]{n137}{n142}
\ncline[]{n137}{n143}
\ncline[]{n137}{n144}
\ncline[]{n137}{n146}
\ncline[]{n138}{n140}
\ncline[]{n138}{n141}
\ncline[]{n138}{n147}
\ncline[]{n138}{n148}
\ncline[]{n139}{n142}
\ncline[]{n139}{n143}
\ncline[]{n139}{n144}
\ncline[]{n139}{n146}
\ncline[]{n140}{n141}
\ncline[]{n140}{n147}
\ncline[]{n140}{n148}
\ncline[]{n141}{n147}
\ncline[]{n141}{n148}
\ncline[]{n142}{n143}
\ncline[]{n142}{n144}
\ncline[]{n142}{n146}
\ncline[]{n143}{n144}
\ncline[]{n143}{n146}
\ncline[]{n144}{n146}
\ncline[]{n145}{n149}
\ncline[]{n147}{n148}
\psset{fillcolor=gray, linecolor=black}
\dotnode[dotstyle=Bsquare](-0.511, -0.102){n0}
\dotnode[dotstyle=Btriangle](-1.464, 0.504){n1}
\dotnode[dotstyle=Bsquare](-1.043, 0.229){n2}
\dotnode[dotstyle=Bo](2.840, -0.221){n3}
\dotnode[dotstyle=Bsquare](-0.262, -0.548){n4}
\dotnode[dotstyle=Bo](2.890, -0.137){n5}
\dotnode[dotstyle=Btriangle](-2.350, -0.042){n6}
\dotnode[dotstyle=Bsquare](-1.414, -0.575){n7}
\dotnode[dotstyle=Btriangle](-2.314, 0.183){n8}
\dotnode[dotstyle=Btriangle](-2.320, -0.246){n9}
\dotnode[dotstyle=Bo](2.543, 0.440){n10}
\dotnode[dotstyle=Bo](2.587, 0.520){n11}
\dotnode[dotstyle=Bsquare](-0.331, -0.211){n12}
\dotnode[dotstyle=Bsquare](-1.291, -0.116){n13}
\dotnode[dotstyle=Bo](2.674, -0.107){n14}
\dotnode[dotstyle=Bo](2.469, 0.138){n15}
\dotnode[dotstyle=Bo](2.626, 0.170){n16}
\dotnode[dotstyle=Bsquare](-0.298, -0.347){n17}
\dotnode[dotstyle=Bo](2.704, 0.115){n18}
\dotnode[dotstyle=Btriangle](-2.614, 0.558){n19}
\dotnode[dotstyle=Bsquare](-1.390, -0.283){n20}
\dotnode[dotstyle=Btriangle](-1.764, 0.079){n21}
\dotnode[dotstyle=Bo](2.384, 1.345){n22}
\dotnode[dotstyle=Bsquare](-0.165, -0.680){n23}
\dotnode[dotstyle=Bsquare](-0.519, -1.191){n24}
\dotnode[dotstyle=Bo](2.406, 0.196){n25}
\dotnode[dotstyle=Bsquare](-0.181, -0.826){n26}
\dotnode[dotstyle=Bo](2.746, -0.311){n27}
\dotnode[dotstyle=Btriangle](-2.388, 0.463){n28}
\dotnode[dotstyle=Btriangle](-2.563, 0.276){n29}
\dotnode[dotstyle=Bo](2.623, 0.818){n30}
\dotnode[dotstyle=Bo](2.867, 0.077){n31}
\dotnode[dotstyle=Bsquare](-1.345, -0.776){n32}
\dotnode[dotstyle=Bsquare](-0.229, -0.402){n33}
\dotnode[dotstyle=Bsquare](0.908, -0.752){n34}
\dotnode[dotstyle=Bsquare](-0.043, -0.581){n35}
\dotnode[dotstyle=Bsquare](-0.235, -0.332){n36}
\dotnode[dotstyle=Bo](2.209, 0.443){n37}
\dotnode[dotstyle=Bsquare](-0.813, -0.371){n38}
\dotnode[dotstyle=Bsquare](-1.526, -0.375){n39}
\dotnode[dotstyle=Bsquare](-0.660, -0.352){n40}
\dotnode[dotstyle=Bsquare](-0.890, -0.034){n41}
\dotnode[dotstyle=Btriangle](-2.165, 0.215){n42}
\dotnode[dotstyle=Btriangle](-2.932, 0.352){n43}
\dotnode[dotstyle=Bsquare](-0.356, -0.503){n44}
\dotnode[dotstyle=Bo](2.787, -0.228){n45}
\dotnode[dotstyle=Btriangle](-2.616, 0.342){n46}
\dotnode[dotstyle=Bsquare](0.307, -0.365){n47}
\dotnode[dotstyle=Bo](2.543, 0.586){n48}
\dotnode[dotstyle=Bo](2.199, 0.879){n49}
\dotnode[dotstyle=Bsquare](-0.375, -0.292){n50}
\dotnode[dotstyle=Bsquare](-1.198, -0.606){n51}
\dotnode[dotstyle=Bo](2.982, -0.480){n52}
\dotnode[dotstyle=Btriangle](-2.532, -0.012){n53}
\dotnode[dotstyle=Bo](2.410, 0.418){n54}
\dotnode[dotstyle=Btriangle](-2.145, 0.139){n55}
\dotnode[dotstyle=Btriangle](-2.108, 0.371){n56}
\dotnode[dotstyle=Btriangle](-3.077, 0.686){n57}
\dotnode[dotstyle=Bsquare](-0.921, -0.182){n58}
\dotnode[dotstyle=Bsquare](0.175, -0.252){n59}
\dotnode[dotstyle=Btriangle](-1.780, -0.501){n60}
\dotnode[dotstyle=Btriangle](-1.802, -0.216){n61}
\dotnode[dotstyle=Bo](2.770, 0.271){n62}
\dotnode[dotstyle=Bo](2.303, 0.106){n63}
\dotnode[dotstyle=Bsquare](-0.245, -0.267){n64}
\dotnode[dotstyle=Bsquare](-0.587, -0.483){n65}
\dotnode[dotstyle=Bsquare](-1.585, -0.539){n66}
\dotnode[dotstyle=Bo](2.674, -0.107){n67}
\dotnode[dotstyle=Bo](3.216, 0.142){n68}
\dotnode[dotstyle=Bo](2.640, 0.319){n69}
\dotnode[dotstyle=Btriangle](-3.232, 1.371){n70}
\dotnode[dotstyle=Btriangle](-3.795, 0.253){n71}
\dotnode[dotstyle=Bsquare](-0.357, -0.067){n72}
\dotnode[dotstyle=Bo](2.625, 0.607){n73}
\dotnode[dotstyle=Bo](2.821, -0.082){n74}
\dotnode[dotstyle=Bo](2.633, -0.190){n75}
\dotnode[dotstyle=Bo](2.888, -0.571){n76}
\dotnode[dotstyle=Bsquare](-1.258, -0.179){n77}
\dotnode[dotstyle=Bo](2.716, -0.243){n78}
\dotnode[dotstyle=Bsquare](-0.812, -0.162){n79}
\dotnode[dotstyle=Btriangle](-1.902, 0.116){n80}
\dotnode[dotstyle=Bsquare](0.707, -1.008){n81}
\dotnode[dotstyle=Bsquare](-0.932, 0.319){n82}
\dotnode[dotstyle=Btriangle](-1.971, -0.181){n83}
\dotnode[dotstyle=Btriangle](-1.557, 0.267){n84}
\dotnode[dotstyle=Bsquare](0.511, -1.262){n85}
\dotnode[dotstyle=Bsquare](-1.116, -0.084){n86}
\dotnode[dotstyle=Btriangle](-2.419, 0.304){n87}
\dotnode[dotstyle=Bo](2.280, 0.748){n88}
\dotnode[dotstyle=Bsquare](-1.388, -0.204){n89}
\dotnode[dotstyle=Bo](2.613, 0.022){n90}
\dotnode[dotstyle=Bo](2.998, -0.334){n91}
\dotnode[dotstyle=Btriangle](-1.905, 0.119){n92}
\dotnode[dotstyle=Bsquare](-0.942, -0.542){n93}
\dotnode[dotstyle=Bsquare](-1.298, -0.761){n94}
\dotnode[dotstyle=Btriangle](-2.841, 0.373){n95}
\dotnode[dotstyle=Bsquare](0.751, -1.001){n96}
\dotnode[dotstyle=Btriangle](-1.285, 0.685){n97}
\dotnode[dotstyle=Bsquare](0.192, -0.677){n98}
\dotnode[dotstyle=Bsquare](-1.095, 0.284){n99}
\dotnode[dotstyle=Bsquare](-1.331, 0.245){n100}
\dotnode[dotstyle=Bsquare](-0.984, -0.124){n101}
\dotnode[dotstyle=Btriangle](-1.662, 0.242){n102}
\dotnode[dotstyle=Btriangle](-2.276, 0.333){n103}
\dotnode[dotstyle=Bsquare](-0.928, 0.468){n104}
\dotnode[dotstyle=Bo](2.356, -0.031){n105}
\dotnode[dotstyle=Bo](2.715, -0.170){n106}
\dotnode[dotstyle=Btriangle](-2.428, 0.377){n107}
\dotnode[dotstyle=Bo](2.852, -0.933){n108}
\dotnode[dotstyle=Bo](3.225, -0.503){n109}
\dotnode[dotstyle=Bsquare](0.010, -0.721){n110}
\dotnode[dotstyle=Bo](2.729, 0.334){n111}
\dotnode[dotstyle=Bsquare](-0.714, 0.150){n112}
\dotnode[dotstyle=Bo](2.562, 0.375){n113}
\dotnode[dotstyle=Bsquare](-0.135, -0.312){n114}
\dotnode[dotstyle=Bo](2.597, 1.100){n115}
\dotnode[dotstyle=Btriangle](-3.397, 0.547){n116}
\dotnode[dotstyle=Bsquare](-1.443, -0.144){n117}
\dotnode[dotstyle=Btriangle](-1.905, 0.048){n118}
\dotnode[dotstyle=Bsquare](-1.296, -0.328){n119}
\dotnode[dotstyle=Bsquare](-1.414, -0.575){n120}
\dotnode[dotstyle=Bo](2.508, -0.139){n121}
\dotnode[dotstyle=Bsquare](-1.220, 0.408){n122}
\dotnode[dotstyle=Bsquare](-1.379, -0.421){n123}
\dotnode[dotstyle=Bo](2.684, 0.327){n124}
\dotnode[dotstyle=Bo](2.588, -0.197){n125}
\dotnode[dotstyle=Bsquare](-0.640, -0.417){n126}
\dotnode[dotstyle=Bo](2.538, 0.510){n127}
\dotnode[dotstyle=Bsquare](-0.900, 0.330){n128}
\dotnode[dotstyle=Btriangle](-2.123, -0.211){n129}
\dotnode[dotstyle=Bo](2.648, 0.820){n130}
\dotnode[dotstyle=Btriangle](-2.159, -0.218){n131}
\dotnode[dotstyle=Btriangle](-3.499, 0.457){n132}
\dotnode[dotstyle=Bo](2.644, 1.186){n133}
\dotnode[dotstyle=Bo](2.674, -0.107){n134}
\dotnode[dotstyle=Bo](2.507, 0.652){n135}
\dotnode[dotstyle=Bo](2.648, 0.319){n136}
\dotnode[dotstyle=Bsquare](-0.807, 0.195){n137}
\dotnode[dotstyle=Btriangle](-1.949, 0.041){n138}
\dotnode[dotstyle=Bsquare](0.070, -0.703){n139}
\dotnode[dotstyle=Btriangle](-2.918, 0.780){n140}
\dotnode[dotstyle=Btriangle](-1.922, 0.409){n141}
\dotnode[dotstyle=Bsquare](-0.642, 0.019){n142}
\dotnode[dotstyle=Bsquare](-1.087, 0.075){n143}
\dotnode[dotstyle=Bsquare](-1.169, -0.165){n144}
\dotnode[dotstyle=Bo](2.311, 0.398){n145}
\dotnode[dotstyle=Bsquare](-0.463, -0.670){n146}
\dotnode[dotstyle=Btriangle](-1.944, 0.187){n147}
\dotnode[dotstyle=Btriangle](-3.489, 1.172){n148}
\dotnode[dotstyle=Bo](2.590, 0.236){n149}

        \endpsgraph
    }}
\end{figure}
Only intracluster edges shown.
\end{frame}

\begin{frame}[fragile]{Clusterings as Graphs: Iris (Bad Case)}
\setcounter{subfigure}{0}
\begin{figure}
    \captionsetup[subfloat]{captionskip=20pt}
    \def\pshlabel#1{ {\footnotesize $#1$}}
    \def\psvlabel#1{ {\footnotesize $#1$}}
    \psset{xAxisLabel=$\bu_1$,yAxisLabel= $\bu_2$}
    \psset{xunit=0.5in,yunit=0.65in,dotscale=1.5,arrowscale=2,PointName=none}
    \centerline{
	\scalebox{0.55}{
        %\pspicture[](-5,-2)(3.5,1.5)
        %\psaxes[tickstyle=bottom,Dx=1,Ox=-4,Dy=0.5,Oy=-1.5]{->}(-4,-1.5)(3.5,1.5)
        \psgraph[tickstyle=bottom,Dx=1,Ox=-4,Dy=0.5,Oy=-1.5]{->}%
        (-4,-1.5)(3.5,1.5){3.5in}{2in}
        \psset{dotstyle=Bsquare,fillcolor=lightgray}
        \psdot[](2.54,0.44)
\psdot[](2.59,0.52)
\psdot[](2.47,0.14)
\psdot[](2.63,0.17)
\psdot[](2.70,0.12)
\psdot[](2.38,1.34)
\psdot[](2.41,0.20)
\psdot[](2.62,0.82)
\psdot[](2.87,0.08)
\psdot[](2.21,0.44)
\psdot[](2.54,0.59)
\psdot[](2.20,0.88)
\psdot[](2.41,0.42)
\psdot[](2.77,0.27)
\psdot[](2.30,0.11)
\psdot[](3.22,0.14)
\psdot[](2.64,0.32)
\psdot[](2.63,0.61)
\psdot[](2.28,0.75)
\psdot[](2.73,0.33)
\psdot[](2.56,0.37)
\psdot[](2.60,1.10)
\psdot[](2.68,0.33)
\psdot[](2.54,0.51)
\psdot[](2.65,0.82)
\psdot[](2.64,1.19)
\psdot[](2.51,0.65)
\psdot[](2.65,0.32)
\psdot[](2.31,0.40)
\psdot[](2.59,0.24)

        \psset{fillcolor=white}
        \input{CLUST/eval/irisPCbad-W1}
        \psset{dotstyle=Bo,fillcolor=lightgray}
        \psdot[](2.84,-0.22)
\psdot[](2.89,-0.14)
\psdot[](2.67,-0.11)
\psdot[](2.75,-0.31)
\psdot[](2.79,-0.23)
\psdot[](2.98,-0.48)
\psdot[](2.67,-0.11)
\psdot[](2.82,-0.08)
\psdot[](2.63,-0.19)
\psdot[](2.89,-0.57)
\psdot[](2.72,-0.24)
\psdot[](2.61,0.02)
\psdot[](3.00,-0.33)
\psdot[](2.36,-0.03)
\psdot[](2.72,-0.17)
\psdot[](2.85,-0.93)
\psdot[](3.23,-0.50)
\psdot[](2.51,-0.14)
\psdot[](2.59,-0.20)
\psdot[](2.67,-0.11)

        \psset{fillcolor=white}
        \psdot[](0.91,-0.75)
\psdot[](0.71,-1.01)
\psdot[](0.51,-1.26)
\psdot[](0.75,-1.00)

        \psset{dotstyle=Btriangle,fillcolor=lightgray}
        \input{CLUST/eval/irisPCbad-C3}
        \psset{fillcolor=white}
        \input{CLUST/eval/irisPCbad-W3}
        \psset{fillcolor=black}
        \pstGeonode[PointSymbol=Bsquare, dotscale=2](2.562,0.486){A}
        \pstGeonode[PointSymbol=Bo,
        dotscale=2](2.419,-0.379){B}
        \pstGeonode[PointSymbol=Btriangle,
        dotscale=2](-1.405,-0.057){C}
        \psclip{\psframe[](-4,-1.5)(3.5,1.5)}%
        {
        \psset{linestyle=none, PointSymbol=none}
        \pstMediatorAB{A}{B}{K}{KP}
        \pstMediatorAB{C}{A}{J}{JP}
        \pstMediatorAB{B}{C}{I}{IP}
        \pstInterLL[PointSymbol=none]{I}{IP}{J}{JP}{O}
        \psset{linewidth=1pt,linestyle=dashed}
        \pstGeonode[PointSymbol=none](-4,-1.5){a}(-4,1.5){b}(3.5,1.5){c}(3.5,-1.5){d}
        \pstInterLL[PointSymbol=none]{O}{I}{a}{b}{oi}
        \pstLineAB{O}{oi}
        \pstInterLL[PointSymbol=none]{O}{J}{b}{c}{oj}
        \pstLineAB{O}{oj}
        \pstInterLL[PointSymbol=none]{O}{K}{a}{d}{ok}
        \pstLineAB{O}{ok}
        }
        \endpsclip
        %\endpspicture
        \endpsgraph
    }
	\hspace{0.2in}
	\scalebox{0.55}{
        %\pspicture[](-5,-2)(3.5,1.5)
        %\psaxes[tickstyle=bottom,Dx=1,Ox=-4,Dy=0.5,Oy=-1.5]{->}(-4,-1.5)(3.5,1.5)
        \psgraph[tickstyle=bottom,Dx=1,Ox=-4,Dy=0.5,Oy=-1.5]{->}%
        (-4,-1.5)(3.5,1.5){3.5in}{2in}
        \pnode(-0.511, -0.102){n0}
\pnode(-1.464, 0.504){n1}
\pnode(-1.043, 0.229){n2}
\pnode(2.840, -0.221){n3}
\pnode(-0.262, -0.548){n4}
\pnode(2.890, -0.137){n5}
\pnode(-2.350, -0.042){n6}
\pnode(-1.414, -0.575){n7}
\pnode(-2.314, 0.183){n8}
\pnode(-2.320, -0.246){n9}
\pnode(2.543, 0.440){n10}
\pnode(2.587, 0.520){n11}
\pnode(-0.331, -0.211){n12}
\pnode(-1.291, -0.116){n13}
\pnode(2.674, -0.107){n14}
\pnode(2.469, 0.138){n15}
\pnode(2.626, 0.170){n16}
\pnode(-0.298, -0.347){n17}
\pnode(2.704, 0.115){n18}
\pnode(-2.614, 0.558){n19}
\pnode(-1.390, -0.283){n20}
\pnode(-1.764, 0.079){n21}
\pnode(2.384, 1.345){n22}
\pnode(-0.165, -0.680){n23}
\pnode(-0.519, -1.191){n24}
\pnode(2.406, 0.196){n25}
\pnode(-0.181, -0.826){n26}
\pnode(2.746, -0.311){n27}
\pnode(-2.388, 0.463){n28}
\pnode(-2.563, 0.276){n29}
\pnode(2.623, 0.818){n30}
\pnode(2.867, 0.077){n31}
\pnode(-1.345, -0.776){n32}
\pnode(-0.229, -0.402){n33}
\pnode(0.908, -0.752){n34}
\pnode(-0.043, -0.581){n35}
\pnode(-0.235, -0.332){n36}
\pnode(2.209, 0.443){n37}
\pnode(-0.813, -0.371){n38}
\pnode(-1.526, -0.375){n39}
\pnode(-0.660, -0.352){n40}
\pnode(-0.890, -0.034){n41}
\pnode(-2.165, 0.215){n42}
\pnode(-2.932, 0.352){n43}
\pnode(-0.356, -0.503){n44}
\pnode(2.787, -0.228){n45}
\pnode(-2.616, 0.342){n46}
\pnode(0.307, -0.365){n47}
\pnode(2.543, 0.586){n48}
\pnode(2.199, 0.879){n49}
\pnode(-0.375, -0.292){n50}
\pnode(-1.198, -0.606){n51}
\pnode(2.982, -0.480){n52}
\pnode(-2.532, -0.012){n53}
\pnode(2.410, 0.418){n54}
\pnode(-2.145, 0.139){n55}
\pnode(-2.108, 0.371){n56}
\pnode(-3.077, 0.686){n57}
\pnode(-0.921, -0.182){n58}
\pnode(0.175, -0.252){n59}
\pnode(-1.780, -0.501){n60}
\pnode(-1.802, -0.216){n61}
\pnode(2.770, 0.271){n62}
\pnode(2.303, 0.106){n63}
\pnode(-0.245, -0.267){n64}
\pnode(-0.587, -0.483){n65}
\pnode(-1.585, -0.539){n66}
\pnode(2.674, -0.107){n67}
\pnode(3.216, 0.142){n68}
\pnode(2.640, 0.319){n69}
\pnode(-3.232, 1.371){n70}
\pnode(-3.795, 0.253){n71}
\pnode(-0.357, -0.067){n72}
\pnode(2.625, 0.607){n73}
\pnode(2.821, -0.082){n74}
\pnode(2.633, -0.190){n75}
\pnode(2.888, -0.571){n76}
\pnode(-1.258, -0.179){n77}
\pnode(2.716, -0.243){n78}
\pnode(-0.812, -0.162){n79}
\pnode(-1.902, 0.116){n80}
\pnode(0.707, -1.008){n81}
\pnode(-0.932, 0.319){n82}
\pnode(-1.971, -0.181){n83}
\pnode(-1.557, 0.267){n84}
\pnode(0.511, -1.262){n85}
\pnode(-1.116, -0.084){n86}
\pnode(-2.419, 0.304){n87}
\pnode(2.280, 0.748){n88}
\pnode(-1.388, -0.204){n89}
\pnode(2.613, 0.022){n90}
\pnode(2.998, -0.334){n91}
\pnode(-1.905, 0.119){n92}
\pnode(-0.942, -0.542){n93}
\pnode(-1.298, -0.761){n94}
\pnode(-2.841, 0.373){n95}
\pnode(0.751, -1.001){n96}
\pnode(-1.285, 0.685){n97}
\pnode(0.192, -0.677){n98}
\pnode(-1.095, 0.284){n99}
\pnode(-1.331, 0.245){n100}
\pnode(-0.984, -0.124){n101}
\pnode(-1.662, 0.242){n102}
\pnode(-2.276, 0.333){n103}
\pnode(-0.928, 0.468){n104}
\pnode(2.356, -0.031){n105}
\pnode(2.715, -0.170){n106}
\pnode(-2.428, 0.377){n107}
\pnode(2.852, -0.933){n108}
\pnode(3.225, -0.503){n109}
\pnode(0.010, -0.721){n110}
\pnode(2.729, 0.334){n111}
\pnode(-0.714, 0.150){n112}
\pnode(2.562, 0.375){n113}
\pnode(-0.135, -0.312){n114}
\pnode(2.597, 1.100){n115}
\pnode(-3.397, 0.547){n116}
\pnode(-1.443, -0.144){n117}
\pnode(-1.905, 0.048){n118}
\pnode(-1.296, -0.328){n119}
\pnode(-1.414, -0.575){n120}
\pnode(2.508, -0.139){n121}
\pnode(-1.220, 0.408){n122}
\pnode(-1.379, -0.421){n123}
\pnode(2.684, 0.327){n124}
\pnode(2.588, -0.197){n125}
\pnode(-0.640, -0.417){n126}
\pnode(2.538, 0.510){n127}
\pnode(-0.900, 0.330){n128}
\pnode(-2.123, -0.211){n129}
\pnode(2.648, 0.820){n130}
\pnode(-2.159, -0.218){n131}
\pnode(-3.499, 0.457){n132}
\pnode(2.644, 1.186){n133}
\pnode(2.674, -0.107){n134}
\pnode(2.507, 0.652){n135}
\pnode(2.648, 0.319){n136}
\pnode(-0.807, 0.195){n137}
\pnode(-1.949, 0.041){n138}
\pnode(0.070, -0.703){n139}
\pnode(-2.918, 0.780){n140}
\pnode(-1.922, 0.409){n141}
\pnode(-0.642, 0.019){n142}
\pnode(-1.087, 0.075){n143}
\pnode(-1.169, -0.165){n144}
\pnode(2.311, 0.398){n145}
\pnode(-0.463, -0.670){n146}
\pnode(-1.944, 0.187){n147}
\pnode(-3.489, 1.172){n148}
\pnode(2.590, 0.236){n149}
\psset{linewidth=0.01pt,linecolor=lightgray}
\ncline[]{n0}{n1}
\ncline[]{n0}{n2}
\ncline[]{n0}{n4}
\ncline[]{n0}{n6}
\ncline[]{n0}{n7}
\ncline[]{n0}{n8}
\ncline[]{n0}{n9}
\ncline[]{n0}{n12}
\ncline[]{n0}{n13}
\ncline[]{n0}{n17}
\ncline[]{n0}{n19}
\ncline[]{n0}{n20}
\ncline[]{n0}{n21}
\ncline[]{n0}{n23}
\ncline[]{n0}{n24}
\ncline[]{n0}{n26}
\ncline[]{n0}{n28}
\ncline[]{n0}{n29}
\ncline[]{n0}{n32}
\ncline[]{n0}{n33}
\ncline[]{n0}{n35}
\ncline[]{n0}{n36}
\ncline[]{n0}{n38}
\ncline[]{n0}{n39}
\ncline[]{n0}{n40}
\ncline[]{n0}{n41}
\ncline[]{n0}{n42}
\ncline[]{n0}{n43}
\ncline[]{n0}{n44}
\ncline[]{n0}{n46}
\ncline[]{n0}{n47}
\ncline[]{n0}{n50}
\ncline[]{n0}{n51}
\ncline[]{n0}{n53}
\ncline[]{n0}{n55}
\ncline[]{n0}{n56}
\ncline[]{n0}{n57}
\ncline[]{n0}{n58}
\ncline[]{n0}{n59}
\ncline[]{n0}{n60}
\ncline[]{n0}{n61}
\ncline[]{n0}{n64}
\ncline[]{n0}{n65}
\ncline[]{n0}{n66}
\ncline[]{n0}{n70}
\ncline[]{n0}{n71}
\ncline[]{n0}{n72}
\ncline[]{n0}{n77}
\ncline[]{n0}{n79}
\ncline[]{n0}{n80}
\ncline[]{n0}{n82}
\ncline[]{n0}{n83}
\ncline[]{n0}{n84}
\ncline[]{n0}{n86}
\ncline[]{n0}{n87}
\ncline[]{n0}{n89}
\ncline[]{n0}{n92}
\ncline[]{n0}{n93}
\ncline[]{n0}{n94}
\ncline[]{n0}{n95}
\ncline[]{n0}{n97}
\ncline[]{n0}{n98}
\ncline[]{n0}{n99}
\ncline[]{n0}{n100}
\ncline[]{n0}{n101}
\ncline[]{n0}{n102}
\ncline[]{n0}{n103}
\ncline[]{n0}{n104}
\ncline[]{n0}{n107}
\ncline[]{n0}{n110}
\ncline[]{n0}{n112}
\ncline[]{n0}{n114}
\ncline[]{n0}{n116}
\ncline[]{n0}{n117}
\ncline[]{n0}{n118}
\ncline[]{n0}{n119}
\ncline[]{n0}{n120}
\ncline[]{n0}{n122}
\ncline[]{n0}{n123}
\ncline[]{n0}{n126}
\ncline[]{n0}{n128}
\ncline[]{n0}{n129}
\ncline[]{n0}{n131}
\ncline[]{n0}{n132}
\ncline[]{n0}{n137}
\ncline[]{n0}{n138}
\ncline[]{n0}{n139}
\ncline[]{n0}{n140}
\ncline[]{n0}{n141}
\ncline[]{n0}{n142}
\ncline[]{n0}{n143}
\ncline[]{n0}{n144}
\ncline[]{n0}{n146}
\ncline[]{n0}{n147}
\ncline[]{n0}{n148}
\ncline[]{n1}{n2}
\ncline[]{n1}{n4}
\ncline[]{n1}{n6}
\ncline[]{n1}{n7}
\ncline[]{n1}{n8}
\ncline[]{n1}{n9}
\ncline[]{n1}{n12}
\ncline[]{n1}{n13}
\ncline[]{n1}{n17}
\ncline[]{n1}{n19}
\ncline[]{n1}{n20}
\ncline[]{n1}{n21}
\ncline[]{n1}{n23}
\ncline[]{n1}{n24}
\ncline[]{n1}{n26}
\ncline[]{n1}{n28}
\ncline[]{n1}{n29}
\ncline[]{n1}{n32}
\ncline[]{n1}{n33}
\ncline[]{n1}{n35}
\ncline[]{n1}{n36}
\ncline[]{n1}{n38}
\ncline[]{n1}{n39}
\ncline[]{n1}{n40}
\ncline[]{n1}{n41}
\ncline[]{n1}{n42}
\ncline[]{n1}{n43}
\ncline[]{n1}{n44}
\ncline[]{n1}{n46}
\ncline[]{n1}{n47}
\ncline[]{n1}{n50}
\ncline[]{n1}{n51}
\ncline[]{n1}{n53}
\ncline[]{n1}{n55}
\ncline[]{n1}{n56}
\ncline[]{n1}{n57}
\ncline[]{n1}{n58}
\ncline[]{n1}{n59}
\ncline[]{n1}{n60}
\ncline[]{n1}{n61}
\ncline[]{n1}{n64}
\ncline[]{n1}{n65}
\ncline[]{n1}{n66}
\ncline[]{n1}{n70}
\ncline[]{n1}{n71}
\ncline[]{n1}{n72}
\ncline[]{n1}{n77}
\ncline[]{n1}{n79}
\ncline[]{n1}{n80}
\ncline[]{n1}{n82}
\ncline[]{n1}{n83}
\ncline[]{n1}{n84}
\ncline[]{n1}{n86}
\ncline[]{n1}{n87}
\ncline[]{n1}{n89}
\ncline[]{n1}{n92}
\ncline[]{n1}{n93}
\ncline[]{n1}{n94}
\ncline[]{n1}{n95}
\ncline[]{n1}{n97}
\ncline[]{n1}{n98}
\ncline[]{n1}{n99}
\ncline[]{n1}{n100}
\ncline[]{n1}{n101}
\ncline[]{n1}{n102}
\ncline[]{n1}{n103}
\ncline[]{n1}{n104}
\ncline[]{n1}{n107}
\ncline[]{n1}{n110}
\ncline[]{n1}{n112}
\ncline[]{n1}{n114}
\ncline[]{n1}{n116}
\ncline[]{n1}{n117}
\ncline[]{n1}{n118}
\ncline[]{n1}{n119}
\ncline[]{n1}{n120}
\ncline[]{n1}{n122}
\ncline[]{n1}{n123}
\ncline[]{n1}{n126}
\ncline[]{n1}{n128}
\ncline[]{n1}{n129}
\ncline[]{n1}{n131}
\ncline[]{n1}{n132}
\ncline[]{n1}{n137}
\ncline[]{n1}{n138}
\ncline[]{n1}{n139}
\ncline[]{n1}{n140}
\ncline[]{n1}{n141}
\ncline[]{n1}{n142}
\ncline[]{n1}{n143}
\ncline[]{n1}{n144}
\ncline[]{n1}{n146}
\ncline[]{n1}{n147}
\ncline[]{n1}{n148}
\ncline[]{n2}{n4}
\ncline[]{n2}{n6}
\ncline[]{n2}{n7}
\ncline[]{n2}{n8}
\ncline[]{n2}{n9}
\ncline[]{n2}{n12}
\ncline[]{n2}{n13}
\ncline[]{n2}{n17}
\ncline[]{n2}{n19}
\ncline[]{n2}{n20}
\ncline[]{n2}{n21}
\ncline[]{n2}{n23}
\ncline[]{n2}{n24}
\ncline[]{n2}{n26}
\ncline[]{n2}{n28}
\ncline[]{n2}{n29}
\ncline[]{n2}{n32}
\ncline[]{n2}{n33}
\ncline[]{n2}{n35}
\ncline[]{n2}{n36}
\ncline[]{n2}{n38}
\ncline[]{n2}{n39}
\ncline[]{n2}{n40}
\ncline[]{n2}{n41}
\ncline[]{n2}{n42}
\ncline[]{n2}{n43}
\ncline[]{n2}{n44}
\ncline[]{n2}{n46}
\ncline[]{n2}{n47}
\ncline[]{n2}{n50}
\ncline[]{n2}{n51}
\ncline[]{n2}{n53}
\ncline[]{n2}{n55}
\ncline[]{n2}{n56}
\ncline[]{n2}{n57}
\ncline[]{n2}{n58}
\ncline[]{n2}{n59}
\ncline[]{n2}{n60}
\ncline[]{n2}{n61}
\ncline[]{n2}{n64}
\ncline[]{n2}{n65}
\ncline[]{n2}{n66}
\ncline[]{n2}{n70}
\ncline[]{n2}{n71}
\ncline[]{n2}{n72}
\ncline[]{n2}{n77}
\ncline[]{n2}{n79}
\ncline[]{n2}{n80}
\ncline[]{n2}{n82}
\ncline[]{n2}{n83}
\ncline[]{n2}{n84}
\ncline[]{n2}{n86}
\ncline[]{n2}{n87}
\ncline[]{n2}{n89}
\ncline[]{n2}{n92}
\ncline[]{n2}{n93}
\ncline[]{n2}{n94}
\ncline[]{n2}{n95}
\ncline[]{n2}{n97}
\ncline[]{n2}{n98}
\ncline[]{n2}{n99}
\ncline[]{n2}{n100}
\ncline[]{n2}{n101}
\ncline[]{n2}{n102}
\ncline[]{n2}{n103}
\ncline[]{n2}{n104}
\ncline[]{n2}{n107}
\ncline[]{n2}{n110}
\ncline[]{n2}{n112}
\ncline[]{n2}{n114}
\ncline[]{n2}{n116}
\ncline[]{n2}{n117}
\ncline[]{n2}{n118}
\ncline[]{n2}{n119}
\ncline[]{n2}{n120}
\ncline[]{n2}{n122}
\ncline[]{n2}{n123}
\ncline[]{n2}{n126}
\ncline[]{n2}{n128}
\ncline[]{n2}{n129}
\ncline[]{n2}{n131}
\ncline[]{n2}{n132}
\ncline[]{n2}{n137}
\ncline[]{n2}{n138}
\ncline[]{n2}{n139}
\ncline[]{n2}{n140}
\ncline[]{n2}{n141}
\ncline[]{n2}{n142}
\ncline[]{n2}{n143}
\ncline[]{n2}{n144}
\ncline[]{n2}{n146}
\ncline[]{n2}{n147}
\ncline[]{n2}{n148}
\ncline[]{n3}{n5}
\ncline[]{n3}{n14}
\ncline[]{n3}{n27}
\ncline[]{n3}{n34}
\ncline[]{n3}{n45}
\ncline[]{n3}{n52}
\ncline[]{n3}{n67}
\ncline[]{n3}{n74}
\ncline[]{n3}{n75}
\ncline[]{n3}{n76}
\ncline[]{n3}{n78}
\ncline[]{n3}{n81}
\ncline[]{n3}{n85}
\ncline[]{n3}{n90}
\ncline[]{n3}{n91}
\ncline[]{n3}{n96}
\ncline[]{n3}{n105}
\ncline[]{n3}{n106}
\ncline[]{n3}{n108}
\ncline[]{n3}{n109}
\ncline[]{n3}{n121}
\ncline[]{n3}{n125}
\ncline[]{n3}{n134}
\ncline[]{n4}{n6}
\ncline[]{n4}{n7}
\ncline[]{n4}{n8}
\ncline[]{n4}{n9}
\ncline[]{n4}{n12}
\ncline[]{n4}{n13}
\ncline[]{n4}{n17}
\ncline[]{n4}{n19}
\ncline[]{n4}{n20}
\ncline[]{n4}{n21}
\ncline[]{n4}{n23}
\ncline[]{n4}{n24}
\ncline[]{n4}{n26}
\ncline[]{n4}{n28}
\ncline[]{n4}{n29}
\ncline[]{n4}{n32}
\ncline[]{n4}{n33}
\ncline[]{n4}{n35}
\ncline[]{n4}{n36}
\ncline[]{n4}{n38}
\ncline[]{n4}{n39}
\ncline[]{n4}{n40}
\ncline[]{n4}{n41}
\ncline[]{n4}{n42}
\ncline[]{n4}{n43}
\ncline[]{n4}{n44}
\ncline[]{n4}{n46}
\ncline[]{n4}{n47}
\ncline[]{n4}{n50}
\ncline[]{n4}{n51}
\ncline[]{n4}{n53}
\ncline[]{n4}{n55}
\ncline[]{n4}{n56}
\ncline[]{n4}{n57}
\ncline[]{n4}{n58}
\ncline[]{n4}{n59}
\ncline[]{n4}{n60}
\ncline[]{n4}{n61}
\ncline[]{n4}{n64}
\ncline[]{n4}{n65}
\ncline[]{n4}{n66}
\ncline[]{n4}{n70}
\ncline[]{n4}{n71}
\ncline[]{n4}{n72}
\ncline[]{n4}{n77}
\ncline[]{n4}{n79}
\ncline[]{n4}{n80}
\ncline[]{n4}{n82}
\ncline[]{n4}{n83}
\ncline[]{n4}{n84}
\ncline[]{n4}{n86}
\ncline[]{n4}{n87}
\ncline[]{n4}{n89}
\ncline[]{n4}{n92}
\ncline[]{n4}{n93}
\ncline[]{n4}{n94}
\ncline[]{n4}{n95}
\ncline[]{n4}{n97}
\ncline[]{n4}{n98}
\ncline[]{n4}{n99}
\ncline[]{n4}{n100}
\ncline[]{n4}{n101}
\ncline[]{n4}{n102}
\ncline[]{n4}{n103}
\ncline[]{n4}{n104}
\ncline[]{n4}{n107}
\ncline[]{n4}{n110}
\ncline[]{n4}{n112}
\ncline[]{n4}{n114}
\ncline[]{n4}{n116}
\ncline[]{n4}{n117}
\ncline[]{n4}{n118}
\ncline[]{n4}{n119}
\ncline[]{n4}{n120}
\ncline[]{n4}{n122}
\ncline[]{n4}{n123}
\ncline[]{n4}{n126}
\ncline[]{n4}{n128}
\ncline[]{n4}{n129}
\ncline[]{n4}{n131}
\ncline[]{n4}{n132}
\ncline[]{n4}{n137}
\ncline[]{n4}{n138}
\ncline[]{n4}{n139}
\ncline[]{n4}{n140}
\ncline[]{n4}{n141}
\ncline[]{n4}{n142}
\ncline[]{n4}{n143}
\ncline[]{n4}{n144}
\ncline[]{n4}{n146}
\ncline[]{n4}{n147}
\ncline[]{n4}{n148}
\ncline[]{n5}{n14}
\ncline[]{n5}{n27}
\ncline[]{n5}{n34}
\ncline[]{n5}{n45}
\ncline[]{n5}{n52}
\ncline[]{n5}{n67}
\ncline[]{n5}{n74}
\ncline[]{n5}{n75}
\ncline[]{n5}{n76}
\ncline[]{n5}{n78}
\ncline[]{n5}{n81}
\ncline[]{n5}{n85}
\ncline[]{n5}{n90}
\ncline[]{n5}{n91}
\ncline[]{n5}{n96}
\ncline[]{n5}{n105}
\ncline[]{n5}{n106}
\ncline[]{n5}{n108}
\ncline[]{n5}{n109}
\ncline[]{n5}{n121}
\ncline[]{n5}{n125}
\ncline[]{n5}{n134}
\ncline[]{n6}{n7}
\ncline[]{n6}{n8}
\ncline[]{n6}{n9}
\ncline[]{n6}{n12}
\ncline[]{n6}{n13}
\ncline[]{n6}{n17}
\ncline[]{n6}{n19}
\ncline[]{n6}{n20}
\ncline[]{n6}{n21}
\ncline[]{n6}{n23}
\ncline[]{n6}{n24}
\ncline[]{n6}{n26}
\ncline[]{n6}{n28}
\ncline[]{n6}{n29}
\ncline[]{n6}{n32}
\ncline[]{n6}{n33}
\ncline[]{n6}{n35}
\ncline[]{n6}{n36}
\ncline[]{n6}{n38}
\ncline[]{n6}{n39}
\ncline[]{n6}{n40}
\ncline[]{n6}{n41}
\ncline[]{n6}{n42}
\ncline[]{n6}{n43}
\ncline[]{n6}{n44}
\ncline[]{n6}{n46}
\ncline[]{n6}{n47}
\ncline[]{n6}{n50}
\ncline[]{n6}{n51}
\ncline[]{n6}{n53}
\ncline[]{n6}{n55}
\ncline[]{n6}{n56}
\ncline[]{n6}{n57}
\ncline[]{n6}{n58}
\ncline[]{n6}{n59}
\ncline[]{n6}{n60}
\ncline[]{n6}{n61}
\ncline[]{n6}{n64}
\ncline[]{n6}{n65}
\ncline[]{n6}{n66}
\ncline[]{n6}{n70}
\ncline[]{n6}{n71}
\ncline[]{n6}{n72}
\ncline[]{n6}{n77}
\ncline[]{n6}{n79}
\ncline[]{n6}{n80}
\ncline[]{n6}{n82}
\ncline[]{n6}{n83}
\ncline[]{n6}{n84}
\ncline[]{n6}{n86}
\ncline[]{n6}{n87}
\ncline[]{n6}{n89}
\ncline[]{n6}{n92}
\ncline[]{n6}{n93}
\ncline[]{n6}{n94}
\ncline[]{n6}{n95}
\ncline[]{n6}{n97}
\ncline[]{n6}{n98}
\ncline[]{n6}{n99}
\ncline[]{n6}{n100}
\ncline[]{n6}{n101}
\ncline[]{n6}{n102}
\ncline[]{n6}{n103}
\ncline[]{n6}{n104}
\ncline[]{n6}{n107}
\ncline[]{n6}{n110}
\ncline[]{n6}{n112}
\ncline[]{n6}{n114}
\ncline[]{n6}{n116}
\ncline[]{n6}{n117}
\ncline[]{n6}{n118}
\ncline[]{n6}{n119}
\ncline[]{n6}{n120}
\ncline[]{n6}{n122}
\ncline[]{n6}{n123}
\ncline[]{n6}{n126}
\ncline[]{n6}{n128}
\ncline[]{n6}{n129}
\ncline[]{n6}{n131}
\ncline[]{n6}{n132}
\ncline[]{n6}{n137}
\ncline[]{n6}{n138}
\ncline[]{n6}{n139}
\ncline[]{n6}{n140}
\ncline[]{n6}{n141}
\ncline[]{n6}{n142}
\ncline[]{n6}{n143}
\ncline[]{n6}{n144}
\ncline[]{n6}{n146}
\ncline[]{n6}{n147}
\ncline[]{n6}{n148}
\ncline[]{n7}{n8}
\ncline[]{n7}{n9}
\ncline[]{n7}{n12}
\ncline[]{n7}{n13}
\ncline[]{n7}{n17}
\ncline[]{n7}{n19}
\ncline[]{n7}{n20}
\ncline[]{n7}{n21}
\ncline[]{n7}{n23}
\ncline[]{n7}{n24}
\ncline[]{n7}{n26}
\ncline[]{n7}{n28}
\ncline[]{n7}{n29}
\ncline[]{n7}{n32}
\ncline[]{n7}{n33}
\ncline[]{n7}{n35}
\ncline[]{n7}{n36}
\ncline[]{n7}{n38}
\ncline[]{n7}{n39}
\ncline[]{n7}{n40}
\ncline[]{n7}{n41}
\ncline[]{n7}{n42}
\ncline[]{n7}{n43}
\ncline[]{n7}{n44}
\ncline[]{n7}{n46}
\ncline[]{n7}{n47}
\ncline[]{n7}{n50}
\ncline[]{n7}{n51}
\ncline[]{n7}{n53}
\ncline[]{n7}{n55}
\ncline[]{n7}{n56}
\ncline[]{n7}{n57}
\ncline[]{n7}{n58}
\ncline[]{n7}{n59}
\ncline[]{n7}{n60}
\ncline[]{n7}{n61}
\ncline[]{n7}{n64}
\ncline[]{n7}{n65}
\ncline[]{n7}{n66}
\ncline[]{n7}{n70}
\ncline[]{n7}{n71}
\ncline[]{n7}{n72}
\ncline[]{n7}{n77}
\ncline[]{n7}{n79}
\ncline[]{n7}{n80}
\ncline[]{n7}{n82}
\ncline[]{n7}{n83}
\ncline[]{n7}{n84}
\ncline[]{n7}{n86}
\ncline[]{n7}{n87}
\ncline[]{n7}{n89}
\ncline[]{n7}{n92}
\ncline[]{n7}{n93}
\ncline[]{n7}{n94}
\ncline[]{n7}{n95}
\ncline[]{n7}{n97}
\ncline[]{n7}{n98}
\ncline[]{n7}{n99}
\ncline[]{n7}{n100}
\ncline[]{n7}{n101}
\ncline[]{n7}{n102}
\ncline[]{n7}{n103}
\ncline[]{n7}{n104}
\ncline[]{n7}{n107}
\ncline[]{n7}{n110}
\ncline[]{n7}{n112}
\ncline[]{n7}{n114}
\ncline[]{n7}{n116}
\ncline[]{n7}{n117}
\ncline[]{n7}{n118}
\ncline[]{n7}{n119}
\ncline[]{n7}{n120}
\ncline[]{n7}{n122}
\ncline[]{n7}{n123}
\ncline[]{n7}{n126}
\ncline[]{n7}{n128}
\ncline[]{n7}{n129}
\ncline[]{n7}{n131}
\ncline[]{n7}{n132}
\ncline[]{n7}{n137}
\ncline[]{n7}{n138}
\ncline[]{n7}{n139}
\ncline[]{n7}{n140}
\ncline[]{n7}{n141}
\ncline[]{n7}{n142}
\ncline[]{n7}{n143}
\ncline[]{n7}{n144}
\ncline[]{n7}{n146}
\ncline[]{n7}{n147}
\ncline[]{n7}{n148}
\ncline[]{n8}{n9}
\ncline[]{n8}{n12}
\ncline[]{n8}{n13}
\ncline[]{n8}{n17}
\ncline[]{n8}{n19}
\ncline[]{n8}{n20}
\ncline[]{n8}{n21}
\ncline[]{n8}{n23}
\ncline[]{n8}{n24}
\ncline[]{n8}{n26}
\ncline[]{n8}{n28}
\ncline[]{n8}{n29}
\ncline[]{n8}{n32}
\ncline[]{n8}{n33}
\ncline[]{n8}{n35}
\ncline[]{n8}{n36}
\ncline[]{n8}{n38}
\ncline[]{n8}{n39}
\ncline[]{n8}{n40}
\ncline[]{n8}{n41}
\ncline[]{n8}{n42}
\ncline[]{n8}{n43}
\ncline[]{n8}{n44}
\ncline[]{n8}{n46}
\ncline[]{n8}{n47}
\ncline[]{n8}{n50}
\ncline[]{n8}{n51}
\ncline[]{n8}{n53}
\ncline[]{n8}{n55}
\ncline[]{n8}{n56}
\ncline[]{n8}{n57}
\ncline[]{n8}{n58}
\ncline[]{n8}{n59}
\ncline[]{n8}{n60}
\ncline[]{n8}{n61}
\ncline[]{n8}{n64}
\ncline[]{n8}{n65}
\ncline[]{n8}{n66}
\ncline[]{n8}{n70}
\ncline[]{n8}{n71}
\ncline[]{n8}{n72}
\ncline[]{n8}{n77}
\ncline[]{n8}{n79}
\ncline[]{n8}{n80}
\ncline[]{n8}{n82}
\ncline[]{n8}{n83}
\ncline[]{n8}{n84}
\ncline[]{n8}{n86}
\ncline[]{n8}{n87}
\ncline[]{n8}{n89}
\ncline[]{n8}{n92}
\ncline[]{n8}{n93}
\ncline[]{n8}{n94}
\ncline[]{n8}{n95}
\ncline[]{n8}{n97}
\ncline[]{n8}{n98}
\ncline[]{n8}{n99}
\ncline[]{n8}{n100}
\ncline[]{n8}{n101}
\ncline[]{n8}{n102}
\ncline[]{n8}{n103}
\ncline[]{n8}{n104}
\ncline[]{n8}{n107}
\ncline[]{n8}{n110}
\ncline[]{n8}{n112}
\ncline[]{n8}{n114}
\ncline[]{n8}{n116}
\ncline[]{n8}{n117}
\ncline[]{n8}{n118}
\ncline[]{n8}{n119}
\ncline[]{n8}{n120}
\ncline[]{n8}{n122}
\ncline[]{n8}{n123}
\ncline[]{n8}{n126}
\ncline[]{n8}{n128}
\ncline[]{n8}{n129}
\ncline[]{n8}{n131}
\ncline[]{n8}{n132}
\ncline[]{n8}{n137}
\ncline[]{n8}{n138}
\ncline[]{n8}{n139}
\ncline[]{n8}{n140}
\ncline[]{n8}{n141}
\ncline[]{n8}{n142}
\ncline[]{n8}{n143}
\ncline[]{n8}{n144}
\ncline[]{n8}{n146}
\ncline[]{n8}{n147}
\ncline[]{n8}{n148}
\ncline[]{n9}{n12}
\ncline[]{n9}{n13}
\ncline[]{n9}{n17}
\ncline[]{n9}{n19}
\ncline[]{n9}{n20}
\ncline[]{n9}{n21}
\ncline[]{n9}{n23}
\ncline[]{n9}{n24}
\ncline[]{n9}{n26}
\ncline[]{n9}{n28}
\ncline[]{n9}{n29}
\ncline[]{n9}{n32}
\ncline[]{n9}{n33}
\ncline[]{n9}{n35}
\ncline[]{n9}{n36}
\ncline[]{n9}{n38}
\ncline[]{n9}{n39}
\ncline[]{n9}{n40}
\ncline[]{n9}{n41}
\ncline[]{n9}{n42}
\ncline[]{n9}{n43}
\ncline[]{n9}{n44}
\ncline[]{n9}{n46}
\ncline[]{n9}{n47}
\ncline[]{n9}{n50}
\ncline[]{n9}{n51}
\ncline[]{n9}{n53}
\ncline[]{n9}{n55}
\ncline[]{n9}{n56}
\ncline[]{n9}{n57}
\ncline[]{n9}{n58}
\ncline[]{n9}{n59}
\ncline[]{n9}{n60}
\ncline[]{n9}{n61}
\ncline[]{n9}{n64}
\ncline[]{n9}{n65}
\ncline[]{n9}{n66}
\ncline[]{n9}{n70}
\ncline[]{n9}{n71}
\ncline[]{n9}{n72}
\ncline[]{n9}{n77}
\ncline[]{n9}{n79}
\ncline[]{n9}{n80}
\ncline[]{n9}{n82}
\ncline[]{n9}{n83}
\ncline[]{n9}{n84}
\ncline[]{n9}{n86}
\ncline[]{n9}{n87}
\ncline[]{n9}{n89}
\ncline[]{n9}{n92}
\ncline[]{n9}{n93}
\ncline[]{n9}{n94}
\ncline[]{n9}{n95}
\ncline[]{n9}{n97}
\ncline[]{n9}{n98}
\ncline[]{n9}{n99}
\ncline[]{n9}{n100}
\ncline[]{n9}{n101}
\ncline[]{n9}{n102}
\ncline[]{n9}{n103}
\ncline[]{n9}{n104}
\ncline[]{n9}{n107}
\ncline[]{n9}{n110}
\ncline[]{n9}{n112}
\ncline[]{n9}{n114}
\ncline[]{n9}{n116}
\ncline[]{n9}{n117}
\ncline[]{n9}{n118}
\ncline[]{n9}{n119}
\ncline[]{n9}{n120}
\ncline[]{n9}{n122}
\ncline[]{n9}{n123}
\ncline[]{n9}{n126}
\ncline[]{n9}{n128}
\ncline[]{n9}{n129}
\ncline[]{n9}{n131}
\ncline[]{n9}{n132}
\ncline[]{n9}{n137}
\ncline[]{n9}{n138}
\ncline[]{n9}{n139}
\ncline[]{n9}{n140}
\ncline[]{n9}{n141}
\ncline[]{n9}{n142}
\ncline[]{n9}{n143}
\ncline[]{n9}{n144}
\ncline[]{n9}{n146}
\ncline[]{n9}{n147}
\ncline[]{n9}{n148}
\ncline[]{n10}{n11}
\ncline[]{n10}{n15}
\ncline[]{n10}{n16}
\ncline[]{n10}{n18}
\ncline[]{n10}{n22}
\ncline[]{n10}{n25}
\ncline[]{n10}{n30}
\ncline[]{n10}{n31}
\ncline[]{n10}{n37}
\ncline[]{n10}{n48}
\ncline[]{n10}{n49}
\ncline[]{n10}{n54}
\ncline[]{n10}{n62}
\ncline[]{n10}{n63}
\ncline[]{n10}{n68}
\ncline[]{n10}{n69}
\ncline[]{n10}{n73}
\ncline[]{n10}{n88}
\ncline[]{n10}{n111}
\ncline[]{n10}{n113}
\ncline[]{n10}{n115}
\ncline[]{n10}{n124}
\ncline[]{n10}{n127}
\ncline[]{n10}{n130}
\ncline[]{n10}{n133}
\ncline[]{n10}{n135}
\ncline[]{n10}{n136}
\ncline[]{n10}{n145}
\ncline[]{n10}{n149}
\ncline[]{n11}{n15}
\ncline[]{n11}{n16}
\ncline[]{n11}{n18}
\ncline[]{n11}{n22}
\ncline[]{n11}{n25}
\ncline[]{n11}{n30}
\ncline[]{n11}{n31}
\ncline[]{n11}{n37}
\ncline[]{n11}{n48}
\ncline[]{n11}{n49}
\ncline[]{n11}{n54}
\ncline[]{n11}{n62}
\ncline[]{n11}{n63}
\ncline[]{n11}{n68}
\ncline[]{n11}{n69}
\ncline[]{n11}{n73}
\ncline[]{n11}{n88}
\ncline[]{n11}{n111}
\ncline[]{n11}{n113}
\ncline[]{n11}{n115}
\ncline[]{n11}{n124}
\ncline[]{n11}{n127}
\ncline[]{n11}{n130}
\ncline[]{n11}{n133}
\ncline[]{n11}{n135}
\ncline[]{n11}{n136}
\ncline[]{n11}{n145}
\ncline[]{n11}{n149}
\ncline[]{n12}{n13}
\ncline[]{n12}{n17}
\ncline[]{n12}{n19}
\ncline[]{n12}{n20}
\ncline[]{n12}{n21}
\ncline[]{n12}{n23}
\ncline[]{n12}{n24}
\ncline[]{n12}{n26}
\ncline[]{n12}{n28}
\ncline[]{n12}{n29}
\ncline[]{n12}{n32}
\ncline[]{n12}{n33}
\ncline[]{n12}{n35}
\ncline[]{n12}{n36}
\ncline[]{n12}{n38}
\ncline[]{n12}{n39}
\ncline[]{n12}{n40}
\ncline[]{n12}{n41}
\ncline[]{n12}{n42}
\ncline[]{n12}{n43}
\ncline[]{n12}{n44}
\ncline[]{n12}{n46}
\ncline[]{n12}{n47}
\ncline[]{n12}{n50}
\ncline[]{n12}{n51}
\ncline[]{n12}{n53}
\ncline[]{n12}{n55}
\ncline[]{n12}{n56}
\ncline[]{n12}{n57}
\ncline[]{n12}{n58}
\ncline[]{n12}{n59}
\ncline[]{n12}{n60}
\ncline[]{n12}{n61}
\ncline[]{n12}{n64}
\ncline[]{n12}{n65}
\ncline[]{n12}{n66}
\ncline[]{n12}{n70}
\ncline[]{n12}{n71}
\ncline[]{n12}{n72}
\ncline[]{n12}{n77}
\ncline[]{n12}{n79}
\ncline[]{n12}{n80}
\ncline[]{n12}{n82}
\ncline[]{n12}{n83}
\ncline[]{n12}{n84}
\ncline[]{n12}{n86}
\ncline[]{n12}{n87}
\ncline[]{n12}{n89}
\ncline[]{n12}{n92}
\ncline[]{n12}{n93}
\ncline[]{n12}{n94}
\ncline[]{n12}{n95}
\ncline[]{n12}{n97}
\ncline[]{n12}{n98}
\ncline[]{n12}{n99}
\ncline[]{n12}{n100}
\ncline[]{n12}{n101}
\ncline[]{n12}{n102}
\ncline[]{n12}{n103}
\ncline[]{n12}{n104}
\ncline[]{n12}{n107}
\ncline[]{n12}{n110}
\ncline[]{n12}{n112}
\ncline[]{n12}{n114}
\ncline[]{n12}{n116}
\ncline[]{n12}{n117}
\ncline[]{n12}{n118}
\ncline[]{n12}{n119}
\ncline[]{n12}{n120}
\ncline[]{n12}{n122}
\ncline[]{n12}{n123}
\ncline[]{n12}{n126}
\ncline[]{n12}{n128}
\ncline[]{n12}{n129}
\ncline[]{n12}{n131}
\ncline[]{n12}{n132}
\ncline[]{n12}{n137}
\ncline[]{n12}{n138}
\ncline[]{n12}{n139}
\ncline[]{n12}{n140}
\ncline[]{n12}{n141}
\ncline[]{n12}{n142}
\ncline[]{n12}{n143}
\ncline[]{n12}{n144}
\ncline[]{n12}{n146}
\ncline[]{n12}{n147}
\ncline[]{n12}{n148}
\ncline[]{n13}{n17}
\ncline[]{n13}{n19}
\ncline[]{n13}{n20}
\ncline[]{n13}{n21}
\ncline[]{n13}{n23}
\ncline[]{n13}{n24}
\ncline[]{n13}{n26}
\ncline[]{n13}{n28}
\ncline[]{n13}{n29}
\ncline[]{n13}{n32}
\ncline[]{n13}{n33}
\ncline[]{n13}{n35}
\ncline[]{n13}{n36}
\ncline[]{n13}{n38}
\ncline[]{n13}{n39}
\ncline[]{n13}{n40}
\ncline[]{n13}{n41}
\ncline[]{n13}{n42}
\ncline[]{n13}{n43}
\ncline[]{n13}{n44}
\ncline[]{n13}{n46}
\ncline[]{n13}{n47}
\ncline[]{n13}{n50}
\ncline[]{n13}{n51}
\ncline[]{n13}{n53}
\ncline[]{n13}{n55}
\ncline[]{n13}{n56}
\ncline[]{n13}{n57}
\ncline[]{n13}{n58}
\ncline[]{n13}{n59}
\ncline[]{n13}{n60}
\ncline[]{n13}{n61}
\ncline[]{n13}{n64}
\ncline[]{n13}{n65}
\ncline[]{n13}{n66}
\ncline[]{n13}{n70}
\ncline[]{n13}{n71}
\ncline[]{n13}{n72}
\ncline[]{n13}{n77}
\ncline[]{n13}{n79}
\ncline[]{n13}{n80}
\ncline[]{n13}{n82}
\ncline[]{n13}{n83}
\ncline[]{n13}{n84}
\ncline[]{n13}{n86}
\ncline[]{n13}{n87}
\ncline[]{n13}{n89}
\ncline[]{n13}{n92}
\ncline[]{n13}{n93}
\ncline[]{n13}{n94}
\ncline[]{n13}{n95}
\ncline[]{n13}{n97}
\ncline[]{n13}{n98}
\ncline[]{n13}{n99}
\ncline[]{n13}{n100}
\ncline[]{n13}{n101}
\ncline[]{n13}{n102}
\ncline[]{n13}{n103}
\ncline[]{n13}{n104}
\ncline[]{n13}{n107}
\ncline[]{n13}{n110}
\ncline[]{n13}{n112}
\ncline[]{n13}{n114}
\ncline[]{n13}{n116}
\ncline[]{n13}{n117}
\ncline[]{n13}{n118}
\ncline[]{n13}{n119}
\ncline[]{n13}{n120}
\ncline[]{n13}{n122}
\ncline[]{n13}{n123}
\ncline[]{n13}{n126}
\ncline[]{n13}{n128}
\ncline[]{n13}{n129}
\ncline[]{n13}{n131}
\ncline[]{n13}{n132}
\ncline[]{n13}{n137}
\ncline[]{n13}{n138}
\ncline[]{n13}{n139}
\ncline[]{n13}{n140}
\ncline[]{n13}{n141}
\ncline[]{n13}{n142}
\ncline[]{n13}{n143}
\ncline[]{n13}{n144}
\ncline[]{n13}{n146}
\ncline[]{n13}{n147}
\ncline[]{n13}{n148}
\ncline[]{n14}{n27}
\ncline[]{n14}{n34}
\ncline[]{n14}{n45}
\ncline[]{n14}{n52}
\ncline[]{n14}{n67}
\ncline[]{n14}{n74}
\ncline[]{n14}{n75}
\ncline[]{n14}{n76}
\ncline[]{n14}{n78}
\ncline[]{n14}{n81}
\ncline[]{n14}{n85}
\ncline[]{n14}{n90}
\ncline[]{n14}{n91}
\ncline[]{n14}{n96}
\ncline[]{n14}{n105}
\ncline[]{n14}{n106}
\ncline[]{n14}{n108}
\ncline[]{n14}{n109}
\ncline[]{n14}{n121}
\ncline[]{n14}{n125}
\ncline[]{n14}{n134}
\ncline[]{n15}{n16}
\ncline[]{n15}{n18}
\ncline[]{n15}{n22}
\ncline[]{n15}{n25}
\ncline[]{n15}{n30}
\ncline[]{n15}{n31}
\ncline[]{n15}{n37}
\ncline[]{n15}{n48}
\ncline[]{n15}{n49}
\ncline[]{n15}{n54}
\ncline[]{n15}{n62}
\ncline[]{n15}{n63}
\ncline[]{n15}{n68}
\ncline[]{n15}{n69}
\ncline[]{n15}{n73}
\ncline[]{n15}{n88}
\ncline[]{n15}{n111}
\ncline[]{n15}{n113}
\ncline[]{n15}{n115}
\ncline[]{n15}{n124}
\ncline[]{n15}{n127}
\ncline[]{n15}{n130}
\ncline[]{n15}{n133}
\ncline[]{n15}{n135}
\ncline[]{n15}{n136}
\ncline[]{n15}{n145}
\ncline[]{n15}{n149}
\ncline[]{n16}{n18}
\ncline[]{n16}{n22}
\ncline[]{n16}{n25}
\ncline[]{n16}{n30}
\ncline[]{n16}{n31}
\ncline[]{n16}{n37}
\ncline[]{n16}{n48}
\ncline[]{n16}{n49}
\ncline[]{n16}{n54}
\ncline[]{n16}{n62}
\ncline[]{n16}{n63}
\ncline[]{n16}{n68}
\ncline[]{n16}{n69}
\ncline[]{n16}{n73}
\ncline[]{n16}{n88}
\ncline[]{n16}{n111}
\ncline[]{n16}{n113}
\ncline[]{n16}{n115}
\ncline[]{n16}{n124}
\ncline[]{n16}{n127}
\ncline[]{n16}{n130}
\ncline[]{n16}{n133}
\ncline[]{n16}{n135}
\ncline[]{n16}{n136}
\ncline[]{n16}{n145}
\ncline[]{n16}{n149}
\ncline[]{n17}{n19}
\ncline[]{n17}{n20}
\ncline[]{n17}{n21}
\ncline[]{n17}{n23}
\ncline[]{n17}{n24}
\ncline[]{n17}{n26}
\ncline[]{n17}{n28}
\ncline[]{n17}{n29}
\ncline[]{n17}{n32}
\ncline[]{n17}{n33}
\ncline[]{n17}{n35}
\ncline[]{n17}{n36}
\ncline[]{n17}{n38}
\ncline[]{n17}{n39}
\ncline[]{n17}{n40}
\ncline[]{n17}{n41}
\ncline[]{n17}{n42}
\ncline[]{n17}{n43}
\ncline[]{n17}{n44}
\ncline[]{n17}{n46}
\ncline[]{n17}{n47}
\ncline[]{n17}{n50}
\ncline[]{n17}{n51}
\ncline[]{n17}{n53}
\ncline[]{n17}{n55}
\ncline[]{n17}{n56}
\ncline[]{n17}{n57}
\ncline[]{n17}{n58}
\ncline[]{n17}{n59}
\ncline[]{n17}{n60}
\ncline[]{n17}{n61}
\ncline[]{n17}{n64}
\ncline[]{n17}{n65}
\ncline[]{n17}{n66}
\ncline[]{n17}{n70}
\ncline[]{n17}{n71}
\ncline[]{n17}{n72}
\ncline[]{n17}{n77}
\ncline[]{n17}{n79}
\ncline[]{n17}{n80}
\ncline[]{n17}{n82}
\ncline[]{n17}{n83}
\ncline[]{n17}{n84}
\ncline[]{n17}{n86}
\ncline[]{n17}{n87}
\ncline[]{n17}{n89}
\ncline[]{n17}{n92}
\ncline[]{n17}{n93}
\ncline[]{n17}{n94}
\ncline[]{n17}{n95}
\ncline[]{n17}{n97}
\ncline[]{n17}{n98}
\ncline[]{n17}{n99}
\ncline[]{n17}{n100}
\ncline[]{n17}{n101}
\ncline[]{n17}{n102}
\ncline[]{n17}{n103}
\ncline[]{n17}{n104}
\ncline[]{n17}{n107}
\ncline[]{n17}{n110}
\ncline[]{n17}{n112}
\ncline[]{n17}{n114}
\ncline[]{n17}{n116}
\ncline[]{n17}{n117}
\ncline[]{n17}{n118}
\ncline[]{n17}{n119}
\ncline[]{n17}{n120}
\ncline[]{n17}{n122}
\ncline[]{n17}{n123}
\ncline[]{n17}{n126}
\ncline[]{n17}{n128}
\ncline[]{n17}{n129}
\ncline[]{n17}{n131}
\ncline[]{n17}{n132}
\ncline[]{n17}{n137}
\ncline[]{n17}{n138}
\ncline[]{n17}{n139}
\ncline[]{n17}{n140}
\ncline[]{n17}{n141}
\ncline[]{n17}{n142}
\ncline[]{n17}{n143}
\ncline[]{n17}{n144}
\ncline[]{n17}{n146}
\ncline[]{n17}{n147}
\ncline[]{n17}{n148}
\ncline[]{n18}{n22}
\ncline[]{n18}{n25}
\ncline[]{n18}{n30}
\ncline[]{n18}{n31}
\ncline[]{n18}{n37}
\ncline[]{n18}{n48}
\ncline[]{n18}{n49}
\ncline[]{n18}{n54}
\ncline[]{n18}{n62}
\ncline[]{n18}{n63}
\ncline[]{n18}{n68}
\ncline[]{n18}{n69}
\ncline[]{n18}{n73}
\ncline[]{n18}{n88}
\ncline[]{n18}{n111}
\ncline[]{n18}{n113}
\ncline[]{n18}{n115}
\ncline[]{n18}{n124}
\ncline[]{n18}{n127}
\ncline[]{n18}{n130}
\ncline[]{n18}{n133}
\ncline[]{n18}{n135}
\ncline[]{n18}{n136}
\ncline[]{n18}{n145}
\ncline[]{n18}{n149}
\ncline[]{n19}{n20}
\ncline[]{n19}{n21}
\ncline[]{n19}{n23}
\ncline[]{n19}{n24}
\ncline[]{n19}{n26}
\ncline[]{n19}{n28}
\ncline[]{n19}{n29}
\ncline[]{n19}{n32}
\ncline[]{n19}{n33}
\ncline[]{n19}{n35}
\ncline[]{n19}{n36}
\ncline[]{n19}{n38}
\ncline[]{n19}{n39}
\ncline[]{n19}{n40}
\ncline[]{n19}{n41}
\ncline[]{n19}{n42}
\ncline[]{n19}{n43}
\ncline[]{n19}{n44}
\ncline[]{n19}{n46}
\ncline[]{n19}{n47}
\ncline[]{n19}{n50}
\ncline[]{n19}{n51}
\ncline[]{n19}{n53}
\ncline[]{n19}{n55}
\ncline[]{n19}{n56}
\ncline[]{n19}{n57}
\ncline[]{n19}{n58}
\ncline[]{n19}{n59}
\ncline[]{n19}{n60}
\ncline[]{n19}{n61}
\ncline[]{n19}{n64}
\ncline[]{n19}{n65}
\ncline[]{n19}{n66}
\ncline[]{n19}{n70}
\ncline[]{n19}{n71}
\ncline[]{n19}{n72}
\ncline[]{n19}{n77}
\ncline[]{n19}{n79}
\ncline[]{n19}{n80}
\ncline[]{n19}{n82}
\ncline[]{n19}{n83}
\ncline[]{n19}{n84}
\ncline[]{n19}{n86}
\ncline[]{n19}{n87}
\ncline[]{n19}{n89}
\ncline[]{n19}{n92}
\ncline[]{n19}{n93}
\ncline[]{n19}{n94}
\ncline[]{n19}{n95}
\ncline[]{n19}{n97}
\ncline[]{n19}{n98}
\ncline[]{n19}{n99}
\ncline[]{n19}{n100}
\ncline[]{n19}{n101}
\ncline[]{n19}{n102}
\ncline[]{n19}{n103}
\ncline[]{n19}{n104}
\ncline[]{n19}{n107}
\ncline[]{n19}{n110}
\ncline[]{n19}{n112}
\ncline[]{n19}{n114}
\ncline[]{n19}{n116}
\ncline[]{n19}{n117}
\ncline[]{n19}{n118}
\ncline[]{n19}{n119}
\ncline[]{n19}{n120}
\ncline[]{n19}{n122}
\ncline[]{n19}{n123}
\ncline[]{n19}{n126}
\ncline[]{n19}{n128}
\ncline[]{n19}{n129}
\ncline[]{n19}{n131}
\ncline[]{n19}{n132}
\ncline[]{n19}{n137}
\ncline[]{n19}{n138}
\ncline[]{n19}{n139}
\ncline[]{n19}{n140}
\ncline[]{n19}{n141}
\ncline[]{n19}{n142}
\ncline[]{n19}{n143}
\ncline[]{n19}{n144}
\ncline[]{n19}{n146}
\ncline[]{n19}{n147}
\ncline[]{n19}{n148}
\ncline[]{n20}{n21}
\ncline[]{n20}{n23}
\ncline[]{n20}{n24}
\ncline[]{n20}{n26}
\ncline[]{n20}{n28}
\ncline[]{n20}{n29}
\ncline[]{n20}{n32}
\ncline[]{n20}{n33}
\ncline[]{n20}{n35}
\ncline[]{n20}{n36}
\ncline[]{n20}{n38}
\ncline[]{n20}{n39}
\ncline[]{n20}{n40}
\ncline[]{n20}{n41}
\ncline[]{n20}{n42}
\ncline[]{n20}{n43}
\ncline[]{n20}{n44}
\ncline[]{n20}{n46}
\ncline[]{n20}{n47}
\ncline[]{n20}{n50}
\ncline[]{n20}{n51}
\ncline[]{n20}{n53}
\ncline[]{n20}{n55}
\ncline[]{n20}{n56}
\ncline[]{n20}{n57}
\ncline[]{n20}{n58}
\ncline[]{n20}{n59}
\ncline[]{n20}{n60}
\ncline[]{n20}{n61}
\ncline[]{n20}{n64}
\ncline[]{n20}{n65}
\ncline[]{n20}{n66}
\ncline[]{n20}{n70}
\ncline[]{n20}{n71}
\ncline[]{n20}{n72}
\ncline[]{n20}{n77}
\ncline[]{n20}{n79}
\ncline[]{n20}{n80}
\ncline[]{n20}{n82}
\ncline[]{n20}{n83}
\ncline[]{n20}{n84}
\ncline[]{n20}{n86}
\ncline[]{n20}{n87}
\ncline[]{n20}{n89}
\ncline[]{n20}{n92}
\ncline[]{n20}{n93}
\ncline[]{n20}{n94}
\ncline[]{n20}{n95}
\ncline[]{n20}{n97}
\ncline[]{n20}{n98}
\ncline[]{n20}{n99}
\ncline[]{n20}{n100}
\ncline[]{n20}{n101}
\ncline[]{n20}{n102}
\ncline[]{n20}{n103}
\ncline[]{n20}{n104}
\ncline[]{n20}{n107}
\ncline[]{n20}{n110}
\ncline[]{n20}{n112}
\ncline[]{n20}{n114}
\ncline[]{n20}{n116}
\ncline[]{n20}{n117}
\ncline[]{n20}{n118}
\ncline[]{n20}{n119}
\ncline[]{n20}{n120}
\ncline[]{n20}{n122}
\ncline[]{n20}{n123}
\ncline[]{n20}{n126}
\ncline[]{n20}{n128}
\ncline[]{n20}{n129}
\ncline[]{n20}{n131}
\ncline[]{n20}{n132}
\ncline[]{n20}{n137}
\ncline[]{n20}{n138}
\ncline[]{n20}{n139}
\ncline[]{n20}{n140}
\ncline[]{n20}{n141}
\ncline[]{n20}{n142}
\ncline[]{n20}{n143}
\ncline[]{n20}{n144}
\ncline[]{n20}{n146}
\ncline[]{n20}{n147}
\ncline[]{n20}{n148}
\ncline[]{n21}{n23}
\ncline[]{n21}{n24}
\ncline[]{n21}{n26}
\ncline[]{n21}{n28}
\ncline[]{n21}{n29}
\ncline[]{n21}{n32}
\ncline[]{n21}{n33}
\ncline[]{n21}{n35}
\ncline[]{n21}{n36}
\ncline[]{n21}{n38}
\ncline[]{n21}{n39}
\ncline[]{n21}{n40}
\ncline[]{n21}{n41}
\ncline[]{n21}{n42}
\ncline[]{n21}{n43}
\ncline[]{n21}{n44}
\ncline[]{n21}{n46}
\ncline[]{n21}{n47}
\ncline[]{n21}{n50}
\ncline[]{n21}{n51}
\ncline[]{n21}{n53}
\ncline[]{n21}{n55}
\ncline[]{n21}{n56}
\ncline[]{n21}{n57}
\ncline[]{n21}{n58}
\ncline[]{n21}{n59}
\ncline[]{n21}{n60}
\ncline[]{n21}{n61}
\ncline[]{n21}{n64}
\ncline[]{n21}{n65}
\ncline[]{n21}{n66}
\ncline[]{n21}{n70}
\ncline[]{n21}{n71}
\ncline[]{n21}{n72}
\ncline[]{n21}{n77}
\ncline[]{n21}{n79}
\ncline[]{n21}{n80}
\ncline[]{n21}{n82}
\ncline[]{n21}{n83}
\ncline[]{n21}{n84}
\ncline[]{n21}{n86}
\ncline[]{n21}{n87}
\ncline[]{n21}{n89}
\ncline[]{n21}{n92}
\ncline[]{n21}{n93}
\ncline[]{n21}{n94}
\ncline[]{n21}{n95}
\ncline[]{n21}{n97}
\ncline[]{n21}{n98}
\ncline[]{n21}{n99}
\ncline[]{n21}{n100}
\ncline[]{n21}{n101}
\ncline[]{n21}{n102}
\ncline[]{n21}{n103}
\ncline[]{n21}{n104}
\ncline[]{n21}{n107}
\ncline[]{n21}{n110}
\ncline[]{n21}{n112}
\ncline[]{n21}{n114}
\ncline[]{n21}{n116}
\ncline[]{n21}{n117}
\ncline[]{n21}{n118}
\ncline[]{n21}{n119}
\ncline[]{n21}{n120}
\ncline[]{n21}{n122}
\ncline[]{n21}{n123}
\ncline[]{n21}{n126}
\ncline[]{n21}{n128}
\ncline[]{n21}{n129}
\ncline[]{n21}{n131}
\ncline[]{n21}{n132}
\ncline[]{n21}{n137}
\ncline[]{n21}{n138}
\ncline[]{n21}{n139}
\ncline[]{n21}{n140}
\ncline[]{n21}{n141}
\ncline[]{n21}{n142}
\ncline[]{n21}{n143}
\ncline[]{n21}{n144}
\ncline[]{n21}{n146}
\ncline[]{n21}{n147}
\ncline[]{n21}{n148}
\ncline[]{n22}{n25}
\ncline[]{n22}{n30}
\ncline[]{n22}{n31}
\ncline[]{n22}{n37}
\ncline[]{n22}{n48}
\ncline[]{n22}{n49}
\ncline[]{n22}{n54}
\ncline[]{n22}{n62}
\ncline[]{n22}{n63}
\ncline[]{n22}{n68}
\ncline[]{n22}{n69}
\ncline[]{n22}{n73}
\ncline[]{n22}{n88}
\ncline[]{n22}{n111}
\ncline[]{n22}{n113}
\ncline[]{n22}{n115}
\ncline[]{n22}{n124}
\ncline[]{n22}{n127}
\ncline[]{n22}{n130}
\ncline[]{n22}{n133}
\ncline[]{n22}{n135}
\ncline[]{n22}{n136}
\ncline[]{n22}{n145}
\ncline[]{n22}{n149}
\ncline[]{n23}{n24}
\ncline[]{n23}{n26}
\ncline[]{n23}{n28}
\ncline[]{n23}{n29}
\ncline[]{n23}{n32}
\ncline[]{n23}{n33}
\ncline[]{n23}{n35}
\ncline[]{n23}{n36}
\ncline[]{n23}{n38}
\ncline[]{n23}{n39}
\ncline[]{n23}{n40}
\ncline[]{n23}{n41}
\ncline[]{n23}{n42}
\ncline[]{n23}{n43}
\ncline[]{n23}{n44}
\ncline[]{n23}{n46}
\ncline[]{n23}{n47}
\ncline[]{n23}{n50}
\ncline[]{n23}{n51}
\ncline[]{n23}{n53}
\ncline[]{n23}{n55}
\ncline[]{n23}{n56}
\ncline[]{n23}{n57}
\ncline[]{n23}{n58}
\ncline[]{n23}{n59}
\ncline[]{n23}{n60}
\ncline[]{n23}{n61}
\ncline[]{n23}{n64}
\ncline[]{n23}{n65}
\ncline[]{n23}{n66}
\ncline[]{n23}{n70}
\ncline[]{n23}{n71}
\ncline[]{n23}{n72}
\ncline[]{n23}{n77}
\ncline[]{n23}{n79}
\ncline[]{n23}{n80}
\ncline[]{n23}{n82}
\ncline[]{n23}{n83}
\ncline[]{n23}{n84}
\ncline[]{n23}{n86}
\ncline[]{n23}{n87}
\ncline[]{n23}{n89}
\ncline[]{n23}{n92}
\ncline[]{n23}{n93}
\ncline[]{n23}{n94}
\ncline[]{n23}{n95}
\ncline[]{n23}{n97}
\ncline[]{n23}{n98}
\ncline[]{n23}{n99}
\ncline[]{n23}{n100}
\ncline[]{n23}{n101}
\ncline[]{n23}{n102}
\ncline[]{n23}{n103}
\ncline[]{n23}{n104}
\ncline[]{n23}{n107}
\ncline[]{n23}{n110}
\ncline[]{n23}{n112}
\ncline[]{n23}{n114}
\ncline[]{n23}{n116}
\ncline[]{n23}{n117}
\ncline[]{n23}{n118}
\ncline[]{n23}{n119}
\ncline[]{n23}{n120}
\ncline[]{n23}{n122}
\ncline[]{n23}{n123}
\ncline[]{n23}{n126}
\ncline[]{n23}{n128}
\ncline[]{n23}{n129}
\ncline[]{n23}{n131}
\ncline[]{n23}{n132}
\ncline[]{n23}{n137}
\ncline[]{n23}{n138}
\ncline[]{n23}{n139}
\ncline[]{n23}{n140}
\ncline[]{n23}{n141}
\ncline[]{n23}{n142}
\ncline[]{n23}{n143}
\ncline[]{n23}{n144}
\ncline[]{n23}{n146}
\ncline[]{n23}{n147}
\ncline[]{n23}{n148}
\ncline[]{n24}{n26}
\ncline[]{n24}{n28}
\ncline[]{n24}{n29}
\ncline[]{n24}{n32}
\ncline[]{n24}{n33}
\ncline[]{n24}{n35}
\ncline[]{n24}{n36}
\ncline[]{n24}{n38}
\ncline[]{n24}{n39}
\ncline[]{n24}{n40}
\ncline[]{n24}{n41}
\ncline[]{n24}{n42}
\ncline[]{n24}{n43}
\ncline[]{n24}{n44}
\ncline[]{n24}{n46}
\ncline[]{n24}{n47}
\ncline[]{n24}{n50}
\ncline[]{n24}{n51}
\ncline[]{n24}{n53}
\ncline[]{n24}{n55}
\ncline[]{n24}{n56}
\ncline[]{n24}{n57}
\ncline[]{n24}{n58}
\ncline[]{n24}{n59}
\ncline[]{n24}{n60}
\ncline[]{n24}{n61}
\ncline[]{n24}{n64}
\ncline[]{n24}{n65}
\ncline[]{n24}{n66}
\ncline[]{n24}{n70}
\ncline[]{n24}{n71}
\ncline[]{n24}{n72}
\ncline[]{n24}{n77}
\ncline[]{n24}{n79}
\ncline[]{n24}{n80}
\ncline[]{n24}{n82}
\ncline[]{n24}{n83}
\ncline[]{n24}{n84}
\ncline[]{n24}{n86}
\ncline[]{n24}{n87}
\ncline[]{n24}{n89}
\ncline[]{n24}{n92}
\ncline[]{n24}{n93}
\ncline[]{n24}{n94}
\ncline[]{n24}{n95}
\ncline[]{n24}{n97}
\ncline[]{n24}{n98}
\ncline[]{n24}{n99}
\ncline[]{n24}{n100}
\ncline[]{n24}{n101}
\ncline[]{n24}{n102}
\ncline[]{n24}{n103}
\ncline[]{n24}{n104}
\ncline[]{n24}{n107}
\ncline[]{n24}{n110}
\ncline[]{n24}{n112}
\ncline[]{n24}{n114}
\ncline[]{n24}{n116}
\ncline[]{n24}{n117}
\ncline[]{n24}{n118}
\ncline[]{n24}{n119}
\ncline[]{n24}{n120}
\ncline[]{n24}{n122}
\ncline[]{n24}{n123}
\ncline[]{n24}{n126}
\ncline[]{n24}{n128}
\ncline[]{n24}{n129}
\ncline[]{n24}{n131}
\ncline[]{n24}{n132}
\ncline[]{n24}{n137}
\ncline[]{n24}{n138}
\ncline[]{n24}{n139}
\ncline[]{n24}{n140}
\ncline[]{n24}{n141}
\ncline[]{n24}{n142}
\ncline[]{n24}{n143}
\ncline[]{n24}{n144}
\ncline[]{n24}{n146}
\ncline[]{n24}{n147}
\ncline[]{n24}{n148}
\ncline[]{n25}{n30}
\ncline[]{n25}{n31}
\ncline[]{n25}{n37}
\ncline[]{n25}{n48}
\ncline[]{n25}{n49}
\ncline[]{n25}{n54}
\ncline[]{n25}{n62}
\ncline[]{n25}{n63}
\ncline[]{n25}{n68}
\ncline[]{n25}{n69}
\ncline[]{n25}{n73}
\ncline[]{n25}{n88}
\ncline[]{n25}{n111}
\ncline[]{n25}{n113}
\ncline[]{n25}{n115}
\ncline[]{n25}{n124}
\ncline[]{n25}{n127}
\ncline[]{n25}{n130}
\ncline[]{n25}{n133}
\ncline[]{n25}{n135}
\ncline[]{n25}{n136}
\ncline[]{n25}{n145}
\ncline[]{n25}{n149}
\ncline[]{n26}{n28}
\ncline[]{n26}{n29}
\ncline[]{n26}{n32}
\ncline[]{n26}{n33}
\ncline[]{n26}{n35}
\ncline[]{n26}{n36}
\ncline[]{n26}{n38}
\ncline[]{n26}{n39}
\ncline[]{n26}{n40}
\ncline[]{n26}{n41}
\ncline[]{n26}{n42}
\ncline[]{n26}{n43}
\ncline[]{n26}{n44}
\ncline[]{n26}{n46}
\ncline[]{n26}{n47}
\ncline[]{n26}{n50}
\ncline[]{n26}{n51}
\ncline[]{n26}{n53}
\ncline[]{n26}{n55}
\ncline[]{n26}{n56}
\ncline[]{n26}{n57}
\ncline[]{n26}{n58}
\ncline[]{n26}{n59}
\ncline[]{n26}{n60}
\ncline[]{n26}{n61}
\ncline[]{n26}{n64}
\ncline[]{n26}{n65}
\ncline[]{n26}{n66}
\ncline[]{n26}{n70}
\ncline[]{n26}{n71}
\ncline[]{n26}{n72}
\ncline[]{n26}{n77}
\ncline[]{n26}{n79}
\ncline[]{n26}{n80}
\ncline[]{n26}{n82}
\ncline[]{n26}{n83}
\ncline[]{n26}{n84}
\ncline[]{n26}{n86}
\ncline[]{n26}{n87}
\ncline[]{n26}{n89}
\ncline[]{n26}{n92}
\ncline[]{n26}{n93}
\ncline[]{n26}{n94}
\ncline[]{n26}{n95}
\ncline[]{n26}{n97}
\ncline[]{n26}{n98}
\ncline[]{n26}{n99}
\ncline[]{n26}{n100}
\ncline[]{n26}{n101}
\ncline[]{n26}{n102}
\ncline[]{n26}{n103}
\ncline[]{n26}{n104}
\ncline[]{n26}{n107}
\ncline[]{n26}{n110}
\ncline[]{n26}{n112}
\ncline[]{n26}{n114}
\ncline[]{n26}{n116}
\ncline[]{n26}{n117}
\ncline[]{n26}{n118}
\ncline[]{n26}{n119}
\ncline[]{n26}{n120}
\ncline[]{n26}{n122}
\ncline[]{n26}{n123}
\ncline[]{n26}{n126}
\ncline[]{n26}{n128}
\ncline[]{n26}{n129}
\ncline[]{n26}{n131}
\ncline[]{n26}{n132}
\ncline[]{n26}{n137}
\ncline[]{n26}{n138}
\ncline[]{n26}{n139}
\ncline[]{n26}{n140}
\ncline[]{n26}{n141}
\ncline[]{n26}{n142}
\ncline[]{n26}{n143}
\ncline[]{n26}{n144}
\ncline[]{n26}{n146}
\ncline[]{n26}{n147}
\ncline[]{n26}{n148}
\ncline[]{n27}{n34}
\ncline[]{n27}{n45}
\ncline[]{n27}{n52}
\ncline[]{n27}{n67}
\ncline[]{n27}{n74}
\ncline[]{n27}{n75}
\ncline[]{n27}{n76}
\ncline[]{n27}{n78}
\ncline[]{n27}{n81}
\ncline[]{n27}{n85}
\ncline[]{n27}{n90}
\ncline[]{n27}{n91}
\ncline[]{n27}{n96}
\ncline[]{n27}{n105}
\ncline[]{n27}{n106}
\ncline[]{n27}{n108}
\ncline[]{n27}{n109}
\ncline[]{n27}{n121}
\ncline[]{n27}{n125}
\ncline[]{n27}{n134}
\ncline[]{n28}{n29}
\ncline[]{n28}{n32}
\ncline[]{n28}{n33}
\ncline[]{n28}{n35}
\ncline[]{n28}{n36}
\ncline[]{n28}{n38}
\ncline[]{n28}{n39}
\ncline[]{n28}{n40}
\ncline[]{n28}{n41}
\ncline[]{n28}{n42}
\ncline[]{n28}{n43}
\ncline[]{n28}{n44}
\ncline[]{n28}{n46}
\ncline[]{n28}{n47}
\ncline[]{n28}{n50}
\ncline[]{n28}{n51}
\ncline[]{n28}{n53}
\ncline[]{n28}{n55}
\ncline[]{n28}{n56}
\ncline[]{n28}{n57}
\ncline[]{n28}{n58}
\ncline[]{n28}{n59}
\ncline[]{n28}{n60}
\ncline[]{n28}{n61}
\ncline[]{n28}{n64}
\ncline[]{n28}{n65}
\ncline[]{n28}{n66}
\ncline[]{n28}{n70}
\ncline[]{n28}{n71}
\ncline[]{n28}{n72}
\ncline[]{n28}{n77}
\ncline[]{n28}{n79}
\ncline[]{n28}{n80}
\ncline[]{n28}{n82}
\ncline[]{n28}{n83}
\ncline[]{n28}{n84}
\ncline[]{n28}{n86}
\ncline[]{n28}{n87}
\ncline[]{n28}{n89}
\ncline[]{n28}{n92}
\ncline[]{n28}{n93}
\ncline[]{n28}{n94}
\ncline[]{n28}{n95}
\ncline[]{n28}{n97}
\ncline[]{n28}{n98}
\ncline[]{n28}{n99}
\ncline[]{n28}{n100}
\ncline[]{n28}{n101}
\ncline[]{n28}{n102}
\ncline[]{n28}{n103}
\ncline[]{n28}{n104}
\ncline[]{n28}{n107}
\ncline[]{n28}{n110}
\ncline[]{n28}{n112}
\ncline[]{n28}{n114}
\ncline[]{n28}{n116}
\ncline[]{n28}{n117}
\ncline[]{n28}{n118}
\ncline[]{n28}{n119}
\ncline[]{n28}{n120}
\ncline[]{n28}{n122}
\ncline[]{n28}{n123}
\ncline[]{n28}{n126}
\ncline[]{n28}{n128}
\ncline[]{n28}{n129}
\ncline[]{n28}{n131}
\ncline[]{n28}{n132}
\ncline[]{n28}{n137}
\ncline[]{n28}{n138}
\ncline[]{n28}{n139}
\ncline[]{n28}{n140}
\ncline[]{n28}{n141}
\ncline[]{n28}{n142}
\ncline[]{n28}{n143}
\ncline[]{n28}{n144}
\ncline[]{n28}{n146}
\ncline[]{n28}{n147}
\ncline[]{n28}{n148}
\ncline[]{n29}{n32}
\ncline[]{n29}{n33}
\ncline[]{n29}{n35}
\ncline[]{n29}{n36}
\ncline[]{n29}{n38}
\ncline[]{n29}{n39}
\ncline[]{n29}{n40}
\ncline[]{n29}{n41}
\ncline[]{n29}{n42}
\ncline[]{n29}{n43}
\ncline[]{n29}{n44}
\ncline[]{n29}{n46}
\ncline[]{n29}{n47}
\ncline[]{n29}{n50}
\ncline[]{n29}{n51}
\ncline[]{n29}{n53}
\ncline[]{n29}{n55}
\ncline[]{n29}{n56}
\ncline[]{n29}{n57}
\ncline[]{n29}{n58}
\ncline[]{n29}{n59}
\ncline[]{n29}{n60}
\ncline[]{n29}{n61}
\ncline[]{n29}{n64}
\ncline[]{n29}{n65}
\ncline[]{n29}{n66}
\ncline[]{n29}{n70}
\ncline[]{n29}{n71}
\ncline[]{n29}{n72}
\ncline[]{n29}{n77}
\ncline[]{n29}{n79}
\ncline[]{n29}{n80}
\ncline[]{n29}{n82}
\ncline[]{n29}{n83}
\ncline[]{n29}{n84}
\ncline[]{n29}{n86}
\ncline[]{n29}{n87}
\ncline[]{n29}{n89}
\ncline[]{n29}{n92}
\ncline[]{n29}{n93}
\ncline[]{n29}{n94}
\ncline[]{n29}{n95}
\ncline[]{n29}{n97}
\ncline[]{n29}{n98}
\ncline[]{n29}{n99}
\ncline[]{n29}{n100}
\ncline[]{n29}{n101}
\ncline[]{n29}{n102}
\ncline[]{n29}{n103}
\ncline[]{n29}{n104}
\ncline[]{n29}{n107}
\ncline[]{n29}{n110}
\ncline[]{n29}{n112}
\ncline[]{n29}{n114}
\ncline[]{n29}{n116}
\ncline[]{n29}{n117}
\ncline[]{n29}{n118}
\ncline[]{n29}{n119}
\ncline[]{n29}{n120}
\ncline[]{n29}{n122}
\ncline[]{n29}{n123}
\ncline[]{n29}{n126}
\ncline[]{n29}{n128}
\ncline[]{n29}{n129}
\ncline[]{n29}{n131}
\ncline[]{n29}{n132}
\ncline[]{n29}{n137}
\ncline[]{n29}{n138}
\ncline[]{n29}{n139}
\ncline[]{n29}{n140}
\ncline[]{n29}{n141}
\ncline[]{n29}{n142}
\ncline[]{n29}{n143}
\ncline[]{n29}{n144}
\ncline[]{n29}{n146}
\ncline[]{n29}{n147}
\ncline[]{n29}{n148}
\ncline[]{n30}{n31}
\ncline[]{n30}{n37}
\ncline[]{n30}{n48}
\ncline[]{n30}{n49}
\ncline[]{n30}{n54}
\ncline[]{n30}{n62}
\ncline[]{n30}{n63}
\ncline[]{n30}{n68}
\ncline[]{n30}{n69}
\ncline[]{n30}{n73}
\ncline[]{n30}{n88}
\ncline[]{n30}{n111}
\ncline[]{n30}{n113}
\ncline[]{n30}{n115}
\ncline[]{n30}{n124}
\ncline[]{n30}{n127}
\ncline[]{n30}{n130}
\ncline[]{n30}{n133}
\ncline[]{n30}{n135}
\ncline[]{n30}{n136}
\ncline[]{n30}{n145}
\ncline[]{n30}{n149}
\ncline[]{n31}{n37}
\ncline[]{n31}{n48}
\ncline[]{n31}{n49}
\ncline[]{n31}{n54}
\ncline[]{n31}{n62}
\ncline[]{n31}{n63}
\ncline[]{n31}{n68}
\ncline[]{n31}{n69}
\ncline[]{n31}{n73}
\ncline[]{n31}{n88}
\ncline[]{n31}{n111}
\ncline[]{n31}{n113}
\ncline[]{n31}{n115}
\ncline[]{n31}{n124}
\ncline[]{n31}{n127}
\ncline[]{n31}{n130}
\ncline[]{n31}{n133}
\ncline[]{n31}{n135}
\ncline[]{n31}{n136}
\ncline[]{n31}{n145}
\ncline[]{n31}{n149}
\ncline[]{n32}{n33}
\ncline[]{n32}{n35}
\ncline[]{n32}{n36}
\ncline[]{n32}{n38}
\ncline[]{n32}{n39}
\ncline[]{n32}{n40}
\ncline[]{n32}{n41}
\ncline[]{n32}{n42}
\ncline[]{n32}{n43}
\ncline[]{n32}{n44}
\ncline[]{n32}{n46}
\ncline[]{n32}{n47}
\ncline[]{n32}{n50}
\ncline[]{n32}{n51}
\ncline[]{n32}{n53}
\ncline[]{n32}{n55}
\ncline[]{n32}{n56}
\ncline[]{n32}{n57}
\ncline[]{n32}{n58}
\ncline[]{n32}{n59}
\ncline[]{n32}{n60}
\ncline[]{n32}{n61}
\ncline[]{n32}{n64}
\ncline[]{n32}{n65}
\ncline[]{n32}{n66}
\ncline[]{n32}{n70}
\ncline[]{n32}{n71}
\ncline[]{n32}{n72}
\ncline[]{n32}{n77}
\ncline[]{n32}{n79}
\ncline[]{n32}{n80}
\ncline[]{n32}{n82}
\ncline[]{n32}{n83}
\ncline[]{n32}{n84}
\ncline[]{n32}{n86}
\ncline[]{n32}{n87}
\ncline[]{n32}{n89}
\ncline[]{n32}{n92}
\ncline[]{n32}{n93}
\ncline[]{n32}{n94}
\ncline[]{n32}{n95}
\ncline[]{n32}{n97}
\ncline[]{n32}{n98}
\ncline[]{n32}{n99}
\ncline[]{n32}{n100}
\ncline[]{n32}{n101}
\ncline[]{n32}{n102}
\ncline[]{n32}{n103}
\ncline[]{n32}{n104}
\ncline[]{n32}{n107}
\ncline[]{n32}{n110}
\ncline[]{n32}{n112}
\ncline[]{n32}{n114}
\ncline[]{n32}{n116}
\ncline[]{n32}{n117}
\ncline[]{n32}{n118}
\ncline[]{n32}{n119}
\ncline[]{n32}{n120}
\ncline[]{n32}{n122}
\ncline[]{n32}{n123}
\ncline[]{n32}{n126}
\ncline[]{n32}{n128}
\ncline[]{n32}{n129}
\ncline[]{n32}{n131}
\ncline[]{n32}{n132}
\ncline[]{n32}{n137}
\ncline[]{n32}{n138}
\ncline[]{n32}{n139}
\ncline[]{n32}{n140}
\ncline[]{n32}{n141}
\ncline[]{n32}{n142}
\ncline[]{n32}{n143}
\ncline[]{n32}{n144}
\ncline[]{n32}{n146}
\ncline[]{n32}{n147}
\ncline[]{n32}{n148}
\ncline[]{n33}{n35}
\ncline[]{n33}{n36}
\ncline[]{n33}{n38}
\ncline[]{n33}{n39}
\ncline[]{n33}{n40}
\ncline[]{n33}{n41}
\ncline[]{n33}{n42}
\ncline[]{n33}{n43}
\ncline[]{n33}{n44}
\ncline[]{n33}{n46}
\ncline[]{n33}{n47}
\ncline[]{n33}{n50}
\ncline[]{n33}{n51}
\ncline[]{n33}{n53}
\ncline[]{n33}{n55}
\ncline[]{n33}{n56}
\ncline[]{n33}{n57}
\ncline[]{n33}{n58}
\ncline[]{n33}{n59}
\ncline[]{n33}{n60}
\ncline[]{n33}{n61}
\ncline[]{n33}{n64}
\ncline[]{n33}{n65}
\ncline[]{n33}{n66}
\ncline[]{n33}{n70}
\ncline[]{n33}{n71}
\ncline[]{n33}{n72}
\ncline[]{n33}{n77}
\ncline[]{n33}{n79}
\ncline[]{n33}{n80}
\ncline[]{n33}{n82}
\ncline[]{n33}{n83}
\ncline[]{n33}{n84}
\ncline[]{n33}{n86}
\ncline[]{n33}{n87}
\ncline[]{n33}{n89}
\ncline[]{n33}{n92}
\ncline[]{n33}{n93}
\ncline[]{n33}{n94}
\ncline[]{n33}{n95}
\ncline[]{n33}{n97}
\ncline[]{n33}{n98}
\ncline[]{n33}{n99}
\ncline[]{n33}{n100}
\ncline[]{n33}{n101}
\ncline[]{n33}{n102}
\ncline[]{n33}{n103}
\ncline[]{n33}{n104}
\ncline[]{n33}{n107}
\ncline[]{n33}{n110}
\ncline[]{n33}{n112}
\ncline[]{n33}{n114}
\ncline[]{n33}{n116}
\ncline[]{n33}{n117}
\ncline[]{n33}{n118}
\ncline[]{n33}{n119}
\ncline[]{n33}{n120}
\ncline[]{n33}{n122}
\ncline[]{n33}{n123}
\ncline[]{n33}{n126}
\ncline[]{n33}{n128}
\ncline[]{n33}{n129}
\ncline[]{n33}{n131}
\ncline[]{n33}{n132}
\ncline[]{n33}{n137}
\ncline[]{n33}{n138}
\ncline[]{n33}{n139}
\ncline[]{n33}{n140}
\ncline[]{n33}{n141}
\ncline[]{n33}{n142}
\ncline[]{n33}{n143}
\ncline[]{n33}{n144}
\ncline[]{n33}{n146}
\ncline[]{n33}{n147}
\ncline[]{n33}{n148}
\ncline[]{n34}{n45}
\ncline[]{n34}{n52}
\ncline[]{n34}{n67}
\ncline[]{n34}{n74}
\ncline[]{n34}{n75}
\ncline[]{n34}{n76}
\ncline[]{n34}{n78}
\ncline[]{n34}{n81}
\ncline[]{n34}{n85}
\ncline[]{n34}{n90}
\ncline[]{n34}{n91}
\ncline[]{n34}{n96}
\ncline[]{n34}{n105}
\ncline[]{n34}{n106}
\ncline[]{n34}{n108}
\ncline[]{n34}{n109}
\ncline[]{n34}{n121}
\ncline[]{n34}{n125}
\ncline[]{n34}{n134}
\ncline[]{n35}{n36}
\ncline[]{n35}{n38}
\ncline[]{n35}{n39}
\ncline[]{n35}{n40}
\ncline[]{n35}{n41}
\ncline[]{n35}{n42}
\ncline[]{n35}{n43}
\ncline[]{n35}{n44}
\ncline[]{n35}{n46}
\ncline[]{n35}{n47}
\ncline[]{n35}{n50}
\ncline[]{n35}{n51}
\ncline[]{n35}{n53}
\ncline[]{n35}{n55}
\ncline[]{n35}{n56}
\ncline[]{n35}{n57}
\ncline[]{n35}{n58}
\ncline[]{n35}{n59}
\ncline[]{n35}{n60}
\ncline[]{n35}{n61}
\ncline[]{n35}{n64}
\ncline[]{n35}{n65}
\ncline[]{n35}{n66}
\ncline[]{n35}{n70}
\ncline[]{n35}{n71}
\ncline[]{n35}{n72}
\ncline[]{n35}{n77}
\ncline[]{n35}{n79}
\ncline[]{n35}{n80}
\ncline[]{n35}{n82}
\ncline[]{n35}{n83}
\ncline[]{n35}{n84}
\ncline[]{n35}{n86}
\ncline[]{n35}{n87}
\ncline[]{n35}{n89}
\ncline[]{n35}{n92}
\ncline[]{n35}{n93}
\ncline[]{n35}{n94}
\ncline[]{n35}{n95}
\ncline[]{n35}{n97}
\ncline[]{n35}{n98}
\ncline[]{n35}{n99}
\ncline[]{n35}{n100}
\ncline[]{n35}{n101}
\ncline[]{n35}{n102}
\ncline[]{n35}{n103}
\ncline[]{n35}{n104}
\ncline[]{n35}{n107}
\ncline[]{n35}{n110}
\ncline[]{n35}{n112}
\ncline[]{n35}{n114}
\ncline[]{n35}{n116}
\ncline[]{n35}{n117}
\ncline[]{n35}{n118}
\ncline[]{n35}{n119}
\ncline[]{n35}{n120}
\ncline[]{n35}{n122}
\ncline[]{n35}{n123}
\ncline[]{n35}{n126}
\ncline[]{n35}{n128}
\ncline[]{n35}{n129}
\ncline[]{n35}{n131}
\ncline[]{n35}{n132}
\ncline[]{n35}{n137}
\ncline[]{n35}{n138}
\ncline[]{n35}{n139}
\ncline[]{n35}{n140}
\ncline[]{n35}{n141}
\ncline[]{n35}{n142}
\ncline[]{n35}{n143}
\ncline[]{n35}{n144}
\ncline[]{n35}{n146}
\ncline[]{n35}{n147}
\ncline[]{n35}{n148}
\ncline[]{n36}{n38}
\ncline[]{n36}{n39}
\ncline[]{n36}{n40}
\ncline[]{n36}{n41}
\ncline[]{n36}{n42}
\ncline[]{n36}{n43}
\ncline[]{n36}{n44}
\ncline[]{n36}{n46}
\ncline[]{n36}{n47}
\ncline[]{n36}{n50}
\ncline[]{n36}{n51}
\ncline[]{n36}{n53}
\ncline[]{n36}{n55}
\ncline[]{n36}{n56}
\ncline[]{n36}{n57}
\ncline[]{n36}{n58}
\ncline[]{n36}{n59}
\ncline[]{n36}{n60}
\ncline[]{n36}{n61}
\ncline[]{n36}{n64}
\ncline[]{n36}{n65}
\ncline[]{n36}{n66}
\ncline[]{n36}{n70}
\ncline[]{n36}{n71}
\ncline[]{n36}{n72}
\ncline[]{n36}{n77}
\ncline[]{n36}{n79}
\ncline[]{n36}{n80}
\ncline[]{n36}{n82}
\ncline[]{n36}{n83}
\ncline[]{n36}{n84}
\ncline[]{n36}{n86}
\ncline[]{n36}{n87}
\ncline[]{n36}{n89}
\ncline[]{n36}{n92}
\ncline[]{n36}{n93}
\ncline[]{n36}{n94}
\ncline[]{n36}{n95}
\ncline[]{n36}{n97}
\ncline[]{n36}{n98}
\ncline[]{n36}{n99}
\ncline[]{n36}{n100}
\ncline[]{n36}{n101}
\ncline[]{n36}{n102}
\ncline[]{n36}{n103}
\ncline[]{n36}{n104}
\ncline[]{n36}{n107}
\ncline[]{n36}{n110}
\ncline[]{n36}{n112}
\ncline[]{n36}{n114}
\ncline[]{n36}{n116}
\ncline[]{n36}{n117}
\ncline[]{n36}{n118}
\ncline[]{n36}{n119}
\ncline[]{n36}{n120}
\ncline[]{n36}{n122}
\ncline[]{n36}{n123}
\ncline[]{n36}{n126}
\ncline[]{n36}{n128}
\ncline[]{n36}{n129}
\ncline[]{n36}{n131}
\ncline[]{n36}{n132}
\ncline[]{n36}{n137}
\ncline[]{n36}{n138}
\ncline[]{n36}{n139}
\ncline[]{n36}{n140}
\ncline[]{n36}{n141}
\ncline[]{n36}{n142}
\ncline[]{n36}{n143}
\ncline[]{n36}{n144}
\ncline[]{n36}{n146}
\ncline[]{n36}{n147}
\ncline[]{n36}{n148}
\ncline[]{n37}{n48}
\ncline[]{n37}{n49}
\ncline[]{n37}{n54}
\ncline[]{n37}{n62}
\ncline[]{n37}{n63}
\ncline[]{n37}{n68}
\ncline[]{n37}{n69}
\ncline[]{n37}{n73}
\ncline[]{n37}{n88}
\ncline[]{n37}{n111}
\ncline[]{n37}{n113}
\ncline[]{n37}{n115}
\ncline[]{n37}{n124}
\ncline[]{n37}{n127}
\ncline[]{n37}{n130}
\ncline[]{n37}{n133}
\ncline[]{n37}{n135}
\ncline[]{n37}{n136}
\ncline[]{n37}{n145}
\ncline[]{n37}{n149}
\ncline[]{n38}{n39}
\ncline[]{n38}{n40}
\ncline[]{n38}{n41}
\ncline[]{n38}{n42}
\ncline[]{n38}{n43}
\ncline[]{n38}{n44}
\ncline[]{n38}{n46}
\ncline[]{n38}{n47}
\ncline[]{n38}{n50}
\ncline[]{n38}{n51}
\ncline[]{n38}{n53}
\ncline[]{n38}{n55}
\ncline[]{n38}{n56}
\ncline[]{n38}{n57}
\ncline[]{n38}{n58}
\ncline[]{n38}{n59}
\ncline[]{n38}{n60}
\ncline[]{n38}{n61}
\ncline[]{n38}{n64}
\ncline[]{n38}{n65}
\ncline[]{n38}{n66}
\ncline[]{n38}{n70}
\ncline[]{n38}{n71}
\ncline[]{n38}{n72}
\ncline[]{n38}{n77}
\ncline[]{n38}{n79}
\ncline[]{n38}{n80}
\ncline[]{n38}{n82}
\ncline[]{n38}{n83}
\ncline[]{n38}{n84}
\ncline[]{n38}{n86}
\ncline[]{n38}{n87}
\ncline[]{n38}{n89}
\ncline[]{n38}{n92}
\ncline[]{n38}{n93}
\ncline[]{n38}{n94}
\ncline[]{n38}{n95}
\ncline[]{n38}{n97}
\ncline[]{n38}{n98}
\ncline[]{n38}{n99}
\ncline[]{n38}{n100}
\ncline[]{n38}{n101}
\ncline[]{n38}{n102}
\ncline[]{n38}{n103}
\ncline[]{n38}{n104}
\ncline[]{n38}{n107}
\ncline[]{n38}{n110}
\ncline[]{n38}{n112}
\ncline[]{n38}{n114}
\ncline[]{n38}{n116}
\ncline[]{n38}{n117}
\ncline[]{n38}{n118}
\ncline[]{n38}{n119}
\ncline[]{n38}{n120}
\ncline[]{n38}{n122}
\ncline[]{n38}{n123}
\ncline[]{n38}{n126}
\ncline[]{n38}{n128}
\ncline[]{n38}{n129}
\ncline[]{n38}{n131}
\ncline[]{n38}{n132}
\ncline[]{n38}{n137}
\ncline[]{n38}{n138}
\ncline[]{n38}{n139}
\ncline[]{n38}{n140}
\ncline[]{n38}{n141}
\ncline[]{n38}{n142}
\ncline[]{n38}{n143}
\ncline[]{n38}{n144}
\ncline[]{n38}{n146}
\ncline[]{n38}{n147}
\ncline[]{n38}{n148}
\ncline[]{n39}{n40}
\ncline[]{n39}{n41}
\ncline[]{n39}{n42}
\ncline[]{n39}{n43}
\ncline[]{n39}{n44}
\ncline[]{n39}{n46}
\ncline[]{n39}{n47}
\ncline[]{n39}{n50}
\ncline[]{n39}{n51}
\ncline[]{n39}{n53}
\ncline[]{n39}{n55}
\ncline[]{n39}{n56}
\ncline[]{n39}{n57}
\ncline[]{n39}{n58}
\ncline[]{n39}{n59}
\ncline[]{n39}{n60}
\ncline[]{n39}{n61}
\ncline[]{n39}{n64}
\ncline[]{n39}{n65}
\ncline[]{n39}{n66}
\ncline[]{n39}{n70}
\ncline[]{n39}{n71}
\ncline[]{n39}{n72}
\ncline[]{n39}{n77}
\ncline[]{n39}{n79}
\ncline[]{n39}{n80}
\ncline[]{n39}{n82}
\ncline[]{n39}{n83}
\ncline[]{n39}{n84}
\ncline[]{n39}{n86}
\ncline[]{n39}{n87}
\ncline[]{n39}{n89}
\ncline[]{n39}{n92}
\ncline[]{n39}{n93}
\ncline[]{n39}{n94}
\ncline[]{n39}{n95}
\ncline[]{n39}{n97}
\ncline[]{n39}{n98}
\ncline[]{n39}{n99}
\ncline[]{n39}{n100}
\ncline[]{n39}{n101}
\ncline[]{n39}{n102}
\ncline[]{n39}{n103}
\ncline[]{n39}{n104}
\ncline[]{n39}{n107}
\ncline[]{n39}{n110}
\ncline[]{n39}{n112}
\ncline[]{n39}{n114}
\ncline[]{n39}{n116}
\ncline[]{n39}{n117}
\ncline[]{n39}{n118}
\ncline[]{n39}{n119}
\ncline[]{n39}{n120}
\ncline[]{n39}{n122}
\ncline[]{n39}{n123}
\ncline[]{n39}{n126}
\ncline[]{n39}{n128}
\ncline[]{n39}{n129}
\ncline[]{n39}{n131}
\ncline[]{n39}{n132}
\ncline[]{n39}{n137}
\ncline[]{n39}{n138}
\ncline[]{n39}{n139}
\ncline[]{n39}{n140}
\ncline[]{n39}{n141}
\ncline[]{n39}{n142}
\ncline[]{n39}{n143}
\ncline[]{n39}{n144}
\ncline[]{n39}{n146}
\ncline[]{n39}{n147}
\ncline[]{n39}{n148}
\ncline[]{n40}{n41}
\ncline[]{n40}{n42}
\ncline[]{n40}{n43}
\ncline[]{n40}{n44}
\ncline[]{n40}{n46}
\ncline[]{n40}{n47}
\ncline[]{n40}{n50}
\ncline[]{n40}{n51}
\ncline[]{n40}{n53}
\ncline[]{n40}{n55}
\ncline[]{n40}{n56}
\ncline[]{n40}{n57}
\ncline[]{n40}{n58}
\ncline[]{n40}{n59}
\ncline[]{n40}{n60}
\ncline[]{n40}{n61}
\ncline[]{n40}{n64}
\ncline[]{n40}{n65}
\ncline[]{n40}{n66}
\ncline[]{n40}{n70}
\ncline[]{n40}{n71}
\ncline[]{n40}{n72}
\ncline[]{n40}{n77}
\ncline[]{n40}{n79}
\ncline[]{n40}{n80}
\ncline[]{n40}{n82}
\ncline[]{n40}{n83}
\ncline[]{n40}{n84}
\ncline[]{n40}{n86}
\ncline[]{n40}{n87}
\ncline[]{n40}{n89}
\ncline[]{n40}{n92}
\ncline[]{n40}{n93}
\ncline[]{n40}{n94}
\ncline[]{n40}{n95}
\ncline[]{n40}{n97}
\ncline[]{n40}{n98}
\ncline[]{n40}{n99}
\ncline[]{n40}{n100}
\ncline[]{n40}{n101}
\ncline[]{n40}{n102}
\ncline[]{n40}{n103}
\ncline[]{n40}{n104}
\ncline[]{n40}{n107}
\ncline[]{n40}{n110}
\ncline[]{n40}{n112}
\ncline[]{n40}{n114}
\ncline[]{n40}{n116}
\ncline[]{n40}{n117}
\ncline[]{n40}{n118}
\ncline[]{n40}{n119}
\ncline[]{n40}{n120}
\ncline[]{n40}{n122}
\ncline[]{n40}{n123}
\ncline[]{n40}{n126}
\ncline[]{n40}{n128}
\ncline[]{n40}{n129}
\ncline[]{n40}{n131}
\ncline[]{n40}{n132}
\ncline[]{n40}{n137}
\ncline[]{n40}{n138}
\ncline[]{n40}{n139}
\ncline[]{n40}{n140}
\ncline[]{n40}{n141}
\ncline[]{n40}{n142}
\ncline[]{n40}{n143}
\ncline[]{n40}{n144}
\ncline[]{n40}{n146}
\ncline[]{n40}{n147}
\ncline[]{n40}{n148}
\ncline[]{n41}{n42}
\ncline[]{n41}{n43}
\ncline[]{n41}{n44}
\ncline[]{n41}{n46}
\ncline[]{n41}{n47}
\ncline[]{n41}{n50}
\ncline[]{n41}{n51}
\ncline[]{n41}{n53}
\ncline[]{n41}{n55}
\ncline[]{n41}{n56}
\ncline[]{n41}{n57}
\ncline[]{n41}{n58}
\ncline[]{n41}{n59}
\ncline[]{n41}{n60}
\ncline[]{n41}{n61}
\ncline[]{n41}{n64}
\ncline[]{n41}{n65}
\ncline[]{n41}{n66}
\ncline[]{n41}{n70}
\ncline[]{n41}{n71}
\ncline[]{n41}{n72}
\ncline[]{n41}{n77}
\ncline[]{n41}{n79}
\ncline[]{n41}{n80}
\ncline[]{n41}{n82}
\ncline[]{n41}{n83}
\ncline[]{n41}{n84}
\ncline[]{n41}{n86}
\ncline[]{n41}{n87}
\ncline[]{n41}{n89}
\ncline[]{n41}{n92}
\ncline[]{n41}{n93}
\ncline[]{n41}{n94}
\ncline[]{n41}{n95}
\ncline[]{n41}{n97}
\ncline[]{n41}{n98}
\ncline[]{n41}{n99}
\ncline[]{n41}{n100}
\ncline[]{n41}{n101}
\ncline[]{n41}{n102}
\ncline[]{n41}{n103}
\ncline[]{n41}{n104}
\ncline[]{n41}{n107}
\ncline[]{n41}{n110}
\ncline[]{n41}{n112}
\ncline[]{n41}{n114}
\ncline[]{n41}{n116}
\ncline[]{n41}{n117}
\ncline[]{n41}{n118}
\ncline[]{n41}{n119}
\ncline[]{n41}{n120}
\ncline[]{n41}{n122}
\ncline[]{n41}{n123}
\ncline[]{n41}{n126}
\ncline[]{n41}{n128}
\ncline[]{n41}{n129}
\ncline[]{n41}{n131}
\ncline[]{n41}{n132}
\ncline[]{n41}{n137}
\ncline[]{n41}{n138}
\ncline[]{n41}{n139}
\ncline[]{n41}{n140}
\ncline[]{n41}{n141}
\ncline[]{n41}{n142}
\ncline[]{n41}{n143}
\ncline[]{n41}{n144}
\ncline[]{n41}{n146}
\ncline[]{n41}{n147}
\ncline[]{n41}{n148}
\ncline[]{n42}{n43}
\ncline[]{n42}{n44}
\ncline[]{n42}{n46}
\ncline[]{n42}{n47}
\ncline[]{n42}{n50}
\ncline[]{n42}{n51}
\ncline[]{n42}{n53}
\ncline[]{n42}{n55}
\ncline[]{n42}{n56}
\ncline[]{n42}{n57}
\ncline[]{n42}{n58}
\ncline[]{n42}{n59}
\ncline[]{n42}{n60}
\ncline[]{n42}{n61}
\ncline[]{n42}{n64}
\ncline[]{n42}{n65}
\ncline[]{n42}{n66}
\ncline[]{n42}{n70}
\ncline[]{n42}{n71}
\ncline[]{n42}{n72}
\ncline[]{n42}{n77}
\ncline[]{n42}{n79}
\ncline[]{n42}{n80}
\ncline[]{n42}{n82}
\ncline[]{n42}{n83}
\ncline[]{n42}{n84}
\ncline[]{n42}{n86}
\ncline[]{n42}{n87}
\ncline[]{n42}{n89}
\ncline[]{n42}{n92}
\ncline[]{n42}{n93}
\ncline[]{n42}{n94}
\ncline[]{n42}{n95}
\ncline[]{n42}{n97}
\ncline[]{n42}{n98}
\ncline[]{n42}{n99}
\ncline[]{n42}{n100}
\ncline[]{n42}{n101}
\ncline[]{n42}{n102}
\ncline[]{n42}{n103}
\ncline[]{n42}{n104}
\ncline[]{n42}{n107}
\ncline[]{n42}{n110}
\ncline[]{n42}{n112}
\ncline[]{n42}{n114}
\ncline[]{n42}{n116}
\ncline[]{n42}{n117}
\ncline[]{n42}{n118}
\ncline[]{n42}{n119}
\ncline[]{n42}{n120}
\ncline[]{n42}{n122}
\ncline[]{n42}{n123}
\ncline[]{n42}{n126}
\ncline[]{n42}{n128}
\ncline[]{n42}{n129}
\ncline[]{n42}{n131}
\ncline[]{n42}{n132}
\ncline[]{n42}{n137}
\ncline[]{n42}{n138}
\ncline[]{n42}{n139}
\ncline[]{n42}{n140}
\ncline[]{n42}{n141}
\ncline[]{n42}{n142}
\ncline[]{n42}{n143}
\ncline[]{n42}{n144}
\ncline[]{n42}{n146}
\ncline[]{n42}{n147}
\ncline[]{n42}{n148}
\ncline[]{n43}{n44}
\ncline[]{n43}{n46}
\ncline[]{n43}{n47}
\ncline[]{n43}{n50}
\ncline[]{n43}{n51}
\ncline[]{n43}{n53}
\ncline[]{n43}{n55}
\ncline[]{n43}{n56}
\ncline[]{n43}{n57}
\ncline[]{n43}{n58}
\ncline[]{n43}{n59}
\ncline[]{n43}{n60}
\ncline[]{n43}{n61}
\ncline[]{n43}{n64}
\ncline[]{n43}{n65}
\ncline[]{n43}{n66}
\ncline[]{n43}{n70}
\ncline[]{n43}{n71}
\ncline[]{n43}{n72}
\ncline[]{n43}{n77}
\ncline[]{n43}{n79}
\ncline[]{n43}{n80}
\ncline[]{n43}{n82}
\ncline[]{n43}{n83}
\ncline[]{n43}{n84}
\ncline[]{n43}{n86}
\ncline[]{n43}{n87}
\ncline[]{n43}{n89}
\ncline[]{n43}{n92}
\ncline[]{n43}{n93}
\ncline[]{n43}{n94}
\ncline[]{n43}{n95}
\ncline[]{n43}{n97}
\ncline[]{n43}{n98}
\ncline[]{n43}{n99}
\ncline[]{n43}{n100}
\ncline[]{n43}{n101}
\ncline[]{n43}{n102}
\ncline[]{n43}{n103}
\ncline[]{n43}{n104}
\ncline[]{n43}{n107}
\ncline[]{n43}{n110}
\ncline[]{n43}{n112}
\ncline[]{n43}{n114}
\ncline[]{n43}{n116}
\ncline[]{n43}{n117}
\ncline[]{n43}{n118}
\ncline[]{n43}{n119}
\ncline[]{n43}{n120}
\ncline[]{n43}{n122}
\ncline[]{n43}{n123}
\ncline[]{n43}{n126}
\ncline[]{n43}{n128}
\ncline[]{n43}{n129}
\ncline[]{n43}{n131}
\ncline[]{n43}{n132}
\ncline[]{n43}{n137}
\ncline[]{n43}{n138}
\ncline[]{n43}{n139}
\ncline[]{n43}{n140}
\ncline[]{n43}{n141}
\ncline[]{n43}{n142}
\ncline[]{n43}{n143}
\ncline[]{n43}{n144}
\ncline[]{n43}{n146}
\ncline[]{n43}{n147}
\ncline[]{n43}{n148}
\ncline[]{n44}{n46}
\ncline[]{n44}{n47}
\ncline[]{n44}{n50}
\ncline[]{n44}{n51}
\ncline[]{n44}{n53}
\ncline[]{n44}{n55}
\ncline[]{n44}{n56}
\ncline[]{n44}{n57}
\ncline[]{n44}{n58}
\ncline[]{n44}{n59}
\ncline[]{n44}{n60}
\ncline[]{n44}{n61}
\ncline[]{n44}{n64}
\ncline[]{n44}{n65}
\ncline[]{n44}{n66}
\ncline[]{n44}{n70}
\ncline[]{n44}{n71}
\ncline[]{n44}{n72}
\ncline[]{n44}{n77}
\ncline[]{n44}{n79}
\ncline[]{n44}{n80}
\ncline[]{n44}{n82}
\ncline[]{n44}{n83}
\ncline[]{n44}{n84}
\ncline[]{n44}{n86}
\ncline[]{n44}{n87}
\ncline[]{n44}{n89}
\ncline[]{n44}{n92}
\ncline[]{n44}{n93}
\ncline[]{n44}{n94}
\ncline[]{n44}{n95}
\ncline[]{n44}{n97}
\ncline[]{n44}{n98}
\ncline[]{n44}{n99}
\ncline[]{n44}{n100}
\ncline[]{n44}{n101}
\ncline[]{n44}{n102}
\ncline[]{n44}{n103}
\ncline[]{n44}{n104}
\ncline[]{n44}{n107}
\ncline[]{n44}{n110}
\ncline[]{n44}{n112}
\ncline[]{n44}{n114}
\ncline[]{n44}{n116}
\ncline[]{n44}{n117}
\ncline[]{n44}{n118}
\ncline[]{n44}{n119}
\ncline[]{n44}{n120}
\ncline[]{n44}{n122}
\ncline[]{n44}{n123}
\ncline[]{n44}{n126}
\ncline[]{n44}{n128}
\ncline[]{n44}{n129}
\ncline[]{n44}{n131}
\ncline[]{n44}{n132}
\ncline[]{n44}{n137}
\ncline[]{n44}{n138}
\ncline[]{n44}{n139}
\ncline[]{n44}{n140}
\ncline[]{n44}{n141}
\ncline[]{n44}{n142}
\ncline[]{n44}{n143}
\ncline[]{n44}{n144}
\ncline[]{n44}{n146}
\ncline[]{n44}{n147}
\ncline[]{n44}{n148}
\ncline[]{n45}{n52}
\ncline[]{n45}{n67}
\ncline[]{n45}{n74}
\ncline[]{n45}{n75}
\ncline[]{n45}{n76}
\ncline[]{n45}{n78}
\ncline[]{n45}{n81}
\ncline[]{n45}{n85}
\ncline[]{n45}{n90}
\ncline[]{n45}{n91}
\ncline[]{n45}{n96}
\ncline[]{n45}{n105}
\ncline[]{n45}{n106}
\ncline[]{n45}{n108}
\ncline[]{n45}{n109}
\ncline[]{n45}{n121}
\ncline[]{n45}{n125}
\ncline[]{n45}{n134}
\ncline[]{n46}{n47}
\ncline[]{n46}{n50}
\ncline[]{n46}{n51}
\ncline[]{n46}{n53}
\ncline[]{n46}{n55}
\ncline[]{n46}{n56}
\ncline[]{n46}{n57}
\ncline[]{n46}{n58}
\ncline[]{n46}{n59}
\ncline[]{n46}{n60}
\ncline[]{n46}{n61}
\ncline[]{n46}{n64}
\ncline[]{n46}{n65}
\ncline[]{n46}{n66}
\ncline[]{n46}{n70}
\ncline[]{n46}{n71}
\ncline[]{n46}{n72}
\ncline[]{n46}{n77}
\ncline[]{n46}{n79}
\ncline[]{n46}{n80}
\ncline[]{n46}{n82}
\ncline[]{n46}{n83}
\ncline[]{n46}{n84}
\ncline[]{n46}{n86}
\ncline[]{n46}{n87}
\ncline[]{n46}{n89}
\ncline[]{n46}{n92}
\ncline[]{n46}{n93}
\ncline[]{n46}{n94}
\ncline[]{n46}{n95}
\ncline[]{n46}{n97}
\ncline[]{n46}{n98}
\ncline[]{n46}{n99}
\ncline[]{n46}{n100}
\ncline[]{n46}{n101}
\ncline[]{n46}{n102}
\ncline[]{n46}{n103}
\ncline[]{n46}{n104}
\ncline[]{n46}{n107}
\ncline[]{n46}{n110}
\ncline[]{n46}{n112}
\ncline[]{n46}{n114}
\ncline[]{n46}{n116}
\ncline[]{n46}{n117}
\ncline[]{n46}{n118}
\ncline[]{n46}{n119}
\ncline[]{n46}{n120}
\ncline[]{n46}{n122}
\ncline[]{n46}{n123}
\ncline[]{n46}{n126}
\ncline[]{n46}{n128}
\ncline[]{n46}{n129}
\ncline[]{n46}{n131}
\ncline[]{n46}{n132}
\ncline[]{n46}{n137}
\ncline[]{n46}{n138}
\ncline[]{n46}{n139}
\ncline[]{n46}{n140}
\ncline[]{n46}{n141}
\ncline[]{n46}{n142}
\ncline[]{n46}{n143}
\ncline[]{n46}{n144}
\ncline[]{n46}{n146}
\ncline[]{n46}{n147}
\ncline[]{n46}{n148}
\ncline[]{n47}{n50}
\ncline[]{n47}{n51}
\ncline[]{n47}{n53}
\ncline[]{n47}{n55}
\ncline[]{n47}{n56}
\ncline[]{n47}{n57}
\ncline[]{n47}{n58}
\ncline[]{n47}{n59}
\ncline[]{n47}{n60}
\ncline[]{n47}{n61}
\ncline[]{n47}{n64}
\ncline[]{n47}{n65}
\ncline[]{n47}{n66}
\ncline[]{n47}{n70}
\ncline[]{n47}{n71}
\ncline[]{n47}{n72}
\ncline[]{n47}{n77}
\ncline[]{n47}{n79}
\ncline[]{n47}{n80}
\ncline[]{n47}{n82}
\ncline[]{n47}{n83}
\ncline[]{n47}{n84}
\ncline[]{n47}{n86}
\ncline[]{n47}{n87}
\ncline[]{n47}{n89}
\ncline[]{n47}{n92}
\ncline[]{n47}{n93}
\ncline[]{n47}{n94}
\ncline[]{n47}{n95}
\ncline[]{n47}{n97}
\ncline[]{n47}{n98}
\ncline[]{n47}{n99}
\ncline[]{n47}{n100}
\ncline[]{n47}{n101}
\ncline[]{n47}{n102}
\ncline[]{n47}{n103}
\ncline[]{n47}{n104}
\ncline[]{n47}{n107}
\ncline[]{n47}{n110}
\ncline[]{n47}{n112}
\ncline[]{n47}{n114}
\ncline[]{n47}{n116}
\ncline[]{n47}{n117}
\ncline[]{n47}{n118}
\ncline[]{n47}{n119}
\ncline[]{n47}{n120}
\ncline[]{n47}{n122}
\ncline[]{n47}{n123}
\ncline[]{n47}{n126}
\ncline[]{n47}{n128}
\ncline[]{n47}{n129}
\ncline[]{n47}{n131}
\ncline[]{n47}{n132}
\ncline[]{n47}{n137}
\ncline[]{n47}{n138}
\ncline[]{n47}{n139}
\ncline[]{n47}{n140}
\ncline[]{n47}{n141}
\ncline[]{n47}{n142}
\ncline[]{n47}{n143}
\ncline[]{n47}{n144}
\ncline[]{n47}{n146}
\ncline[]{n47}{n147}
\ncline[]{n47}{n148}
\ncline[]{n48}{n49}
\ncline[]{n48}{n54}
\ncline[]{n48}{n62}
\ncline[]{n48}{n63}
\ncline[]{n48}{n68}
\ncline[]{n48}{n69}
\ncline[]{n48}{n73}
\ncline[]{n48}{n88}
\ncline[]{n48}{n111}
\ncline[]{n48}{n113}
\ncline[]{n48}{n115}
\ncline[]{n48}{n124}
\ncline[]{n48}{n127}
\ncline[]{n48}{n130}
\ncline[]{n48}{n133}
\ncline[]{n48}{n135}
\ncline[]{n48}{n136}
\ncline[]{n48}{n145}
\ncline[]{n48}{n149}
\ncline[]{n49}{n54}
\ncline[]{n49}{n62}
\ncline[]{n49}{n63}
\ncline[]{n49}{n68}
\ncline[]{n49}{n69}
\ncline[]{n49}{n73}
\ncline[]{n49}{n88}
\ncline[]{n49}{n111}
\ncline[]{n49}{n113}
\ncline[]{n49}{n115}
\ncline[]{n49}{n124}
\ncline[]{n49}{n127}
\ncline[]{n49}{n130}
\ncline[]{n49}{n133}
\ncline[]{n49}{n135}
\ncline[]{n49}{n136}
\ncline[]{n49}{n145}
\ncline[]{n49}{n149}
\ncline[]{n50}{n51}
\ncline[]{n50}{n53}
\ncline[]{n50}{n55}
\ncline[]{n50}{n56}
\ncline[]{n50}{n57}
\ncline[]{n50}{n58}
\ncline[]{n50}{n59}
\ncline[]{n50}{n60}
\ncline[]{n50}{n61}
\ncline[]{n50}{n64}
\ncline[]{n50}{n65}
\ncline[]{n50}{n66}
\ncline[]{n50}{n70}
\ncline[]{n50}{n71}
\ncline[]{n50}{n72}
\ncline[]{n50}{n77}
\ncline[]{n50}{n79}
\ncline[]{n50}{n80}
\ncline[]{n50}{n82}
\ncline[]{n50}{n83}
\ncline[]{n50}{n84}
\ncline[]{n50}{n86}
\ncline[]{n50}{n87}
\ncline[]{n50}{n89}
\ncline[]{n50}{n92}
\ncline[]{n50}{n93}
\ncline[]{n50}{n94}
\ncline[]{n50}{n95}
\ncline[]{n50}{n97}
\ncline[]{n50}{n98}
\ncline[]{n50}{n99}
\ncline[]{n50}{n100}
\ncline[]{n50}{n101}
\ncline[]{n50}{n102}
\ncline[]{n50}{n103}
\ncline[]{n50}{n104}
\ncline[]{n50}{n107}
\ncline[]{n50}{n110}
\ncline[]{n50}{n112}
\ncline[]{n50}{n114}
\ncline[]{n50}{n116}
\ncline[]{n50}{n117}
\ncline[]{n50}{n118}
\ncline[]{n50}{n119}
\ncline[]{n50}{n120}
\ncline[]{n50}{n122}
\ncline[]{n50}{n123}
\ncline[]{n50}{n126}
\ncline[]{n50}{n128}
\ncline[]{n50}{n129}
\ncline[]{n50}{n131}
\ncline[]{n50}{n132}
\ncline[]{n50}{n137}
\ncline[]{n50}{n138}
\ncline[]{n50}{n139}
\ncline[]{n50}{n140}
\ncline[]{n50}{n141}
\ncline[]{n50}{n142}
\ncline[]{n50}{n143}
\ncline[]{n50}{n144}
\ncline[]{n50}{n146}
\ncline[]{n50}{n147}
\ncline[]{n50}{n148}
\ncline[]{n51}{n53}
\ncline[]{n51}{n55}
\ncline[]{n51}{n56}
\ncline[]{n51}{n57}
\ncline[]{n51}{n58}
\ncline[]{n51}{n59}
\ncline[]{n51}{n60}
\ncline[]{n51}{n61}
\ncline[]{n51}{n64}
\ncline[]{n51}{n65}
\ncline[]{n51}{n66}
\ncline[]{n51}{n70}
\ncline[]{n51}{n71}
\ncline[]{n51}{n72}
\ncline[]{n51}{n77}
\ncline[]{n51}{n79}
\ncline[]{n51}{n80}
\ncline[]{n51}{n82}
\ncline[]{n51}{n83}
\ncline[]{n51}{n84}
\ncline[]{n51}{n86}
\ncline[]{n51}{n87}
\ncline[]{n51}{n89}
\ncline[]{n51}{n92}
\ncline[]{n51}{n93}
\ncline[]{n51}{n94}
\ncline[]{n51}{n95}
\ncline[]{n51}{n97}
\ncline[]{n51}{n98}
\ncline[]{n51}{n99}
\ncline[]{n51}{n100}
\ncline[]{n51}{n101}
\ncline[]{n51}{n102}
\ncline[]{n51}{n103}
\ncline[]{n51}{n104}
\ncline[]{n51}{n107}
\ncline[]{n51}{n110}
\ncline[]{n51}{n112}
\ncline[]{n51}{n114}
\ncline[]{n51}{n116}
\ncline[]{n51}{n117}
\ncline[]{n51}{n118}
\ncline[]{n51}{n119}
\ncline[]{n51}{n120}
\ncline[]{n51}{n122}
\ncline[]{n51}{n123}
\ncline[]{n51}{n126}
\ncline[]{n51}{n128}
\ncline[]{n51}{n129}
\ncline[]{n51}{n131}
\ncline[]{n51}{n132}
\ncline[]{n51}{n137}
\ncline[]{n51}{n138}
\ncline[]{n51}{n139}
\ncline[]{n51}{n140}
\ncline[]{n51}{n141}
\ncline[]{n51}{n142}
\ncline[]{n51}{n143}
\ncline[]{n51}{n144}
\ncline[]{n51}{n146}
\ncline[]{n51}{n147}
\ncline[]{n51}{n148}
\ncline[]{n52}{n67}
\ncline[]{n52}{n74}
\ncline[]{n52}{n75}
\ncline[]{n52}{n76}
\ncline[]{n52}{n78}
\ncline[]{n52}{n81}
\ncline[]{n52}{n85}
\ncline[]{n52}{n90}
\ncline[]{n52}{n91}
\ncline[]{n52}{n96}
\ncline[]{n52}{n105}
\ncline[]{n52}{n106}
\ncline[]{n52}{n108}
\ncline[]{n52}{n109}
\ncline[]{n52}{n121}
\ncline[]{n52}{n125}
\ncline[]{n52}{n134}
\ncline[]{n53}{n55}
\ncline[]{n53}{n56}
\ncline[]{n53}{n57}
\ncline[]{n53}{n58}
\ncline[]{n53}{n59}
\ncline[]{n53}{n60}
\ncline[]{n53}{n61}
\ncline[]{n53}{n64}
\ncline[]{n53}{n65}
\ncline[]{n53}{n66}
\ncline[]{n53}{n70}
\ncline[]{n53}{n71}
\ncline[]{n53}{n72}
\ncline[]{n53}{n77}
\ncline[]{n53}{n79}
\ncline[]{n53}{n80}
\ncline[]{n53}{n82}
\ncline[]{n53}{n83}
\ncline[]{n53}{n84}
\ncline[]{n53}{n86}
\ncline[]{n53}{n87}
\ncline[]{n53}{n89}
\ncline[]{n53}{n92}
\ncline[]{n53}{n93}
\ncline[]{n53}{n94}
\ncline[]{n53}{n95}
\ncline[]{n53}{n97}
\ncline[]{n53}{n98}
\ncline[]{n53}{n99}
\ncline[]{n53}{n100}
\ncline[]{n53}{n101}
\ncline[]{n53}{n102}
\ncline[]{n53}{n103}
\ncline[]{n53}{n104}
\ncline[]{n53}{n107}
\ncline[]{n53}{n110}
\ncline[]{n53}{n112}
\ncline[]{n53}{n114}
\ncline[]{n53}{n116}
\ncline[]{n53}{n117}
\ncline[]{n53}{n118}
\ncline[]{n53}{n119}
\ncline[]{n53}{n120}
\ncline[]{n53}{n122}
\ncline[]{n53}{n123}
\ncline[]{n53}{n126}
\ncline[]{n53}{n128}
\ncline[]{n53}{n129}
\ncline[]{n53}{n131}
\ncline[]{n53}{n132}
\ncline[]{n53}{n137}
\ncline[]{n53}{n138}
\ncline[]{n53}{n139}
\ncline[]{n53}{n140}
\ncline[]{n53}{n141}
\ncline[]{n53}{n142}
\ncline[]{n53}{n143}
\ncline[]{n53}{n144}
\ncline[]{n53}{n146}
\ncline[]{n53}{n147}
\ncline[]{n53}{n148}
\ncline[]{n54}{n62}
\ncline[]{n54}{n63}
\ncline[]{n54}{n68}
\ncline[]{n54}{n69}
\ncline[]{n54}{n73}
\ncline[]{n54}{n88}
\ncline[]{n54}{n111}
\ncline[]{n54}{n113}
\ncline[]{n54}{n115}
\ncline[]{n54}{n124}
\ncline[]{n54}{n127}
\ncline[]{n54}{n130}
\ncline[]{n54}{n133}
\ncline[]{n54}{n135}
\ncline[]{n54}{n136}
\ncline[]{n54}{n145}
\ncline[]{n54}{n149}
\ncline[]{n55}{n56}
\ncline[]{n55}{n57}
\ncline[]{n55}{n58}
\ncline[]{n55}{n59}
\ncline[]{n55}{n60}
\ncline[]{n55}{n61}
\ncline[]{n55}{n64}
\ncline[]{n55}{n65}
\ncline[]{n55}{n66}
\ncline[]{n55}{n70}
\ncline[]{n55}{n71}
\ncline[]{n55}{n72}
\ncline[]{n55}{n77}
\ncline[]{n55}{n79}
\ncline[]{n55}{n80}
\ncline[]{n55}{n82}
\ncline[]{n55}{n83}
\ncline[]{n55}{n84}
\ncline[]{n55}{n86}
\ncline[]{n55}{n87}
\ncline[]{n55}{n89}
\ncline[]{n55}{n92}
\ncline[]{n55}{n93}
\ncline[]{n55}{n94}
\ncline[]{n55}{n95}
\ncline[]{n55}{n97}
\ncline[]{n55}{n98}
\ncline[]{n55}{n99}
\ncline[]{n55}{n100}
\ncline[]{n55}{n101}
\ncline[]{n55}{n102}
\ncline[]{n55}{n103}
\ncline[]{n55}{n104}
\ncline[]{n55}{n107}
\ncline[]{n55}{n110}
\ncline[]{n55}{n112}
\ncline[]{n55}{n114}
\ncline[]{n55}{n116}
\ncline[]{n55}{n117}
\ncline[]{n55}{n118}
\ncline[]{n55}{n119}
\ncline[]{n55}{n120}
\ncline[]{n55}{n122}
\ncline[]{n55}{n123}
\ncline[]{n55}{n126}
\ncline[]{n55}{n128}
\ncline[]{n55}{n129}
\ncline[]{n55}{n131}
\ncline[]{n55}{n132}
\ncline[]{n55}{n137}
\ncline[]{n55}{n138}
\ncline[]{n55}{n139}
\ncline[]{n55}{n140}
\ncline[]{n55}{n141}
\ncline[]{n55}{n142}
\ncline[]{n55}{n143}
\ncline[]{n55}{n144}
\ncline[]{n55}{n146}
\ncline[]{n55}{n147}
\ncline[]{n55}{n148}
\ncline[]{n56}{n57}
\ncline[]{n56}{n58}
\ncline[]{n56}{n59}
\ncline[]{n56}{n60}
\ncline[]{n56}{n61}
\ncline[]{n56}{n64}
\ncline[]{n56}{n65}
\ncline[]{n56}{n66}
\ncline[]{n56}{n70}
\ncline[]{n56}{n71}
\ncline[]{n56}{n72}
\ncline[]{n56}{n77}
\ncline[]{n56}{n79}
\ncline[]{n56}{n80}
\ncline[]{n56}{n82}
\ncline[]{n56}{n83}
\ncline[]{n56}{n84}
\ncline[]{n56}{n86}
\ncline[]{n56}{n87}
\ncline[]{n56}{n89}
\ncline[]{n56}{n92}
\ncline[]{n56}{n93}
\ncline[]{n56}{n94}
\ncline[]{n56}{n95}
\ncline[]{n56}{n97}
\ncline[]{n56}{n98}
\ncline[]{n56}{n99}
\ncline[]{n56}{n100}
\ncline[]{n56}{n101}
\ncline[]{n56}{n102}
\ncline[]{n56}{n103}
\ncline[]{n56}{n104}
\ncline[]{n56}{n107}
\ncline[]{n56}{n110}
\ncline[]{n56}{n112}
\ncline[]{n56}{n114}
\ncline[]{n56}{n116}
\ncline[]{n56}{n117}
\ncline[]{n56}{n118}
\ncline[]{n56}{n119}
\ncline[]{n56}{n120}
\ncline[]{n56}{n122}
\ncline[]{n56}{n123}
\ncline[]{n56}{n126}
\ncline[]{n56}{n128}
\ncline[]{n56}{n129}
\ncline[]{n56}{n131}
\ncline[]{n56}{n132}
\ncline[]{n56}{n137}
\ncline[]{n56}{n138}
\ncline[]{n56}{n139}
\ncline[]{n56}{n140}
\ncline[]{n56}{n141}
\ncline[]{n56}{n142}
\ncline[]{n56}{n143}
\ncline[]{n56}{n144}
\ncline[]{n56}{n146}
\ncline[]{n56}{n147}
\ncline[]{n56}{n148}
\ncline[]{n57}{n58}
\ncline[]{n57}{n59}
\ncline[]{n57}{n60}
\ncline[]{n57}{n61}
\ncline[]{n57}{n64}
\ncline[]{n57}{n65}
\ncline[]{n57}{n66}
\ncline[]{n57}{n70}
\ncline[]{n57}{n71}
\ncline[]{n57}{n72}
\ncline[]{n57}{n77}
\ncline[]{n57}{n79}
\ncline[]{n57}{n80}
\ncline[]{n57}{n82}
\ncline[]{n57}{n83}
\ncline[]{n57}{n84}
\ncline[]{n57}{n86}
\ncline[]{n57}{n87}
\ncline[]{n57}{n89}
\ncline[]{n57}{n92}
\ncline[]{n57}{n93}
\ncline[]{n57}{n94}
\ncline[]{n57}{n95}
\ncline[]{n57}{n97}
\ncline[]{n57}{n98}
\ncline[]{n57}{n99}
\ncline[]{n57}{n100}
\ncline[]{n57}{n101}
\ncline[]{n57}{n102}
\ncline[]{n57}{n103}
\ncline[]{n57}{n104}
\ncline[]{n57}{n107}
\ncline[]{n57}{n110}
\ncline[]{n57}{n112}
\ncline[]{n57}{n114}
\ncline[]{n57}{n116}
\ncline[]{n57}{n117}
\ncline[]{n57}{n118}
\ncline[]{n57}{n119}
\ncline[]{n57}{n120}
\ncline[]{n57}{n122}
\ncline[]{n57}{n123}
\ncline[]{n57}{n126}
\ncline[]{n57}{n128}
\ncline[]{n57}{n129}
\ncline[]{n57}{n131}
\ncline[]{n57}{n132}
\ncline[]{n57}{n137}
\ncline[]{n57}{n138}
\ncline[]{n57}{n139}
\ncline[]{n57}{n140}
\ncline[]{n57}{n141}
\ncline[]{n57}{n142}
\ncline[]{n57}{n143}
\ncline[]{n57}{n144}
\ncline[]{n57}{n146}
\ncline[]{n57}{n147}
\ncline[]{n57}{n148}
\ncline[]{n58}{n59}
\ncline[]{n58}{n60}
\ncline[]{n58}{n61}
\ncline[]{n58}{n64}
\ncline[]{n58}{n65}
\ncline[]{n58}{n66}
\ncline[]{n58}{n70}
\ncline[]{n58}{n71}
\ncline[]{n58}{n72}
\ncline[]{n58}{n77}
\ncline[]{n58}{n79}
\ncline[]{n58}{n80}
\ncline[]{n58}{n82}
\ncline[]{n58}{n83}
\ncline[]{n58}{n84}
\ncline[]{n58}{n86}
\ncline[]{n58}{n87}
\ncline[]{n58}{n89}
\ncline[]{n58}{n92}
\ncline[]{n58}{n93}
\ncline[]{n58}{n94}
\ncline[]{n58}{n95}
\ncline[]{n58}{n97}
\ncline[]{n58}{n98}
\ncline[]{n58}{n99}
\ncline[]{n58}{n100}
\ncline[]{n58}{n101}
\ncline[]{n58}{n102}
\ncline[]{n58}{n103}
\ncline[]{n58}{n104}
\ncline[]{n58}{n107}
\ncline[]{n58}{n110}
\ncline[]{n58}{n112}
\ncline[]{n58}{n114}
\ncline[]{n58}{n116}
\ncline[]{n58}{n117}
\ncline[]{n58}{n118}
\ncline[]{n58}{n119}
\ncline[]{n58}{n120}
\ncline[]{n58}{n122}
\ncline[]{n58}{n123}
\ncline[]{n58}{n126}
\ncline[]{n58}{n128}
\ncline[]{n58}{n129}
\ncline[]{n58}{n131}
\ncline[]{n58}{n132}
\ncline[]{n58}{n137}
\ncline[]{n58}{n138}
\ncline[]{n58}{n139}
\ncline[]{n58}{n140}
\ncline[]{n58}{n141}
\ncline[]{n58}{n142}
\ncline[]{n58}{n143}
\ncline[]{n58}{n144}
\ncline[]{n58}{n146}
\ncline[]{n58}{n147}
\ncline[]{n58}{n148}
\ncline[]{n59}{n60}
\ncline[]{n59}{n61}
\ncline[]{n59}{n64}
\ncline[]{n59}{n65}
\ncline[]{n59}{n66}
\ncline[]{n59}{n70}
\ncline[]{n59}{n71}
\ncline[]{n59}{n72}
\ncline[]{n59}{n77}
\ncline[]{n59}{n79}
\ncline[]{n59}{n80}
\ncline[]{n59}{n82}
\ncline[]{n59}{n83}
\ncline[]{n59}{n84}
\ncline[]{n59}{n86}
\ncline[]{n59}{n87}
\ncline[]{n59}{n89}
\ncline[]{n59}{n92}
\ncline[]{n59}{n93}
\ncline[]{n59}{n94}
\ncline[]{n59}{n95}
\ncline[]{n59}{n97}
\ncline[]{n59}{n98}
\ncline[]{n59}{n99}
\ncline[]{n59}{n100}
\ncline[]{n59}{n101}
\ncline[]{n59}{n102}
\ncline[]{n59}{n103}
\ncline[]{n59}{n104}
\ncline[]{n59}{n107}
\ncline[]{n59}{n110}
\ncline[]{n59}{n112}
\ncline[]{n59}{n114}
\ncline[]{n59}{n116}
\ncline[]{n59}{n117}
\ncline[]{n59}{n118}
\ncline[]{n59}{n119}
\ncline[]{n59}{n120}
\ncline[]{n59}{n122}
\ncline[]{n59}{n123}
\ncline[]{n59}{n126}
\ncline[]{n59}{n128}
\ncline[]{n59}{n129}
\ncline[]{n59}{n131}
\ncline[]{n59}{n132}
\ncline[]{n59}{n137}
\ncline[]{n59}{n138}
\ncline[]{n59}{n139}
\ncline[]{n59}{n140}
\ncline[]{n59}{n141}
\ncline[]{n59}{n142}
\ncline[]{n59}{n143}
\ncline[]{n59}{n144}
\ncline[]{n59}{n146}
\ncline[]{n59}{n147}
\ncline[]{n59}{n148}
\ncline[]{n60}{n61}
\ncline[]{n60}{n64}
\ncline[]{n60}{n65}
\ncline[]{n60}{n66}
\ncline[]{n60}{n70}
\ncline[]{n60}{n71}
\ncline[]{n60}{n72}
\ncline[]{n60}{n77}
\ncline[]{n60}{n79}
\ncline[]{n60}{n80}
\ncline[]{n60}{n82}
\ncline[]{n60}{n83}
\ncline[]{n60}{n84}
\ncline[]{n60}{n86}
\ncline[]{n60}{n87}
\ncline[]{n60}{n89}
\ncline[]{n60}{n92}
\ncline[]{n60}{n93}
\ncline[]{n60}{n94}
\ncline[]{n60}{n95}
\ncline[]{n60}{n97}
\ncline[]{n60}{n98}
\ncline[]{n60}{n99}
\ncline[]{n60}{n100}
\ncline[]{n60}{n101}
\ncline[]{n60}{n102}
\ncline[]{n60}{n103}
\ncline[]{n60}{n104}
\ncline[]{n60}{n107}
\ncline[]{n60}{n110}
\ncline[]{n60}{n112}
\ncline[]{n60}{n114}
\ncline[]{n60}{n116}
\ncline[]{n60}{n117}
\ncline[]{n60}{n118}
\ncline[]{n60}{n119}
\ncline[]{n60}{n120}
\ncline[]{n60}{n122}
\ncline[]{n60}{n123}
\ncline[]{n60}{n126}
\ncline[]{n60}{n128}
\ncline[]{n60}{n129}
\ncline[]{n60}{n131}
\ncline[]{n60}{n132}
\ncline[]{n60}{n137}
\ncline[]{n60}{n138}
\ncline[]{n60}{n139}
\ncline[]{n60}{n140}
\ncline[]{n60}{n141}
\ncline[]{n60}{n142}
\ncline[]{n60}{n143}
\ncline[]{n60}{n144}
\ncline[]{n60}{n146}
\ncline[]{n60}{n147}
\ncline[]{n60}{n148}
\ncline[]{n61}{n64}
\ncline[]{n61}{n65}
\ncline[]{n61}{n66}
\ncline[]{n61}{n70}
\ncline[]{n61}{n71}
\ncline[]{n61}{n72}
\ncline[]{n61}{n77}
\ncline[]{n61}{n79}
\ncline[]{n61}{n80}
\ncline[]{n61}{n82}
\ncline[]{n61}{n83}
\ncline[]{n61}{n84}
\ncline[]{n61}{n86}
\ncline[]{n61}{n87}
\ncline[]{n61}{n89}
\ncline[]{n61}{n92}
\ncline[]{n61}{n93}
\ncline[]{n61}{n94}
\ncline[]{n61}{n95}
\ncline[]{n61}{n97}
\ncline[]{n61}{n98}
\ncline[]{n61}{n99}
\ncline[]{n61}{n100}
\ncline[]{n61}{n101}
\ncline[]{n61}{n102}
\ncline[]{n61}{n103}
\ncline[]{n61}{n104}
\ncline[]{n61}{n107}
\ncline[]{n61}{n110}
\ncline[]{n61}{n112}
\ncline[]{n61}{n114}
\ncline[]{n61}{n116}
\ncline[]{n61}{n117}
\ncline[]{n61}{n118}
\ncline[]{n61}{n119}
\ncline[]{n61}{n120}
\ncline[]{n61}{n122}
\ncline[]{n61}{n123}
\ncline[]{n61}{n126}
\ncline[]{n61}{n128}
\ncline[]{n61}{n129}
\ncline[]{n61}{n131}
\ncline[]{n61}{n132}
\ncline[]{n61}{n137}
\ncline[]{n61}{n138}
\ncline[]{n61}{n139}
\ncline[]{n61}{n140}
\ncline[]{n61}{n141}
\ncline[]{n61}{n142}
\ncline[]{n61}{n143}
\ncline[]{n61}{n144}
\ncline[]{n61}{n146}
\ncline[]{n61}{n147}
\ncline[]{n61}{n148}
\ncline[]{n62}{n63}
\ncline[]{n62}{n68}
\ncline[]{n62}{n69}
\ncline[]{n62}{n73}
\ncline[]{n62}{n88}
\ncline[]{n62}{n111}
\ncline[]{n62}{n113}
\ncline[]{n62}{n115}
\ncline[]{n62}{n124}
\ncline[]{n62}{n127}
\ncline[]{n62}{n130}
\ncline[]{n62}{n133}
\ncline[]{n62}{n135}
\ncline[]{n62}{n136}
\ncline[]{n62}{n145}
\ncline[]{n62}{n149}
\ncline[]{n63}{n68}
\ncline[]{n63}{n69}
\ncline[]{n63}{n73}
\ncline[]{n63}{n88}
\ncline[]{n63}{n111}
\ncline[]{n63}{n113}
\ncline[]{n63}{n115}
\ncline[]{n63}{n124}
\ncline[]{n63}{n127}
\ncline[]{n63}{n130}
\ncline[]{n63}{n133}
\ncline[]{n63}{n135}
\ncline[]{n63}{n136}
\ncline[]{n63}{n145}
\ncline[]{n63}{n149}
\ncline[]{n64}{n65}
\ncline[]{n64}{n66}
\ncline[]{n64}{n70}
\ncline[]{n64}{n71}
\ncline[]{n64}{n72}
\ncline[]{n64}{n77}
\ncline[]{n64}{n79}
\ncline[]{n64}{n80}
\ncline[]{n64}{n82}
\ncline[]{n64}{n83}
\ncline[]{n64}{n84}
\ncline[]{n64}{n86}
\ncline[]{n64}{n87}
\ncline[]{n64}{n89}
\ncline[]{n64}{n92}
\ncline[]{n64}{n93}
\ncline[]{n64}{n94}
\ncline[]{n64}{n95}
\ncline[]{n64}{n97}
\ncline[]{n64}{n98}
\ncline[]{n64}{n99}
\ncline[]{n64}{n100}
\ncline[]{n64}{n101}
\ncline[]{n64}{n102}
\ncline[]{n64}{n103}
\ncline[]{n64}{n104}
\ncline[]{n64}{n107}
\ncline[]{n64}{n110}
\ncline[]{n64}{n112}
\ncline[]{n64}{n114}
\ncline[]{n64}{n116}
\ncline[]{n64}{n117}
\ncline[]{n64}{n118}
\ncline[]{n64}{n119}
\ncline[]{n64}{n120}
\ncline[]{n64}{n122}
\ncline[]{n64}{n123}
\ncline[]{n64}{n126}
\ncline[]{n64}{n128}
\ncline[]{n64}{n129}
\ncline[]{n64}{n131}
\ncline[]{n64}{n132}
\ncline[]{n64}{n137}
\ncline[]{n64}{n138}
\ncline[]{n64}{n139}
\ncline[]{n64}{n140}
\ncline[]{n64}{n141}
\ncline[]{n64}{n142}
\ncline[]{n64}{n143}
\ncline[]{n64}{n144}
\ncline[]{n64}{n146}
\ncline[]{n64}{n147}
\ncline[]{n64}{n148}
\ncline[]{n65}{n66}
\ncline[]{n65}{n70}
\ncline[]{n65}{n71}
\ncline[]{n65}{n72}
\ncline[]{n65}{n77}
\ncline[]{n65}{n79}
\ncline[]{n65}{n80}
\ncline[]{n65}{n82}
\ncline[]{n65}{n83}
\ncline[]{n65}{n84}
\ncline[]{n65}{n86}
\ncline[]{n65}{n87}
\ncline[]{n65}{n89}
\ncline[]{n65}{n92}
\ncline[]{n65}{n93}
\ncline[]{n65}{n94}
\ncline[]{n65}{n95}
\ncline[]{n65}{n97}
\ncline[]{n65}{n98}
\ncline[]{n65}{n99}
\ncline[]{n65}{n100}
\ncline[]{n65}{n101}
\ncline[]{n65}{n102}
\ncline[]{n65}{n103}
\ncline[]{n65}{n104}
\ncline[]{n65}{n107}
\ncline[]{n65}{n110}
\ncline[]{n65}{n112}
\ncline[]{n65}{n114}
\ncline[]{n65}{n116}
\ncline[]{n65}{n117}
\ncline[]{n65}{n118}
\ncline[]{n65}{n119}
\ncline[]{n65}{n120}
\ncline[]{n65}{n122}
\ncline[]{n65}{n123}
\ncline[]{n65}{n126}
\ncline[]{n65}{n128}
\ncline[]{n65}{n129}
\ncline[]{n65}{n131}
\ncline[]{n65}{n132}
\ncline[]{n65}{n137}
\ncline[]{n65}{n138}
\ncline[]{n65}{n139}
\ncline[]{n65}{n140}
\ncline[]{n65}{n141}
\ncline[]{n65}{n142}
\ncline[]{n65}{n143}
\ncline[]{n65}{n144}
\ncline[]{n65}{n146}
\ncline[]{n65}{n147}
\ncline[]{n65}{n148}
\ncline[]{n66}{n70}
\ncline[]{n66}{n71}
\ncline[]{n66}{n72}
\ncline[]{n66}{n77}
\ncline[]{n66}{n79}
\ncline[]{n66}{n80}
\ncline[]{n66}{n82}
\ncline[]{n66}{n83}
\ncline[]{n66}{n84}
\ncline[]{n66}{n86}
\ncline[]{n66}{n87}
\ncline[]{n66}{n89}
\ncline[]{n66}{n92}
\ncline[]{n66}{n93}
\ncline[]{n66}{n94}
\ncline[]{n66}{n95}
\ncline[]{n66}{n97}
\ncline[]{n66}{n98}
\ncline[]{n66}{n99}
\ncline[]{n66}{n100}
\ncline[]{n66}{n101}
\ncline[]{n66}{n102}
\ncline[]{n66}{n103}
\ncline[]{n66}{n104}
\ncline[]{n66}{n107}
\ncline[]{n66}{n110}
\ncline[]{n66}{n112}
\ncline[]{n66}{n114}
\ncline[]{n66}{n116}
\ncline[]{n66}{n117}
\ncline[]{n66}{n118}
\ncline[]{n66}{n119}
\ncline[]{n66}{n120}
\ncline[]{n66}{n122}
\ncline[]{n66}{n123}
\ncline[]{n66}{n126}
\ncline[]{n66}{n128}
\ncline[]{n66}{n129}
\ncline[]{n66}{n131}
\ncline[]{n66}{n132}
\ncline[]{n66}{n137}
\ncline[]{n66}{n138}
\ncline[]{n66}{n139}
\ncline[]{n66}{n140}
\ncline[]{n66}{n141}
\ncline[]{n66}{n142}
\ncline[]{n66}{n143}
\ncline[]{n66}{n144}
\ncline[]{n66}{n146}
\ncline[]{n66}{n147}
\ncline[]{n66}{n148}
\ncline[]{n67}{n74}
\ncline[]{n67}{n75}
\ncline[]{n67}{n76}
\ncline[]{n67}{n78}
\ncline[]{n67}{n81}
\ncline[]{n67}{n85}
\ncline[]{n67}{n90}
\ncline[]{n67}{n91}
\ncline[]{n67}{n96}
\ncline[]{n67}{n105}
\ncline[]{n67}{n106}
\ncline[]{n67}{n108}
\ncline[]{n67}{n109}
\ncline[]{n67}{n121}
\ncline[]{n67}{n125}
\ncline[]{n67}{n134}
\ncline[]{n68}{n69}
\ncline[]{n68}{n73}
\ncline[]{n68}{n88}
\ncline[]{n68}{n111}
\ncline[]{n68}{n113}
\ncline[]{n68}{n115}
\ncline[]{n68}{n124}
\ncline[]{n68}{n127}
\ncline[]{n68}{n130}
\ncline[]{n68}{n133}
\ncline[]{n68}{n135}
\ncline[]{n68}{n136}
\ncline[]{n68}{n145}
\ncline[]{n68}{n149}
\ncline[]{n69}{n73}
\ncline[]{n69}{n88}
\ncline[]{n69}{n111}
\ncline[]{n69}{n113}
\ncline[]{n69}{n115}
\ncline[]{n69}{n124}
\ncline[]{n69}{n127}
\ncline[]{n69}{n130}
\ncline[]{n69}{n133}
\ncline[]{n69}{n135}
\ncline[]{n69}{n136}
\ncline[]{n69}{n145}
\ncline[]{n69}{n149}
\ncline[]{n70}{n71}
\ncline[]{n70}{n72}
\ncline[]{n70}{n77}
\ncline[]{n70}{n79}
\ncline[]{n70}{n80}
\ncline[]{n70}{n82}
\ncline[]{n70}{n83}
\ncline[]{n70}{n84}
\ncline[]{n70}{n86}
\ncline[]{n70}{n87}
\ncline[]{n70}{n89}
\ncline[]{n70}{n92}
\ncline[]{n70}{n93}
\ncline[]{n70}{n94}
\ncline[]{n70}{n95}
\ncline[]{n70}{n97}
\ncline[]{n70}{n98}
\ncline[]{n70}{n99}
\ncline[]{n70}{n100}
\ncline[]{n70}{n101}
\ncline[]{n70}{n102}
\ncline[]{n70}{n103}
\ncline[]{n70}{n104}
\ncline[]{n70}{n107}
\ncline[]{n70}{n110}
\ncline[]{n70}{n112}
\ncline[]{n70}{n114}
\ncline[]{n70}{n116}
\ncline[]{n70}{n117}
\ncline[]{n70}{n118}
\ncline[]{n70}{n119}
\ncline[]{n70}{n120}
\ncline[]{n70}{n122}
\ncline[]{n70}{n123}
\ncline[]{n70}{n126}
\ncline[]{n70}{n128}
\ncline[]{n70}{n129}
\ncline[]{n70}{n131}
\ncline[]{n70}{n132}
\ncline[]{n70}{n137}
\ncline[]{n70}{n138}
\ncline[]{n70}{n139}
\ncline[]{n70}{n140}
\ncline[]{n70}{n141}
\ncline[]{n70}{n142}
\ncline[]{n70}{n143}
\ncline[]{n70}{n144}
\ncline[]{n70}{n146}
\ncline[]{n70}{n147}
\ncline[]{n70}{n148}
\ncline[]{n71}{n72}
\ncline[]{n71}{n77}
\ncline[]{n71}{n79}
\ncline[]{n71}{n80}
\ncline[]{n71}{n82}
\ncline[]{n71}{n83}
\ncline[]{n71}{n84}
\ncline[]{n71}{n86}
\ncline[]{n71}{n87}
\ncline[]{n71}{n89}
\ncline[]{n71}{n92}
\ncline[]{n71}{n93}
\ncline[]{n71}{n94}
\ncline[]{n71}{n95}
\ncline[]{n71}{n97}
\ncline[]{n71}{n98}
\ncline[]{n71}{n99}
\ncline[]{n71}{n100}
\ncline[]{n71}{n101}
\ncline[]{n71}{n102}
\ncline[]{n71}{n103}
\ncline[]{n71}{n104}
\ncline[]{n71}{n107}
\ncline[]{n71}{n110}
\ncline[]{n71}{n112}
\ncline[]{n71}{n114}
\ncline[]{n71}{n116}
\ncline[]{n71}{n117}
\ncline[]{n71}{n118}
\ncline[]{n71}{n119}
\ncline[]{n71}{n120}
\ncline[]{n71}{n122}
\ncline[]{n71}{n123}
\ncline[]{n71}{n126}
\ncline[]{n71}{n128}
\ncline[]{n71}{n129}
\ncline[]{n71}{n131}
\ncline[]{n71}{n132}
\ncline[]{n71}{n137}
\ncline[]{n71}{n138}
\ncline[]{n71}{n139}
\ncline[]{n71}{n140}
\ncline[]{n71}{n141}
\ncline[]{n71}{n142}
\ncline[]{n71}{n143}
\ncline[]{n71}{n144}
\ncline[]{n71}{n146}
\ncline[]{n71}{n147}
\ncline[]{n71}{n148}
\ncline[]{n72}{n77}
\ncline[]{n72}{n79}
\ncline[]{n72}{n80}
\ncline[]{n72}{n82}
\ncline[]{n72}{n83}
\ncline[]{n72}{n84}
\ncline[]{n72}{n86}
\ncline[]{n72}{n87}
\ncline[]{n72}{n89}
\ncline[]{n72}{n92}
\ncline[]{n72}{n93}
\ncline[]{n72}{n94}
\ncline[]{n72}{n95}
\ncline[]{n72}{n97}
\ncline[]{n72}{n98}
\ncline[]{n72}{n99}
\ncline[]{n72}{n100}
\ncline[]{n72}{n101}
\ncline[]{n72}{n102}
\ncline[]{n72}{n103}
\ncline[]{n72}{n104}
\ncline[]{n72}{n107}
\ncline[]{n72}{n110}
\ncline[]{n72}{n112}
\ncline[]{n72}{n114}
\ncline[]{n72}{n116}
\ncline[]{n72}{n117}
\ncline[]{n72}{n118}
\ncline[]{n72}{n119}
\ncline[]{n72}{n120}
\ncline[]{n72}{n122}
\ncline[]{n72}{n123}
\ncline[]{n72}{n126}
\ncline[]{n72}{n128}
\ncline[]{n72}{n129}
\ncline[]{n72}{n131}
\ncline[]{n72}{n132}
\ncline[]{n72}{n137}
\ncline[]{n72}{n138}
\ncline[]{n72}{n139}
\ncline[]{n72}{n140}
\ncline[]{n72}{n141}
\ncline[]{n72}{n142}
\ncline[]{n72}{n143}
\ncline[]{n72}{n144}
\ncline[]{n72}{n146}
\ncline[]{n72}{n147}
\ncline[]{n72}{n148}
\ncline[]{n73}{n88}
\ncline[]{n73}{n111}
\ncline[]{n73}{n113}
\ncline[]{n73}{n115}
\ncline[]{n73}{n124}
\ncline[]{n73}{n127}
\ncline[]{n73}{n130}
\ncline[]{n73}{n133}
\ncline[]{n73}{n135}
\ncline[]{n73}{n136}
\ncline[]{n73}{n145}
\ncline[]{n73}{n149}
\ncline[]{n74}{n75}
\ncline[]{n74}{n76}
\ncline[]{n74}{n78}
\ncline[]{n74}{n81}
\ncline[]{n74}{n85}
\ncline[]{n74}{n90}
\ncline[]{n74}{n91}
\ncline[]{n74}{n96}
\ncline[]{n74}{n105}
\ncline[]{n74}{n106}
\ncline[]{n74}{n108}
\ncline[]{n74}{n109}
\ncline[]{n74}{n121}
\ncline[]{n74}{n125}
\ncline[]{n74}{n134}
\ncline[]{n75}{n76}
\ncline[]{n75}{n78}
\ncline[]{n75}{n81}
\ncline[]{n75}{n85}
\ncline[]{n75}{n90}
\ncline[]{n75}{n91}
\ncline[]{n75}{n96}
\ncline[]{n75}{n105}
\ncline[]{n75}{n106}
\ncline[]{n75}{n108}
\ncline[]{n75}{n109}
\ncline[]{n75}{n121}
\ncline[]{n75}{n125}
\ncline[]{n75}{n134}
\ncline[]{n76}{n78}
\ncline[]{n76}{n81}
\ncline[]{n76}{n85}
\ncline[]{n76}{n90}
\ncline[]{n76}{n91}
\ncline[]{n76}{n96}
\ncline[]{n76}{n105}
\ncline[]{n76}{n106}
\ncline[]{n76}{n108}
\ncline[]{n76}{n109}
\ncline[]{n76}{n121}
\ncline[]{n76}{n125}
\ncline[]{n76}{n134}
\ncline[]{n77}{n79}
\ncline[]{n77}{n80}
\ncline[]{n77}{n82}
\ncline[]{n77}{n83}
\ncline[]{n77}{n84}
\ncline[]{n77}{n86}
\ncline[]{n77}{n87}
\ncline[]{n77}{n89}
\ncline[]{n77}{n92}
\ncline[]{n77}{n93}
\ncline[]{n77}{n94}
\ncline[]{n77}{n95}
\ncline[]{n77}{n97}
\ncline[]{n77}{n98}
\ncline[]{n77}{n99}
\ncline[]{n77}{n100}
\ncline[]{n77}{n101}
\ncline[]{n77}{n102}
\ncline[]{n77}{n103}
\ncline[]{n77}{n104}
\ncline[]{n77}{n107}
\ncline[]{n77}{n110}
\ncline[]{n77}{n112}
\ncline[]{n77}{n114}
\ncline[]{n77}{n116}
\ncline[]{n77}{n117}
\ncline[]{n77}{n118}
\ncline[]{n77}{n119}
\ncline[]{n77}{n120}
\ncline[]{n77}{n122}
\ncline[]{n77}{n123}
\ncline[]{n77}{n126}
\ncline[]{n77}{n128}
\ncline[]{n77}{n129}
\ncline[]{n77}{n131}
\ncline[]{n77}{n132}
\ncline[]{n77}{n137}
\ncline[]{n77}{n138}
\ncline[]{n77}{n139}
\ncline[]{n77}{n140}
\ncline[]{n77}{n141}
\ncline[]{n77}{n142}
\ncline[]{n77}{n143}
\ncline[]{n77}{n144}
\ncline[]{n77}{n146}
\ncline[]{n77}{n147}
\ncline[]{n77}{n148}
\ncline[]{n78}{n81}
\ncline[]{n78}{n85}
\ncline[]{n78}{n90}
\ncline[]{n78}{n91}
\ncline[]{n78}{n96}
\ncline[]{n78}{n105}
\ncline[]{n78}{n106}
\ncline[]{n78}{n108}
\ncline[]{n78}{n109}
\ncline[]{n78}{n121}
\ncline[]{n78}{n125}
\ncline[]{n78}{n134}
\ncline[]{n79}{n80}
\ncline[]{n79}{n82}
\ncline[]{n79}{n83}
\ncline[]{n79}{n84}
\ncline[]{n79}{n86}
\ncline[]{n79}{n87}
\ncline[]{n79}{n89}
\ncline[]{n79}{n92}
\ncline[]{n79}{n93}
\ncline[]{n79}{n94}
\ncline[]{n79}{n95}
\ncline[]{n79}{n97}
\ncline[]{n79}{n98}
\ncline[]{n79}{n99}
\ncline[]{n79}{n100}
\ncline[]{n79}{n101}
\ncline[]{n79}{n102}
\ncline[]{n79}{n103}
\ncline[]{n79}{n104}
\ncline[]{n79}{n107}
\ncline[]{n79}{n110}
\ncline[]{n79}{n112}
\ncline[]{n79}{n114}
\ncline[]{n79}{n116}
\ncline[]{n79}{n117}
\ncline[]{n79}{n118}
\ncline[]{n79}{n119}
\ncline[]{n79}{n120}
\ncline[]{n79}{n122}
\ncline[]{n79}{n123}
\ncline[]{n79}{n126}
\ncline[]{n79}{n128}
\ncline[]{n79}{n129}
\ncline[]{n79}{n131}
\ncline[]{n79}{n132}
\ncline[]{n79}{n137}
\ncline[]{n79}{n138}
\ncline[]{n79}{n139}
\ncline[]{n79}{n140}
\ncline[]{n79}{n141}
\ncline[]{n79}{n142}
\ncline[]{n79}{n143}
\ncline[]{n79}{n144}
\ncline[]{n79}{n146}
\ncline[]{n79}{n147}
\ncline[]{n79}{n148}
\ncline[]{n80}{n82}
\ncline[]{n80}{n83}
\ncline[]{n80}{n84}
\ncline[]{n80}{n86}
\ncline[]{n80}{n87}
\ncline[]{n80}{n89}
\ncline[]{n80}{n92}
\ncline[]{n80}{n93}
\ncline[]{n80}{n94}
\ncline[]{n80}{n95}
\ncline[]{n80}{n97}
\ncline[]{n80}{n98}
\ncline[]{n80}{n99}
\ncline[]{n80}{n100}
\ncline[]{n80}{n101}
\ncline[]{n80}{n102}
\ncline[]{n80}{n103}
\ncline[]{n80}{n104}
\ncline[]{n80}{n107}
\ncline[]{n80}{n110}
\ncline[]{n80}{n112}
\ncline[]{n80}{n114}
\ncline[]{n80}{n116}
\ncline[]{n80}{n117}
\ncline[]{n80}{n118}
\ncline[]{n80}{n119}
\ncline[]{n80}{n120}
\ncline[]{n80}{n122}
\ncline[]{n80}{n123}
\ncline[]{n80}{n126}
\ncline[]{n80}{n128}
\ncline[]{n80}{n129}
\ncline[]{n80}{n131}
\ncline[]{n80}{n132}
\ncline[]{n80}{n137}
\ncline[]{n80}{n138}
\ncline[]{n80}{n139}
\ncline[]{n80}{n140}
\ncline[]{n80}{n141}
\ncline[]{n80}{n142}
\ncline[]{n80}{n143}
\ncline[]{n80}{n144}
\ncline[]{n80}{n146}
\ncline[]{n80}{n147}
\ncline[]{n80}{n148}
\ncline[]{n81}{n85}
\ncline[]{n81}{n90}
\ncline[]{n81}{n91}
\ncline[]{n81}{n96}
\ncline[]{n81}{n105}
\ncline[]{n81}{n106}
\ncline[]{n81}{n108}
\ncline[]{n81}{n109}
\ncline[]{n81}{n121}
\ncline[]{n81}{n125}
\ncline[]{n81}{n134}
\ncline[]{n82}{n83}
\ncline[]{n82}{n84}
\ncline[]{n82}{n86}
\ncline[]{n82}{n87}
\ncline[]{n82}{n89}
\ncline[]{n82}{n92}
\ncline[]{n82}{n93}
\ncline[]{n82}{n94}
\ncline[]{n82}{n95}
\ncline[]{n82}{n97}
\ncline[]{n82}{n98}
\ncline[]{n82}{n99}
\ncline[]{n82}{n100}
\ncline[]{n82}{n101}
\ncline[]{n82}{n102}
\ncline[]{n82}{n103}
\ncline[]{n82}{n104}
\ncline[]{n82}{n107}
\ncline[]{n82}{n110}
\ncline[]{n82}{n112}
\ncline[]{n82}{n114}
\ncline[]{n82}{n116}
\ncline[]{n82}{n117}
\ncline[]{n82}{n118}
\ncline[]{n82}{n119}
\ncline[]{n82}{n120}
\ncline[]{n82}{n122}
\ncline[]{n82}{n123}
\ncline[]{n82}{n126}
\ncline[]{n82}{n128}
\ncline[]{n82}{n129}
\ncline[]{n82}{n131}
\ncline[]{n82}{n132}
\ncline[]{n82}{n137}
\ncline[]{n82}{n138}
\ncline[]{n82}{n139}
\ncline[]{n82}{n140}
\ncline[]{n82}{n141}
\ncline[]{n82}{n142}
\ncline[]{n82}{n143}
\ncline[]{n82}{n144}
\ncline[]{n82}{n146}
\ncline[]{n82}{n147}
\ncline[]{n82}{n148}
\ncline[]{n83}{n84}
\ncline[]{n83}{n86}
\ncline[]{n83}{n87}
\ncline[]{n83}{n89}
\ncline[]{n83}{n92}
\ncline[]{n83}{n93}
\ncline[]{n83}{n94}
\ncline[]{n83}{n95}
\ncline[]{n83}{n97}
\ncline[]{n83}{n98}
\ncline[]{n83}{n99}
\ncline[]{n83}{n100}
\ncline[]{n83}{n101}
\ncline[]{n83}{n102}
\ncline[]{n83}{n103}
\ncline[]{n83}{n104}
\ncline[]{n83}{n107}
\ncline[]{n83}{n110}
\ncline[]{n83}{n112}
\ncline[]{n83}{n114}
\ncline[]{n83}{n116}
\ncline[]{n83}{n117}
\ncline[]{n83}{n118}
\ncline[]{n83}{n119}
\ncline[]{n83}{n120}
\ncline[]{n83}{n122}
\ncline[]{n83}{n123}
\ncline[]{n83}{n126}
\ncline[]{n83}{n128}
\ncline[]{n83}{n129}
\ncline[]{n83}{n131}
\ncline[]{n83}{n132}
\ncline[]{n83}{n137}
\ncline[]{n83}{n138}
\ncline[]{n83}{n139}
\ncline[]{n83}{n140}
\ncline[]{n83}{n141}
\ncline[]{n83}{n142}
\ncline[]{n83}{n143}
\ncline[]{n83}{n144}
\ncline[]{n83}{n146}
\ncline[]{n83}{n147}
\ncline[]{n83}{n148}
\ncline[]{n84}{n86}
\ncline[]{n84}{n87}
\ncline[]{n84}{n89}
\ncline[]{n84}{n92}
\ncline[]{n84}{n93}
\ncline[]{n84}{n94}
\ncline[]{n84}{n95}
\ncline[]{n84}{n97}
\ncline[]{n84}{n98}
\ncline[]{n84}{n99}
\ncline[]{n84}{n100}
\ncline[]{n84}{n101}
\ncline[]{n84}{n102}
\ncline[]{n84}{n103}
\ncline[]{n84}{n104}
\ncline[]{n84}{n107}
\ncline[]{n84}{n110}
\ncline[]{n84}{n112}
\ncline[]{n84}{n114}
\ncline[]{n84}{n116}
\ncline[]{n84}{n117}
\ncline[]{n84}{n118}
\ncline[]{n84}{n119}
\ncline[]{n84}{n120}
\ncline[]{n84}{n122}
\ncline[]{n84}{n123}
\ncline[]{n84}{n126}
\ncline[]{n84}{n128}
\ncline[]{n84}{n129}
\ncline[]{n84}{n131}
\ncline[]{n84}{n132}
\ncline[]{n84}{n137}
\ncline[]{n84}{n138}
\ncline[]{n84}{n139}
\ncline[]{n84}{n140}
\ncline[]{n84}{n141}
\ncline[]{n84}{n142}
\ncline[]{n84}{n143}
\ncline[]{n84}{n144}
\ncline[]{n84}{n146}
\ncline[]{n84}{n147}
\ncline[]{n84}{n148}
\ncline[]{n85}{n90}
\ncline[]{n85}{n91}
\ncline[]{n85}{n96}
\ncline[]{n85}{n105}
\ncline[]{n85}{n106}
\ncline[]{n85}{n108}
\ncline[]{n85}{n109}
\ncline[]{n85}{n121}
\ncline[]{n85}{n125}
\ncline[]{n85}{n134}
\ncline[]{n86}{n87}
\ncline[]{n86}{n89}
\ncline[]{n86}{n92}
\ncline[]{n86}{n93}
\ncline[]{n86}{n94}
\ncline[]{n86}{n95}
\ncline[]{n86}{n97}
\ncline[]{n86}{n98}
\ncline[]{n86}{n99}
\ncline[]{n86}{n100}
\ncline[]{n86}{n101}
\ncline[]{n86}{n102}
\ncline[]{n86}{n103}
\ncline[]{n86}{n104}
\ncline[]{n86}{n107}
\ncline[]{n86}{n110}
\ncline[]{n86}{n112}
\ncline[]{n86}{n114}
\ncline[]{n86}{n116}
\ncline[]{n86}{n117}
\ncline[]{n86}{n118}
\ncline[]{n86}{n119}
\ncline[]{n86}{n120}
\ncline[]{n86}{n122}
\ncline[]{n86}{n123}
\ncline[]{n86}{n126}
\ncline[]{n86}{n128}
\ncline[]{n86}{n129}
\ncline[]{n86}{n131}
\ncline[]{n86}{n132}
\ncline[]{n86}{n137}
\ncline[]{n86}{n138}
\ncline[]{n86}{n139}
\ncline[]{n86}{n140}
\ncline[]{n86}{n141}
\ncline[]{n86}{n142}
\ncline[]{n86}{n143}
\ncline[]{n86}{n144}
\ncline[]{n86}{n146}
\ncline[]{n86}{n147}
\ncline[]{n86}{n148}
\ncline[]{n87}{n89}
\ncline[]{n87}{n92}
\ncline[]{n87}{n93}
\ncline[]{n87}{n94}
\ncline[]{n87}{n95}
\ncline[]{n87}{n97}
\ncline[]{n87}{n98}
\ncline[]{n87}{n99}
\ncline[]{n87}{n100}
\ncline[]{n87}{n101}
\ncline[]{n87}{n102}
\ncline[]{n87}{n103}
\ncline[]{n87}{n104}
\ncline[]{n87}{n107}
\ncline[]{n87}{n110}
\ncline[]{n87}{n112}
\ncline[]{n87}{n114}
\ncline[]{n87}{n116}
\ncline[]{n87}{n117}
\ncline[]{n87}{n118}
\ncline[]{n87}{n119}
\ncline[]{n87}{n120}
\ncline[]{n87}{n122}
\ncline[]{n87}{n123}
\ncline[]{n87}{n126}
\ncline[]{n87}{n128}
\ncline[]{n87}{n129}
\ncline[]{n87}{n131}
\ncline[]{n87}{n132}
\ncline[]{n87}{n137}
\ncline[]{n87}{n138}
\ncline[]{n87}{n139}
\ncline[]{n87}{n140}
\ncline[]{n87}{n141}
\ncline[]{n87}{n142}
\ncline[]{n87}{n143}
\ncline[]{n87}{n144}
\ncline[]{n87}{n146}
\ncline[]{n87}{n147}
\ncline[]{n87}{n148}
\ncline[]{n88}{n111}
\ncline[]{n88}{n113}
\ncline[]{n88}{n115}
\ncline[]{n88}{n124}
\ncline[]{n88}{n127}
\ncline[]{n88}{n130}
\ncline[]{n88}{n133}
\ncline[]{n88}{n135}
\ncline[]{n88}{n136}
\ncline[]{n88}{n145}
\ncline[]{n88}{n149}
\ncline[]{n89}{n92}
\ncline[]{n89}{n93}
\ncline[]{n89}{n94}
\ncline[]{n89}{n95}
\ncline[]{n89}{n97}
\ncline[]{n89}{n98}
\ncline[]{n89}{n99}
\ncline[]{n89}{n100}
\ncline[]{n89}{n101}
\ncline[]{n89}{n102}
\ncline[]{n89}{n103}
\ncline[]{n89}{n104}
\ncline[]{n89}{n107}
\ncline[]{n89}{n110}
\ncline[]{n89}{n112}
\ncline[]{n89}{n114}
\ncline[]{n89}{n116}
\ncline[]{n89}{n117}
\ncline[]{n89}{n118}
\ncline[]{n89}{n119}
\ncline[]{n89}{n120}
\ncline[]{n89}{n122}
\ncline[]{n89}{n123}
\ncline[]{n89}{n126}
\ncline[]{n89}{n128}
\ncline[]{n89}{n129}
\ncline[]{n89}{n131}
\ncline[]{n89}{n132}
\ncline[]{n89}{n137}
\ncline[]{n89}{n138}
\ncline[]{n89}{n139}
\ncline[]{n89}{n140}
\ncline[]{n89}{n141}
\ncline[]{n89}{n142}
\ncline[]{n89}{n143}
\ncline[]{n89}{n144}
\ncline[]{n89}{n146}
\ncline[]{n89}{n147}
\ncline[]{n89}{n148}
\ncline[]{n90}{n91}
\ncline[]{n90}{n96}
\ncline[]{n90}{n105}
\ncline[]{n90}{n106}
\ncline[]{n90}{n108}
\ncline[]{n90}{n109}
\ncline[]{n90}{n121}
\ncline[]{n90}{n125}
\ncline[]{n90}{n134}
\ncline[]{n91}{n96}
\ncline[]{n91}{n105}
\ncline[]{n91}{n106}
\ncline[]{n91}{n108}
\ncline[]{n91}{n109}
\ncline[]{n91}{n121}
\ncline[]{n91}{n125}
\ncline[]{n91}{n134}
\ncline[]{n92}{n93}
\ncline[]{n92}{n94}
\ncline[]{n92}{n95}
\ncline[]{n92}{n97}
\ncline[]{n92}{n98}
\ncline[]{n92}{n99}
\ncline[]{n92}{n100}
\ncline[]{n92}{n101}
\ncline[]{n92}{n102}
\ncline[]{n92}{n103}
\ncline[]{n92}{n104}
\ncline[]{n92}{n107}
\ncline[]{n92}{n110}
\ncline[]{n92}{n112}
\ncline[]{n92}{n114}
\ncline[]{n92}{n116}
\ncline[]{n92}{n117}
\ncline[]{n92}{n118}
\ncline[]{n92}{n119}
\ncline[]{n92}{n120}
\ncline[]{n92}{n122}
\ncline[]{n92}{n123}
\ncline[]{n92}{n126}
\ncline[]{n92}{n128}
\ncline[]{n92}{n129}
\ncline[]{n92}{n131}
\ncline[]{n92}{n132}
\ncline[]{n92}{n137}
\ncline[]{n92}{n138}
\ncline[]{n92}{n139}
\ncline[]{n92}{n140}
\ncline[]{n92}{n141}
\ncline[]{n92}{n142}
\ncline[]{n92}{n143}
\ncline[]{n92}{n144}
\ncline[]{n92}{n146}
\ncline[]{n92}{n147}
\ncline[]{n92}{n148}
\ncline[]{n93}{n94}
\ncline[]{n93}{n95}
\ncline[]{n93}{n97}
\ncline[]{n93}{n98}
\ncline[]{n93}{n99}
\ncline[]{n93}{n100}
\ncline[]{n93}{n101}
\ncline[]{n93}{n102}
\ncline[]{n93}{n103}
\ncline[]{n93}{n104}
\ncline[]{n93}{n107}
\ncline[]{n93}{n110}
\ncline[]{n93}{n112}
\ncline[]{n93}{n114}
\ncline[]{n93}{n116}
\ncline[]{n93}{n117}
\ncline[]{n93}{n118}
\ncline[]{n93}{n119}
\ncline[]{n93}{n120}
\ncline[]{n93}{n122}
\ncline[]{n93}{n123}
\ncline[]{n93}{n126}
\ncline[]{n93}{n128}
\ncline[]{n93}{n129}
\ncline[]{n93}{n131}
\ncline[]{n93}{n132}
\ncline[]{n93}{n137}
\ncline[]{n93}{n138}
\ncline[]{n93}{n139}
\ncline[]{n93}{n140}
\ncline[]{n93}{n141}
\ncline[]{n93}{n142}
\ncline[]{n93}{n143}
\ncline[]{n93}{n144}
\ncline[]{n93}{n146}
\ncline[]{n93}{n147}
\ncline[]{n93}{n148}
\ncline[]{n94}{n95}
\ncline[]{n94}{n97}
\ncline[]{n94}{n98}
\ncline[]{n94}{n99}
\ncline[]{n94}{n100}
\ncline[]{n94}{n101}
\ncline[]{n94}{n102}
\ncline[]{n94}{n103}
\ncline[]{n94}{n104}
\ncline[]{n94}{n107}
\ncline[]{n94}{n110}
\ncline[]{n94}{n112}
\ncline[]{n94}{n114}
\ncline[]{n94}{n116}
\ncline[]{n94}{n117}
\ncline[]{n94}{n118}
\ncline[]{n94}{n119}
\ncline[]{n94}{n120}
\ncline[]{n94}{n122}
\ncline[]{n94}{n123}
\ncline[]{n94}{n126}
\ncline[]{n94}{n128}
\ncline[]{n94}{n129}
\ncline[]{n94}{n131}
\ncline[]{n94}{n132}
\ncline[]{n94}{n137}
\ncline[]{n94}{n138}
\ncline[]{n94}{n139}
\ncline[]{n94}{n140}
\ncline[]{n94}{n141}
\ncline[]{n94}{n142}
\ncline[]{n94}{n143}
\ncline[]{n94}{n144}
\ncline[]{n94}{n146}
\ncline[]{n94}{n147}
\ncline[]{n94}{n148}
\ncline[]{n95}{n97}
\ncline[]{n95}{n98}
\ncline[]{n95}{n99}
\ncline[]{n95}{n100}
\ncline[]{n95}{n101}
\ncline[]{n95}{n102}
\ncline[]{n95}{n103}
\ncline[]{n95}{n104}
\ncline[]{n95}{n107}
\ncline[]{n95}{n110}
\ncline[]{n95}{n112}
\ncline[]{n95}{n114}
\ncline[]{n95}{n116}
\ncline[]{n95}{n117}
\ncline[]{n95}{n118}
\ncline[]{n95}{n119}
\ncline[]{n95}{n120}
\ncline[]{n95}{n122}
\ncline[]{n95}{n123}
\ncline[]{n95}{n126}
\ncline[]{n95}{n128}
\ncline[]{n95}{n129}
\ncline[]{n95}{n131}
\ncline[]{n95}{n132}
\ncline[]{n95}{n137}
\ncline[]{n95}{n138}
\ncline[]{n95}{n139}
\ncline[]{n95}{n140}
\ncline[]{n95}{n141}
\ncline[]{n95}{n142}
\ncline[]{n95}{n143}
\ncline[]{n95}{n144}
\ncline[]{n95}{n146}
\ncline[]{n95}{n147}
\ncline[]{n95}{n148}
\ncline[]{n96}{n105}
\ncline[]{n96}{n106}
\ncline[]{n96}{n108}
\ncline[]{n96}{n109}
\ncline[]{n96}{n121}
\ncline[]{n96}{n125}
\ncline[]{n96}{n134}
\ncline[]{n97}{n98}
\ncline[]{n97}{n99}
\ncline[]{n97}{n100}
\ncline[]{n97}{n101}
\ncline[]{n97}{n102}
\ncline[]{n97}{n103}
\ncline[]{n97}{n104}
\ncline[]{n97}{n107}
\ncline[]{n97}{n110}
\ncline[]{n97}{n112}
\ncline[]{n97}{n114}
\ncline[]{n97}{n116}
\ncline[]{n97}{n117}
\ncline[]{n97}{n118}
\ncline[]{n97}{n119}
\ncline[]{n97}{n120}
\ncline[]{n97}{n122}
\ncline[]{n97}{n123}
\ncline[]{n97}{n126}
\ncline[]{n97}{n128}
\ncline[]{n97}{n129}
\ncline[]{n97}{n131}
\ncline[]{n97}{n132}
\ncline[]{n97}{n137}
\ncline[]{n97}{n138}
\ncline[]{n97}{n139}
\ncline[]{n97}{n140}
\ncline[]{n97}{n141}
\ncline[]{n97}{n142}
\ncline[]{n97}{n143}
\ncline[]{n97}{n144}
\ncline[]{n97}{n146}
\ncline[]{n97}{n147}
\ncline[]{n97}{n148}
\ncline[]{n98}{n99}
\ncline[]{n98}{n100}
\ncline[]{n98}{n101}
\ncline[]{n98}{n102}
\ncline[]{n98}{n103}
\ncline[]{n98}{n104}
\ncline[]{n98}{n107}
\ncline[]{n98}{n110}
\ncline[]{n98}{n112}
\ncline[]{n98}{n114}
\ncline[]{n98}{n116}
\ncline[]{n98}{n117}
\ncline[]{n98}{n118}
\ncline[]{n98}{n119}
\ncline[]{n98}{n120}
\ncline[]{n98}{n122}
\ncline[]{n98}{n123}
\ncline[]{n98}{n126}
\ncline[]{n98}{n128}
\ncline[]{n98}{n129}
\ncline[]{n98}{n131}
\ncline[]{n98}{n132}
\ncline[]{n98}{n137}
\ncline[]{n98}{n138}
\ncline[]{n98}{n139}
\ncline[]{n98}{n140}
\ncline[]{n98}{n141}
\ncline[]{n98}{n142}
\ncline[]{n98}{n143}
\ncline[]{n98}{n144}
\ncline[]{n98}{n146}
\ncline[]{n98}{n147}
\ncline[]{n98}{n148}
\ncline[]{n99}{n100}
\ncline[]{n99}{n101}
\ncline[]{n99}{n102}
\ncline[]{n99}{n103}
\ncline[]{n99}{n104}
\ncline[]{n99}{n107}
\ncline[]{n99}{n110}
\ncline[]{n99}{n112}
\ncline[]{n99}{n114}
\ncline[]{n99}{n116}
\ncline[]{n99}{n117}
\ncline[]{n99}{n118}
\ncline[]{n99}{n119}
\ncline[]{n99}{n120}
\ncline[]{n99}{n122}
\ncline[]{n99}{n123}
\ncline[]{n99}{n126}
\ncline[]{n99}{n128}
\ncline[]{n99}{n129}
\ncline[]{n99}{n131}
\ncline[]{n99}{n132}
\ncline[]{n99}{n137}
\ncline[]{n99}{n138}
\ncline[]{n99}{n139}
\ncline[]{n99}{n140}
\ncline[]{n99}{n141}
\ncline[]{n99}{n142}
\ncline[]{n99}{n143}
\ncline[]{n99}{n144}
\ncline[]{n99}{n146}
\ncline[]{n99}{n147}
\ncline[]{n99}{n148}
\ncline[]{n100}{n101}
\ncline[]{n100}{n102}
\ncline[]{n100}{n103}
\ncline[]{n100}{n104}
\ncline[]{n100}{n107}
\ncline[]{n100}{n110}
\ncline[]{n100}{n112}
\ncline[]{n100}{n114}
\ncline[]{n100}{n116}
\ncline[]{n100}{n117}
\ncline[]{n100}{n118}
\ncline[]{n100}{n119}
\ncline[]{n100}{n120}
\ncline[]{n100}{n122}
\ncline[]{n100}{n123}
\ncline[]{n100}{n126}
\ncline[]{n100}{n128}
\ncline[]{n100}{n129}
\ncline[]{n100}{n131}
\ncline[]{n100}{n132}
\ncline[]{n100}{n137}
\ncline[]{n100}{n138}
\ncline[]{n100}{n139}
\ncline[]{n100}{n140}
\ncline[]{n100}{n141}
\ncline[]{n100}{n142}
\ncline[]{n100}{n143}
\ncline[]{n100}{n144}
\ncline[]{n100}{n146}
\ncline[]{n100}{n147}
\ncline[]{n100}{n148}
\ncline[]{n101}{n102}
\ncline[]{n101}{n103}
\ncline[]{n101}{n104}
\ncline[]{n101}{n107}
\ncline[]{n101}{n110}
\ncline[]{n101}{n112}
\ncline[]{n101}{n114}
\ncline[]{n101}{n116}
\ncline[]{n101}{n117}
\ncline[]{n101}{n118}
\ncline[]{n101}{n119}
\ncline[]{n101}{n120}
\ncline[]{n101}{n122}
\ncline[]{n101}{n123}
\ncline[]{n101}{n126}
\ncline[]{n101}{n128}
\ncline[]{n101}{n129}
\ncline[]{n101}{n131}
\ncline[]{n101}{n132}
\ncline[]{n101}{n137}
\ncline[]{n101}{n138}
\ncline[]{n101}{n139}
\ncline[]{n101}{n140}
\ncline[]{n101}{n141}
\ncline[]{n101}{n142}
\ncline[]{n101}{n143}
\ncline[]{n101}{n144}
\ncline[]{n101}{n146}
\ncline[]{n101}{n147}
\ncline[]{n101}{n148}
\ncline[]{n102}{n103}
\ncline[]{n102}{n104}
\ncline[]{n102}{n107}
\ncline[]{n102}{n110}
\ncline[]{n102}{n112}
\ncline[]{n102}{n114}
\ncline[]{n102}{n116}
\ncline[]{n102}{n117}
\ncline[]{n102}{n118}
\ncline[]{n102}{n119}
\ncline[]{n102}{n120}
\ncline[]{n102}{n122}
\ncline[]{n102}{n123}
\ncline[]{n102}{n126}
\ncline[]{n102}{n128}
\ncline[]{n102}{n129}
\ncline[]{n102}{n131}
\ncline[]{n102}{n132}
\ncline[]{n102}{n137}
\ncline[]{n102}{n138}
\ncline[]{n102}{n139}
\ncline[]{n102}{n140}
\ncline[]{n102}{n141}
\ncline[]{n102}{n142}
\ncline[]{n102}{n143}
\ncline[]{n102}{n144}
\ncline[]{n102}{n146}
\ncline[]{n102}{n147}
\ncline[]{n102}{n148}
\ncline[]{n103}{n104}
\ncline[]{n103}{n107}
\ncline[]{n103}{n110}
\ncline[]{n103}{n112}
\ncline[]{n103}{n114}
\ncline[]{n103}{n116}
\ncline[]{n103}{n117}
\ncline[]{n103}{n118}
\ncline[]{n103}{n119}
\ncline[]{n103}{n120}
\ncline[]{n103}{n122}
\ncline[]{n103}{n123}
\ncline[]{n103}{n126}
\ncline[]{n103}{n128}
\ncline[]{n103}{n129}
\ncline[]{n103}{n131}
\ncline[]{n103}{n132}
\ncline[]{n103}{n137}
\ncline[]{n103}{n138}
\ncline[]{n103}{n139}
\ncline[]{n103}{n140}
\ncline[]{n103}{n141}
\ncline[]{n103}{n142}
\ncline[]{n103}{n143}
\ncline[]{n103}{n144}
\ncline[]{n103}{n146}
\ncline[]{n103}{n147}
\ncline[]{n103}{n148}
\ncline[]{n104}{n107}
\ncline[]{n104}{n110}
\ncline[]{n104}{n112}
\ncline[]{n104}{n114}
\ncline[]{n104}{n116}
\ncline[]{n104}{n117}
\ncline[]{n104}{n118}
\ncline[]{n104}{n119}
\ncline[]{n104}{n120}
\ncline[]{n104}{n122}
\ncline[]{n104}{n123}
\ncline[]{n104}{n126}
\ncline[]{n104}{n128}
\ncline[]{n104}{n129}
\ncline[]{n104}{n131}
\ncline[]{n104}{n132}
\ncline[]{n104}{n137}
\ncline[]{n104}{n138}
\ncline[]{n104}{n139}
\ncline[]{n104}{n140}
\ncline[]{n104}{n141}
\ncline[]{n104}{n142}
\ncline[]{n104}{n143}
\ncline[]{n104}{n144}
\ncline[]{n104}{n146}
\ncline[]{n104}{n147}
\ncline[]{n104}{n148}
\ncline[]{n105}{n106}
\ncline[]{n105}{n108}
\ncline[]{n105}{n109}
\ncline[]{n105}{n121}
\ncline[]{n105}{n125}
\ncline[]{n105}{n134}
\ncline[]{n106}{n108}
\ncline[]{n106}{n109}
\ncline[]{n106}{n121}
\ncline[]{n106}{n125}
\ncline[]{n106}{n134}
\ncline[]{n107}{n110}
\ncline[]{n107}{n112}
\ncline[]{n107}{n114}
\ncline[]{n107}{n116}
\ncline[]{n107}{n117}
\ncline[]{n107}{n118}
\ncline[]{n107}{n119}
\ncline[]{n107}{n120}
\ncline[]{n107}{n122}
\ncline[]{n107}{n123}
\ncline[]{n107}{n126}
\ncline[]{n107}{n128}
\ncline[]{n107}{n129}
\ncline[]{n107}{n131}
\ncline[]{n107}{n132}
\ncline[]{n107}{n137}
\ncline[]{n107}{n138}
\ncline[]{n107}{n139}
\ncline[]{n107}{n140}
\ncline[]{n107}{n141}
\ncline[]{n107}{n142}
\ncline[]{n107}{n143}
\ncline[]{n107}{n144}
\ncline[]{n107}{n146}
\ncline[]{n107}{n147}
\ncline[]{n107}{n148}
\ncline[]{n108}{n109}
\ncline[]{n108}{n121}
\ncline[]{n108}{n125}
\ncline[]{n108}{n134}
\ncline[]{n109}{n121}
\ncline[]{n109}{n125}
\ncline[]{n109}{n134}
\ncline[]{n110}{n112}
\ncline[]{n110}{n114}
\ncline[]{n110}{n116}
\ncline[]{n110}{n117}
\ncline[]{n110}{n118}
\ncline[]{n110}{n119}
\ncline[]{n110}{n120}
\ncline[]{n110}{n122}
\ncline[]{n110}{n123}
\ncline[]{n110}{n126}
\ncline[]{n110}{n128}
\ncline[]{n110}{n129}
\ncline[]{n110}{n131}
\ncline[]{n110}{n132}
\ncline[]{n110}{n137}
\ncline[]{n110}{n138}
\ncline[]{n110}{n139}
\ncline[]{n110}{n140}
\ncline[]{n110}{n141}
\ncline[]{n110}{n142}
\ncline[]{n110}{n143}
\ncline[]{n110}{n144}
\ncline[]{n110}{n146}
\ncline[]{n110}{n147}
\ncline[]{n110}{n148}
\ncline[]{n111}{n113}
\ncline[]{n111}{n115}
\ncline[]{n111}{n124}
\ncline[]{n111}{n127}
\ncline[]{n111}{n130}
\ncline[]{n111}{n133}
\ncline[]{n111}{n135}
\ncline[]{n111}{n136}
\ncline[]{n111}{n145}
\ncline[]{n111}{n149}
\ncline[]{n112}{n114}
\ncline[]{n112}{n116}
\ncline[]{n112}{n117}
\ncline[]{n112}{n118}
\ncline[]{n112}{n119}
\ncline[]{n112}{n120}
\ncline[]{n112}{n122}
\ncline[]{n112}{n123}
\ncline[]{n112}{n126}
\ncline[]{n112}{n128}
\ncline[]{n112}{n129}
\ncline[]{n112}{n131}
\ncline[]{n112}{n132}
\ncline[]{n112}{n137}
\ncline[]{n112}{n138}
\ncline[]{n112}{n139}
\ncline[]{n112}{n140}
\ncline[]{n112}{n141}
\ncline[]{n112}{n142}
\ncline[]{n112}{n143}
\ncline[]{n112}{n144}
\ncline[]{n112}{n146}
\ncline[]{n112}{n147}
\ncline[]{n112}{n148}
\ncline[]{n113}{n115}
\ncline[]{n113}{n124}
\ncline[]{n113}{n127}
\ncline[]{n113}{n130}
\ncline[]{n113}{n133}
\ncline[]{n113}{n135}
\ncline[]{n113}{n136}
\ncline[]{n113}{n145}
\ncline[]{n113}{n149}
\ncline[]{n114}{n116}
\ncline[]{n114}{n117}
\ncline[]{n114}{n118}
\ncline[]{n114}{n119}
\ncline[]{n114}{n120}
\ncline[]{n114}{n122}
\ncline[]{n114}{n123}
\ncline[]{n114}{n126}
\ncline[]{n114}{n128}
\ncline[]{n114}{n129}
\ncline[]{n114}{n131}
\ncline[]{n114}{n132}
\ncline[]{n114}{n137}
\ncline[]{n114}{n138}
\ncline[]{n114}{n139}
\ncline[]{n114}{n140}
\ncline[]{n114}{n141}
\ncline[]{n114}{n142}
\ncline[]{n114}{n143}
\ncline[]{n114}{n144}
\ncline[]{n114}{n146}
\ncline[]{n114}{n147}
\ncline[]{n114}{n148}
\ncline[]{n115}{n124}
\ncline[]{n115}{n127}
\ncline[]{n115}{n130}
\ncline[]{n115}{n133}
\ncline[]{n115}{n135}
\ncline[]{n115}{n136}
\ncline[]{n115}{n145}
\ncline[]{n115}{n149}
\ncline[]{n116}{n117}
\ncline[]{n116}{n118}
\ncline[]{n116}{n119}
\ncline[]{n116}{n120}
\ncline[]{n116}{n122}
\ncline[]{n116}{n123}
\ncline[]{n116}{n126}
\ncline[]{n116}{n128}
\ncline[]{n116}{n129}
\ncline[]{n116}{n131}
\ncline[]{n116}{n132}
\ncline[]{n116}{n137}
\ncline[]{n116}{n138}
\ncline[]{n116}{n139}
\ncline[]{n116}{n140}
\ncline[]{n116}{n141}
\ncline[]{n116}{n142}
\ncline[]{n116}{n143}
\ncline[]{n116}{n144}
\ncline[]{n116}{n146}
\ncline[]{n116}{n147}
\ncline[]{n116}{n148}
\ncline[]{n117}{n118}
\ncline[]{n117}{n119}
\ncline[]{n117}{n120}
\ncline[]{n117}{n122}
\ncline[]{n117}{n123}
\ncline[]{n117}{n126}
\ncline[]{n117}{n128}
\ncline[]{n117}{n129}
\ncline[]{n117}{n131}
\ncline[]{n117}{n132}
\ncline[]{n117}{n137}
\ncline[]{n117}{n138}
\ncline[]{n117}{n139}
\ncline[]{n117}{n140}
\ncline[]{n117}{n141}
\ncline[]{n117}{n142}
\ncline[]{n117}{n143}
\ncline[]{n117}{n144}
\ncline[]{n117}{n146}
\ncline[]{n117}{n147}
\ncline[]{n117}{n148}
\ncline[]{n118}{n119}
\ncline[]{n118}{n120}
\ncline[]{n118}{n122}
\ncline[]{n118}{n123}
\ncline[]{n118}{n126}
\ncline[]{n118}{n128}
\ncline[]{n118}{n129}
\ncline[]{n118}{n131}
\ncline[]{n118}{n132}
\ncline[]{n118}{n137}
\ncline[]{n118}{n138}
\ncline[]{n118}{n139}
\ncline[]{n118}{n140}
\ncline[]{n118}{n141}
\ncline[]{n118}{n142}
\ncline[]{n118}{n143}
\ncline[]{n118}{n144}
\ncline[]{n118}{n146}
\ncline[]{n118}{n147}
\ncline[]{n118}{n148}
\ncline[]{n119}{n120}
\ncline[]{n119}{n122}
\ncline[]{n119}{n123}
\ncline[]{n119}{n126}
\ncline[]{n119}{n128}
\ncline[]{n119}{n129}
\ncline[]{n119}{n131}
\ncline[]{n119}{n132}
\ncline[]{n119}{n137}
\ncline[]{n119}{n138}
\ncline[]{n119}{n139}
\ncline[]{n119}{n140}
\ncline[]{n119}{n141}
\ncline[]{n119}{n142}
\ncline[]{n119}{n143}
\ncline[]{n119}{n144}
\ncline[]{n119}{n146}
\ncline[]{n119}{n147}
\ncline[]{n119}{n148}
\ncline[]{n120}{n122}
\ncline[]{n120}{n123}
\ncline[]{n120}{n126}
\ncline[]{n120}{n128}
\ncline[]{n120}{n129}
\ncline[]{n120}{n131}
\ncline[]{n120}{n132}
\ncline[]{n120}{n137}
\ncline[]{n120}{n138}
\ncline[]{n120}{n139}
\ncline[]{n120}{n140}
\ncline[]{n120}{n141}
\ncline[]{n120}{n142}
\ncline[]{n120}{n143}
\ncline[]{n120}{n144}
\ncline[]{n120}{n146}
\ncline[]{n120}{n147}
\ncline[]{n120}{n148}
\ncline[]{n121}{n125}
\ncline[]{n121}{n134}
\ncline[]{n122}{n123}
\ncline[]{n122}{n126}
\ncline[]{n122}{n128}
\ncline[]{n122}{n129}
\ncline[]{n122}{n131}
\ncline[]{n122}{n132}
\ncline[]{n122}{n137}
\ncline[]{n122}{n138}
\ncline[]{n122}{n139}
\ncline[]{n122}{n140}
\ncline[]{n122}{n141}
\ncline[]{n122}{n142}
\ncline[]{n122}{n143}
\ncline[]{n122}{n144}
\ncline[]{n122}{n146}
\ncline[]{n122}{n147}
\ncline[]{n122}{n148}
\ncline[]{n123}{n126}
\ncline[]{n123}{n128}
\ncline[]{n123}{n129}
\ncline[]{n123}{n131}
\ncline[]{n123}{n132}
\ncline[]{n123}{n137}
\ncline[]{n123}{n138}
\ncline[]{n123}{n139}
\ncline[]{n123}{n140}
\ncline[]{n123}{n141}
\ncline[]{n123}{n142}
\ncline[]{n123}{n143}
\ncline[]{n123}{n144}
\ncline[]{n123}{n146}
\ncline[]{n123}{n147}
\ncline[]{n123}{n148}
\ncline[]{n124}{n127}
\ncline[]{n124}{n130}
\ncline[]{n124}{n133}
\ncline[]{n124}{n135}
\ncline[]{n124}{n136}
\ncline[]{n124}{n145}
\ncline[]{n124}{n149}
\ncline[]{n125}{n134}
\ncline[]{n126}{n128}
\ncline[]{n126}{n129}
\ncline[]{n126}{n131}
\ncline[]{n126}{n132}
\ncline[]{n126}{n137}
\ncline[]{n126}{n138}
\ncline[]{n126}{n139}
\ncline[]{n126}{n140}
\ncline[]{n126}{n141}
\ncline[]{n126}{n142}
\ncline[]{n126}{n143}
\ncline[]{n126}{n144}
\ncline[]{n126}{n146}
\ncline[]{n126}{n147}
\ncline[]{n126}{n148}
\ncline[]{n127}{n130}
\ncline[]{n127}{n133}
\ncline[]{n127}{n135}
\ncline[]{n127}{n136}
\ncline[]{n127}{n145}
\ncline[]{n127}{n149}
\ncline[]{n128}{n129}
\ncline[]{n128}{n131}
\ncline[]{n128}{n132}
\ncline[]{n128}{n137}
\ncline[]{n128}{n138}
\ncline[]{n128}{n139}
\ncline[]{n128}{n140}
\ncline[]{n128}{n141}
\ncline[]{n128}{n142}
\ncline[]{n128}{n143}
\ncline[]{n128}{n144}
\ncline[]{n128}{n146}
\ncline[]{n128}{n147}
\ncline[]{n128}{n148}
\ncline[]{n129}{n131}
\ncline[]{n129}{n132}
\ncline[]{n129}{n137}
\ncline[]{n129}{n138}
\ncline[]{n129}{n139}
\ncline[]{n129}{n140}
\ncline[]{n129}{n141}
\ncline[]{n129}{n142}
\ncline[]{n129}{n143}
\ncline[]{n129}{n144}
\ncline[]{n129}{n146}
\ncline[]{n129}{n147}
\ncline[]{n129}{n148}
\ncline[]{n130}{n133}
\ncline[]{n130}{n135}
\ncline[]{n130}{n136}
\ncline[]{n130}{n145}
\ncline[]{n130}{n149}
\ncline[]{n131}{n132}
\ncline[]{n131}{n137}
\ncline[]{n131}{n138}
\ncline[]{n131}{n139}
\ncline[]{n131}{n140}
\ncline[]{n131}{n141}
\ncline[]{n131}{n142}
\ncline[]{n131}{n143}
\ncline[]{n131}{n144}
\ncline[]{n131}{n146}
\ncline[]{n131}{n147}
\ncline[]{n131}{n148}
\ncline[]{n132}{n137}
\ncline[]{n132}{n138}
\ncline[]{n132}{n139}
\ncline[]{n132}{n140}
\ncline[]{n132}{n141}
\ncline[]{n132}{n142}
\ncline[]{n132}{n143}
\ncline[]{n132}{n144}
\ncline[]{n132}{n146}
\ncline[]{n132}{n147}
\ncline[]{n132}{n148}
\ncline[]{n133}{n135}
\ncline[]{n133}{n136}
\ncline[]{n133}{n145}
\ncline[]{n133}{n149}
\ncline[]{n135}{n136}
\ncline[]{n135}{n145}
\ncline[]{n135}{n149}
\ncline[]{n136}{n145}
\ncline[]{n136}{n149}
\ncline[]{n137}{n138}
\ncline[]{n137}{n139}
\ncline[]{n137}{n140}
\ncline[]{n137}{n141}
\ncline[]{n137}{n142}
\ncline[]{n137}{n143}
\ncline[]{n137}{n144}
\ncline[]{n137}{n146}
\ncline[]{n137}{n147}
\ncline[]{n137}{n148}
\ncline[]{n138}{n139}
\ncline[]{n138}{n140}
\ncline[]{n138}{n141}
\ncline[]{n138}{n142}
\ncline[]{n138}{n143}
\ncline[]{n138}{n144}
\ncline[]{n138}{n146}
\ncline[]{n138}{n147}
\ncline[]{n138}{n148}
\ncline[]{n139}{n140}
\ncline[]{n139}{n141}
\ncline[]{n139}{n142}
\ncline[]{n139}{n143}
\ncline[]{n139}{n144}
\ncline[]{n139}{n146}
\ncline[]{n139}{n147}
\ncline[]{n139}{n148}
\ncline[]{n140}{n141}
\ncline[]{n140}{n142}
\ncline[]{n140}{n143}
\ncline[]{n140}{n144}
\ncline[]{n140}{n146}
\ncline[]{n140}{n147}
\ncline[]{n140}{n148}
\ncline[]{n141}{n142}
\ncline[]{n141}{n143}
\ncline[]{n141}{n144}
\ncline[]{n141}{n146}
\ncline[]{n141}{n147}
\ncline[]{n141}{n148}
\ncline[]{n142}{n143}
\ncline[]{n142}{n144}
\ncline[]{n142}{n146}
\ncline[]{n142}{n147}
\ncline[]{n142}{n148}
\ncline[]{n143}{n144}
\ncline[]{n143}{n146}
\ncline[]{n143}{n147}
\ncline[]{n143}{n148}
\ncline[]{n144}{n146}
\ncline[]{n144}{n147}
\ncline[]{n144}{n148}
\ncline[]{n145}{n149}
\ncline[]{n146}{n147}
\ncline[]{n146}{n148}
\ncline[]{n147}{n148}
\psset{fillcolor=gray, linecolor=black}
\dotnode[dotstyle=Btriangle](-0.511, -0.102){n0}
\dotnode[dotstyle=Btriangle](-1.464, 0.504){n1}
\dotnode[dotstyle=Btriangle](-1.043, 0.229){n2}
\dotnode[dotstyle=Bo](2.840, -0.221){n3}
\dotnode[dotstyle=Btriangle](-0.262, -0.548){n4}
\dotnode[dotstyle=Bo](2.890, -0.137){n5}
\dotnode[dotstyle=Btriangle](-2.350, -0.042){n6}
\dotnode[dotstyle=Btriangle](-1.414, -0.575){n7}
\dotnode[dotstyle=Btriangle](-2.314, 0.183){n8}
\dotnode[dotstyle=Btriangle](-2.320, -0.246){n9}
\dotnode[dotstyle=Bsquare](2.543, 0.440){n10}
\dotnode[dotstyle=Bsquare](2.587, 0.520){n11}
\dotnode[dotstyle=Btriangle](-0.331, -0.211){n12}
\dotnode[dotstyle=Btriangle](-1.291, -0.116){n13}
\dotnode[dotstyle=Bo](2.674, -0.107){n14}
\dotnode[dotstyle=Bsquare](2.469, 0.138){n15}
\dotnode[dotstyle=Bsquare](2.626, 0.170){n16}
\dotnode[dotstyle=Btriangle](-0.298, -0.347){n17}
\dotnode[dotstyle=Bsquare](2.704, 0.115){n18}
\dotnode[dotstyle=Btriangle](-2.614, 0.558){n19}
\dotnode[dotstyle=Btriangle](-1.390, -0.283){n20}
\dotnode[dotstyle=Btriangle](-1.764, 0.079){n21}
\dotnode[dotstyle=Bsquare](2.384, 1.345){n22}
\dotnode[dotstyle=Btriangle](-0.165, -0.680){n23}
\dotnode[dotstyle=Btriangle](-0.519, -1.191){n24}
\dotnode[dotstyle=Bsquare](2.406, 0.196){n25}
\dotnode[dotstyle=Btriangle](-0.181, -0.826){n26}
\dotnode[dotstyle=Bo](2.746, -0.311){n27}
\dotnode[dotstyle=Btriangle](-2.388, 0.463){n28}
\dotnode[dotstyle=Btriangle](-2.563, 0.276){n29}
\dotnode[dotstyle=Bsquare](2.623, 0.818){n30}
\dotnode[dotstyle=Bsquare](2.867, 0.077){n31}
\dotnode[dotstyle=Btriangle](-1.345, -0.776){n32}
\dotnode[dotstyle=Btriangle](-0.229, -0.402){n33}
\dotnode[dotstyle=Bo](0.908, -0.752){n34}
\dotnode[dotstyle=Btriangle](-0.043, -0.581){n35}
\dotnode[dotstyle=Btriangle](-0.235, -0.332){n36}
\dotnode[dotstyle=Bsquare](2.209, 0.443){n37}
\dotnode[dotstyle=Btriangle](-0.813, -0.371){n38}
\dotnode[dotstyle=Btriangle](-1.526, -0.375){n39}
\dotnode[dotstyle=Btriangle](-0.660, -0.352){n40}
\dotnode[dotstyle=Btriangle](-0.890, -0.034){n41}
\dotnode[dotstyle=Btriangle](-2.165, 0.215){n42}
\dotnode[dotstyle=Btriangle](-2.932, 0.352){n43}
\dotnode[dotstyle=Btriangle](-0.356, -0.503){n44}
\dotnode[dotstyle=Bo](2.787, -0.228){n45}
\dotnode[dotstyle=Btriangle](-2.616, 0.342){n46}
\dotnode[dotstyle=Btriangle](0.307, -0.365){n47}
\dotnode[dotstyle=Bsquare](2.543, 0.586){n48}
\dotnode[dotstyle=Bsquare](2.199, 0.879){n49}
\dotnode[dotstyle=Btriangle](-0.375, -0.292){n50}
\dotnode[dotstyle=Btriangle](-1.198, -0.606){n51}
\dotnode[dotstyle=Bo](2.982, -0.480){n52}
\dotnode[dotstyle=Btriangle](-2.532, -0.012){n53}
\dotnode[dotstyle=Bsquare](2.410, 0.418){n54}
\dotnode[dotstyle=Btriangle](-2.145, 0.139){n55}
\dotnode[dotstyle=Btriangle](-2.108, 0.371){n56}
\dotnode[dotstyle=Btriangle](-3.077, 0.686){n57}
\dotnode[dotstyle=Btriangle](-0.921, -0.182){n58}
\dotnode[dotstyle=Btriangle](0.175, -0.252){n59}
\dotnode[dotstyle=Btriangle](-1.780, -0.501){n60}
\dotnode[dotstyle=Btriangle](-1.802, -0.216){n61}
\dotnode[dotstyle=Bsquare](2.770, 0.271){n62}
\dotnode[dotstyle=Bsquare](2.303, 0.106){n63}
\dotnode[dotstyle=Btriangle](-0.245, -0.267){n64}
\dotnode[dotstyle=Btriangle](-0.587, -0.483){n65}
\dotnode[dotstyle=Btriangle](-1.585, -0.539){n66}
\dotnode[dotstyle=Bo](2.674, -0.107){n67}
\dotnode[dotstyle=Bsquare](3.216, 0.142){n68}
\dotnode[dotstyle=Bsquare](2.640, 0.319){n69}
\dotnode[dotstyle=Btriangle](-3.232, 1.371){n70}
\dotnode[dotstyle=Btriangle](-3.795, 0.253){n71}
\dotnode[dotstyle=Btriangle](-0.357, -0.067){n72}
\dotnode[dotstyle=Bsquare](2.625, 0.607){n73}
\dotnode[dotstyle=Bo](2.821, -0.082){n74}
\dotnode[dotstyle=Bo](2.633, -0.190){n75}
\dotnode[dotstyle=Bo](2.888, -0.571){n76}
\dotnode[dotstyle=Btriangle](-1.258, -0.179){n77}
\dotnode[dotstyle=Bo](2.716, -0.243){n78}
\dotnode[dotstyle=Btriangle](-0.812, -0.162){n79}
\dotnode[dotstyle=Btriangle](-1.902, 0.116){n80}
\dotnode[dotstyle=Bo](0.707, -1.008){n81}
\dotnode[dotstyle=Btriangle](-0.932, 0.319){n82}
\dotnode[dotstyle=Btriangle](-1.971, -0.181){n83}
\dotnode[dotstyle=Btriangle](-1.557, 0.267){n84}
\dotnode[dotstyle=Bo](0.511, -1.262){n85}
\dotnode[dotstyle=Btriangle](-1.116, -0.084){n86}
\dotnode[dotstyle=Btriangle](-2.419, 0.304){n87}
\dotnode[dotstyle=Bsquare](2.280, 0.748){n88}
\dotnode[dotstyle=Btriangle](-1.388, -0.204){n89}
\dotnode[dotstyle=Bo](2.613, 0.022){n90}
\dotnode[dotstyle=Bo](2.998, -0.334){n91}
\dotnode[dotstyle=Btriangle](-1.905, 0.119){n92}
\dotnode[dotstyle=Btriangle](-0.942, -0.542){n93}
\dotnode[dotstyle=Btriangle](-1.298, -0.761){n94}
\dotnode[dotstyle=Btriangle](-2.841, 0.373){n95}
\dotnode[dotstyle=Bo](0.751, -1.001){n96}
\dotnode[dotstyle=Btriangle](-1.285, 0.685){n97}
\dotnode[dotstyle=Btriangle](0.192, -0.677){n98}
\dotnode[dotstyle=Btriangle](-1.095, 0.284){n99}
\dotnode[dotstyle=Btriangle](-1.331, 0.245){n100}
\dotnode[dotstyle=Btriangle](-0.984, -0.124){n101}
\dotnode[dotstyle=Btriangle](-1.662, 0.242){n102}
\dotnode[dotstyle=Btriangle](-2.276, 0.333){n103}
\dotnode[dotstyle=Btriangle](-0.928, 0.468){n104}
\dotnode[dotstyle=Bo](2.356, -0.031){n105}
\dotnode[dotstyle=Bo](2.715, -0.170){n106}
\dotnode[dotstyle=Btriangle](-2.428, 0.377){n107}
\dotnode[dotstyle=Bo](2.852, -0.933){n108}
\dotnode[dotstyle=Bo](3.225, -0.503){n109}
\dotnode[dotstyle=Btriangle](0.010, -0.721){n110}
\dotnode[dotstyle=Bsquare](2.729, 0.334){n111}
\dotnode[dotstyle=Btriangle](-0.714, 0.150){n112}
\dotnode[dotstyle=Bsquare](2.562, 0.375){n113}
\dotnode[dotstyle=Btriangle](-0.135, -0.312){n114}
\dotnode[dotstyle=Bsquare](2.597, 1.100){n115}
\dotnode[dotstyle=Btriangle](-3.397, 0.547){n116}
\dotnode[dotstyle=Btriangle](-1.443, -0.144){n117}
\dotnode[dotstyle=Btriangle](-1.905, 0.048){n118}
\dotnode[dotstyle=Btriangle](-1.296, -0.328){n119}
\dotnode[dotstyle=Btriangle](-1.414, -0.575){n120}
\dotnode[dotstyle=Bo](2.508, -0.139){n121}
\dotnode[dotstyle=Btriangle](-1.220, 0.408){n122}
\dotnode[dotstyle=Btriangle](-1.379, -0.421){n123}
\dotnode[dotstyle=Bsquare](2.684, 0.327){n124}
\dotnode[dotstyle=Bo](2.588, -0.197){n125}
\dotnode[dotstyle=Btriangle](-0.640, -0.417){n126}
\dotnode[dotstyle=Bsquare](2.538, 0.510){n127}
\dotnode[dotstyle=Btriangle](-0.900, 0.330){n128}
\dotnode[dotstyle=Btriangle](-2.123, -0.211){n129}
\dotnode[dotstyle=Bsquare](2.648, 0.820){n130}
\dotnode[dotstyle=Btriangle](-2.159, -0.218){n131}
\dotnode[dotstyle=Btriangle](-3.499, 0.457){n132}
\dotnode[dotstyle=Bsquare](2.644, 1.186){n133}
\dotnode[dotstyle=Bo](2.674, -0.107){n134}
\dotnode[dotstyle=Bsquare](2.507, 0.652){n135}
\dotnode[dotstyle=Bsquare](2.648, 0.319){n136}
\dotnode[dotstyle=Btriangle](-0.807, 0.195){n137}
\dotnode[dotstyle=Btriangle](-1.949, 0.041){n138}
\dotnode[dotstyle=Btriangle](0.070, -0.703){n139}
\dotnode[dotstyle=Btriangle](-2.918, 0.780){n140}
\dotnode[dotstyle=Btriangle](-1.922, 0.409){n141}
\dotnode[dotstyle=Btriangle](-0.642, 0.019){n142}
\dotnode[dotstyle=Btriangle](-1.087, 0.075){n143}
\dotnode[dotstyle=Btriangle](-1.169, -0.165){n144}
\dotnode[dotstyle=Bsquare](2.311, 0.398){n145}
\dotnode[dotstyle=Btriangle](-0.463, -0.670){n146}
\dotnode[dotstyle=Btriangle](-1.944, 0.187){n147}
\dotnode[dotstyle=Btriangle](-3.489, 1.172){n148}
\dotnode[dotstyle=Bsquare](2.590, 0.236){n149}

        \endpsgraph
    }}
\end{figure}
Only intracluster edges shown.
\end{frame}


\begin{frame}{Internal Measures: BetaCV and C-index}

{\bf BetaCV Measure:} The BetaCV measure is the
ratio of the mean intracluster distance to the mean intercluster
distance:
\begin{align*}
\tcbhighmath{
  BetaCV = \frac{W_{in}/N_{in}}{W_{out}/N_{out}} =
  \frac{N_{out}}{N_{in}} \cdot \frac{W_{in}}{W_{out}}
  = \frac{N_{out}}{N_{in}}
  \frac{\sum_{i=1}^k W(C_i, C_i)}{\sum_{i=1}^k
  W(C_i, \ol{C_i})}
}
\end{align*}
The smaller the BetaCV ratio, the better the clustering.

\bigskip
{\bf C-index:} Let $W_{\min}(N_{in})$ be the sum of the
smallest $N_{in}$ distances in the proximity matrix $\bW$, where
$N_{in}$ is the total number of intracluster edges, or point
pairs. Let $W_{\max}(N_{in})$ be the sum of the largest $N_{in}$
distances in $\bW$.

\smallskip
The C-index measures to what extent the clustering puts together the
$N_{in}$
points that are the closest across the $k$ clusters.
It is def\/{i}ned as
\begin{align*}
\tcbhighmath{
  Cindex = \frac{W_{in} -
  W_{\min}(N_{in})}{W_{\max}(N_{in})-W_{\min}(N_{in})}
}
\end{align*}
The C-index lies in the range $[0,1]$.
The smaller the C-index, the better the clustering.
\end{frame}


\begin{frame}{Internal Measures: Normalized Cut and Modularity}

  {\bf Normalized Cut Measure:} The normalized cut
objective for graph clustering can
also be used as an internal clustering evaluation measure:
\begin{align*}
\tcbhighmath{
NC = \sum_{i=1}^k {W(C_i, \ol{C_i}) \over vol(C_i)} =
\sum_{i=1}^k {W(C_i, \ol{C_i}) \over W(C_i, V)}
}
\end{align*}
where $vol(C_i) = W(C_i,V)$ is the volume of cluster $C_i$.
The higher the normalized cut value the better.

\bigskip
{\bf Modularity:}
The modularity objective is given as
\begin{align*}
\tcbhighmath{
Q = \sum_{i=1}^k \Biggl(
    {W(C_i,C_i) \over W(V,V)} -
    \lB({W(C_i,V) \over W(V,V)}\rB)^2
    \Biggr)
}
\end{align*}
The smaller the modularity measure the better the clustering.
\end{frame}


\begin{frame}{Internal Measures: Dunn Index}
The Dunn index is def\/{i}ned as the ratio between the
minimum distance between point pairs from different clusters and
the maximum distance between point pairs from the same cluster
\begin{align*}
\tcbhighmath{
  Dunn = \frac{\displaystyle W_{out}^{\min}}{\displaystyle W_{in}^{\max}}
}
\end{align*}
where $W_{out}^{\min}$ is the minimum intercluster distance:
\begin{align*}
  \displaystyle
  W_{out}^{\min} = \min_{i, j>i} \; \bigl\{
  w_{ab} | \bx_a \in C_i, \bx_b \in C_{j}\bigr\}
\end{align*}
and $W_{in}^{\max}$ is the maximum intracluster distance:
\begin{align*}
  \displaystyle
  W_{in}^{\max} = \max_{i} \; \bigl\{ w_{ab} | \bx_a,\bx_b \in C_i   \bigr\}
\end{align*}
The larger the Dunn index the better the clustering because it means even
the closest distance between points in different clusters is much larger
than the farthest distance between points in the same cluster.
\end{frame}



\begin{frame}{Internal Measures: Davies-Bouldin Index}
Let $\mu_i$ denote the cluster mean
\begin{align*}
\mu_i = \frac{1}{n_i} \sum_{\bx_{j} \in C_i} \bx_{j}
\end{align*}
Let
$\sigma_{\mu_i}$ denote the dispersion or spread of the points around the cluster
mean
\begin{align*}
\sigma_{\mu_i} = \sqrt{\frac{\sum_{\bx_{j} \in C_i} \dist(\bx_{j},
\mu_i)^2}{n_i}} = \sqrt{var(C_i)}
\end{align*}

The Davies--Bouldin measure for a pair of clusters $C_i$ and $C_{j}$ is
def\/{i}ned as the ratio
\begin{align*}
\tcbhighmath{
\mathit{DB}_{ij} = \frac{\sigma_{\mu_i}+\sigma_{\mu_{j}}}{\dist(\mu_i,\mu_{j})}
}
\end{align*}
$\mathit{DB}_{ij}$ measures how compact the clusters are compared to the distance
between the cluster means.
The Davies--Bouldin index is then def\/{i}ned as
\begin{align*}
  \mathit{DB} = \frac{1}{k} \sum_{i=1}^k \max_{j\ne i} \{\mathit{DB}_{ij} \}
\end{align*}
The smaller the DB value the better the clustering.
\end{frame}


\begin{frame}{Silhouette Coefficient}
  \small
Define the
silhoutte coeff\/{i}cient of a point $\bx_i$ as
\begin{align*}
\tcbhighmath{
  s_i = \frac{\displaystyle \mu_{out}^{\min}(\bx_i)-\mu_{in}(\bx_i)}
  {\displaystyle \max
  \Bigl\{\mu_{out}^{\min}(\bx_i), \mu_{in}(\bx_i)\Bigr\} }
}
\end{align*}
where
$\mu_{in}(\bx_i)$ is the mean distance from $\bx_i$ to points in its own cluster $\hy_i$:
\begin{align*}
  \mu_{in}(\bx_i) = \frac{\sum_{\bx_{j} \in C_{\hy_i}, j \ne i}
  \dist(\bx_i,\bx_{j})}{n_{\hy_i}-1}
\end{align*}
and $\mu_{out}^{\min}(\bx_i)$ is the mean of the distances from
$\bx_i$ to points in the closest cluster:
\begin{align*}
  \mu_{out}^{\min}(\bx_i) = \min_{j\ne \hy_i} \lB\{
  \frac{\sum_{\by \in C_{j}} \dist(\bx_i,\by)}{n_{j}}
  \rB\}
\end{align*}

\medskip
The $s_i$ value lies in the interval $[-1, +1]$. A value
close to
$+1$ indicates that $\bx_i$ is much closer to points in its own
cluster, a value close to zero indicates $\bx_i$ is
close to the boundary, and a value close to $-1$
indicates that $\bx_i$ is much closer to another cluster, 
and therefore
may be mis-clustered.

\medskip
The silhouette coeff\/{i}cient is the mean $s_i$ value:
  $SC = \frac{1}{n} \sum_{i=1}^n s_i$.
A value close to $+1$ indicates a good clustering.
\end{frame}


\begin{frame}[fragile]{Iris Data: Good vs.\ Bad Clustering}
\setcounter{subfigure}{0}
\begin{figure}
    \centering
    \captionsetup[subfloat]{captionskip=20pt}
    \def\pshlabel#1{ {\footnotesize $#1$}}
    \def\psvlabel#1{ {\footnotesize $#1$}}
    \psset{xAxisLabel=$\bu_1$,yAxisLabel= $\bu_2$}
    \psset{xunit=0.5in,yunit=0.65in,dotscale=1.5,arrowscale=2,PointName=none}
    \centerline{
	\subfloat[Good]{
	\scalebox{0.55}{
        \psgraph[tickstyle=bottom,Dx=1,Ox=-4,Dy=0.5,Oy=-1.5]{->}%
        (-4,-1.5)(3.5,1.5){3.5in}{2in}
        \pnode(-0.511, -0.102){n0}
\pnode(-1.464, 0.504){n1}
\pnode(-1.043, 0.229){n2}
\pnode(2.840, -0.221){n3}
\pnode(-0.262, -0.548){n4}
\pnode(2.890, -0.137){n5}
\pnode(-2.350, -0.042){n6}
\pnode(-1.414, -0.575){n7}
\pnode(-2.314, 0.183){n8}
\pnode(-2.320, -0.246){n9}
\pnode(2.543, 0.440){n10}
\pnode(2.587, 0.520){n11}
\pnode(-0.331, -0.211){n12}
\pnode(-1.291, -0.116){n13}
\pnode(2.674, -0.107){n14}
\pnode(2.469, 0.138){n15}
\pnode(2.626, 0.170){n16}
\pnode(-0.298, -0.347){n17}
\pnode(2.704, 0.115){n18}
\pnode(-2.614, 0.558){n19}
\pnode(-1.390, -0.283){n20}
\pnode(-1.764, 0.079){n21}
\pnode(2.384, 1.345){n22}
\pnode(-0.165, -0.680){n23}
\pnode(-0.519, -1.191){n24}
\pnode(2.406, 0.196){n25}
\pnode(-0.181, -0.826){n26}
\pnode(2.746, -0.311){n27}
\pnode(-2.388, 0.463){n28}
\pnode(-2.563, 0.276){n29}
\pnode(2.623, 0.818){n30}
\pnode(2.867, 0.077){n31}
\pnode(-1.345, -0.776){n32}
\pnode(-0.229, -0.402){n33}
\pnode(0.908, -0.752){n34}
\pnode(-0.043, -0.581){n35}
\pnode(-0.235, -0.332){n36}
\pnode(2.209, 0.443){n37}
\pnode(-0.813, -0.371){n38}
\pnode(-1.526, -0.375){n39}
\pnode(-0.660, -0.352){n40}
\pnode(-0.890, -0.034){n41}
\pnode(-2.165, 0.215){n42}
\pnode(-2.932, 0.352){n43}
\pnode(-0.356, -0.503){n44}
\pnode(2.787, -0.228){n45}
\pnode(-2.616, 0.342){n46}
\pnode(0.307, -0.365){n47}
\pnode(2.543, 0.586){n48}
\pnode(2.199, 0.879){n49}
\pnode(-0.375, -0.292){n50}
\pnode(-1.198, -0.606){n51}
\pnode(2.982, -0.480){n52}
\pnode(-2.532, -0.012){n53}
\pnode(2.410, 0.418){n54}
\pnode(-2.145, 0.139){n55}
\pnode(-2.108, 0.371){n56}
\pnode(-3.077, 0.686){n57}
\pnode(-0.921, -0.182){n58}
\pnode(0.175, -0.252){n59}
\pnode(-1.780, -0.501){n60}
\pnode(-1.802, -0.216){n61}
\pnode(2.770, 0.271){n62}
\pnode(2.303, 0.106){n63}
\pnode(-0.245, -0.267){n64}
\pnode(-0.587, -0.483){n65}
\pnode(-1.585, -0.539){n66}
\pnode(2.674, -0.107){n67}
\pnode(3.216, 0.142){n68}
\pnode(2.640, 0.319){n69}
\pnode(-3.232, 1.371){n70}
\pnode(-3.795, 0.253){n71}
\pnode(-0.357, -0.067){n72}
\pnode(2.625, 0.607){n73}
\pnode(2.821, -0.082){n74}
\pnode(2.633, -0.190){n75}
\pnode(2.888, -0.571){n76}
\pnode(-1.258, -0.179){n77}
\pnode(2.716, -0.243){n78}
\pnode(-0.812, -0.162){n79}
\pnode(-1.902, 0.116){n80}
\pnode(0.707, -1.008){n81}
\pnode(-0.932, 0.319){n82}
\pnode(-1.971, -0.181){n83}
\pnode(-1.557, 0.267){n84}
\pnode(0.511, -1.262){n85}
\pnode(-1.116, -0.084){n86}
\pnode(-2.419, 0.304){n87}
\pnode(2.280, 0.748){n88}
\pnode(-1.388, -0.204){n89}
\pnode(2.613, 0.022){n90}
\pnode(2.998, -0.334){n91}
\pnode(-1.905, 0.119){n92}
\pnode(-0.942, -0.542){n93}
\pnode(-1.298, -0.761){n94}
\pnode(-2.841, 0.373){n95}
\pnode(0.751, -1.001){n96}
\pnode(-1.285, 0.685){n97}
\pnode(0.192, -0.677){n98}
\pnode(-1.095, 0.284){n99}
\pnode(-1.331, 0.245){n100}
\pnode(-0.984, -0.124){n101}
\pnode(-1.662, 0.242){n102}
\pnode(-2.276, 0.333){n103}
\pnode(-0.928, 0.468){n104}
\pnode(2.356, -0.031){n105}
\pnode(2.715, -0.170){n106}
\pnode(-2.428, 0.377){n107}
\pnode(2.852, -0.933){n108}
\pnode(3.225, -0.503){n109}
\pnode(0.010, -0.721){n110}
\pnode(2.729, 0.334){n111}
\pnode(-0.714, 0.150){n112}
\pnode(2.562, 0.375){n113}
\pnode(-0.135, -0.312){n114}
\pnode(2.597, 1.100){n115}
\pnode(-3.397, 0.547){n116}
\pnode(-1.443, -0.144){n117}
\pnode(-1.905, 0.048){n118}
\pnode(-1.296, -0.328){n119}
\pnode(-1.414, -0.575){n120}
\pnode(2.508, -0.139){n121}
\pnode(-1.220, 0.408){n122}
\pnode(-1.379, -0.421){n123}
\pnode(2.684, 0.327){n124}
\pnode(2.588, -0.197){n125}
\pnode(-0.640, -0.417){n126}
\pnode(2.538, 0.510){n127}
\pnode(-0.900, 0.330){n128}
\pnode(-2.123, -0.211){n129}
\pnode(2.648, 0.820){n130}
\pnode(-2.159, -0.218){n131}
\pnode(-3.499, 0.457){n132}
\pnode(2.644, 1.186){n133}
\pnode(2.674, -0.107){n134}
\pnode(2.507, 0.652){n135}
\pnode(2.648, 0.319){n136}
\pnode(-0.807, 0.195){n137}
\pnode(-1.949, 0.041){n138}
\pnode(0.070, -0.703){n139}
\pnode(-2.918, 0.780){n140}
\pnode(-1.922, 0.409){n141}
\pnode(-0.642, 0.019){n142}
\pnode(-1.087, 0.075){n143}
\pnode(-1.169, -0.165){n144}
\pnode(2.311, 0.398){n145}
\pnode(-0.463, -0.670){n146}
\pnode(-1.944, 0.187){n147}
\pnode(-3.489, 1.172){n148}
\pnode(2.590, 0.236){n149}
\psset{linewidth=0.01pt,linecolor=lightgray}
\ncline[]{n0}{n2}
\ncline[]{n0}{n4}
\ncline[]{n0}{n7}
\ncline[]{n0}{n12}
\ncline[]{n0}{n13}
\ncline[]{n0}{n17}
\ncline[]{n0}{n20}
\ncline[]{n0}{n23}
\ncline[]{n0}{n24}
\ncline[]{n0}{n26}
\ncline[]{n0}{n32}
\ncline[]{n0}{n33}
\ncline[]{n0}{n34}
\ncline[]{n0}{n35}
\ncline[]{n0}{n36}
\ncline[]{n0}{n38}
\ncline[]{n0}{n39}
\ncline[]{n0}{n40}
\ncline[]{n0}{n41}
\ncline[]{n0}{n44}
\ncline[]{n0}{n47}
\ncline[]{n0}{n50}
\ncline[]{n0}{n51}
\ncline[]{n0}{n58}
\ncline[]{n0}{n59}
\ncline[]{n0}{n64}
\ncline[]{n0}{n65}
\ncline[]{n0}{n66}
\ncline[]{n0}{n72}
\ncline[]{n0}{n77}
\ncline[]{n0}{n79}
\ncline[]{n0}{n81}
\ncline[]{n0}{n82}
\ncline[]{n0}{n85}
\ncline[]{n0}{n86}
\ncline[]{n0}{n89}
\ncline[]{n0}{n93}
\ncline[]{n0}{n94}
\ncline[]{n0}{n96}
\ncline[]{n0}{n98}
\ncline[]{n0}{n99}
\ncline[]{n0}{n100}
\ncline[]{n0}{n101}
\ncline[]{n0}{n104}
\ncline[]{n0}{n110}
\ncline[]{n0}{n112}
\ncline[]{n0}{n114}
\ncline[]{n0}{n117}
\ncline[]{n0}{n119}
\ncline[]{n0}{n120}
\ncline[]{n0}{n122}
\ncline[]{n0}{n123}
\ncline[]{n0}{n126}
\ncline[]{n0}{n128}
\ncline[]{n0}{n137}
\ncline[]{n0}{n139}
\ncline[]{n0}{n142}
\ncline[]{n0}{n143}
\ncline[]{n0}{n144}
\ncline[]{n0}{n146}
\ncline[]{n1}{n6}
\ncline[]{n1}{n8}
\ncline[]{n1}{n9}
\ncline[]{n1}{n19}
\ncline[]{n1}{n21}
\ncline[]{n1}{n28}
\ncline[]{n1}{n29}
\ncline[]{n1}{n42}
\ncline[]{n1}{n43}
\ncline[]{n1}{n46}
\ncline[]{n1}{n53}
\ncline[]{n1}{n55}
\ncline[]{n1}{n56}
\ncline[]{n1}{n57}
\ncline[]{n1}{n60}
\ncline[]{n1}{n61}
\ncline[]{n1}{n70}
\ncline[]{n1}{n71}
\ncline[]{n1}{n80}
\ncline[]{n1}{n83}
\ncline[]{n1}{n84}
\ncline[]{n1}{n87}
\ncline[]{n1}{n92}
\ncline[]{n1}{n95}
\ncline[]{n1}{n97}
\ncline[]{n1}{n102}
\ncline[]{n1}{n103}
\ncline[]{n1}{n107}
\ncline[]{n1}{n116}
\ncline[]{n1}{n118}
\ncline[]{n1}{n129}
\ncline[]{n1}{n131}
\ncline[]{n1}{n132}
\ncline[]{n1}{n138}
\ncline[]{n1}{n140}
\ncline[]{n1}{n141}
\ncline[]{n1}{n147}
\ncline[]{n1}{n148}
\ncline[]{n2}{n4}
\ncline[]{n2}{n7}
\ncline[]{n2}{n12}
\ncline[]{n2}{n13}
\ncline[]{n2}{n17}
\ncline[]{n2}{n20}
\ncline[]{n2}{n23}
\ncline[]{n2}{n24}
\ncline[]{n2}{n26}
\ncline[]{n2}{n32}
\ncline[]{n2}{n33}
\ncline[]{n2}{n34}
\ncline[]{n2}{n35}
\ncline[]{n2}{n36}
\ncline[]{n2}{n38}
\ncline[]{n2}{n39}
\ncline[]{n2}{n40}
\ncline[]{n2}{n41}
\ncline[]{n2}{n44}
\ncline[]{n2}{n47}
\ncline[]{n2}{n50}
\ncline[]{n2}{n51}
\ncline[]{n2}{n58}
\ncline[]{n2}{n59}
\ncline[]{n2}{n64}
\ncline[]{n2}{n65}
\ncline[]{n2}{n66}
\ncline[]{n2}{n72}
\ncline[]{n2}{n77}
\ncline[]{n2}{n79}
\ncline[]{n2}{n81}
\ncline[]{n2}{n82}
\ncline[]{n2}{n85}
\ncline[]{n2}{n86}
\ncline[]{n2}{n89}
\ncline[]{n2}{n93}
\ncline[]{n2}{n94}
\ncline[]{n2}{n96}
\ncline[]{n2}{n98}
\ncline[]{n2}{n99}
\ncline[]{n2}{n100}
\ncline[]{n2}{n101}
\ncline[]{n2}{n104}
\ncline[]{n2}{n110}
\ncline[]{n2}{n112}
\ncline[]{n2}{n114}
\ncline[]{n2}{n117}
\ncline[]{n2}{n119}
\ncline[]{n2}{n120}
\ncline[]{n2}{n122}
\ncline[]{n2}{n123}
\ncline[]{n2}{n126}
\ncline[]{n2}{n128}
\ncline[]{n2}{n137}
\ncline[]{n2}{n139}
\ncline[]{n2}{n142}
\ncline[]{n2}{n143}
\ncline[]{n2}{n144}
\ncline[]{n2}{n146}
\ncline[]{n3}{n5}
\ncline[]{n3}{n10}
\ncline[]{n3}{n11}
\ncline[]{n3}{n14}
\ncline[]{n3}{n15}
\ncline[]{n3}{n16}
\ncline[]{n3}{n18}
\ncline[]{n3}{n22}
\ncline[]{n3}{n25}
\ncline[]{n3}{n27}
\ncline[]{n3}{n30}
\ncline[]{n3}{n31}
\ncline[]{n3}{n37}
\ncline[]{n3}{n45}
\ncline[]{n3}{n48}
\ncline[]{n3}{n49}
\ncline[]{n3}{n52}
\ncline[]{n3}{n54}
\ncline[]{n3}{n62}
\ncline[]{n3}{n63}
\ncline[]{n3}{n67}
\ncline[]{n3}{n68}
\ncline[]{n3}{n69}
\ncline[]{n3}{n73}
\ncline[]{n3}{n74}
\ncline[]{n3}{n75}
\ncline[]{n3}{n76}
\ncline[]{n3}{n78}
\ncline[]{n3}{n88}
\ncline[]{n3}{n90}
\ncline[]{n3}{n91}
\ncline[]{n3}{n105}
\ncline[]{n3}{n106}
\ncline[]{n3}{n108}
\ncline[]{n3}{n109}
\ncline[]{n3}{n111}
\ncline[]{n3}{n113}
\ncline[]{n3}{n115}
\ncline[]{n3}{n121}
\ncline[]{n3}{n124}
\ncline[]{n3}{n125}
\ncline[]{n3}{n127}
\ncline[]{n3}{n130}
\ncline[]{n3}{n133}
\ncline[]{n3}{n134}
\ncline[]{n3}{n135}
\ncline[]{n3}{n136}
\ncline[]{n3}{n145}
\ncline[]{n3}{n149}
\ncline[]{n4}{n7}
\ncline[]{n4}{n12}
\ncline[]{n4}{n13}
\ncline[]{n4}{n17}
\ncline[]{n4}{n20}
\ncline[]{n4}{n23}
\ncline[]{n4}{n24}
\ncline[]{n4}{n26}
\ncline[]{n4}{n32}
\ncline[]{n4}{n33}
\ncline[]{n4}{n34}
\ncline[]{n4}{n35}
\ncline[]{n4}{n36}
\ncline[]{n4}{n38}
\ncline[]{n4}{n39}
\ncline[]{n4}{n40}
\ncline[]{n4}{n41}
\ncline[]{n4}{n44}
\ncline[]{n4}{n47}
\ncline[]{n4}{n50}
\ncline[]{n4}{n51}
\ncline[]{n4}{n58}
\ncline[]{n4}{n59}
\ncline[]{n4}{n64}
\ncline[]{n4}{n65}
\ncline[]{n4}{n66}
\ncline[]{n4}{n72}
\ncline[]{n4}{n77}
\ncline[]{n4}{n79}
\ncline[]{n4}{n81}
\ncline[]{n4}{n82}
\ncline[]{n4}{n85}
\ncline[]{n4}{n86}
\ncline[]{n4}{n89}
\ncline[]{n4}{n93}
\ncline[]{n4}{n94}
\ncline[]{n4}{n96}
\ncline[]{n4}{n98}
\ncline[]{n4}{n99}
\ncline[]{n4}{n100}
\ncline[]{n4}{n101}
\ncline[]{n4}{n104}
\ncline[]{n4}{n110}
\ncline[]{n4}{n112}
\ncline[]{n4}{n114}
\ncline[]{n4}{n117}
\ncline[]{n4}{n119}
\ncline[]{n4}{n120}
\ncline[]{n4}{n122}
\ncline[]{n4}{n123}
\ncline[]{n4}{n126}
\ncline[]{n4}{n128}
\ncline[]{n4}{n137}
\ncline[]{n4}{n139}
\ncline[]{n4}{n142}
\ncline[]{n4}{n143}
\ncline[]{n4}{n144}
\ncline[]{n4}{n146}
\ncline[]{n5}{n10}
\ncline[]{n5}{n11}
\ncline[]{n5}{n14}
\ncline[]{n5}{n15}
\ncline[]{n5}{n16}
\ncline[]{n5}{n18}
\ncline[]{n5}{n22}
\ncline[]{n5}{n25}
\ncline[]{n5}{n27}
\ncline[]{n5}{n30}
\ncline[]{n5}{n31}
\ncline[]{n5}{n37}
\ncline[]{n5}{n45}
\ncline[]{n5}{n48}
\ncline[]{n5}{n49}
\ncline[]{n5}{n52}
\ncline[]{n5}{n54}
\ncline[]{n5}{n62}
\ncline[]{n5}{n63}
\ncline[]{n5}{n67}
\ncline[]{n5}{n68}
\ncline[]{n5}{n69}
\ncline[]{n5}{n73}
\ncline[]{n5}{n74}
\ncline[]{n5}{n75}
\ncline[]{n5}{n76}
\ncline[]{n5}{n78}
\ncline[]{n5}{n88}
\ncline[]{n5}{n90}
\ncline[]{n5}{n91}
\ncline[]{n5}{n105}
\ncline[]{n5}{n106}
\ncline[]{n5}{n108}
\ncline[]{n5}{n109}
\ncline[]{n5}{n111}
\ncline[]{n5}{n113}
\ncline[]{n5}{n115}
\ncline[]{n5}{n121}
\ncline[]{n5}{n124}
\ncline[]{n5}{n125}
\ncline[]{n5}{n127}
\ncline[]{n5}{n130}
\ncline[]{n5}{n133}
\ncline[]{n5}{n134}
\ncline[]{n5}{n135}
\ncline[]{n5}{n136}
\ncline[]{n5}{n145}
\ncline[]{n5}{n149}
\ncline[]{n6}{n8}
\ncline[]{n6}{n9}
\ncline[]{n6}{n19}
\ncline[]{n6}{n21}
\ncline[]{n6}{n28}
\ncline[]{n6}{n29}
\ncline[]{n6}{n42}
\ncline[]{n6}{n43}
\ncline[]{n6}{n46}
\ncline[]{n6}{n53}
\ncline[]{n6}{n55}
\ncline[]{n6}{n56}
\ncline[]{n6}{n57}
\ncline[]{n6}{n60}
\ncline[]{n6}{n61}
\ncline[]{n6}{n70}
\ncline[]{n6}{n71}
\ncline[]{n6}{n80}
\ncline[]{n6}{n83}
\ncline[]{n6}{n84}
\ncline[]{n6}{n87}
\ncline[]{n6}{n92}
\ncline[]{n6}{n95}
\ncline[]{n6}{n97}
\ncline[]{n6}{n102}
\ncline[]{n6}{n103}
\ncline[]{n6}{n107}
\ncline[]{n6}{n116}
\ncline[]{n6}{n118}
\ncline[]{n6}{n129}
\ncline[]{n6}{n131}
\ncline[]{n6}{n132}
\ncline[]{n6}{n138}
\ncline[]{n6}{n140}
\ncline[]{n6}{n141}
\ncline[]{n6}{n147}
\ncline[]{n6}{n148}
\ncline[]{n7}{n12}
\ncline[]{n7}{n13}
\ncline[]{n7}{n17}
\ncline[]{n7}{n20}
\ncline[]{n7}{n23}
\ncline[]{n7}{n24}
\ncline[]{n7}{n26}
\ncline[]{n7}{n32}
\ncline[]{n7}{n33}
\ncline[]{n7}{n34}
\ncline[]{n7}{n35}
\ncline[]{n7}{n36}
\ncline[]{n7}{n38}
\ncline[]{n7}{n39}
\ncline[]{n7}{n40}
\ncline[]{n7}{n41}
\ncline[]{n7}{n44}
\ncline[]{n7}{n47}
\ncline[]{n7}{n50}
\ncline[]{n7}{n51}
\ncline[]{n7}{n58}
\ncline[]{n7}{n59}
\ncline[]{n7}{n64}
\ncline[]{n7}{n65}
\ncline[]{n7}{n66}
\ncline[]{n7}{n72}
\ncline[]{n7}{n77}
\ncline[]{n7}{n79}
\ncline[]{n7}{n81}
\ncline[]{n7}{n82}
\ncline[]{n7}{n85}
\ncline[]{n7}{n86}
\ncline[]{n7}{n89}
\ncline[]{n7}{n93}
\ncline[]{n7}{n94}
\ncline[]{n7}{n96}
\ncline[]{n7}{n98}
\ncline[]{n7}{n99}
\ncline[]{n7}{n100}
\ncline[]{n7}{n101}
\ncline[]{n7}{n104}
\ncline[]{n7}{n110}
\ncline[]{n7}{n112}
\ncline[]{n7}{n114}
\ncline[]{n7}{n117}
\ncline[]{n7}{n119}
\ncline[]{n7}{n120}
\ncline[]{n7}{n122}
\ncline[]{n7}{n123}
\ncline[]{n7}{n126}
\ncline[]{n7}{n128}
\ncline[]{n7}{n137}
\ncline[]{n7}{n139}
\ncline[]{n7}{n142}
\ncline[]{n7}{n143}
\ncline[]{n7}{n144}
\ncline[]{n7}{n146}
\ncline[]{n8}{n9}
\ncline[]{n8}{n19}
\ncline[]{n8}{n21}
\ncline[]{n8}{n28}
\ncline[]{n8}{n29}
\ncline[]{n8}{n42}
\ncline[]{n8}{n43}
\ncline[]{n8}{n46}
\ncline[]{n8}{n53}
\ncline[]{n8}{n55}
\ncline[]{n8}{n56}
\ncline[]{n8}{n57}
\ncline[]{n8}{n60}
\ncline[]{n8}{n61}
\ncline[]{n8}{n70}
\ncline[]{n8}{n71}
\ncline[]{n8}{n80}
\ncline[]{n8}{n83}
\ncline[]{n8}{n84}
\ncline[]{n8}{n87}
\ncline[]{n8}{n92}
\ncline[]{n8}{n95}
\ncline[]{n8}{n97}
\ncline[]{n8}{n102}
\ncline[]{n8}{n103}
\ncline[]{n8}{n107}
\ncline[]{n8}{n116}
\ncline[]{n8}{n118}
\ncline[]{n8}{n129}
\ncline[]{n8}{n131}
\ncline[]{n8}{n132}
\ncline[]{n8}{n138}
\ncline[]{n8}{n140}
\ncline[]{n8}{n141}
\ncline[]{n8}{n147}
\ncline[]{n8}{n148}
\ncline[]{n9}{n19}
\ncline[]{n9}{n21}
\ncline[]{n9}{n28}
\ncline[]{n9}{n29}
\ncline[]{n9}{n42}
\ncline[]{n9}{n43}
\ncline[]{n9}{n46}
\ncline[]{n9}{n53}
\ncline[]{n9}{n55}
\ncline[]{n9}{n56}
\ncline[]{n9}{n57}
\ncline[]{n9}{n60}
\ncline[]{n9}{n61}
\ncline[]{n9}{n70}
\ncline[]{n9}{n71}
\ncline[]{n9}{n80}
\ncline[]{n9}{n83}
\ncline[]{n9}{n84}
\ncline[]{n9}{n87}
\ncline[]{n9}{n92}
\ncline[]{n9}{n95}
\ncline[]{n9}{n97}
\ncline[]{n9}{n102}
\ncline[]{n9}{n103}
\ncline[]{n9}{n107}
\ncline[]{n9}{n116}
\ncline[]{n9}{n118}
\ncline[]{n9}{n129}
\ncline[]{n9}{n131}
\ncline[]{n9}{n132}
\ncline[]{n9}{n138}
\ncline[]{n9}{n140}
\ncline[]{n9}{n141}
\ncline[]{n9}{n147}
\ncline[]{n9}{n148}
\ncline[]{n10}{n11}
\ncline[]{n10}{n14}
\ncline[]{n10}{n15}
\ncline[]{n10}{n16}
\ncline[]{n10}{n18}
\ncline[]{n10}{n22}
\ncline[]{n10}{n25}
\ncline[]{n10}{n27}
\ncline[]{n10}{n30}
\ncline[]{n10}{n31}
\ncline[]{n10}{n37}
\ncline[]{n10}{n45}
\ncline[]{n10}{n48}
\ncline[]{n10}{n49}
\ncline[]{n10}{n52}
\ncline[]{n10}{n54}
\ncline[]{n10}{n62}
\ncline[]{n10}{n63}
\ncline[]{n10}{n67}
\ncline[]{n10}{n68}
\ncline[]{n10}{n69}
\ncline[]{n10}{n73}
\ncline[]{n10}{n74}
\ncline[]{n10}{n75}
\ncline[]{n10}{n76}
\ncline[]{n10}{n78}
\ncline[]{n10}{n88}
\ncline[]{n10}{n90}
\ncline[]{n10}{n91}
\ncline[]{n10}{n105}
\ncline[]{n10}{n106}
\ncline[]{n10}{n108}
\ncline[]{n10}{n109}
\ncline[]{n10}{n111}
\ncline[]{n10}{n113}
\ncline[]{n10}{n115}
\ncline[]{n10}{n121}
\ncline[]{n10}{n124}
\ncline[]{n10}{n125}
\ncline[]{n10}{n127}
\ncline[]{n10}{n130}
\ncline[]{n10}{n133}
\ncline[]{n10}{n134}
\ncline[]{n10}{n135}
\ncline[]{n10}{n136}
\ncline[]{n10}{n145}
\ncline[]{n10}{n149}
\ncline[]{n11}{n14}
\ncline[]{n11}{n15}
\ncline[]{n11}{n16}
\ncline[]{n11}{n18}
\ncline[]{n11}{n22}
\ncline[]{n11}{n25}
\ncline[]{n11}{n27}
\ncline[]{n11}{n30}
\ncline[]{n11}{n31}
\ncline[]{n11}{n37}
\ncline[]{n11}{n45}
\ncline[]{n11}{n48}
\ncline[]{n11}{n49}
\ncline[]{n11}{n52}
\ncline[]{n11}{n54}
\ncline[]{n11}{n62}
\ncline[]{n11}{n63}
\ncline[]{n11}{n67}
\ncline[]{n11}{n68}
\ncline[]{n11}{n69}
\ncline[]{n11}{n73}
\ncline[]{n11}{n74}
\ncline[]{n11}{n75}
\ncline[]{n11}{n76}
\ncline[]{n11}{n78}
\ncline[]{n11}{n88}
\ncline[]{n11}{n90}
\ncline[]{n11}{n91}
\ncline[]{n11}{n105}
\ncline[]{n11}{n106}
\ncline[]{n11}{n108}
\ncline[]{n11}{n109}
\ncline[]{n11}{n111}
\ncline[]{n11}{n113}
\ncline[]{n11}{n115}
\ncline[]{n11}{n121}
\ncline[]{n11}{n124}
\ncline[]{n11}{n125}
\ncline[]{n11}{n127}
\ncline[]{n11}{n130}
\ncline[]{n11}{n133}
\ncline[]{n11}{n134}
\ncline[]{n11}{n135}
\ncline[]{n11}{n136}
\ncline[]{n11}{n145}
\ncline[]{n11}{n149}
\ncline[]{n12}{n13}
\ncline[]{n12}{n17}
\ncline[]{n12}{n20}
\ncline[]{n12}{n23}
\ncline[]{n12}{n24}
\ncline[]{n12}{n26}
\ncline[]{n12}{n32}
\ncline[]{n12}{n33}
\ncline[]{n12}{n34}
\ncline[]{n12}{n35}
\ncline[]{n12}{n36}
\ncline[]{n12}{n38}
\ncline[]{n12}{n39}
\ncline[]{n12}{n40}
\ncline[]{n12}{n41}
\ncline[]{n12}{n44}
\ncline[]{n12}{n47}
\ncline[]{n12}{n50}
\ncline[]{n12}{n51}
\ncline[]{n12}{n58}
\ncline[]{n12}{n59}
\ncline[]{n12}{n64}
\ncline[]{n12}{n65}
\ncline[]{n12}{n66}
\ncline[]{n12}{n72}
\ncline[]{n12}{n77}
\ncline[]{n12}{n79}
\ncline[]{n12}{n81}
\ncline[]{n12}{n82}
\ncline[]{n12}{n85}
\ncline[]{n12}{n86}
\ncline[]{n12}{n89}
\ncline[]{n12}{n93}
\ncline[]{n12}{n94}
\ncline[]{n12}{n96}
\ncline[]{n12}{n98}
\ncline[]{n12}{n99}
\ncline[]{n12}{n100}
\ncline[]{n12}{n101}
\ncline[]{n12}{n104}
\ncline[]{n12}{n110}
\ncline[]{n12}{n112}
\ncline[]{n12}{n114}
\ncline[]{n12}{n117}
\ncline[]{n12}{n119}
\ncline[]{n12}{n120}
\ncline[]{n12}{n122}
\ncline[]{n12}{n123}
\ncline[]{n12}{n126}
\ncline[]{n12}{n128}
\ncline[]{n12}{n137}
\ncline[]{n12}{n139}
\ncline[]{n12}{n142}
\ncline[]{n12}{n143}
\ncline[]{n12}{n144}
\ncline[]{n12}{n146}
\ncline[]{n13}{n17}
\ncline[]{n13}{n20}
\ncline[]{n13}{n23}
\ncline[]{n13}{n24}
\ncline[]{n13}{n26}
\ncline[]{n13}{n32}
\ncline[]{n13}{n33}
\ncline[]{n13}{n34}
\ncline[]{n13}{n35}
\ncline[]{n13}{n36}
\ncline[]{n13}{n38}
\ncline[]{n13}{n39}
\ncline[]{n13}{n40}
\ncline[]{n13}{n41}
\ncline[]{n13}{n44}
\ncline[]{n13}{n47}
\ncline[]{n13}{n50}
\ncline[]{n13}{n51}
\ncline[]{n13}{n58}
\ncline[]{n13}{n59}
\ncline[]{n13}{n64}
\ncline[]{n13}{n65}
\ncline[]{n13}{n66}
\ncline[]{n13}{n72}
\ncline[]{n13}{n77}
\ncline[]{n13}{n79}
\ncline[]{n13}{n81}
\ncline[]{n13}{n82}
\ncline[]{n13}{n85}
\ncline[]{n13}{n86}
\ncline[]{n13}{n89}
\ncline[]{n13}{n93}
\ncline[]{n13}{n94}
\ncline[]{n13}{n96}
\ncline[]{n13}{n98}
\ncline[]{n13}{n99}
\ncline[]{n13}{n100}
\ncline[]{n13}{n101}
\ncline[]{n13}{n104}
\ncline[]{n13}{n110}
\ncline[]{n13}{n112}
\ncline[]{n13}{n114}
\ncline[]{n13}{n117}
\ncline[]{n13}{n119}
\ncline[]{n13}{n120}
\ncline[]{n13}{n122}
\ncline[]{n13}{n123}
\ncline[]{n13}{n126}
\ncline[]{n13}{n128}
\ncline[]{n13}{n137}
\ncline[]{n13}{n139}
\ncline[]{n13}{n142}
\ncline[]{n13}{n143}
\ncline[]{n13}{n144}
\ncline[]{n13}{n146}
\ncline[]{n14}{n15}
\ncline[]{n14}{n16}
\ncline[]{n14}{n18}
\ncline[]{n14}{n22}
\ncline[]{n14}{n25}
\ncline[]{n14}{n27}
\ncline[]{n14}{n30}
\ncline[]{n14}{n31}
\ncline[]{n14}{n37}
\ncline[]{n14}{n45}
\ncline[]{n14}{n48}
\ncline[]{n14}{n49}
\ncline[]{n14}{n52}
\ncline[]{n14}{n54}
\ncline[]{n14}{n62}
\ncline[]{n14}{n63}
\ncline[]{n14}{n67}
\ncline[]{n14}{n68}
\ncline[]{n14}{n69}
\ncline[]{n14}{n73}
\ncline[]{n14}{n74}
\ncline[]{n14}{n75}
\ncline[]{n14}{n76}
\ncline[]{n14}{n78}
\ncline[]{n14}{n88}
\ncline[]{n14}{n90}
\ncline[]{n14}{n91}
\ncline[]{n14}{n105}
\ncline[]{n14}{n106}
\ncline[]{n14}{n108}
\ncline[]{n14}{n109}
\ncline[]{n14}{n111}
\ncline[]{n14}{n113}
\ncline[]{n14}{n115}
\ncline[]{n14}{n121}
\ncline[]{n14}{n124}
\ncline[]{n14}{n125}
\ncline[]{n14}{n127}
\ncline[]{n14}{n130}
\ncline[]{n14}{n133}
\ncline[]{n14}{n134}
\ncline[]{n14}{n135}
\ncline[]{n14}{n136}
\ncline[]{n14}{n145}
\ncline[]{n14}{n149}
\ncline[]{n15}{n16}
\ncline[]{n15}{n18}
\ncline[]{n15}{n22}
\ncline[]{n15}{n25}
\ncline[]{n15}{n27}
\ncline[]{n15}{n30}
\ncline[]{n15}{n31}
\ncline[]{n15}{n37}
\ncline[]{n15}{n45}
\ncline[]{n15}{n48}
\ncline[]{n15}{n49}
\ncline[]{n15}{n52}
\ncline[]{n15}{n54}
\ncline[]{n15}{n62}
\ncline[]{n15}{n63}
\ncline[]{n15}{n67}
\ncline[]{n15}{n68}
\ncline[]{n15}{n69}
\ncline[]{n15}{n73}
\ncline[]{n15}{n74}
\ncline[]{n15}{n75}
\ncline[]{n15}{n76}
\ncline[]{n15}{n78}
\ncline[]{n15}{n88}
\ncline[]{n15}{n90}
\ncline[]{n15}{n91}
\ncline[]{n15}{n105}
\ncline[]{n15}{n106}
\ncline[]{n15}{n108}
\ncline[]{n15}{n109}
\ncline[]{n15}{n111}
\ncline[]{n15}{n113}
\ncline[]{n15}{n115}
\ncline[]{n15}{n121}
\ncline[]{n15}{n124}
\ncline[]{n15}{n125}
\ncline[]{n15}{n127}
\ncline[]{n15}{n130}
\ncline[]{n15}{n133}
\ncline[]{n15}{n134}
\ncline[]{n15}{n135}
\ncline[]{n15}{n136}
\ncline[]{n15}{n145}
\ncline[]{n15}{n149}
\ncline[]{n16}{n18}
\ncline[]{n16}{n22}
\ncline[]{n16}{n25}
\ncline[]{n16}{n27}
\ncline[]{n16}{n30}
\ncline[]{n16}{n31}
\ncline[]{n16}{n37}
\ncline[]{n16}{n45}
\ncline[]{n16}{n48}
\ncline[]{n16}{n49}
\ncline[]{n16}{n52}
\ncline[]{n16}{n54}
\ncline[]{n16}{n62}
\ncline[]{n16}{n63}
\ncline[]{n16}{n67}
\ncline[]{n16}{n68}
\ncline[]{n16}{n69}
\ncline[]{n16}{n73}
\ncline[]{n16}{n74}
\ncline[]{n16}{n75}
\ncline[]{n16}{n76}
\ncline[]{n16}{n78}
\ncline[]{n16}{n88}
\ncline[]{n16}{n90}
\ncline[]{n16}{n91}
\ncline[]{n16}{n105}
\ncline[]{n16}{n106}
\ncline[]{n16}{n108}
\ncline[]{n16}{n109}
\ncline[]{n16}{n111}
\ncline[]{n16}{n113}
\ncline[]{n16}{n115}
\ncline[]{n16}{n121}
\ncline[]{n16}{n124}
\ncline[]{n16}{n125}
\ncline[]{n16}{n127}
\ncline[]{n16}{n130}
\ncline[]{n16}{n133}
\ncline[]{n16}{n134}
\ncline[]{n16}{n135}
\ncline[]{n16}{n136}
\ncline[]{n16}{n145}
\ncline[]{n16}{n149}
\ncline[]{n17}{n20}
\ncline[]{n17}{n23}
\ncline[]{n17}{n24}
\ncline[]{n17}{n26}
\ncline[]{n17}{n32}
\ncline[]{n17}{n33}
\ncline[]{n17}{n34}
\ncline[]{n17}{n35}
\ncline[]{n17}{n36}
\ncline[]{n17}{n38}
\ncline[]{n17}{n39}
\ncline[]{n17}{n40}
\ncline[]{n17}{n41}
\ncline[]{n17}{n44}
\ncline[]{n17}{n47}
\ncline[]{n17}{n50}
\ncline[]{n17}{n51}
\ncline[]{n17}{n58}
\ncline[]{n17}{n59}
\ncline[]{n17}{n64}
\ncline[]{n17}{n65}
\ncline[]{n17}{n66}
\ncline[]{n17}{n72}
\ncline[]{n17}{n77}
\ncline[]{n17}{n79}
\ncline[]{n17}{n81}
\ncline[]{n17}{n82}
\ncline[]{n17}{n85}
\ncline[]{n17}{n86}
\ncline[]{n17}{n89}
\ncline[]{n17}{n93}
\ncline[]{n17}{n94}
\ncline[]{n17}{n96}
\ncline[]{n17}{n98}
\ncline[]{n17}{n99}
\ncline[]{n17}{n100}
\ncline[]{n17}{n101}
\ncline[]{n17}{n104}
\ncline[]{n17}{n110}
\ncline[]{n17}{n112}
\ncline[]{n17}{n114}
\ncline[]{n17}{n117}
\ncline[]{n17}{n119}
\ncline[]{n17}{n120}
\ncline[]{n17}{n122}
\ncline[]{n17}{n123}
\ncline[]{n17}{n126}
\ncline[]{n17}{n128}
\ncline[]{n17}{n137}
\ncline[]{n17}{n139}
\ncline[]{n17}{n142}
\ncline[]{n17}{n143}
\ncline[]{n17}{n144}
\ncline[]{n17}{n146}
\ncline[]{n18}{n22}
\ncline[]{n18}{n25}
\ncline[]{n18}{n27}
\ncline[]{n18}{n30}
\ncline[]{n18}{n31}
\ncline[]{n18}{n37}
\ncline[]{n18}{n45}
\ncline[]{n18}{n48}
\ncline[]{n18}{n49}
\ncline[]{n18}{n52}
\ncline[]{n18}{n54}
\ncline[]{n18}{n62}
\ncline[]{n18}{n63}
\ncline[]{n18}{n67}
\ncline[]{n18}{n68}
\ncline[]{n18}{n69}
\ncline[]{n18}{n73}
\ncline[]{n18}{n74}
\ncline[]{n18}{n75}
\ncline[]{n18}{n76}
\ncline[]{n18}{n78}
\ncline[]{n18}{n88}
\ncline[]{n18}{n90}
\ncline[]{n18}{n91}
\ncline[]{n18}{n105}
\ncline[]{n18}{n106}
\ncline[]{n18}{n108}
\ncline[]{n18}{n109}
\ncline[]{n18}{n111}
\ncline[]{n18}{n113}
\ncline[]{n18}{n115}
\ncline[]{n18}{n121}
\ncline[]{n18}{n124}
\ncline[]{n18}{n125}
\ncline[]{n18}{n127}
\ncline[]{n18}{n130}
\ncline[]{n18}{n133}
\ncline[]{n18}{n134}
\ncline[]{n18}{n135}
\ncline[]{n18}{n136}
\ncline[]{n18}{n145}
\ncline[]{n18}{n149}
\ncline[]{n19}{n21}
\ncline[]{n19}{n28}
\ncline[]{n19}{n29}
\ncline[]{n19}{n42}
\ncline[]{n19}{n43}
\ncline[]{n19}{n46}
\ncline[]{n19}{n53}
\ncline[]{n19}{n55}
\ncline[]{n19}{n56}
\ncline[]{n19}{n57}
\ncline[]{n19}{n60}
\ncline[]{n19}{n61}
\ncline[]{n19}{n70}
\ncline[]{n19}{n71}
\ncline[]{n19}{n80}
\ncline[]{n19}{n83}
\ncline[]{n19}{n84}
\ncline[]{n19}{n87}
\ncline[]{n19}{n92}
\ncline[]{n19}{n95}
\ncline[]{n19}{n97}
\ncline[]{n19}{n102}
\ncline[]{n19}{n103}
\ncline[]{n19}{n107}
\ncline[]{n19}{n116}
\ncline[]{n19}{n118}
\ncline[]{n19}{n129}
\ncline[]{n19}{n131}
\ncline[]{n19}{n132}
\ncline[]{n19}{n138}
\ncline[]{n19}{n140}
\ncline[]{n19}{n141}
\ncline[]{n19}{n147}
\ncline[]{n19}{n148}
\ncline[]{n20}{n23}
\ncline[]{n20}{n24}
\ncline[]{n20}{n26}
\ncline[]{n20}{n32}
\ncline[]{n20}{n33}
\ncline[]{n20}{n34}
\ncline[]{n20}{n35}
\ncline[]{n20}{n36}
\ncline[]{n20}{n38}
\ncline[]{n20}{n39}
\ncline[]{n20}{n40}
\ncline[]{n20}{n41}
\ncline[]{n20}{n44}
\ncline[]{n20}{n47}
\ncline[]{n20}{n50}
\ncline[]{n20}{n51}
\ncline[]{n20}{n58}
\ncline[]{n20}{n59}
\ncline[]{n20}{n64}
\ncline[]{n20}{n65}
\ncline[]{n20}{n66}
\ncline[]{n20}{n72}
\ncline[]{n20}{n77}
\ncline[]{n20}{n79}
\ncline[]{n20}{n81}
\ncline[]{n20}{n82}
\ncline[]{n20}{n85}
\ncline[]{n20}{n86}
\ncline[]{n20}{n89}
\ncline[]{n20}{n93}
\ncline[]{n20}{n94}
\ncline[]{n20}{n96}
\ncline[]{n20}{n98}
\ncline[]{n20}{n99}
\ncline[]{n20}{n100}
\ncline[]{n20}{n101}
\ncline[]{n20}{n104}
\ncline[]{n20}{n110}
\ncline[]{n20}{n112}
\ncline[]{n20}{n114}
\ncline[]{n20}{n117}
\ncline[]{n20}{n119}
\ncline[]{n20}{n120}
\ncline[]{n20}{n122}
\ncline[]{n20}{n123}
\ncline[]{n20}{n126}
\ncline[]{n20}{n128}
\ncline[]{n20}{n137}
\ncline[]{n20}{n139}
\ncline[]{n20}{n142}
\ncline[]{n20}{n143}
\ncline[]{n20}{n144}
\ncline[]{n20}{n146}
\ncline[]{n21}{n28}
\ncline[]{n21}{n29}
\ncline[]{n21}{n42}
\ncline[]{n21}{n43}
\ncline[]{n21}{n46}
\ncline[]{n21}{n53}
\ncline[]{n21}{n55}
\ncline[]{n21}{n56}
\ncline[]{n21}{n57}
\ncline[]{n21}{n60}
\ncline[]{n21}{n61}
\ncline[]{n21}{n70}
\ncline[]{n21}{n71}
\ncline[]{n21}{n80}
\ncline[]{n21}{n83}
\ncline[]{n21}{n84}
\ncline[]{n21}{n87}
\ncline[]{n21}{n92}
\ncline[]{n21}{n95}
\ncline[]{n21}{n97}
\ncline[]{n21}{n102}
\ncline[]{n21}{n103}
\ncline[]{n21}{n107}
\ncline[]{n21}{n116}
\ncline[]{n21}{n118}
\ncline[]{n21}{n129}
\ncline[]{n21}{n131}
\ncline[]{n21}{n132}
\ncline[]{n21}{n138}
\ncline[]{n21}{n140}
\ncline[]{n21}{n141}
\ncline[]{n21}{n147}
\ncline[]{n21}{n148}
\ncline[]{n22}{n25}
\ncline[]{n22}{n27}
\ncline[]{n22}{n30}
\ncline[]{n22}{n31}
\ncline[]{n22}{n37}
\ncline[]{n22}{n45}
\ncline[]{n22}{n48}
\ncline[]{n22}{n49}
\ncline[]{n22}{n52}
\ncline[]{n22}{n54}
\ncline[]{n22}{n62}
\ncline[]{n22}{n63}
\ncline[]{n22}{n67}
\ncline[]{n22}{n68}
\ncline[]{n22}{n69}
\ncline[]{n22}{n73}
\ncline[]{n22}{n74}
\ncline[]{n22}{n75}
\ncline[]{n22}{n76}
\ncline[]{n22}{n78}
\ncline[]{n22}{n88}
\ncline[]{n22}{n90}
\ncline[]{n22}{n91}
\ncline[]{n22}{n105}
\ncline[]{n22}{n106}
\ncline[]{n22}{n108}
\ncline[]{n22}{n109}
\ncline[]{n22}{n111}
\ncline[]{n22}{n113}
\ncline[]{n22}{n115}
\ncline[]{n22}{n121}
\ncline[]{n22}{n124}
\ncline[]{n22}{n125}
\ncline[]{n22}{n127}
\ncline[]{n22}{n130}
\ncline[]{n22}{n133}
\ncline[]{n22}{n134}
\ncline[]{n22}{n135}
\ncline[]{n22}{n136}
\ncline[]{n22}{n145}
\ncline[]{n22}{n149}
\ncline[]{n23}{n24}
\ncline[]{n23}{n26}
\ncline[]{n23}{n32}
\ncline[]{n23}{n33}
\ncline[]{n23}{n34}
\ncline[]{n23}{n35}
\ncline[]{n23}{n36}
\ncline[]{n23}{n38}
\ncline[]{n23}{n39}
\ncline[]{n23}{n40}
\ncline[]{n23}{n41}
\ncline[]{n23}{n44}
\ncline[]{n23}{n47}
\ncline[]{n23}{n50}
\ncline[]{n23}{n51}
\ncline[]{n23}{n58}
\ncline[]{n23}{n59}
\ncline[]{n23}{n64}
\ncline[]{n23}{n65}
\ncline[]{n23}{n66}
\ncline[]{n23}{n72}
\ncline[]{n23}{n77}
\ncline[]{n23}{n79}
\ncline[]{n23}{n81}
\ncline[]{n23}{n82}
\ncline[]{n23}{n85}
\ncline[]{n23}{n86}
\ncline[]{n23}{n89}
\ncline[]{n23}{n93}
\ncline[]{n23}{n94}
\ncline[]{n23}{n96}
\ncline[]{n23}{n98}
\ncline[]{n23}{n99}
\ncline[]{n23}{n100}
\ncline[]{n23}{n101}
\ncline[]{n23}{n104}
\ncline[]{n23}{n110}
\ncline[]{n23}{n112}
\ncline[]{n23}{n114}
\ncline[]{n23}{n117}
\ncline[]{n23}{n119}
\ncline[]{n23}{n120}
\ncline[]{n23}{n122}
\ncline[]{n23}{n123}
\ncline[]{n23}{n126}
\ncline[]{n23}{n128}
\ncline[]{n23}{n137}
\ncline[]{n23}{n139}
\ncline[]{n23}{n142}
\ncline[]{n23}{n143}
\ncline[]{n23}{n144}
\ncline[]{n23}{n146}
\ncline[]{n24}{n26}
\ncline[]{n24}{n32}
\ncline[]{n24}{n33}
\ncline[]{n24}{n34}
\ncline[]{n24}{n35}
\ncline[]{n24}{n36}
\ncline[]{n24}{n38}
\ncline[]{n24}{n39}
\ncline[]{n24}{n40}
\ncline[]{n24}{n41}
\ncline[]{n24}{n44}
\ncline[]{n24}{n47}
\ncline[]{n24}{n50}
\ncline[]{n24}{n51}
\ncline[]{n24}{n58}
\ncline[]{n24}{n59}
\ncline[]{n24}{n64}
\ncline[]{n24}{n65}
\ncline[]{n24}{n66}
\ncline[]{n24}{n72}
\ncline[]{n24}{n77}
\ncline[]{n24}{n79}
\ncline[]{n24}{n81}
\ncline[]{n24}{n82}
\ncline[]{n24}{n85}
\ncline[]{n24}{n86}
\ncline[]{n24}{n89}
\ncline[]{n24}{n93}
\ncline[]{n24}{n94}
\ncline[]{n24}{n96}
\ncline[]{n24}{n98}
\ncline[]{n24}{n99}
\ncline[]{n24}{n100}
\ncline[]{n24}{n101}
\ncline[]{n24}{n104}
\ncline[]{n24}{n110}
\ncline[]{n24}{n112}
\ncline[]{n24}{n114}
\ncline[]{n24}{n117}
\ncline[]{n24}{n119}
\ncline[]{n24}{n120}
\ncline[]{n24}{n122}
\ncline[]{n24}{n123}
\ncline[]{n24}{n126}
\ncline[]{n24}{n128}
\ncline[]{n24}{n137}
\ncline[]{n24}{n139}
\ncline[]{n24}{n142}
\ncline[]{n24}{n143}
\ncline[]{n24}{n144}
\ncline[]{n24}{n146}
\ncline[]{n25}{n27}
\ncline[]{n25}{n30}
\ncline[]{n25}{n31}
\ncline[]{n25}{n37}
\ncline[]{n25}{n45}
\ncline[]{n25}{n48}
\ncline[]{n25}{n49}
\ncline[]{n25}{n52}
\ncline[]{n25}{n54}
\ncline[]{n25}{n62}
\ncline[]{n25}{n63}
\ncline[]{n25}{n67}
\ncline[]{n25}{n68}
\ncline[]{n25}{n69}
\ncline[]{n25}{n73}
\ncline[]{n25}{n74}
\ncline[]{n25}{n75}
\ncline[]{n25}{n76}
\ncline[]{n25}{n78}
\ncline[]{n25}{n88}
\ncline[]{n25}{n90}
\ncline[]{n25}{n91}
\ncline[]{n25}{n105}
\ncline[]{n25}{n106}
\ncline[]{n25}{n108}
\ncline[]{n25}{n109}
\ncline[]{n25}{n111}
\ncline[]{n25}{n113}
\ncline[]{n25}{n115}
\ncline[]{n25}{n121}
\ncline[]{n25}{n124}
\ncline[]{n25}{n125}
\ncline[]{n25}{n127}
\ncline[]{n25}{n130}
\ncline[]{n25}{n133}
\ncline[]{n25}{n134}
\ncline[]{n25}{n135}
\ncline[]{n25}{n136}
\ncline[]{n25}{n145}
\ncline[]{n25}{n149}
\ncline[]{n26}{n32}
\ncline[]{n26}{n33}
\ncline[]{n26}{n34}
\ncline[]{n26}{n35}
\ncline[]{n26}{n36}
\ncline[]{n26}{n38}
\ncline[]{n26}{n39}
\ncline[]{n26}{n40}
\ncline[]{n26}{n41}
\ncline[]{n26}{n44}
\ncline[]{n26}{n47}
\ncline[]{n26}{n50}
\ncline[]{n26}{n51}
\ncline[]{n26}{n58}
\ncline[]{n26}{n59}
\ncline[]{n26}{n64}
\ncline[]{n26}{n65}
\ncline[]{n26}{n66}
\ncline[]{n26}{n72}
\ncline[]{n26}{n77}
\ncline[]{n26}{n79}
\ncline[]{n26}{n81}
\ncline[]{n26}{n82}
\ncline[]{n26}{n85}
\ncline[]{n26}{n86}
\ncline[]{n26}{n89}
\ncline[]{n26}{n93}
\ncline[]{n26}{n94}
\ncline[]{n26}{n96}
\ncline[]{n26}{n98}
\ncline[]{n26}{n99}
\ncline[]{n26}{n100}
\ncline[]{n26}{n101}
\ncline[]{n26}{n104}
\ncline[]{n26}{n110}
\ncline[]{n26}{n112}
\ncline[]{n26}{n114}
\ncline[]{n26}{n117}
\ncline[]{n26}{n119}
\ncline[]{n26}{n120}
\ncline[]{n26}{n122}
\ncline[]{n26}{n123}
\ncline[]{n26}{n126}
\ncline[]{n26}{n128}
\ncline[]{n26}{n137}
\ncline[]{n26}{n139}
\ncline[]{n26}{n142}
\ncline[]{n26}{n143}
\ncline[]{n26}{n144}
\ncline[]{n26}{n146}
\ncline[]{n27}{n30}
\ncline[]{n27}{n31}
\ncline[]{n27}{n37}
\ncline[]{n27}{n45}
\ncline[]{n27}{n48}
\ncline[]{n27}{n49}
\ncline[]{n27}{n52}
\ncline[]{n27}{n54}
\ncline[]{n27}{n62}
\ncline[]{n27}{n63}
\ncline[]{n27}{n67}
\ncline[]{n27}{n68}
\ncline[]{n27}{n69}
\ncline[]{n27}{n73}
\ncline[]{n27}{n74}
\ncline[]{n27}{n75}
\ncline[]{n27}{n76}
\ncline[]{n27}{n78}
\ncline[]{n27}{n88}
\ncline[]{n27}{n90}
\ncline[]{n27}{n91}
\ncline[]{n27}{n105}
\ncline[]{n27}{n106}
\ncline[]{n27}{n108}
\ncline[]{n27}{n109}
\ncline[]{n27}{n111}
\ncline[]{n27}{n113}
\ncline[]{n27}{n115}
\ncline[]{n27}{n121}
\ncline[]{n27}{n124}
\ncline[]{n27}{n125}
\ncline[]{n27}{n127}
\ncline[]{n27}{n130}
\ncline[]{n27}{n133}
\ncline[]{n27}{n134}
\ncline[]{n27}{n135}
\ncline[]{n27}{n136}
\ncline[]{n27}{n145}
\ncline[]{n27}{n149}
\ncline[]{n28}{n29}
\ncline[]{n28}{n42}
\ncline[]{n28}{n43}
\ncline[]{n28}{n46}
\ncline[]{n28}{n53}
\ncline[]{n28}{n55}
\ncline[]{n28}{n56}
\ncline[]{n28}{n57}
\ncline[]{n28}{n60}
\ncline[]{n28}{n61}
\ncline[]{n28}{n70}
\ncline[]{n28}{n71}
\ncline[]{n28}{n80}
\ncline[]{n28}{n83}
\ncline[]{n28}{n84}
\ncline[]{n28}{n87}
\ncline[]{n28}{n92}
\ncline[]{n28}{n95}
\ncline[]{n28}{n97}
\ncline[]{n28}{n102}
\ncline[]{n28}{n103}
\ncline[]{n28}{n107}
\ncline[]{n28}{n116}
\ncline[]{n28}{n118}
\ncline[]{n28}{n129}
\ncline[]{n28}{n131}
\ncline[]{n28}{n132}
\ncline[]{n28}{n138}
\ncline[]{n28}{n140}
\ncline[]{n28}{n141}
\ncline[]{n28}{n147}
\ncline[]{n28}{n148}
\ncline[]{n29}{n42}
\ncline[]{n29}{n43}
\ncline[]{n29}{n46}
\ncline[]{n29}{n53}
\ncline[]{n29}{n55}
\ncline[]{n29}{n56}
\ncline[]{n29}{n57}
\ncline[]{n29}{n60}
\ncline[]{n29}{n61}
\ncline[]{n29}{n70}
\ncline[]{n29}{n71}
\ncline[]{n29}{n80}
\ncline[]{n29}{n83}
\ncline[]{n29}{n84}
\ncline[]{n29}{n87}
\ncline[]{n29}{n92}
\ncline[]{n29}{n95}
\ncline[]{n29}{n97}
\ncline[]{n29}{n102}
\ncline[]{n29}{n103}
\ncline[]{n29}{n107}
\ncline[]{n29}{n116}
\ncline[]{n29}{n118}
\ncline[]{n29}{n129}
\ncline[]{n29}{n131}
\ncline[]{n29}{n132}
\ncline[]{n29}{n138}
\ncline[]{n29}{n140}
\ncline[]{n29}{n141}
\ncline[]{n29}{n147}
\ncline[]{n29}{n148}
\ncline[]{n30}{n31}
\ncline[]{n30}{n37}
\ncline[]{n30}{n45}
\ncline[]{n30}{n48}
\ncline[]{n30}{n49}
\ncline[]{n30}{n52}
\ncline[]{n30}{n54}
\ncline[]{n30}{n62}
\ncline[]{n30}{n63}
\ncline[]{n30}{n67}
\ncline[]{n30}{n68}
\ncline[]{n30}{n69}
\ncline[]{n30}{n73}
\ncline[]{n30}{n74}
\ncline[]{n30}{n75}
\ncline[]{n30}{n76}
\ncline[]{n30}{n78}
\ncline[]{n30}{n88}
\ncline[]{n30}{n90}
\ncline[]{n30}{n91}
\ncline[]{n30}{n105}
\ncline[]{n30}{n106}
\ncline[]{n30}{n108}
\ncline[]{n30}{n109}
\ncline[]{n30}{n111}
\ncline[]{n30}{n113}
\ncline[]{n30}{n115}
\ncline[]{n30}{n121}
\ncline[]{n30}{n124}
\ncline[]{n30}{n125}
\ncline[]{n30}{n127}
\ncline[]{n30}{n130}
\ncline[]{n30}{n133}
\ncline[]{n30}{n134}
\ncline[]{n30}{n135}
\ncline[]{n30}{n136}
\ncline[]{n30}{n145}
\ncline[]{n30}{n149}
\ncline[]{n31}{n37}
\ncline[]{n31}{n45}
\ncline[]{n31}{n48}
\ncline[]{n31}{n49}
\ncline[]{n31}{n52}
\ncline[]{n31}{n54}
\ncline[]{n31}{n62}
\ncline[]{n31}{n63}
\ncline[]{n31}{n67}
\ncline[]{n31}{n68}
\ncline[]{n31}{n69}
\ncline[]{n31}{n73}
\ncline[]{n31}{n74}
\ncline[]{n31}{n75}
\ncline[]{n31}{n76}
\ncline[]{n31}{n78}
\ncline[]{n31}{n88}
\ncline[]{n31}{n90}
\ncline[]{n31}{n91}
\ncline[]{n31}{n105}
\ncline[]{n31}{n106}
\ncline[]{n31}{n108}
\ncline[]{n31}{n109}
\ncline[]{n31}{n111}
\ncline[]{n31}{n113}
\ncline[]{n31}{n115}
\ncline[]{n31}{n121}
\ncline[]{n31}{n124}
\ncline[]{n31}{n125}
\ncline[]{n31}{n127}
\ncline[]{n31}{n130}
\ncline[]{n31}{n133}
\ncline[]{n31}{n134}
\ncline[]{n31}{n135}
\ncline[]{n31}{n136}
\ncline[]{n31}{n145}
\ncline[]{n31}{n149}
\ncline[]{n32}{n33}
\ncline[]{n32}{n34}
\ncline[]{n32}{n35}
\ncline[]{n32}{n36}
\ncline[]{n32}{n38}
\ncline[]{n32}{n39}
\ncline[]{n32}{n40}
\ncline[]{n32}{n41}
\ncline[]{n32}{n44}
\ncline[]{n32}{n47}
\ncline[]{n32}{n50}
\ncline[]{n32}{n51}
\ncline[]{n32}{n58}
\ncline[]{n32}{n59}
\ncline[]{n32}{n64}
\ncline[]{n32}{n65}
\ncline[]{n32}{n66}
\ncline[]{n32}{n72}
\ncline[]{n32}{n77}
\ncline[]{n32}{n79}
\ncline[]{n32}{n81}
\ncline[]{n32}{n82}
\ncline[]{n32}{n85}
\ncline[]{n32}{n86}
\ncline[]{n32}{n89}
\ncline[]{n32}{n93}
\ncline[]{n32}{n94}
\ncline[]{n32}{n96}
\ncline[]{n32}{n98}
\ncline[]{n32}{n99}
\ncline[]{n32}{n100}
\ncline[]{n32}{n101}
\ncline[]{n32}{n104}
\ncline[]{n32}{n110}
\ncline[]{n32}{n112}
\ncline[]{n32}{n114}
\ncline[]{n32}{n117}
\ncline[]{n32}{n119}
\ncline[]{n32}{n120}
\ncline[]{n32}{n122}
\ncline[]{n32}{n123}
\ncline[]{n32}{n126}
\ncline[]{n32}{n128}
\ncline[]{n32}{n137}
\ncline[]{n32}{n139}
\ncline[]{n32}{n142}
\ncline[]{n32}{n143}
\ncline[]{n32}{n144}
\ncline[]{n32}{n146}
\ncline[]{n33}{n34}
\ncline[]{n33}{n35}
\ncline[]{n33}{n36}
\ncline[]{n33}{n38}
\ncline[]{n33}{n39}
\ncline[]{n33}{n40}
\ncline[]{n33}{n41}
\ncline[]{n33}{n44}
\ncline[]{n33}{n47}
\ncline[]{n33}{n50}
\ncline[]{n33}{n51}
\ncline[]{n33}{n58}
\ncline[]{n33}{n59}
\ncline[]{n33}{n64}
\ncline[]{n33}{n65}
\ncline[]{n33}{n66}
\ncline[]{n33}{n72}
\ncline[]{n33}{n77}
\ncline[]{n33}{n79}
\ncline[]{n33}{n81}
\ncline[]{n33}{n82}
\ncline[]{n33}{n85}
\ncline[]{n33}{n86}
\ncline[]{n33}{n89}
\ncline[]{n33}{n93}
\ncline[]{n33}{n94}
\ncline[]{n33}{n96}
\ncline[]{n33}{n98}
\ncline[]{n33}{n99}
\ncline[]{n33}{n100}
\ncline[]{n33}{n101}
\ncline[]{n33}{n104}
\ncline[]{n33}{n110}
\ncline[]{n33}{n112}
\ncline[]{n33}{n114}
\ncline[]{n33}{n117}
\ncline[]{n33}{n119}
\ncline[]{n33}{n120}
\ncline[]{n33}{n122}
\ncline[]{n33}{n123}
\ncline[]{n33}{n126}
\ncline[]{n33}{n128}
\ncline[]{n33}{n137}
\ncline[]{n33}{n139}
\ncline[]{n33}{n142}
\ncline[]{n33}{n143}
\ncline[]{n33}{n144}
\ncline[]{n33}{n146}
\ncline[]{n34}{n35}
\ncline[]{n34}{n36}
\ncline[]{n34}{n38}
\ncline[]{n34}{n39}
\ncline[]{n34}{n40}
\ncline[]{n34}{n41}
\ncline[]{n34}{n44}
\ncline[]{n34}{n47}
\ncline[]{n34}{n50}
\ncline[]{n34}{n51}
\ncline[]{n34}{n58}
\ncline[]{n34}{n59}
\ncline[]{n34}{n64}
\ncline[]{n34}{n65}
\ncline[]{n34}{n66}
\ncline[]{n34}{n72}
\ncline[]{n34}{n77}
\ncline[]{n34}{n79}
\ncline[]{n34}{n81}
\ncline[]{n34}{n82}
\ncline[]{n34}{n85}
\ncline[]{n34}{n86}
\ncline[]{n34}{n89}
\ncline[]{n34}{n93}
\ncline[]{n34}{n94}
\ncline[]{n34}{n96}
\ncline[]{n34}{n98}
\ncline[]{n34}{n99}
\ncline[]{n34}{n100}
\ncline[]{n34}{n101}
\ncline[]{n34}{n104}
\ncline[]{n34}{n110}
\ncline[]{n34}{n112}
\ncline[]{n34}{n114}
\ncline[]{n34}{n117}
\ncline[]{n34}{n119}
\ncline[]{n34}{n120}
\ncline[]{n34}{n122}
\ncline[]{n34}{n123}
\ncline[]{n34}{n126}
\ncline[]{n34}{n128}
\ncline[]{n34}{n137}
\ncline[]{n34}{n139}
\ncline[]{n34}{n142}
\ncline[]{n34}{n143}
\ncline[]{n34}{n144}
\ncline[]{n34}{n146}
\ncline[]{n35}{n36}
\ncline[]{n35}{n38}
\ncline[]{n35}{n39}
\ncline[]{n35}{n40}
\ncline[]{n35}{n41}
\ncline[]{n35}{n44}
\ncline[]{n35}{n47}
\ncline[]{n35}{n50}
\ncline[]{n35}{n51}
\ncline[]{n35}{n58}
\ncline[]{n35}{n59}
\ncline[]{n35}{n64}
\ncline[]{n35}{n65}
\ncline[]{n35}{n66}
\ncline[]{n35}{n72}
\ncline[]{n35}{n77}
\ncline[]{n35}{n79}
\ncline[]{n35}{n81}
\ncline[]{n35}{n82}
\ncline[]{n35}{n85}
\ncline[]{n35}{n86}
\ncline[]{n35}{n89}
\ncline[]{n35}{n93}
\ncline[]{n35}{n94}
\ncline[]{n35}{n96}
\ncline[]{n35}{n98}
\ncline[]{n35}{n99}
\ncline[]{n35}{n100}
\ncline[]{n35}{n101}
\ncline[]{n35}{n104}
\ncline[]{n35}{n110}
\ncline[]{n35}{n112}
\ncline[]{n35}{n114}
\ncline[]{n35}{n117}
\ncline[]{n35}{n119}
\ncline[]{n35}{n120}
\ncline[]{n35}{n122}
\ncline[]{n35}{n123}
\ncline[]{n35}{n126}
\ncline[]{n35}{n128}
\ncline[]{n35}{n137}
\ncline[]{n35}{n139}
\ncline[]{n35}{n142}
\ncline[]{n35}{n143}
\ncline[]{n35}{n144}
\ncline[]{n35}{n146}
\ncline[]{n36}{n38}
\ncline[]{n36}{n39}
\ncline[]{n36}{n40}
\ncline[]{n36}{n41}
\ncline[]{n36}{n44}
\ncline[]{n36}{n47}
\ncline[]{n36}{n50}
\ncline[]{n36}{n51}
\ncline[]{n36}{n58}
\ncline[]{n36}{n59}
\ncline[]{n36}{n64}
\ncline[]{n36}{n65}
\ncline[]{n36}{n66}
\ncline[]{n36}{n72}
\ncline[]{n36}{n77}
\ncline[]{n36}{n79}
\ncline[]{n36}{n81}
\ncline[]{n36}{n82}
\ncline[]{n36}{n85}
\ncline[]{n36}{n86}
\ncline[]{n36}{n89}
\ncline[]{n36}{n93}
\ncline[]{n36}{n94}
\ncline[]{n36}{n96}
\ncline[]{n36}{n98}
\ncline[]{n36}{n99}
\ncline[]{n36}{n100}
\ncline[]{n36}{n101}
\ncline[]{n36}{n104}
\ncline[]{n36}{n110}
\ncline[]{n36}{n112}
\ncline[]{n36}{n114}
\ncline[]{n36}{n117}
\ncline[]{n36}{n119}
\ncline[]{n36}{n120}
\ncline[]{n36}{n122}
\ncline[]{n36}{n123}
\ncline[]{n36}{n126}
\ncline[]{n36}{n128}
\ncline[]{n36}{n137}
\ncline[]{n36}{n139}
\ncline[]{n36}{n142}
\ncline[]{n36}{n143}
\ncline[]{n36}{n144}
\ncline[]{n36}{n146}
\ncline[]{n37}{n45}
\ncline[]{n37}{n48}
\ncline[]{n37}{n49}
\ncline[]{n37}{n52}
\ncline[]{n37}{n54}
\ncline[]{n37}{n62}
\ncline[]{n37}{n63}
\ncline[]{n37}{n67}
\ncline[]{n37}{n68}
\ncline[]{n37}{n69}
\ncline[]{n37}{n73}
\ncline[]{n37}{n74}
\ncline[]{n37}{n75}
\ncline[]{n37}{n76}
\ncline[]{n37}{n78}
\ncline[]{n37}{n88}
\ncline[]{n37}{n90}
\ncline[]{n37}{n91}
\ncline[]{n37}{n105}
\ncline[]{n37}{n106}
\ncline[]{n37}{n108}
\ncline[]{n37}{n109}
\ncline[]{n37}{n111}
\ncline[]{n37}{n113}
\ncline[]{n37}{n115}
\ncline[]{n37}{n121}
\ncline[]{n37}{n124}
\ncline[]{n37}{n125}
\ncline[]{n37}{n127}
\ncline[]{n37}{n130}
\ncline[]{n37}{n133}
\ncline[]{n37}{n134}
\ncline[]{n37}{n135}
\ncline[]{n37}{n136}
\ncline[]{n37}{n145}
\ncline[]{n37}{n149}
\ncline[]{n38}{n39}
\ncline[]{n38}{n40}
\ncline[]{n38}{n41}
\ncline[]{n38}{n44}
\ncline[]{n38}{n47}
\ncline[]{n38}{n50}
\ncline[]{n38}{n51}
\ncline[]{n38}{n58}
\ncline[]{n38}{n59}
\ncline[]{n38}{n64}
\ncline[]{n38}{n65}
\ncline[]{n38}{n66}
\ncline[]{n38}{n72}
\ncline[]{n38}{n77}
\ncline[]{n38}{n79}
\ncline[]{n38}{n81}
\ncline[]{n38}{n82}
\ncline[]{n38}{n85}
\ncline[]{n38}{n86}
\ncline[]{n38}{n89}
\ncline[]{n38}{n93}
\ncline[]{n38}{n94}
\ncline[]{n38}{n96}
\ncline[]{n38}{n98}
\ncline[]{n38}{n99}
\ncline[]{n38}{n100}
\ncline[]{n38}{n101}
\ncline[]{n38}{n104}
\ncline[]{n38}{n110}
\ncline[]{n38}{n112}
\ncline[]{n38}{n114}
\ncline[]{n38}{n117}
\ncline[]{n38}{n119}
\ncline[]{n38}{n120}
\ncline[]{n38}{n122}
\ncline[]{n38}{n123}
\ncline[]{n38}{n126}
\ncline[]{n38}{n128}
\ncline[]{n38}{n137}
\ncline[]{n38}{n139}
\ncline[]{n38}{n142}
\ncline[]{n38}{n143}
\ncline[]{n38}{n144}
\ncline[]{n38}{n146}
\ncline[]{n39}{n40}
\ncline[]{n39}{n41}
\ncline[]{n39}{n44}
\ncline[]{n39}{n47}
\ncline[]{n39}{n50}
\ncline[]{n39}{n51}
\ncline[]{n39}{n58}
\ncline[]{n39}{n59}
\ncline[]{n39}{n64}
\ncline[]{n39}{n65}
\ncline[]{n39}{n66}
\ncline[]{n39}{n72}
\ncline[]{n39}{n77}
\ncline[]{n39}{n79}
\ncline[]{n39}{n81}
\ncline[]{n39}{n82}
\ncline[]{n39}{n85}
\ncline[]{n39}{n86}
\ncline[]{n39}{n89}
\ncline[]{n39}{n93}
\ncline[]{n39}{n94}
\ncline[]{n39}{n96}
\ncline[]{n39}{n98}
\ncline[]{n39}{n99}
\ncline[]{n39}{n100}
\ncline[]{n39}{n101}
\ncline[]{n39}{n104}
\ncline[]{n39}{n110}
\ncline[]{n39}{n112}
\ncline[]{n39}{n114}
\ncline[]{n39}{n117}
\ncline[]{n39}{n119}
\ncline[]{n39}{n120}
\ncline[]{n39}{n122}
\ncline[]{n39}{n123}
\ncline[]{n39}{n126}
\ncline[]{n39}{n128}
\ncline[]{n39}{n137}
\ncline[]{n39}{n139}
\ncline[]{n39}{n142}
\ncline[]{n39}{n143}
\ncline[]{n39}{n144}
\ncline[]{n39}{n146}
\ncline[]{n40}{n41}
\ncline[]{n40}{n44}
\ncline[]{n40}{n47}
\ncline[]{n40}{n50}
\ncline[]{n40}{n51}
\ncline[]{n40}{n58}
\ncline[]{n40}{n59}
\ncline[]{n40}{n64}
\ncline[]{n40}{n65}
\ncline[]{n40}{n66}
\ncline[]{n40}{n72}
\ncline[]{n40}{n77}
\ncline[]{n40}{n79}
\ncline[]{n40}{n81}
\ncline[]{n40}{n82}
\ncline[]{n40}{n85}
\ncline[]{n40}{n86}
\ncline[]{n40}{n89}
\ncline[]{n40}{n93}
\ncline[]{n40}{n94}
\ncline[]{n40}{n96}
\ncline[]{n40}{n98}
\ncline[]{n40}{n99}
\ncline[]{n40}{n100}
\ncline[]{n40}{n101}
\ncline[]{n40}{n104}
\ncline[]{n40}{n110}
\ncline[]{n40}{n112}
\ncline[]{n40}{n114}
\ncline[]{n40}{n117}
\ncline[]{n40}{n119}
\ncline[]{n40}{n120}
\ncline[]{n40}{n122}
\ncline[]{n40}{n123}
\ncline[]{n40}{n126}
\ncline[]{n40}{n128}
\ncline[]{n40}{n137}
\ncline[]{n40}{n139}
\ncline[]{n40}{n142}
\ncline[]{n40}{n143}
\ncline[]{n40}{n144}
\ncline[]{n40}{n146}
\ncline[]{n41}{n44}
\ncline[]{n41}{n47}
\ncline[]{n41}{n50}
\ncline[]{n41}{n51}
\ncline[]{n41}{n58}
\ncline[]{n41}{n59}
\ncline[]{n41}{n64}
\ncline[]{n41}{n65}
\ncline[]{n41}{n66}
\ncline[]{n41}{n72}
\ncline[]{n41}{n77}
\ncline[]{n41}{n79}
\ncline[]{n41}{n81}
\ncline[]{n41}{n82}
\ncline[]{n41}{n85}
\ncline[]{n41}{n86}
\ncline[]{n41}{n89}
\ncline[]{n41}{n93}
\ncline[]{n41}{n94}
\ncline[]{n41}{n96}
\ncline[]{n41}{n98}
\ncline[]{n41}{n99}
\ncline[]{n41}{n100}
\ncline[]{n41}{n101}
\ncline[]{n41}{n104}
\ncline[]{n41}{n110}
\ncline[]{n41}{n112}
\ncline[]{n41}{n114}
\ncline[]{n41}{n117}
\ncline[]{n41}{n119}
\ncline[]{n41}{n120}
\ncline[]{n41}{n122}
\ncline[]{n41}{n123}
\ncline[]{n41}{n126}
\ncline[]{n41}{n128}
\ncline[]{n41}{n137}
\ncline[]{n41}{n139}
\ncline[]{n41}{n142}
\ncline[]{n41}{n143}
\ncline[]{n41}{n144}
\ncline[]{n41}{n146}
\ncline[]{n42}{n43}
\ncline[]{n42}{n46}
\ncline[]{n42}{n53}
\ncline[]{n42}{n55}
\ncline[]{n42}{n56}
\ncline[]{n42}{n57}
\ncline[]{n42}{n60}
\ncline[]{n42}{n61}
\ncline[]{n42}{n70}
\ncline[]{n42}{n71}
\ncline[]{n42}{n80}
\ncline[]{n42}{n83}
\ncline[]{n42}{n84}
\ncline[]{n42}{n87}
\ncline[]{n42}{n92}
\ncline[]{n42}{n95}
\ncline[]{n42}{n97}
\ncline[]{n42}{n102}
\ncline[]{n42}{n103}
\ncline[]{n42}{n107}
\ncline[]{n42}{n116}
\ncline[]{n42}{n118}
\ncline[]{n42}{n129}
\ncline[]{n42}{n131}
\ncline[]{n42}{n132}
\ncline[]{n42}{n138}
\ncline[]{n42}{n140}
\ncline[]{n42}{n141}
\ncline[]{n42}{n147}
\ncline[]{n42}{n148}
\ncline[]{n43}{n46}
\ncline[]{n43}{n53}
\ncline[]{n43}{n55}
\ncline[]{n43}{n56}
\ncline[]{n43}{n57}
\ncline[]{n43}{n60}
\ncline[]{n43}{n61}
\ncline[]{n43}{n70}
\ncline[]{n43}{n71}
\ncline[]{n43}{n80}
\ncline[]{n43}{n83}
\ncline[]{n43}{n84}
\ncline[]{n43}{n87}
\ncline[]{n43}{n92}
\ncline[]{n43}{n95}
\ncline[]{n43}{n97}
\ncline[]{n43}{n102}
\ncline[]{n43}{n103}
\ncline[]{n43}{n107}
\ncline[]{n43}{n116}
\ncline[]{n43}{n118}
\ncline[]{n43}{n129}
\ncline[]{n43}{n131}
\ncline[]{n43}{n132}
\ncline[]{n43}{n138}
\ncline[]{n43}{n140}
\ncline[]{n43}{n141}
\ncline[]{n43}{n147}
\ncline[]{n43}{n148}
\ncline[]{n44}{n47}
\ncline[]{n44}{n50}
\ncline[]{n44}{n51}
\ncline[]{n44}{n58}
\ncline[]{n44}{n59}
\ncline[]{n44}{n64}
\ncline[]{n44}{n65}
\ncline[]{n44}{n66}
\ncline[]{n44}{n72}
\ncline[]{n44}{n77}
\ncline[]{n44}{n79}
\ncline[]{n44}{n81}
\ncline[]{n44}{n82}
\ncline[]{n44}{n85}
\ncline[]{n44}{n86}
\ncline[]{n44}{n89}
\ncline[]{n44}{n93}
\ncline[]{n44}{n94}
\ncline[]{n44}{n96}
\ncline[]{n44}{n98}
\ncline[]{n44}{n99}
\ncline[]{n44}{n100}
\ncline[]{n44}{n101}
\ncline[]{n44}{n104}
\ncline[]{n44}{n110}
\ncline[]{n44}{n112}
\ncline[]{n44}{n114}
\ncline[]{n44}{n117}
\ncline[]{n44}{n119}
\ncline[]{n44}{n120}
\ncline[]{n44}{n122}
\ncline[]{n44}{n123}
\ncline[]{n44}{n126}
\ncline[]{n44}{n128}
\ncline[]{n44}{n137}
\ncline[]{n44}{n139}
\ncline[]{n44}{n142}
\ncline[]{n44}{n143}
\ncline[]{n44}{n144}
\ncline[]{n44}{n146}
\ncline[]{n45}{n48}
\ncline[]{n45}{n49}
\ncline[]{n45}{n52}
\ncline[]{n45}{n54}
\ncline[]{n45}{n62}
\ncline[]{n45}{n63}
\ncline[]{n45}{n67}
\ncline[]{n45}{n68}
\ncline[]{n45}{n69}
\ncline[]{n45}{n73}
\ncline[]{n45}{n74}
\ncline[]{n45}{n75}
\ncline[]{n45}{n76}
\ncline[]{n45}{n78}
\ncline[]{n45}{n88}
\ncline[]{n45}{n90}
\ncline[]{n45}{n91}
\ncline[]{n45}{n105}
\ncline[]{n45}{n106}
\ncline[]{n45}{n108}
\ncline[]{n45}{n109}
\ncline[]{n45}{n111}
\ncline[]{n45}{n113}
\ncline[]{n45}{n115}
\ncline[]{n45}{n121}
\ncline[]{n45}{n124}
\ncline[]{n45}{n125}
\ncline[]{n45}{n127}
\ncline[]{n45}{n130}
\ncline[]{n45}{n133}
\ncline[]{n45}{n134}
\ncline[]{n45}{n135}
\ncline[]{n45}{n136}
\ncline[]{n45}{n145}
\ncline[]{n45}{n149}
\ncline[]{n46}{n53}
\ncline[]{n46}{n55}
\ncline[]{n46}{n56}
\ncline[]{n46}{n57}
\ncline[]{n46}{n60}
\ncline[]{n46}{n61}
\ncline[]{n46}{n70}
\ncline[]{n46}{n71}
\ncline[]{n46}{n80}
\ncline[]{n46}{n83}
\ncline[]{n46}{n84}
\ncline[]{n46}{n87}
\ncline[]{n46}{n92}
\ncline[]{n46}{n95}
\ncline[]{n46}{n97}
\ncline[]{n46}{n102}
\ncline[]{n46}{n103}
\ncline[]{n46}{n107}
\ncline[]{n46}{n116}
\ncline[]{n46}{n118}
\ncline[]{n46}{n129}
\ncline[]{n46}{n131}
\ncline[]{n46}{n132}
\ncline[]{n46}{n138}
\ncline[]{n46}{n140}
\ncline[]{n46}{n141}
\ncline[]{n46}{n147}
\ncline[]{n46}{n148}
\ncline[]{n47}{n50}
\ncline[]{n47}{n51}
\ncline[]{n47}{n58}
\ncline[]{n47}{n59}
\ncline[]{n47}{n64}
\ncline[]{n47}{n65}
\ncline[]{n47}{n66}
\ncline[]{n47}{n72}
\ncline[]{n47}{n77}
\ncline[]{n47}{n79}
\ncline[]{n47}{n81}
\ncline[]{n47}{n82}
\ncline[]{n47}{n85}
\ncline[]{n47}{n86}
\ncline[]{n47}{n89}
\ncline[]{n47}{n93}
\ncline[]{n47}{n94}
\ncline[]{n47}{n96}
\ncline[]{n47}{n98}
\ncline[]{n47}{n99}
\ncline[]{n47}{n100}
\ncline[]{n47}{n101}
\ncline[]{n47}{n104}
\ncline[]{n47}{n110}
\ncline[]{n47}{n112}
\ncline[]{n47}{n114}
\ncline[]{n47}{n117}
\ncline[]{n47}{n119}
\ncline[]{n47}{n120}
\ncline[]{n47}{n122}
\ncline[]{n47}{n123}
\ncline[]{n47}{n126}
\ncline[]{n47}{n128}
\ncline[]{n47}{n137}
\ncline[]{n47}{n139}
\ncline[]{n47}{n142}
\ncline[]{n47}{n143}
\ncline[]{n47}{n144}
\ncline[]{n47}{n146}
\ncline[]{n48}{n49}
\ncline[]{n48}{n52}
\ncline[]{n48}{n54}
\ncline[]{n48}{n62}
\ncline[]{n48}{n63}
\ncline[]{n48}{n67}
\ncline[]{n48}{n68}
\ncline[]{n48}{n69}
\ncline[]{n48}{n73}
\ncline[]{n48}{n74}
\ncline[]{n48}{n75}
\ncline[]{n48}{n76}
\ncline[]{n48}{n78}
\ncline[]{n48}{n88}
\ncline[]{n48}{n90}
\ncline[]{n48}{n91}
\ncline[]{n48}{n105}
\ncline[]{n48}{n106}
\ncline[]{n48}{n108}
\ncline[]{n48}{n109}
\ncline[]{n48}{n111}
\ncline[]{n48}{n113}
\ncline[]{n48}{n115}
\ncline[]{n48}{n121}
\ncline[]{n48}{n124}
\ncline[]{n48}{n125}
\ncline[]{n48}{n127}
\ncline[]{n48}{n130}
\ncline[]{n48}{n133}
\ncline[]{n48}{n134}
\ncline[]{n48}{n135}
\ncline[]{n48}{n136}
\ncline[]{n48}{n145}
\ncline[]{n48}{n149}
\ncline[]{n49}{n52}
\ncline[]{n49}{n54}
\ncline[]{n49}{n62}
\ncline[]{n49}{n63}
\ncline[]{n49}{n67}
\ncline[]{n49}{n68}
\ncline[]{n49}{n69}
\ncline[]{n49}{n73}
\ncline[]{n49}{n74}
\ncline[]{n49}{n75}
\ncline[]{n49}{n76}
\ncline[]{n49}{n78}
\ncline[]{n49}{n88}
\ncline[]{n49}{n90}
\ncline[]{n49}{n91}
\ncline[]{n49}{n105}
\ncline[]{n49}{n106}
\ncline[]{n49}{n108}
\ncline[]{n49}{n109}
\ncline[]{n49}{n111}
\ncline[]{n49}{n113}
\ncline[]{n49}{n115}
\ncline[]{n49}{n121}
\ncline[]{n49}{n124}
\ncline[]{n49}{n125}
\ncline[]{n49}{n127}
\ncline[]{n49}{n130}
\ncline[]{n49}{n133}
\ncline[]{n49}{n134}
\ncline[]{n49}{n135}
\ncline[]{n49}{n136}
\ncline[]{n49}{n145}
\ncline[]{n49}{n149}
\ncline[]{n50}{n51}
\ncline[]{n50}{n58}
\ncline[]{n50}{n59}
\ncline[]{n50}{n64}
\ncline[]{n50}{n65}
\ncline[]{n50}{n66}
\ncline[]{n50}{n72}
\ncline[]{n50}{n77}
\ncline[]{n50}{n79}
\ncline[]{n50}{n81}
\ncline[]{n50}{n82}
\ncline[]{n50}{n85}
\ncline[]{n50}{n86}
\ncline[]{n50}{n89}
\ncline[]{n50}{n93}
\ncline[]{n50}{n94}
\ncline[]{n50}{n96}
\ncline[]{n50}{n98}
\ncline[]{n50}{n99}
\ncline[]{n50}{n100}
\ncline[]{n50}{n101}
\ncline[]{n50}{n104}
\ncline[]{n50}{n110}
\ncline[]{n50}{n112}
\ncline[]{n50}{n114}
\ncline[]{n50}{n117}
\ncline[]{n50}{n119}
\ncline[]{n50}{n120}
\ncline[]{n50}{n122}
\ncline[]{n50}{n123}
\ncline[]{n50}{n126}
\ncline[]{n50}{n128}
\ncline[]{n50}{n137}
\ncline[]{n50}{n139}
\ncline[]{n50}{n142}
\ncline[]{n50}{n143}
\ncline[]{n50}{n144}
\ncline[]{n50}{n146}
\ncline[]{n51}{n58}
\ncline[]{n51}{n59}
\ncline[]{n51}{n64}
\ncline[]{n51}{n65}
\ncline[]{n51}{n66}
\ncline[]{n51}{n72}
\ncline[]{n51}{n77}
\ncline[]{n51}{n79}
\ncline[]{n51}{n81}
\ncline[]{n51}{n82}
\ncline[]{n51}{n85}
\ncline[]{n51}{n86}
\ncline[]{n51}{n89}
\ncline[]{n51}{n93}
\ncline[]{n51}{n94}
\ncline[]{n51}{n96}
\ncline[]{n51}{n98}
\ncline[]{n51}{n99}
\ncline[]{n51}{n100}
\ncline[]{n51}{n101}
\ncline[]{n51}{n104}
\ncline[]{n51}{n110}
\ncline[]{n51}{n112}
\ncline[]{n51}{n114}
\ncline[]{n51}{n117}
\ncline[]{n51}{n119}
\ncline[]{n51}{n120}
\ncline[]{n51}{n122}
\ncline[]{n51}{n123}
\ncline[]{n51}{n126}
\ncline[]{n51}{n128}
\ncline[]{n51}{n137}
\ncline[]{n51}{n139}
\ncline[]{n51}{n142}
\ncline[]{n51}{n143}
\ncline[]{n51}{n144}
\ncline[]{n51}{n146}
\ncline[]{n52}{n54}
\ncline[]{n52}{n62}
\ncline[]{n52}{n63}
\ncline[]{n52}{n67}
\ncline[]{n52}{n68}
\ncline[]{n52}{n69}
\ncline[]{n52}{n73}
\ncline[]{n52}{n74}
\ncline[]{n52}{n75}
\ncline[]{n52}{n76}
\ncline[]{n52}{n78}
\ncline[]{n52}{n88}
\ncline[]{n52}{n90}
\ncline[]{n52}{n91}
\ncline[]{n52}{n105}
\ncline[]{n52}{n106}
\ncline[]{n52}{n108}
\ncline[]{n52}{n109}
\ncline[]{n52}{n111}
\ncline[]{n52}{n113}
\ncline[]{n52}{n115}
\ncline[]{n52}{n121}
\ncline[]{n52}{n124}
\ncline[]{n52}{n125}
\ncline[]{n52}{n127}
\ncline[]{n52}{n130}
\ncline[]{n52}{n133}
\ncline[]{n52}{n134}
\ncline[]{n52}{n135}
\ncline[]{n52}{n136}
\ncline[]{n52}{n145}
\ncline[]{n52}{n149}
\ncline[]{n53}{n55}
\ncline[]{n53}{n56}
\ncline[]{n53}{n57}
\ncline[]{n53}{n60}
\ncline[]{n53}{n61}
\ncline[]{n53}{n70}
\ncline[]{n53}{n71}
\ncline[]{n53}{n80}
\ncline[]{n53}{n83}
\ncline[]{n53}{n84}
\ncline[]{n53}{n87}
\ncline[]{n53}{n92}
\ncline[]{n53}{n95}
\ncline[]{n53}{n97}
\ncline[]{n53}{n102}
\ncline[]{n53}{n103}
\ncline[]{n53}{n107}
\ncline[]{n53}{n116}
\ncline[]{n53}{n118}
\ncline[]{n53}{n129}
\ncline[]{n53}{n131}
\ncline[]{n53}{n132}
\ncline[]{n53}{n138}
\ncline[]{n53}{n140}
\ncline[]{n53}{n141}
\ncline[]{n53}{n147}
\ncline[]{n53}{n148}
\ncline[]{n54}{n62}
\ncline[]{n54}{n63}
\ncline[]{n54}{n67}
\ncline[]{n54}{n68}
\ncline[]{n54}{n69}
\ncline[]{n54}{n73}
\ncline[]{n54}{n74}
\ncline[]{n54}{n75}
\ncline[]{n54}{n76}
\ncline[]{n54}{n78}
\ncline[]{n54}{n88}
\ncline[]{n54}{n90}
\ncline[]{n54}{n91}
\ncline[]{n54}{n105}
\ncline[]{n54}{n106}
\ncline[]{n54}{n108}
\ncline[]{n54}{n109}
\ncline[]{n54}{n111}
\ncline[]{n54}{n113}
\ncline[]{n54}{n115}
\ncline[]{n54}{n121}
\ncline[]{n54}{n124}
\ncline[]{n54}{n125}
\ncline[]{n54}{n127}
\ncline[]{n54}{n130}
\ncline[]{n54}{n133}
\ncline[]{n54}{n134}
\ncline[]{n54}{n135}
\ncline[]{n54}{n136}
\ncline[]{n54}{n145}
\ncline[]{n54}{n149}
\ncline[]{n55}{n56}
\ncline[]{n55}{n57}
\ncline[]{n55}{n60}
\ncline[]{n55}{n61}
\ncline[]{n55}{n70}
\ncline[]{n55}{n71}
\ncline[]{n55}{n80}
\ncline[]{n55}{n83}
\ncline[]{n55}{n84}
\ncline[]{n55}{n87}
\ncline[]{n55}{n92}
\ncline[]{n55}{n95}
\ncline[]{n55}{n97}
\ncline[]{n55}{n102}
\ncline[]{n55}{n103}
\ncline[]{n55}{n107}
\ncline[]{n55}{n116}
\ncline[]{n55}{n118}
\ncline[]{n55}{n129}
\ncline[]{n55}{n131}
\ncline[]{n55}{n132}
\ncline[]{n55}{n138}
\ncline[]{n55}{n140}
\ncline[]{n55}{n141}
\ncline[]{n55}{n147}
\ncline[]{n55}{n148}
\ncline[]{n56}{n57}
\ncline[]{n56}{n60}
\ncline[]{n56}{n61}
\ncline[]{n56}{n70}
\ncline[]{n56}{n71}
\ncline[]{n56}{n80}
\ncline[]{n56}{n83}
\ncline[]{n56}{n84}
\ncline[]{n56}{n87}
\ncline[]{n56}{n92}
\ncline[]{n56}{n95}
\ncline[]{n56}{n97}
\ncline[]{n56}{n102}
\ncline[]{n56}{n103}
\ncline[]{n56}{n107}
\ncline[]{n56}{n116}
\ncline[]{n56}{n118}
\ncline[]{n56}{n129}
\ncline[]{n56}{n131}
\ncline[]{n56}{n132}
\ncline[]{n56}{n138}
\ncline[]{n56}{n140}
\ncline[]{n56}{n141}
\ncline[]{n56}{n147}
\ncline[]{n56}{n148}
\ncline[]{n57}{n60}
\ncline[]{n57}{n61}
\ncline[]{n57}{n70}
\ncline[]{n57}{n71}
\ncline[]{n57}{n80}
\ncline[]{n57}{n83}
\ncline[]{n57}{n84}
\ncline[]{n57}{n87}
\ncline[]{n57}{n92}
\ncline[]{n57}{n95}
\ncline[]{n57}{n97}
\ncline[]{n57}{n102}
\ncline[]{n57}{n103}
\ncline[]{n57}{n107}
\ncline[]{n57}{n116}
\ncline[]{n57}{n118}
\ncline[]{n57}{n129}
\ncline[]{n57}{n131}
\ncline[]{n57}{n132}
\ncline[]{n57}{n138}
\ncline[]{n57}{n140}
\ncline[]{n57}{n141}
\ncline[]{n57}{n147}
\ncline[]{n57}{n148}
\ncline[]{n58}{n59}
\ncline[]{n58}{n64}
\ncline[]{n58}{n65}
\ncline[]{n58}{n66}
\ncline[]{n58}{n72}
\ncline[]{n58}{n77}
\ncline[]{n58}{n79}
\ncline[]{n58}{n81}
\ncline[]{n58}{n82}
\ncline[]{n58}{n85}
\ncline[]{n58}{n86}
\ncline[]{n58}{n89}
\ncline[]{n58}{n93}
\ncline[]{n58}{n94}
\ncline[]{n58}{n96}
\ncline[]{n58}{n98}
\ncline[]{n58}{n99}
\ncline[]{n58}{n100}
\ncline[]{n58}{n101}
\ncline[]{n58}{n104}
\ncline[]{n58}{n110}
\ncline[]{n58}{n112}
\ncline[]{n58}{n114}
\ncline[]{n58}{n117}
\ncline[]{n58}{n119}
\ncline[]{n58}{n120}
\ncline[]{n58}{n122}
\ncline[]{n58}{n123}
\ncline[]{n58}{n126}
\ncline[]{n58}{n128}
\ncline[]{n58}{n137}
\ncline[]{n58}{n139}
\ncline[]{n58}{n142}
\ncline[]{n58}{n143}
\ncline[]{n58}{n144}
\ncline[]{n58}{n146}
\ncline[]{n59}{n64}
\ncline[]{n59}{n65}
\ncline[]{n59}{n66}
\ncline[]{n59}{n72}
\ncline[]{n59}{n77}
\ncline[]{n59}{n79}
\ncline[]{n59}{n81}
\ncline[]{n59}{n82}
\ncline[]{n59}{n85}
\ncline[]{n59}{n86}
\ncline[]{n59}{n89}
\ncline[]{n59}{n93}
\ncline[]{n59}{n94}
\ncline[]{n59}{n96}
\ncline[]{n59}{n98}
\ncline[]{n59}{n99}
\ncline[]{n59}{n100}
\ncline[]{n59}{n101}
\ncline[]{n59}{n104}
\ncline[]{n59}{n110}
\ncline[]{n59}{n112}
\ncline[]{n59}{n114}
\ncline[]{n59}{n117}
\ncline[]{n59}{n119}
\ncline[]{n59}{n120}
\ncline[]{n59}{n122}
\ncline[]{n59}{n123}
\ncline[]{n59}{n126}
\ncline[]{n59}{n128}
\ncline[]{n59}{n137}
\ncline[]{n59}{n139}
\ncline[]{n59}{n142}
\ncline[]{n59}{n143}
\ncline[]{n59}{n144}
\ncline[]{n59}{n146}
\ncline[]{n60}{n61}
\ncline[]{n60}{n70}
\ncline[]{n60}{n71}
\ncline[]{n60}{n80}
\ncline[]{n60}{n83}
\ncline[]{n60}{n84}
\ncline[]{n60}{n87}
\ncline[]{n60}{n92}
\ncline[]{n60}{n95}
\ncline[]{n60}{n97}
\ncline[]{n60}{n102}
\ncline[]{n60}{n103}
\ncline[]{n60}{n107}
\ncline[]{n60}{n116}
\ncline[]{n60}{n118}
\ncline[]{n60}{n129}
\ncline[]{n60}{n131}
\ncline[]{n60}{n132}
\ncline[]{n60}{n138}
\ncline[]{n60}{n140}
\ncline[]{n60}{n141}
\ncline[]{n60}{n147}
\ncline[]{n60}{n148}
\ncline[]{n61}{n70}
\ncline[]{n61}{n71}
\ncline[]{n61}{n80}
\ncline[]{n61}{n83}
\ncline[]{n61}{n84}
\ncline[]{n61}{n87}
\ncline[]{n61}{n92}
\ncline[]{n61}{n95}
\ncline[]{n61}{n97}
\ncline[]{n61}{n102}
\ncline[]{n61}{n103}
\ncline[]{n61}{n107}
\ncline[]{n61}{n116}
\ncline[]{n61}{n118}
\ncline[]{n61}{n129}
\ncline[]{n61}{n131}
\ncline[]{n61}{n132}
\ncline[]{n61}{n138}
\ncline[]{n61}{n140}
\ncline[]{n61}{n141}
\ncline[]{n61}{n147}
\ncline[]{n61}{n148}
\ncline[]{n62}{n63}
\ncline[]{n62}{n67}
\ncline[]{n62}{n68}
\ncline[]{n62}{n69}
\ncline[]{n62}{n73}
\ncline[]{n62}{n74}
\ncline[]{n62}{n75}
\ncline[]{n62}{n76}
\ncline[]{n62}{n78}
\ncline[]{n62}{n88}
\ncline[]{n62}{n90}
\ncline[]{n62}{n91}
\ncline[]{n62}{n105}
\ncline[]{n62}{n106}
\ncline[]{n62}{n108}
\ncline[]{n62}{n109}
\ncline[]{n62}{n111}
\ncline[]{n62}{n113}
\ncline[]{n62}{n115}
\ncline[]{n62}{n121}
\ncline[]{n62}{n124}
\ncline[]{n62}{n125}
\ncline[]{n62}{n127}
\ncline[]{n62}{n130}
\ncline[]{n62}{n133}
\ncline[]{n62}{n134}
\ncline[]{n62}{n135}
\ncline[]{n62}{n136}
\ncline[]{n62}{n145}
\ncline[]{n62}{n149}
\ncline[]{n63}{n67}
\ncline[]{n63}{n68}
\ncline[]{n63}{n69}
\ncline[]{n63}{n73}
\ncline[]{n63}{n74}
\ncline[]{n63}{n75}
\ncline[]{n63}{n76}
\ncline[]{n63}{n78}
\ncline[]{n63}{n88}
\ncline[]{n63}{n90}
\ncline[]{n63}{n91}
\ncline[]{n63}{n105}
\ncline[]{n63}{n106}
\ncline[]{n63}{n108}
\ncline[]{n63}{n109}
\ncline[]{n63}{n111}
\ncline[]{n63}{n113}
\ncline[]{n63}{n115}
\ncline[]{n63}{n121}
\ncline[]{n63}{n124}
\ncline[]{n63}{n125}
\ncline[]{n63}{n127}
\ncline[]{n63}{n130}
\ncline[]{n63}{n133}
\ncline[]{n63}{n134}
\ncline[]{n63}{n135}
\ncline[]{n63}{n136}
\ncline[]{n63}{n145}
\ncline[]{n63}{n149}
\ncline[]{n64}{n65}
\ncline[]{n64}{n66}
\ncline[]{n64}{n72}
\ncline[]{n64}{n77}
\ncline[]{n64}{n79}
\ncline[]{n64}{n81}
\ncline[]{n64}{n82}
\ncline[]{n64}{n85}
\ncline[]{n64}{n86}
\ncline[]{n64}{n89}
\ncline[]{n64}{n93}
\ncline[]{n64}{n94}
\ncline[]{n64}{n96}
\ncline[]{n64}{n98}
\ncline[]{n64}{n99}
\ncline[]{n64}{n100}
\ncline[]{n64}{n101}
\ncline[]{n64}{n104}
\ncline[]{n64}{n110}
\ncline[]{n64}{n112}
\ncline[]{n64}{n114}
\ncline[]{n64}{n117}
\ncline[]{n64}{n119}
\ncline[]{n64}{n120}
\ncline[]{n64}{n122}
\ncline[]{n64}{n123}
\ncline[]{n64}{n126}
\ncline[]{n64}{n128}
\ncline[]{n64}{n137}
\ncline[]{n64}{n139}
\ncline[]{n64}{n142}
\ncline[]{n64}{n143}
\ncline[]{n64}{n144}
\ncline[]{n64}{n146}
\ncline[]{n65}{n66}
\ncline[]{n65}{n72}
\ncline[]{n65}{n77}
\ncline[]{n65}{n79}
\ncline[]{n65}{n81}
\ncline[]{n65}{n82}
\ncline[]{n65}{n85}
\ncline[]{n65}{n86}
\ncline[]{n65}{n89}
\ncline[]{n65}{n93}
\ncline[]{n65}{n94}
\ncline[]{n65}{n96}
\ncline[]{n65}{n98}
\ncline[]{n65}{n99}
\ncline[]{n65}{n100}
\ncline[]{n65}{n101}
\ncline[]{n65}{n104}
\ncline[]{n65}{n110}
\ncline[]{n65}{n112}
\ncline[]{n65}{n114}
\ncline[]{n65}{n117}
\ncline[]{n65}{n119}
\ncline[]{n65}{n120}
\ncline[]{n65}{n122}
\ncline[]{n65}{n123}
\ncline[]{n65}{n126}
\ncline[]{n65}{n128}
\ncline[]{n65}{n137}
\ncline[]{n65}{n139}
\ncline[]{n65}{n142}
\ncline[]{n65}{n143}
\ncline[]{n65}{n144}
\ncline[]{n65}{n146}
\ncline[]{n66}{n72}
\ncline[]{n66}{n77}
\ncline[]{n66}{n79}
\ncline[]{n66}{n81}
\ncline[]{n66}{n82}
\ncline[]{n66}{n85}
\ncline[]{n66}{n86}
\ncline[]{n66}{n89}
\ncline[]{n66}{n93}
\ncline[]{n66}{n94}
\ncline[]{n66}{n96}
\ncline[]{n66}{n98}
\ncline[]{n66}{n99}
\ncline[]{n66}{n100}
\ncline[]{n66}{n101}
\ncline[]{n66}{n104}
\ncline[]{n66}{n110}
\ncline[]{n66}{n112}
\ncline[]{n66}{n114}
\ncline[]{n66}{n117}
\ncline[]{n66}{n119}
\ncline[]{n66}{n120}
\ncline[]{n66}{n122}
\ncline[]{n66}{n123}
\ncline[]{n66}{n126}
\ncline[]{n66}{n128}
\ncline[]{n66}{n137}
\ncline[]{n66}{n139}
\ncline[]{n66}{n142}
\ncline[]{n66}{n143}
\ncline[]{n66}{n144}
\ncline[]{n66}{n146}
\ncline[]{n67}{n68}
\ncline[]{n67}{n69}
\ncline[]{n67}{n73}
\ncline[]{n67}{n74}
\ncline[]{n67}{n75}
\ncline[]{n67}{n76}
\ncline[]{n67}{n78}
\ncline[]{n67}{n88}
\ncline[]{n67}{n90}
\ncline[]{n67}{n91}
\ncline[]{n67}{n105}
\ncline[]{n67}{n106}
\ncline[]{n67}{n108}
\ncline[]{n67}{n109}
\ncline[]{n67}{n111}
\ncline[]{n67}{n113}
\ncline[]{n67}{n115}
\ncline[]{n67}{n121}
\ncline[]{n67}{n124}
\ncline[]{n67}{n125}
\ncline[]{n67}{n127}
\ncline[]{n67}{n130}
\ncline[]{n67}{n133}
\ncline[]{n67}{n134}
\ncline[]{n67}{n135}
\ncline[]{n67}{n136}
\ncline[]{n67}{n145}
\ncline[]{n67}{n149}
\ncline[]{n68}{n69}
\ncline[]{n68}{n73}
\ncline[]{n68}{n74}
\ncline[]{n68}{n75}
\ncline[]{n68}{n76}
\ncline[]{n68}{n78}
\ncline[]{n68}{n88}
\ncline[]{n68}{n90}
\ncline[]{n68}{n91}
\ncline[]{n68}{n105}
\ncline[]{n68}{n106}
\ncline[]{n68}{n108}
\ncline[]{n68}{n109}
\ncline[]{n68}{n111}
\ncline[]{n68}{n113}
\ncline[]{n68}{n115}
\ncline[]{n68}{n121}
\ncline[]{n68}{n124}
\ncline[]{n68}{n125}
\ncline[]{n68}{n127}
\ncline[]{n68}{n130}
\ncline[]{n68}{n133}
\ncline[]{n68}{n134}
\ncline[]{n68}{n135}
\ncline[]{n68}{n136}
\ncline[]{n68}{n145}
\ncline[]{n68}{n149}
\ncline[]{n69}{n73}
\ncline[]{n69}{n74}
\ncline[]{n69}{n75}
\ncline[]{n69}{n76}
\ncline[]{n69}{n78}
\ncline[]{n69}{n88}
\ncline[]{n69}{n90}
\ncline[]{n69}{n91}
\ncline[]{n69}{n105}
\ncline[]{n69}{n106}
\ncline[]{n69}{n108}
\ncline[]{n69}{n109}
\ncline[]{n69}{n111}
\ncline[]{n69}{n113}
\ncline[]{n69}{n115}
\ncline[]{n69}{n121}
\ncline[]{n69}{n124}
\ncline[]{n69}{n125}
\ncline[]{n69}{n127}
\ncline[]{n69}{n130}
\ncline[]{n69}{n133}
\ncline[]{n69}{n134}
\ncline[]{n69}{n135}
\ncline[]{n69}{n136}
\ncline[]{n69}{n145}
\ncline[]{n69}{n149}
\ncline[]{n70}{n71}
\ncline[]{n70}{n80}
\ncline[]{n70}{n83}
\ncline[]{n70}{n84}
\ncline[]{n70}{n87}
\ncline[]{n70}{n92}
\ncline[]{n70}{n95}
\ncline[]{n70}{n97}
\ncline[]{n70}{n102}
\ncline[]{n70}{n103}
\ncline[]{n70}{n107}
\ncline[]{n70}{n116}
\ncline[]{n70}{n118}
\ncline[]{n70}{n129}
\ncline[]{n70}{n131}
\ncline[]{n70}{n132}
\ncline[]{n70}{n138}
\ncline[]{n70}{n140}
\ncline[]{n70}{n141}
\ncline[]{n70}{n147}
\ncline[]{n70}{n148}
\ncline[]{n71}{n80}
\ncline[]{n71}{n83}
\ncline[]{n71}{n84}
\ncline[]{n71}{n87}
\ncline[]{n71}{n92}
\ncline[]{n71}{n95}
\ncline[]{n71}{n97}
\ncline[]{n71}{n102}
\ncline[]{n71}{n103}
\ncline[]{n71}{n107}
\ncline[]{n71}{n116}
\ncline[]{n71}{n118}
\ncline[]{n71}{n129}
\ncline[]{n71}{n131}
\ncline[]{n71}{n132}
\ncline[]{n71}{n138}
\ncline[]{n71}{n140}
\ncline[]{n71}{n141}
\ncline[]{n71}{n147}
\ncline[]{n71}{n148}
\ncline[]{n72}{n77}
\ncline[]{n72}{n79}
\ncline[]{n72}{n81}
\ncline[]{n72}{n82}
\ncline[]{n72}{n85}
\ncline[]{n72}{n86}
\ncline[]{n72}{n89}
\ncline[]{n72}{n93}
\ncline[]{n72}{n94}
\ncline[]{n72}{n96}
\ncline[]{n72}{n98}
\ncline[]{n72}{n99}
\ncline[]{n72}{n100}
\ncline[]{n72}{n101}
\ncline[]{n72}{n104}
\ncline[]{n72}{n110}
\ncline[]{n72}{n112}
\ncline[]{n72}{n114}
\ncline[]{n72}{n117}
\ncline[]{n72}{n119}
\ncline[]{n72}{n120}
\ncline[]{n72}{n122}
\ncline[]{n72}{n123}
\ncline[]{n72}{n126}
\ncline[]{n72}{n128}
\ncline[]{n72}{n137}
\ncline[]{n72}{n139}
\ncline[]{n72}{n142}
\ncline[]{n72}{n143}
\ncline[]{n72}{n144}
\ncline[]{n72}{n146}
\ncline[]{n73}{n74}
\ncline[]{n73}{n75}
\ncline[]{n73}{n76}
\ncline[]{n73}{n78}
\ncline[]{n73}{n88}
\ncline[]{n73}{n90}
\ncline[]{n73}{n91}
\ncline[]{n73}{n105}
\ncline[]{n73}{n106}
\ncline[]{n73}{n108}
\ncline[]{n73}{n109}
\ncline[]{n73}{n111}
\ncline[]{n73}{n113}
\ncline[]{n73}{n115}
\ncline[]{n73}{n121}
\ncline[]{n73}{n124}
\ncline[]{n73}{n125}
\ncline[]{n73}{n127}
\ncline[]{n73}{n130}
\ncline[]{n73}{n133}
\ncline[]{n73}{n134}
\ncline[]{n73}{n135}
\ncline[]{n73}{n136}
\ncline[]{n73}{n145}
\ncline[]{n73}{n149}
\ncline[]{n74}{n75}
\ncline[]{n74}{n76}
\ncline[]{n74}{n78}
\ncline[]{n74}{n88}
\ncline[]{n74}{n90}
\ncline[]{n74}{n91}
\ncline[]{n74}{n105}
\ncline[]{n74}{n106}
\ncline[]{n74}{n108}
\ncline[]{n74}{n109}
\ncline[]{n74}{n111}
\ncline[]{n74}{n113}
\ncline[]{n74}{n115}
\ncline[]{n74}{n121}
\ncline[]{n74}{n124}
\ncline[]{n74}{n125}
\ncline[]{n74}{n127}
\ncline[]{n74}{n130}
\ncline[]{n74}{n133}
\ncline[]{n74}{n134}
\ncline[]{n74}{n135}
\ncline[]{n74}{n136}
\ncline[]{n74}{n145}
\ncline[]{n74}{n149}
\ncline[]{n75}{n76}
\ncline[]{n75}{n78}
\ncline[]{n75}{n88}
\ncline[]{n75}{n90}
\ncline[]{n75}{n91}
\ncline[]{n75}{n105}
\ncline[]{n75}{n106}
\ncline[]{n75}{n108}
\ncline[]{n75}{n109}
\ncline[]{n75}{n111}
\ncline[]{n75}{n113}
\ncline[]{n75}{n115}
\ncline[]{n75}{n121}
\ncline[]{n75}{n124}
\ncline[]{n75}{n125}
\ncline[]{n75}{n127}
\ncline[]{n75}{n130}
\ncline[]{n75}{n133}
\ncline[]{n75}{n134}
\ncline[]{n75}{n135}
\ncline[]{n75}{n136}
\ncline[]{n75}{n145}
\ncline[]{n75}{n149}
\ncline[]{n76}{n78}
\ncline[]{n76}{n88}
\ncline[]{n76}{n90}
\ncline[]{n76}{n91}
\ncline[]{n76}{n105}
\ncline[]{n76}{n106}
\ncline[]{n76}{n108}
\ncline[]{n76}{n109}
\ncline[]{n76}{n111}
\ncline[]{n76}{n113}
\ncline[]{n76}{n115}
\ncline[]{n76}{n121}
\ncline[]{n76}{n124}
\ncline[]{n76}{n125}
\ncline[]{n76}{n127}
\ncline[]{n76}{n130}
\ncline[]{n76}{n133}
\ncline[]{n76}{n134}
\ncline[]{n76}{n135}
\ncline[]{n76}{n136}
\ncline[]{n76}{n145}
\ncline[]{n76}{n149}
\ncline[]{n77}{n79}
\ncline[]{n77}{n81}
\ncline[]{n77}{n82}
\ncline[]{n77}{n85}
\ncline[]{n77}{n86}
\ncline[]{n77}{n89}
\ncline[]{n77}{n93}
\ncline[]{n77}{n94}
\ncline[]{n77}{n96}
\ncline[]{n77}{n98}
\ncline[]{n77}{n99}
\ncline[]{n77}{n100}
\ncline[]{n77}{n101}
\ncline[]{n77}{n104}
\ncline[]{n77}{n110}
\ncline[]{n77}{n112}
\ncline[]{n77}{n114}
\ncline[]{n77}{n117}
\ncline[]{n77}{n119}
\ncline[]{n77}{n120}
\ncline[]{n77}{n122}
\ncline[]{n77}{n123}
\ncline[]{n77}{n126}
\ncline[]{n77}{n128}
\ncline[]{n77}{n137}
\ncline[]{n77}{n139}
\ncline[]{n77}{n142}
\ncline[]{n77}{n143}
\ncline[]{n77}{n144}
\ncline[]{n77}{n146}
\ncline[]{n78}{n88}
\ncline[]{n78}{n90}
\ncline[]{n78}{n91}
\ncline[]{n78}{n105}
\ncline[]{n78}{n106}
\ncline[]{n78}{n108}
\ncline[]{n78}{n109}
\ncline[]{n78}{n111}
\ncline[]{n78}{n113}
\ncline[]{n78}{n115}
\ncline[]{n78}{n121}
\ncline[]{n78}{n124}
\ncline[]{n78}{n125}
\ncline[]{n78}{n127}
\ncline[]{n78}{n130}
\ncline[]{n78}{n133}
\ncline[]{n78}{n134}
\ncline[]{n78}{n135}
\ncline[]{n78}{n136}
\ncline[]{n78}{n145}
\ncline[]{n78}{n149}
\ncline[]{n79}{n81}
\ncline[]{n79}{n82}
\ncline[]{n79}{n85}
\ncline[]{n79}{n86}
\ncline[]{n79}{n89}
\ncline[]{n79}{n93}
\ncline[]{n79}{n94}
\ncline[]{n79}{n96}
\ncline[]{n79}{n98}
\ncline[]{n79}{n99}
\ncline[]{n79}{n100}
\ncline[]{n79}{n101}
\ncline[]{n79}{n104}
\ncline[]{n79}{n110}
\ncline[]{n79}{n112}
\ncline[]{n79}{n114}
\ncline[]{n79}{n117}
\ncline[]{n79}{n119}
\ncline[]{n79}{n120}
\ncline[]{n79}{n122}
\ncline[]{n79}{n123}
\ncline[]{n79}{n126}
\ncline[]{n79}{n128}
\ncline[]{n79}{n137}
\ncline[]{n79}{n139}
\ncline[]{n79}{n142}
\ncline[]{n79}{n143}
\ncline[]{n79}{n144}
\ncline[]{n79}{n146}
\ncline[]{n80}{n83}
\ncline[]{n80}{n84}
\ncline[]{n80}{n87}
\ncline[]{n80}{n92}
\ncline[]{n80}{n95}
\ncline[]{n80}{n97}
\ncline[]{n80}{n102}
\ncline[]{n80}{n103}
\ncline[]{n80}{n107}
\ncline[]{n80}{n116}
\ncline[]{n80}{n118}
\ncline[]{n80}{n129}
\ncline[]{n80}{n131}
\ncline[]{n80}{n132}
\ncline[]{n80}{n138}
\ncline[]{n80}{n140}
\ncline[]{n80}{n141}
\ncline[]{n80}{n147}
\ncline[]{n80}{n148}
\ncline[]{n81}{n82}
\ncline[]{n81}{n85}
\ncline[]{n81}{n86}
\ncline[]{n81}{n89}
\ncline[]{n81}{n93}
\ncline[]{n81}{n94}
\ncline[]{n81}{n96}
\ncline[]{n81}{n98}
\ncline[]{n81}{n99}
\ncline[]{n81}{n100}
\ncline[]{n81}{n101}
\ncline[]{n81}{n104}
\ncline[]{n81}{n110}
\ncline[]{n81}{n112}
\ncline[]{n81}{n114}
\ncline[]{n81}{n117}
\ncline[]{n81}{n119}
\ncline[]{n81}{n120}
\ncline[]{n81}{n122}
\ncline[]{n81}{n123}
\ncline[]{n81}{n126}
\ncline[]{n81}{n128}
\ncline[]{n81}{n137}
\ncline[]{n81}{n139}
\ncline[]{n81}{n142}
\ncline[]{n81}{n143}
\ncline[]{n81}{n144}
\ncline[]{n81}{n146}
\ncline[]{n82}{n85}
\ncline[]{n82}{n86}
\ncline[]{n82}{n89}
\ncline[]{n82}{n93}
\ncline[]{n82}{n94}
\ncline[]{n82}{n96}
\ncline[]{n82}{n98}
\ncline[]{n82}{n99}
\ncline[]{n82}{n100}
\ncline[]{n82}{n101}
\ncline[]{n82}{n104}
\ncline[]{n82}{n110}
\ncline[]{n82}{n112}
\ncline[]{n82}{n114}
\ncline[]{n82}{n117}
\ncline[]{n82}{n119}
\ncline[]{n82}{n120}
\ncline[]{n82}{n122}
\ncline[]{n82}{n123}
\ncline[]{n82}{n126}
\ncline[]{n82}{n128}
\ncline[]{n82}{n137}
\ncline[]{n82}{n139}
\ncline[]{n82}{n142}
\ncline[]{n82}{n143}
\ncline[]{n82}{n144}
\ncline[]{n82}{n146}
\ncline[]{n83}{n84}
\ncline[]{n83}{n87}
\ncline[]{n83}{n92}
\ncline[]{n83}{n95}
\ncline[]{n83}{n97}
\ncline[]{n83}{n102}
\ncline[]{n83}{n103}
\ncline[]{n83}{n107}
\ncline[]{n83}{n116}
\ncline[]{n83}{n118}
\ncline[]{n83}{n129}
\ncline[]{n83}{n131}
\ncline[]{n83}{n132}
\ncline[]{n83}{n138}
\ncline[]{n83}{n140}
\ncline[]{n83}{n141}
\ncline[]{n83}{n147}
\ncline[]{n83}{n148}
\ncline[]{n84}{n87}
\ncline[]{n84}{n92}
\ncline[]{n84}{n95}
\ncline[]{n84}{n97}
\ncline[]{n84}{n102}
\ncline[]{n84}{n103}
\ncline[]{n84}{n107}
\ncline[]{n84}{n116}
\ncline[]{n84}{n118}
\ncline[]{n84}{n129}
\ncline[]{n84}{n131}
\ncline[]{n84}{n132}
\ncline[]{n84}{n138}
\ncline[]{n84}{n140}
\ncline[]{n84}{n141}
\ncline[]{n84}{n147}
\ncline[]{n84}{n148}
\ncline[]{n85}{n86}
\ncline[]{n85}{n89}
\ncline[]{n85}{n93}
\ncline[]{n85}{n94}
\ncline[]{n85}{n96}
\ncline[]{n85}{n98}
\ncline[]{n85}{n99}
\ncline[]{n85}{n100}
\ncline[]{n85}{n101}
\ncline[]{n85}{n104}
\ncline[]{n85}{n110}
\ncline[]{n85}{n112}
\ncline[]{n85}{n114}
\ncline[]{n85}{n117}
\ncline[]{n85}{n119}
\ncline[]{n85}{n120}
\ncline[]{n85}{n122}
\ncline[]{n85}{n123}
\ncline[]{n85}{n126}
\ncline[]{n85}{n128}
\ncline[]{n85}{n137}
\ncline[]{n85}{n139}
\ncline[]{n85}{n142}
\ncline[]{n85}{n143}
\ncline[]{n85}{n144}
\ncline[]{n85}{n146}
\ncline[]{n86}{n89}
\ncline[]{n86}{n93}
\ncline[]{n86}{n94}
\ncline[]{n86}{n96}
\ncline[]{n86}{n98}
\ncline[]{n86}{n99}
\ncline[]{n86}{n100}
\ncline[]{n86}{n101}
\ncline[]{n86}{n104}
\ncline[]{n86}{n110}
\ncline[]{n86}{n112}
\ncline[]{n86}{n114}
\ncline[]{n86}{n117}
\ncline[]{n86}{n119}
\ncline[]{n86}{n120}
\ncline[]{n86}{n122}
\ncline[]{n86}{n123}
\ncline[]{n86}{n126}
\ncline[]{n86}{n128}
\ncline[]{n86}{n137}
\ncline[]{n86}{n139}
\ncline[]{n86}{n142}
\ncline[]{n86}{n143}
\ncline[]{n86}{n144}
\ncline[]{n86}{n146}
\ncline[]{n87}{n92}
\ncline[]{n87}{n95}
\ncline[]{n87}{n97}
\ncline[]{n87}{n102}
\ncline[]{n87}{n103}
\ncline[]{n87}{n107}
\ncline[]{n87}{n116}
\ncline[]{n87}{n118}
\ncline[]{n87}{n129}
\ncline[]{n87}{n131}
\ncline[]{n87}{n132}
\ncline[]{n87}{n138}
\ncline[]{n87}{n140}
\ncline[]{n87}{n141}
\ncline[]{n87}{n147}
\ncline[]{n87}{n148}
\ncline[]{n88}{n90}
\ncline[]{n88}{n91}
\ncline[]{n88}{n105}
\ncline[]{n88}{n106}
\ncline[]{n88}{n108}
\ncline[]{n88}{n109}
\ncline[]{n88}{n111}
\ncline[]{n88}{n113}
\ncline[]{n88}{n115}
\ncline[]{n88}{n121}
\ncline[]{n88}{n124}
\ncline[]{n88}{n125}
\ncline[]{n88}{n127}
\ncline[]{n88}{n130}
\ncline[]{n88}{n133}
\ncline[]{n88}{n134}
\ncline[]{n88}{n135}
\ncline[]{n88}{n136}
\ncline[]{n88}{n145}
\ncline[]{n88}{n149}
\ncline[]{n89}{n93}
\ncline[]{n89}{n94}
\ncline[]{n89}{n96}
\ncline[]{n89}{n98}
\ncline[]{n89}{n99}
\ncline[]{n89}{n100}
\ncline[]{n89}{n101}
\ncline[]{n89}{n104}
\ncline[]{n89}{n110}
\ncline[]{n89}{n112}
\ncline[]{n89}{n114}
\ncline[]{n89}{n117}
\ncline[]{n89}{n119}
\ncline[]{n89}{n120}
\ncline[]{n89}{n122}
\ncline[]{n89}{n123}
\ncline[]{n89}{n126}
\ncline[]{n89}{n128}
\ncline[]{n89}{n137}
\ncline[]{n89}{n139}
\ncline[]{n89}{n142}
\ncline[]{n89}{n143}
\ncline[]{n89}{n144}
\ncline[]{n89}{n146}
\ncline[]{n90}{n91}
\ncline[]{n90}{n105}
\ncline[]{n90}{n106}
\ncline[]{n90}{n108}
\ncline[]{n90}{n109}
\ncline[]{n90}{n111}
\ncline[]{n90}{n113}
\ncline[]{n90}{n115}
\ncline[]{n90}{n121}
\ncline[]{n90}{n124}
\ncline[]{n90}{n125}
\ncline[]{n90}{n127}
\ncline[]{n90}{n130}
\ncline[]{n90}{n133}
\ncline[]{n90}{n134}
\ncline[]{n90}{n135}
\ncline[]{n90}{n136}
\ncline[]{n90}{n145}
\ncline[]{n90}{n149}
\ncline[]{n91}{n105}
\ncline[]{n91}{n106}
\ncline[]{n91}{n108}
\ncline[]{n91}{n109}
\ncline[]{n91}{n111}
\ncline[]{n91}{n113}
\ncline[]{n91}{n115}
\ncline[]{n91}{n121}
\ncline[]{n91}{n124}
\ncline[]{n91}{n125}
\ncline[]{n91}{n127}
\ncline[]{n91}{n130}
\ncline[]{n91}{n133}
\ncline[]{n91}{n134}
\ncline[]{n91}{n135}
\ncline[]{n91}{n136}
\ncline[]{n91}{n145}
\ncline[]{n91}{n149}
\ncline[]{n92}{n95}
\ncline[]{n92}{n97}
\ncline[]{n92}{n102}
\ncline[]{n92}{n103}
\ncline[]{n92}{n107}
\ncline[]{n92}{n116}
\ncline[]{n92}{n118}
\ncline[]{n92}{n129}
\ncline[]{n92}{n131}
\ncline[]{n92}{n132}
\ncline[]{n92}{n138}
\ncline[]{n92}{n140}
\ncline[]{n92}{n141}
\ncline[]{n92}{n147}
\ncline[]{n92}{n148}
\ncline[]{n93}{n94}
\ncline[]{n93}{n96}
\ncline[]{n93}{n98}
\ncline[]{n93}{n99}
\ncline[]{n93}{n100}
\ncline[]{n93}{n101}
\ncline[]{n93}{n104}
\ncline[]{n93}{n110}
\ncline[]{n93}{n112}
\ncline[]{n93}{n114}
\ncline[]{n93}{n117}
\ncline[]{n93}{n119}
\ncline[]{n93}{n120}
\ncline[]{n93}{n122}
\ncline[]{n93}{n123}
\ncline[]{n93}{n126}
\ncline[]{n93}{n128}
\ncline[]{n93}{n137}
\ncline[]{n93}{n139}
\ncline[]{n93}{n142}
\ncline[]{n93}{n143}
\ncline[]{n93}{n144}
\ncline[]{n93}{n146}
\ncline[]{n94}{n96}
\ncline[]{n94}{n98}
\ncline[]{n94}{n99}
\ncline[]{n94}{n100}
\ncline[]{n94}{n101}
\ncline[]{n94}{n104}
\ncline[]{n94}{n110}
\ncline[]{n94}{n112}
\ncline[]{n94}{n114}
\ncline[]{n94}{n117}
\ncline[]{n94}{n119}
\ncline[]{n94}{n120}
\ncline[]{n94}{n122}
\ncline[]{n94}{n123}
\ncline[]{n94}{n126}
\ncline[]{n94}{n128}
\ncline[]{n94}{n137}
\ncline[]{n94}{n139}
\ncline[]{n94}{n142}
\ncline[]{n94}{n143}
\ncline[]{n94}{n144}
\ncline[]{n94}{n146}
\ncline[]{n95}{n97}
\ncline[]{n95}{n102}
\ncline[]{n95}{n103}
\ncline[]{n95}{n107}
\ncline[]{n95}{n116}
\ncline[]{n95}{n118}
\ncline[]{n95}{n129}
\ncline[]{n95}{n131}
\ncline[]{n95}{n132}
\ncline[]{n95}{n138}
\ncline[]{n95}{n140}
\ncline[]{n95}{n141}
\ncline[]{n95}{n147}
\ncline[]{n95}{n148}
\ncline[]{n96}{n98}
\ncline[]{n96}{n99}
\ncline[]{n96}{n100}
\ncline[]{n96}{n101}
\ncline[]{n96}{n104}
\ncline[]{n96}{n110}
\ncline[]{n96}{n112}
\ncline[]{n96}{n114}
\ncline[]{n96}{n117}
\ncline[]{n96}{n119}
\ncline[]{n96}{n120}
\ncline[]{n96}{n122}
\ncline[]{n96}{n123}
\ncline[]{n96}{n126}
\ncline[]{n96}{n128}
\ncline[]{n96}{n137}
\ncline[]{n96}{n139}
\ncline[]{n96}{n142}
\ncline[]{n96}{n143}
\ncline[]{n96}{n144}
\ncline[]{n96}{n146}
\ncline[]{n97}{n102}
\ncline[]{n97}{n103}
\ncline[]{n97}{n107}
\ncline[]{n97}{n116}
\ncline[]{n97}{n118}
\ncline[]{n97}{n129}
\ncline[]{n97}{n131}
\ncline[]{n97}{n132}
\ncline[]{n97}{n138}
\ncline[]{n97}{n140}
\ncline[]{n97}{n141}
\ncline[]{n97}{n147}
\ncline[]{n97}{n148}
\ncline[]{n98}{n99}
\ncline[]{n98}{n100}
\ncline[]{n98}{n101}
\ncline[]{n98}{n104}
\ncline[]{n98}{n110}
\ncline[]{n98}{n112}
\ncline[]{n98}{n114}
\ncline[]{n98}{n117}
\ncline[]{n98}{n119}
\ncline[]{n98}{n120}
\ncline[]{n98}{n122}
\ncline[]{n98}{n123}
\ncline[]{n98}{n126}
\ncline[]{n98}{n128}
\ncline[]{n98}{n137}
\ncline[]{n98}{n139}
\ncline[]{n98}{n142}
\ncline[]{n98}{n143}
\ncline[]{n98}{n144}
\ncline[]{n98}{n146}
\ncline[]{n99}{n100}
\ncline[]{n99}{n101}
\ncline[]{n99}{n104}
\ncline[]{n99}{n110}
\ncline[]{n99}{n112}
\ncline[]{n99}{n114}
\ncline[]{n99}{n117}
\ncline[]{n99}{n119}
\ncline[]{n99}{n120}
\ncline[]{n99}{n122}
\ncline[]{n99}{n123}
\ncline[]{n99}{n126}
\ncline[]{n99}{n128}
\ncline[]{n99}{n137}
\ncline[]{n99}{n139}
\ncline[]{n99}{n142}
\ncline[]{n99}{n143}
\ncline[]{n99}{n144}
\ncline[]{n99}{n146}
\ncline[]{n100}{n101}
\ncline[]{n100}{n104}
\ncline[]{n100}{n110}
\ncline[]{n100}{n112}
\ncline[]{n100}{n114}
\ncline[]{n100}{n117}
\ncline[]{n100}{n119}
\ncline[]{n100}{n120}
\ncline[]{n100}{n122}
\ncline[]{n100}{n123}
\ncline[]{n100}{n126}
\ncline[]{n100}{n128}
\ncline[]{n100}{n137}
\ncline[]{n100}{n139}
\ncline[]{n100}{n142}
\ncline[]{n100}{n143}
\ncline[]{n100}{n144}
\ncline[]{n100}{n146}
\ncline[]{n101}{n104}
\ncline[]{n101}{n110}
\ncline[]{n101}{n112}
\ncline[]{n101}{n114}
\ncline[]{n101}{n117}
\ncline[]{n101}{n119}
\ncline[]{n101}{n120}
\ncline[]{n101}{n122}
\ncline[]{n101}{n123}
\ncline[]{n101}{n126}
\ncline[]{n101}{n128}
\ncline[]{n101}{n137}
\ncline[]{n101}{n139}
\ncline[]{n101}{n142}
\ncline[]{n101}{n143}
\ncline[]{n101}{n144}
\ncline[]{n101}{n146}
\ncline[]{n102}{n103}
\ncline[]{n102}{n107}
\ncline[]{n102}{n116}
\ncline[]{n102}{n118}
\ncline[]{n102}{n129}
\ncline[]{n102}{n131}
\ncline[]{n102}{n132}
\ncline[]{n102}{n138}
\ncline[]{n102}{n140}
\ncline[]{n102}{n141}
\ncline[]{n102}{n147}
\ncline[]{n102}{n148}
\ncline[]{n103}{n107}
\ncline[]{n103}{n116}
\ncline[]{n103}{n118}
\ncline[]{n103}{n129}
\ncline[]{n103}{n131}
\ncline[]{n103}{n132}
\ncline[]{n103}{n138}
\ncline[]{n103}{n140}
\ncline[]{n103}{n141}
\ncline[]{n103}{n147}
\ncline[]{n103}{n148}
\ncline[]{n104}{n110}
\ncline[]{n104}{n112}
\ncline[]{n104}{n114}
\ncline[]{n104}{n117}
\ncline[]{n104}{n119}
\ncline[]{n104}{n120}
\ncline[]{n104}{n122}
\ncline[]{n104}{n123}
\ncline[]{n104}{n126}
\ncline[]{n104}{n128}
\ncline[]{n104}{n137}
\ncline[]{n104}{n139}
\ncline[]{n104}{n142}
\ncline[]{n104}{n143}
\ncline[]{n104}{n144}
\ncline[]{n104}{n146}
\ncline[]{n105}{n106}
\ncline[]{n105}{n108}
\ncline[]{n105}{n109}
\ncline[]{n105}{n111}
\ncline[]{n105}{n113}
\ncline[]{n105}{n115}
\ncline[]{n105}{n121}
\ncline[]{n105}{n124}
\ncline[]{n105}{n125}
\ncline[]{n105}{n127}
\ncline[]{n105}{n130}
\ncline[]{n105}{n133}
\ncline[]{n105}{n134}
\ncline[]{n105}{n135}
\ncline[]{n105}{n136}
\ncline[]{n105}{n145}
\ncline[]{n105}{n149}
\ncline[]{n106}{n108}
\ncline[]{n106}{n109}
\ncline[]{n106}{n111}
\ncline[]{n106}{n113}
\ncline[]{n106}{n115}
\ncline[]{n106}{n121}
\ncline[]{n106}{n124}
\ncline[]{n106}{n125}
\ncline[]{n106}{n127}
\ncline[]{n106}{n130}
\ncline[]{n106}{n133}
\ncline[]{n106}{n134}
\ncline[]{n106}{n135}
\ncline[]{n106}{n136}
\ncline[]{n106}{n145}
\ncline[]{n106}{n149}
\ncline[]{n107}{n116}
\ncline[]{n107}{n118}
\ncline[]{n107}{n129}
\ncline[]{n107}{n131}
\ncline[]{n107}{n132}
\ncline[]{n107}{n138}
\ncline[]{n107}{n140}
\ncline[]{n107}{n141}
\ncline[]{n107}{n147}
\ncline[]{n107}{n148}
\ncline[]{n108}{n109}
\ncline[]{n108}{n111}
\ncline[]{n108}{n113}
\ncline[]{n108}{n115}
\ncline[]{n108}{n121}
\ncline[]{n108}{n124}
\ncline[]{n108}{n125}
\ncline[]{n108}{n127}
\ncline[]{n108}{n130}
\ncline[]{n108}{n133}
\ncline[]{n108}{n134}
\ncline[]{n108}{n135}
\ncline[]{n108}{n136}
\ncline[]{n108}{n145}
\ncline[]{n108}{n149}
\ncline[]{n109}{n111}
\ncline[]{n109}{n113}
\ncline[]{n109}{n115}
\ncline[]{n109}{n121}
\ncline[]{n109}{n124}
\ncline[]{n109}{n125}
\ncline[]{n109}{n127}
\ncline[]{n109}{n130}
\ncline[]{n109}{n133}
\ncline[]{n109}{n134}
\ncline[]{n109}{n135}
\ncline[]{n109}{n136}
\ncline[]{n109}{n145}
\ncline[]{n109}{n149}
\ncline[]{n110}{n112}
\ncline[]{n110}{n114}
\ncline[]{n110}{n117}
\ncline[]{n110}{n119}
\ncline[]{n110}{n120}
\ncline[]{n110}{n122}
\ncline[]{n110}{n123}
\ncline[]{n110}{n126}
\ncline[]{n110}{n128}
\ncline[]{n110}{n137}
\ncline[]{n110}{n139}
\ncline[]{n110}{n142}
\ncline[]{n110}{n143}
\ncline[]{n110}{n144}
\ncline[]{n110}{n146}
\ncline[]{n111}{n113}
\ncline[]{n111}{n115}
\ncline[]{n111}{n121}
\ncline[]{n111}{n124}
\ncline[]{n111}{n125}
\ncline[]{n111}{n127}
\ncline[]{n111}{n130}
\ncline[]{n111}{n133}
\ncline[]{n111}{n134}
\ncline[]{n111}{n135}
\ncline[]{n111}{n136}
\ncline[]{n111}{n145}
\ncline[]{n111}{n149}
\ncline[]{n112}{n114}
\ncline[]{n112}{n117}
\ncline[]{n112}{n119}
\ncline[]{n112}{n120}
\ncline[]{n112}{n122}
\ncline[]{n112}{n123}
\ncline[]{n112}{n126}
\ncline[]{n112}{n128}
\ncline[]{n112}{n137}
\ncline[]{n112}{n139}
\ncline[]{n112}{n142}
\ncline[]{n112}{n143}
\ncline[]{n112}{n144}
\ncline[]{n112}{n146}
\ncline[]{n113}{n115}
\ncline[]{n113}{n121}
\ncline[]{n113}{n124}
\ncline[]{n113}{n125}
\ncline[]{n113}{n127}
\ncline[]{n113}{n130}
\ncline[]{n113}{n133}
\ncline[]{n113}{n134}
\ncline[]{n113}{n135}
\ncline[]{n113}{n136}
\ncline[]{n113}{n145}
\ncline[]{n113}{n149}
\ncline[]{n114}{n117}
\ncline[]{n114}{n119}
\ncline[]{n114}{n120}
\ncline[]{n114}{n122}
\ncline[]{n114}{n123}
\ncline[]{n114}{n126}
\ncline[]{n114}{n128}
\ncline[]{n114}{n137}
\ncline[]{n114}{n139}
\ncline[]{n114}{n142}
\ncline[]{n114}{n143}
\ncline[]{n114}{n144}
\ncline[]{n114}{n146}
\ncline[]{n115}{n121}
\ncline[]{n115}{n124}
\ncline[]{n115}{n125}
\ncline[]{n115}{n127}
\ncline[]{n115}{n130}
\ncline[]{n115}{n133}
\ncline[]{n115}{n134}
\ncline[]{n115}{n135}
\ncline[]{n115}{n136}
\ncline[]{n115}{n145}
\ncline[]{n115}{n149}
\ncline[]{n116}{n118}
\ncline[]{n116}{n129}
\ncline[]{n116}{n131}
\ncline[]{n116}{n132}
\ncline[]{n116}{n138}
\ncline[]{n116}{n140}
\ncline[]{n116}{n141}
\ncline[]{n116}{n147}
\ncline[]{n116}{n148}
\ncline[]{n117}{n119}
\ncline[]{n117}{n120}
\ncline[]{n117}{n122}
\ncline[]{n117}{n123}
\ncline[]{n117}{n126}
\ncline[]{n117}{n128}
\ncline[]{n117}{n137}
\ncline[]{n117}{n139}
\ncline[]{n117}{n142}
\ncline[]{n117}{n143}
\ncline[]{n117}{n144}
\ncline[]{n117}{n146}
\ncline[]{n118}{n129}
\ncline[]{n118}{n131}
\ncline[]{n118}{n132}
\ncline[]{n118}{n138}
\ncline[]{n118}{n140}
\ncline[]{n118}{n141}
\ncline[]{n118}{n147}
\ncline[]{n118}{n148}
\ncline[]{n119}{n120}
\ncline[]{n119}{n122}
\ncline[]{n119}{n123}
\ncline[]{n119}{n126}
\ncline[]{n119}{n128}
\ncline[]{n119}{n137}
\ncline[]{n119}{n139}
\ncline[]{n119}{n142}
\ncline[]{n119}{n143}
\ncline[]{n119}{n144}
\ncline[]{n119}{n146}
\ncline[]{n120}{n122}
\ncline[]{n120}{n123}
\ncline[]{n120}{n126}
\ncline[]{n120}{n128}
\ncline[]{n120}{n137}
\ncline[]{n120}{n139}
\ncline[]{n120}{n142}
\ncline[]{n120}{n143}
\ncline[]{n120}{n144}
\ncline[]{n120}{n146}
\ncline[]{n121}{n124}
\ncline[]{n121}{n125}
\ncline[]{n121}{n127}
\ncline[]{n121}{n130}
\ncline[]{n121}{n133}
\ncline[]{n121}{n134}
\ncline[]{n121}{n135}
\ncline[]{n121}{n136}
\ncline[]{n121}{n145}
\ncline[]{n121}{n149}
\ncline[]{n122}{n123}
\ncline[]{n122}{n126}
\ncline[]{n122}{n128}
\ncline[]{n122}{n137}
\ncline[]{n122}{n139}
\ncline[]{n122}{n142}
\ncline[]{n122}{n143}
\ncline[]{n122}{n144}
\ncline[]{n122}{n146}
\ncline[]{n123}{n126}
\ncline[]{n123}{n128}
\ncline[]{n123}{n137}
\ncline[]{n123}{n139}
\ncline[]{n123}{n142}
\ncline[]{n123}{n143}
\ncline[]{n123}{n144}
\ncline[]{n123}{n146}
\ncline[]{n124}{n125}
\ncline[]{n124}{n127}
\ncline[]{n124}{n130}
\ncline[]{n124}{n133}
\ncline[]{n124}{n134}
\ncline[]{n124}{n135}
\ncline[]{n124}{n136}
\ncline[]{n124}{n145}
\ncline[]{n124}{n149}
\ncline[]{n125}{n127}
\ncline[]{n125}{n130}
\ncline[]{n125}{n133}
\ncline[]{n125}{n134}
\ncline[]{n125}{n135}
\ncline[]{n125}{n136}
\ncline[]{n125}{n145}
\ncline[]{n125}{n149}
\ncline[]{n126}{n128}
\ncline[]{n126}{n137}
\ncline[]{n126}{n139}
\ncline[]{n126}{n142}
\ncline[]{n126}{n143}
\ncline[]{n126}{n144}
\ncline[]{n126}{n146}
\ncline[]{n127}{n130}
\ncline[]{n127}{n133}
\ncline[]{n127}{n134}
\ncline[]{n127}{n135}
\ncline[]{n127}{n136}
\ncline[]{n127}{n145}
\ncline[]{n127}{n149}
\ncline[]{n128}{n137}
\ncline[]{n128}{n139}
\ncline[]{n128}{n142}
\ncline[]{n128}{n143}
\ncline[]{n128}{n144}
\ncline[]{n128}{n146}
\ncline[]{n129}{n131}
\ncline[]{n129}{n132}
\ncline[]{n129}{n138}
\ncline[]{n129}{n140}
\ncline[]{n129}{n141}
\ncline[]{n129}{n147}
\ncline[]{n129}{n148}
\ncline[]{n130}{n133}
\ncline[]{n130}{n134}
\ncline[]{n130}{n135}
\ncline[]{n130}{n136}
\ncline[]{n130}{n145}
\ncline[]{n130}{n149}
\ncline[]{n131}{n132}
\ncline[]{n131}{n138}
\ncline[]{n131}{n140}
\ncline[]{n131}{n141}
\ncline[]{n131}{n147}
\ncline[]{n131}{n148}
\ncline[]{n132}{n138}
\ncline[]{n132}{n140}
\ncline[]{n132}{n141}
\ncline[]{n132}{n147}
\ncline[]{n132}{n148}
\ncline[]{n133}{n134}
\ncline[]{n133}{n135}
\ncline[]{n133}{n136}
\ncline[]{n133}{n145}
\ncline[]{n133}{n149}
\ncline[]{n134}{n135}
\ncline[]{n134}{n136}
\ncline[]{n134}{n145}
\ncline[]{n134}{n149}
\ncline[]{n135}{n136}
\ncline[]{n135}{n145}
\ncline[]{n135}{n149}
\ncline[]{n136}{n145}
\ncline[]{n136}{n149}
\ncline[]{n137}{n139}
\ncline[]{n137}{n142}
\ncline[]{n137}{n143}
\ncline[]{n137}{n144}
\ncline[]{n137}{n146}
\ncline[]{n138}{n140}
\ncline[]{n138}{n141}
\ncline[]{n138}{n147}
\ncline[]{n138}{n148}
\ncline[]{n139}{n142}
\ncline[]{n139}{n143}
\ncline[]{n139}{n144}
\ncline[]{n139}{n146}
\ncline[]{n140}{n141}
\ncline[]{n140}{n147}
\ncline[]{n140}{n148}
\ncline[]{n141}{n147}
\ncline[]{n141}{n148}
\ncline[]{n142}{n143}
\ncline[]{n142}{n144}
\ncline[]{n142}{n146}
\ncline[]{n143}{n144}
\ncline[]{n143}{n146}
\ncline[]{n144}{n146}
\ncline[]{n145}{n149}
\ncline[]{n147}{n148}
\psset{fillcolor=gray, linecolor=black}
\dotnode[dotstyle=Bsquare](-0.511, -0.102){n0}
\dotnode[dotstyle=Btriangle](-1.464, 0.504){n1}
\dotnode[dotstyle=Bsquare](-1.043, 0.229){n2}
\dotnode[dotstyle=Bo](2.840, -0.221){n3}
\dotnode[dotstyle=Bsquare](-0.262, -0.548){n4}
\dotnode[dotstyle=Bo](2.890, -0.137){n5}
\dotnode[dotstyle=Btriangle](-2.350, -0.042){n6}
\dotnode[dotstyle=Bsquare](-1.414, -0.575){n7}
\dotnode[dotstyle=Btriangle](-2.314, 0.183){n8}
\dotnode[dotstyle=Btriangle](-2.320, -0.246){n9}
\dotnode[dotstyle=Bo](2.543, 0.440){n10}
\dotnode[dotstyle=Bo](2.587, 0.520){n11}
\dotnode[dotstyle=Bsquare](-0.331, -0.211){n12}
\dotnode[dotstyle=Bsquare](-1.291, -0.116){n13}
\dotnode[dotstyle=Bo](2.674, -0.107){n14}
\dotnode[dotstyle=Bo](2.469, 0.138){n15}
\dotnode[dotstyle=Bo](2.626, 0.170){n16}
\dotnode[dotstyle=Bsquare](-0.298, -0.347){n17}
\dotnode[dotstyle=Bo](2.704, 0.115){n18}
\dotnode[dotstyle=Btriangle](-2.614, 0.558){n19}
\dotnode[dotstyle=Bsquare](-1.390, -0.283){n20}
\dotnode[dotstyle=Btriangle](-1.764, 0.079){n21}
\dotnode[dotstyle=Bo](2.384, 1.345){n22}
\dotnode[dotstyle=Bsquare](-0.165, -0.680){n23}
\dotnode[dotstyle=Bsquare](-0.519, -1.191){n24}
\dotnode[dotstyle=Bo](2.406, 0.196){n25}
\dotnode[dotstyle=Bsquare](-0.181, -0.826){n26}
\dotnode[dotstyle=Bo](2.746, -0.311){n27}
\dotnode[dotstyle=Btriangle](-2.388, 0.463){n28}
\dotnode[dotstyle=Btriangle](-2.563, 0.276){n29}
\dotnode[dotstyle=Bo](2.623, 0.818){n30}
\dotnode[dotstyle=Bo](2.867, 0.077){n31}
\dotnode[dotstyle=Bsquare](-1.345, -0.776){n32}
\dotnode[dotstyle=Bsquare](-0.229, -0.402){n33}
\dotnode[dotstyle=Bsquare](0.908, -0.752){n34}
\dotnode[dotstyle=Bsquare](-0.043, -0.581){n35}
\dotnode[dotstyle=Bsquare](-0.235, -0.332){n36}
\dotnode[dotstyle=Bo](2.209, 0.443){n37}
\dotnode[dotstyle=Bsquare](-0.813, -0.371){n38}
\dotnode[dotstyle=Bsquare](-1.526, -0.375){n39}
\dotnode[dotstyle=Bsquare](-0.660, -0.352){n40}
\dotnode[dotstyle=Bsquare](-0.890, -0.034){n41}
\dotnode[dotstyle=Btriangle](-2.165, 0.215){n42}
\dotnode[dotstyle=Btriangle](-2.932, 0.352){n43}
\dotnode[dotstyle=Bsquare](-0.356, -0.503){n44}
\dotnode[dotstyle=Bo](2.787, -0.228){n45}
\dotnode[dotstyle=Btriangle](-2.616, 0.342){n46}
\dotnode[dotstyle=Bsquare](0.307, -0.365){n47}
\dotnode[dotstyle=Bo](2.543, 0.586){n48}
\dotnode[dotstyle=Bo](2.199, 0.879){n49}
\dotnode[dotstyle=Bsquare](-0.375, -0.292){n50}
\dotnode[dotstyle=Bsquare](-1.198, -0.606){n51}
\dotnode[dotstyle=Bo](2.982, -0.480){n52}
\dotnode[dotstyle=Btriangle](-2.532, -0.012){n53}
\dotnode[dotstyle=Bo](2.410, 0.418){n54}
\dotnode[dotstyle=Btriangle](-2.145, 0.139){n55}
\dotnode[dotstyle=Btriangle](-2.108, 0.371){n56}
\dotnode[dotstyle=Btriangle](-3.077, 0.686){n57}
\dotnode[dotstyle=Bsquare](-0.921, -0.182){n58}
\dotnode[dotstyle=Bsquare](0.175, -0.252){n59}
\dotnode[dotstyle=Btriangle](-1.780, -0.501){n60}
\dotnode[dotstyle=Btriangle](-1.802, -0.216){n61}
\dotnode[dotstyle=Bo](2.770, 0.271){n62}
\dotnode[dotstyle=Bo](2.303, 0.106){n63}
\dotnode[dotstyle=Bsquare](-0.245, -0.267){n64}
\dotnode[dotstyle=Bsquare](-0.587, -0.483){n65}
\dotnode[dotstyle=Bsquare](-1.585, -0.539){n66}
\dotnode[dotstyle=Bo](2.674, -0.107){n67}
\dotnode[dotstyle=Bo](3.216, 0.142){n68}
\dotnode[dotstyle=Bo](2.640, 0.319){n69}
\dotnode[dotstyle=Btriangle](-3.232, 1.371){n70}
\dotnode[dotstyle=Btriangle](-3.795, 0.253){n71}
\dotnode[dotstyle=Bsquare](-0.357, -0.067){n72}
\dotnode[dotstyle=Bo](2.625, 0.607){n73}
\dotnode[dotstyle=Bo](2.821, -0.082){n74}
\dotnode[dotstyle=Bo](2.633, -0.190){n75}
\dotnode[dotstyle=Bo](2.888, -0.571){n76}
\dotnode[dotstyle=Bsquare](-1.258, -0.179){n77}
\dotnode[dotstyle=Bo](2.716, -0.243){n78}
\dotnode[dotstyle=Bsquare](-0.812, -0.162){n79}
\dotnode[dotstyle=Btriangle](-1.902, 0.116){n80}
\dotnode[dotstyle=Bsquare](0.707, -1.008){n81}
\dotnode[dotstyle=Bsquare](-0.932, 0.319){n82}
\dotnode[dotstyle=Btriangle](-1.971, -0.181){n83}
\dotnode[dotstyle=Btriangle](-1.557, 0.267){n84}
\dotnode[dotstyle=Bsquare](0.511, -1.262){n85}
\dotnode[dotstyle=Bsquare](-1.116, -0.084){n86}
\dotnode[dotstyle=Btriangle](-2.419, 0.304){n87}
\dotnode[dotstyle=Bo](2.280, 0.748){n88}
\dotnode[dotstyle=Bsquare](-1.388, -0.204){n89}
\dotnode[dotstyle=Bo](2.613, 0.022){n90}
\dotnode[dotstyle=Bo](2.998, -0.334){n91}
\dotnode[dotstyle=Btriangle](-1.905, 0.119){n92}
\dotnode[dotstyle=Bsquare](-0.942, -0.542){n93}
\dotnode[dotstyle=Bsquare](-1.298, -0.761){n94}
\dotnode[dotstyle=Btriangle](-2.841, 0.373){n95}
\dotnode[dotstyle=Bsquare](0.751, -1.001){n96}
\dotnode[dotstyle=Btriangle](-1.285, 0.685){n97}
\dotnode[dotstyle=Bsquare](0.192, -0.677){n98}
\dotnode[dotstyle=Bsquare](-1.095, 0.284){n99}
\dotnode[dotstyle=Bsquare](-1.331, 0.245){n100}
\dotnode[dotstyle=Bsquare](-0.984, -0.124){n101}
\dotnode[dotstyle=Btriangle](-1.662, 0.242){n102}
\dotnode[dotstyle=Btriangle](-2.276, 0.333){n103}
\dotnode[dotstyle=Bsquare](-0.928, 0.468){n104}
\dotnode[dotstyle=Bo](2.356, -0.031){n105}
\dotnode[dotstyle=Bo](2.715, -0.170){n106}
\dotnode[dotstyle=Btriangle](-2.428, 0.377){n107}
\dotnode[dotstyle=Bo](2.852, -0.933){n108}
\dotnode[dotstyle=Bo](3.225, -0.503){n109}
\dotnode[dotstyle=Bsquare](0.010, -0.721){n110}
\dotnode[dotstyle=Bo](2.729, 0.334){n111}
\dotnode[dotstyle=Bsquare](-0.714, 0.150){n112}
\dotnode[dotstyle=Bo](2.562, 0.375){n113}
\dotnode[dotstyle=Bsquare](-0.135, -0.312){n114}
\dotnode[dotstyle=Bo](2.597, 1.100){n115}
\dotnode[dotstyle=Btriangle](-3.397, 0.547){n116}
\dotnode[dotstyle=Bsquare](-1.443, -0.144){n117}
\dotnode[dotstyle=Btriangle](-1.905, 0.048){n118}
\dotnode[dotstyle=Bsquare](-1.296, -0.328){n119}
\dotnode[dotstyle=Bsquare](-1.414, -0.575){n120}
\dotnode[dotstyle=Bo](2.508, -0.139){n121}
\dotnode[dotstyle=Bsquare](-1.220, 0.408){n122}
\dotnode[dotstyle=Bsquare](-1.379, -0.421){n123}
\dotnode[dotstyle=Bo](2.684, 0.327){n124}
\dotnode[dotstyle=Bo](2.588, -0.197){n125}
\dotnode[dotstyle=Bsquare](-0.640, -0.417){n126}
\dotnode[dotstyle=Bo](2.538, 0.510){n127}
\dotnode[dotstyle=Bsquare](-0.900, 0.330){n128}
\dotnode[dotstyle=Btriangle](-2.123, -0.211){n129}
\dotnode[dotstyle=Bo](2.648, 0.820){n130}
\dotnode[dotstyle=Btriangle](-2.159, -0.218){n131}
\dotnode[dotstyle=Btriangle](-3.499, 0.457){n132}
\dotnode[dotstyle=Bo](2.644, 1.186){n133}
\dotnode[dotstyle=Bo](2.674, -0.107){n134}
\dotnode[dotstyle=Bo](2.507, 0.652){n135}
\dotnode[dotstyle=Bo](2.648, 0.319){n136}
\dotnode[dotstyle=Bsquare](-0.807, 0.195){n137}
\dotnode[dotstyle=Btriangle](-1.949, 0.041){n138}
\dotnode[dotstyle=Bsquare](0.070, -0.703){n139}
\dotnode[dotstyle=Btriangle](-2.918, 0.780){n140}
\dotnode[dotstyle=Btriangle](-1.922, 0.409){n141}
\dotnode[dotstyle=Bsquare](-0.642, 0.019){n142}
\dotnode[dotstyle=Bsquare](-1.087, 0.075){n143}
\dotnode[dotstyle=Bsquare](-1.169, -0.165){n144}
\dotnode[dotstyle=Bo](2.311, 0.398){n145}
\dotnode[dotstyle=Bsquare](-0.463, -0.670){n146}
\dotnode[dotstyle=Btriangle](-1.944, 0.187){n147}
\dotnode[dotstyle=Btriangle](-3.489, 1.172){n148}
\dotnode[dotstyle=Bo](2.590, 0.236){n149}

        \endpsgraph
		}}
	\hspace{0.4in}
	\subfloat[Bad]{
	\scalebox{0.55}{
        \psgraph[tickstyle=bottom,Dx=1,Ox=-4,Dy=0.5,Oy=-1.5]{->}%
        (-4,-1.5)(3.5,1.5){3.5in}{2in}
        \pnode(-0.511, -0.102){n0}
\pnode(-1.464, 0.504){n1}
\pnode(-1.043, 0.229){n2}
\pnode(2.840, -0.221){n3}
\pnode(-0.262, -0.548){n4}
\pnode(2.890, -0.137){n5}
\pnode(-2.350, -0.042){n6}
\pnode(-1.414, -0.575){n7}
\pnode(-2.314, 0.183){n8}
\pnode(-2.320, -0.246){n9}
\pnode(2.543, 0.440){n10}
\pnode(2.587, 0.520){n11}
\pnode(-0.331, -0.211){n12}
\pnode(-1.291, -0.116){n13}
\pnode(2.674, -0.107){n14}
\pnode(2.469, 0.138){n15}
\pnode(2.626, 0.170){n16}
\pnode(-0.298, -0.347){n17}
\pnode(2.704, 0.115){n18}
\pnode(-2.614, 0.558){n19}
\pnode(-1.390, -0.283){n20}
\pnode(-1.764, 0.079){n21}
\pnode(2.384, 1.345){n22}
\pnode(-0.165, -0.680){n23}
\pnode(-0.519, -1.191){n24}
\pnode(2.406, 0.196){n25}
\pnode(-0.181, -0.826){n26}
\pnode(2.746, -0.311){n27}
\pnode(-2.388, 0.463){n28}
\pnode(-2.563, 0.276){n29}
\pnode(2.623, 0.818){n30}
\pnode(2.867, 0.077){n31}
\pnode(-1.345, -0.776){n32}
\pnode(-0.229, -0.402){n33}
\pnode(0.908, -0.752){n34}
\pnode(-0.043, -0.581){n35}
\pnode(-0.235, -0.332){n36}
\pnode(2.209, 0.443){n37}
\pnode(-0.813, -0.371){n38}
\pnode(-1.526, -0.375){n39}
\pnode(-0.660, -0.352){n40}
\pnode(-0.890, -0.034){n41}
\pnode(-2.165, 0.215){n42}
\pnode(-2.932, 0.352){n43}
\pnode(-0.356, -0.503){n44}
\pnode(2.787, -0.228){n45}
\pnode(-2.616, 0.342){n46}
\pnode(0.307, -0.365){n47}
\pnode(2.543, 0.586){n48}
\pnode(2.199, 0.879){n49}
\pnode(-0.375, -0.292){n50}
\pnode(-1.198, -0.606){n51}
\pnode(2.982, -0.480){n52}
\pnode(-2.532, -0.012){n53}
\pnode(2.410, 0.418){n54}
\pnode(-2.145, 0.139){n55}
\pnode(-2.108, 0.371){n56}
\pnode(-3.077, 0.686){n57}
\pnode(-0.921, -0.182){n58}
\pnode(0.175, -0.252){n59}
\pnode(-1.780, -0.501){n60}
\pnode(-1.802, -0.216){n61}
\pnode(2.770, 0.271){n62}
\pnode(2.303, 0.106){n63}
\pnode(-0.245, -0.267){n64}
\pnode(-0.587, -0.483){n65}
\pnode(-1.585, -0.539){n66}
\pnode(2.674, -0.107){n67}
\pnode(3.216, 0.142){n68}
\pnode(2.640, 0.319){n69}
\pnode(-3.232, 1.371){n70}
\pnode(-3.795, 0.253){n71}
\pnode(-0.357, -0.067){n72}
\pnode(2.625, 0.607){n73}
\pnode(2.821, -0.082){n74}
\pnode(2.633, -0.190){n75}
\pnode(2.888, -0.571){n76}
\pnode(-1.258, -0.179){n77}
\pnode(2.716, -0.243){n78}
\pnode(-0.812, -0.162){n79}
\pnode(-1.902, 0.116){n80}
\pnode(0.707, -1.008){n81}
\pnode(-0.932, 0.319){n82}
\pnode(-1.971, -0.181){n83}
\pnode(-1.557, 0.267){n84}
\pnode(0.511, -1.262){n85}
\pnode(-1.116, -0.084){n86}
\pnode(-2.419, 0.304){n87}
\pnode(2.280, 0.748){n88}
\pnode(-1.388, -0.204){n89}
\pnode(2.613, 0.022){n90}
\pnode(2.998, -0.334){n91}
\pnode(-1.905, 0.119){n92}
\pnode(-0.942, -0.542){n93}
\pnode(-1.298, -0.761){n94}
\pnode(-2.841, 0.373){n95}
\pnode(0.751, -1.001){n96}
\pnode(-1.285, 0.685){n97}
\pnode(0.192, -0.677){n98}
\pnode(-1.095, 0.284){n99}
\pnode(-1.331, 0.245){n100}
\pnode(-0.984, -0.124){n101}
\pnode(-1.662, 0.242){n102}
\pnode(-2.276, 0.333){n103}
\pnode(-0.928, 0.468){n104}
\pnode(2.356, -0.031){n105}
\pnode(2.715, -0.170){n106}
\pnode(-2.428, 0.377){n107}
\pnode(2.852, -0.933){n108}
\pnode(3.225, -0.503){n109}
\pnode(0.010, -0.721){n110}
\pnode(2.729, 0.334){n111}
\pnode(-0.714, 0.150){n112}
\pnode(2.562, 0.375){n113}
\pnode(-0.135, -0.312){n114}
\pnode(2.597, 1.100){n115}
\pnode(-3.397, 0.547){n116}
\pnode(-1.443, -0.144){n117}
\pnode(-1.905, 0.048){n118}
\pnode(-1.296, -0.328){n119}
\pnode(-1.414, -0.575){n120}
\pnode(2.508, -0.139){n121}
\pnode(-1.220, 0.408){n122}
\pnode(-1.379, -0.421){n123}
\pnode(2.684, 0.327){n124}
\pnode(2.588, -0.197){n125}
\pnode(-0.640, -0.417){n126}
\pnode(2.538, 0.510){n127}
\pnode(-0.900, 0.330){n128}
\pnode(-2.123, -0.211){n129}
\pnode(2.648, 0.820){n130}
\pnode(-2.159, -0.218){n131}
\pnode(-3.499, 0.457){n132}
\pnode(2.644, 1.186){n133}
\pnode(2.674, -0.107){n134}
\pnode(2.507, 0.652){n135}
\pnode(2.648, 0.319){n136}
\pnode(-0.807, 0.195){n137}
\pnode(-1.949, 0.041){n138}
\pnode(0.070, -0.703){n139}
\pnode(-2.918, 0.780){n140}
\pnode(-1.922, 0.409){n141}
\pnode(-0.642, 0.019){n142}
\pnode(-1.087, 0.075){n143}
\pnode(-1.169, -0.165){n144}
\pnode(2.311, 0.398){n145}
\pnode(-0.463, -0.670){n146}
\pnode(-1.944, 0.187){n147}
\pnode(-3.489, 1.172){n148}
\pnode(2.590, 0.236){n149}
\psset{linewidth=0.01pt,linecolor=lightgray}
\ncline[]{n0}{n1}
\ncline[]{n0}{n2}
\ncline[]{n0}{n4}
\ncline[]{n0}{n6}
\ncline[]{n0}{n7}
\ncline[]{n0}{n8}
\ncline[]{n0}{n9}
\ncline[]{n0}{n12}
\ncline[]{n0}{n13}
\ncline[]{n0}{n17}
\ncline[]{n0}{n19}
\ncline[]{n0}{n20}
\ncline[]{n0}{n21}
\ncline[]{n0}{n23}
\ncline[]{n0}{n24}
\ncline[]{n0}{n26}
\ncline[]{n0}{n28}
\ncline[]{n0}{n29}
\ncline[]{n0}{n32}
\ncline[]{n0}{n33}
\ncline[]{n0}{n35}
\ncline[]{n0}{n36}
\ncline[]{n0}{n38}
\ncline[]{n0}{n39}
\ncline[]{n0}{n40}
\ncline[]{n0}{n41}
\ncline[]{n0}{n42}
\ncline[]{n0}{n43}
\ncline[]{n0}{n44}
\ncline[]{n0}{n46}
\ncline[]{n0}{n47}
\ncline[]{n0}{n50}
\ncline[]{n0}{n51}
\ncline[]{n0}{n53}
\ncline[]{n0}{n55}
\ncline[]{n0}{n56}
\ncline[]{n0}{n57}
\ncline[]{n0}{n58}
\ncline[]{n0}{n59}
\ncline[]{n0}{n60}
\ncline[]{n0}{n61}
\ncline[]{n0}{n64}
\ncline[]{n0}{n65}
\ncline[]{n0}{n66}
\ncline[]{n0}{n70}
\ncline[]{n0}{n71}
\ncline[]{n0}{n72}
\ncline[]{n0}{n77}
\ncline[]{n0}{n79}
\ncline[]{n0}{n80}
\ncline[]{n0}{n82}
\ncline[]{n0}{n83}
\ncline[]{n0}{n84}
\ncline[]{n0}{n86}
\ncline[]{n0}{n87}
\ncline[]{n0}{n89}
\ncline[]{n0}{n92}
\ncline[]{n0}{n93}
\ncline[]{n0}{n94}
\ncline[]{n0}{n95}
\ncline[]{n0}{n97}
\ncline[]{n0}{n98}
\ncline[]{n0}{n99}
\ncline[]{n0}{n100}
\ncline[]{n0}{n101}
\ncline[]{n0}{n102}
\ncline[]{n0}{n103}
\ncline[]{n0}{n104}
\ncline[]{n0}{n107}
\ncline[]{n0}{n110}
\ncline[]{n0}{n112}
\ncline[]{n0}{n114}
\ncline[]{n0}{n116}
\ncline[]{n0}{n117}
\ncline[]{n0}{n118}
\ncline[]{n0}{n119}
\ncline[]{n0}{n120}
\ncline[]{n0}{n122}
\ncline[]{n0}{n123}
\ncline[]{n0}{n126}
\ncline[]{n0}{n128}
\ncline[]{n0}{n129}
\ncline[]{n0}{n131}
\ncline[]{n0}{n132}
\ncline[]{n0}{n137}
\ncline[]{n0}{n138}
\ncline[]{n0}{n139}
\ncline[]{n0}{n140}
\ncline[]{n0}{n141}
\ncline[]{n0}{n142}
\ncline[]{n0}{n143}
\ncline[]{n0}{n144}
\ncline[]{n0}{n146}
\ncline[]{n0}{n147}
\ncline[]{n0}{n148}
\ncline[]{n1}{n2}
\ncline[]{n1}{n4}
\ncline[]{n1}{n6}
\ncline[]{n1}{n7}
\ncline[]{n1}{n8}
\ncline[]{n1}{n9}
\ncline[]{n1}{n12}
\ncline[]{n1}{n13}
\ncline[]{n1}{n17}
\ncline[]{n1}{n19}
\ncline[]{n1}{n20}
\ncline[]{n1}{n21}
\ncline[]{n1}{n23}
\ncline[]{n1}{n24}
\ncline[]{n1}{n26}
\ncline[]{n1}{n28}
\ncline[]{n1}{n29}
\ncline[]{n1}{n32}
\ncline[]{n1}{n33}
\ncline[]{n1}{n35}
\ncline[]{n1}{n36}
\ncline[]{n1}{n38}
\ncline[]{n1}{n39}
\ncline[]{n1}{n40}
\ncline[]{n1}{n41}
\ncline[]{n1}{n42}
\ncline[]{n1}{n43}
\ncline[]{n1}{n44}
\ncline[]{n1}{n46}
\ncline[]{n1}{n47}
\ncline[]{n1}{n50}
\ncline[]{n1}{n51}
\ncline[]{n1}{n53}
\ncline[]{n1}{n55}
\ncline[]{n1}{n56}
\ncline[]{n1}{n57}
\ncline[]{n1}{n58}
\ncline[]{n1}{n59}
\ncline[]{n1}{n60}
\ncline[]{n1}{n61}
\ncline[]{n1}{n64}
\ncline[]{n1}{n65}
\ncline[]{n1}{n66}
\ncline[]{n1}{n70}
\ncline[]{n1}{n71}
\ncline[]{n1}{n72}
\ncline[]{n1}{n77}
\ncline[]{n1}{n79}
\ncline[]{n1}{n80}
\ncline[]{n1}{n82}
\ncline[]{n1}{n83}
\ncline[]{n1}{n84}
\ncline[]{n1}{n86}
\ncline[]{n1}{n87}
\ncline[]{n1}{n89}
\ncline[]{n1}{n92}
\ncline[]{n1}{n93}
\ncline[]{n1}{n94}
\ncline[]{n1}{n95}
\ncline[]{n1}{n97}
\ncline[]{n1}{n98}
\ncline[]{n1}{n99}
\ncline[]{n1}{n100}
\ncline[]{n1}{n101}
\ncline[]{n1}{n102}
\ncline[]{n1}{n103}
\ncline[]{n1}{n104}
\ncline[]{n1}{n107}
\ncline[]{n1}{n110}
\ncline[]{n1}{n112}
\ncline[]{n1}{n114}
\ncline[]{n1}{n116}
\ncline[]{n1}{n117}
\ncline[]{n1}{n118}
\ncline[]{n1}{n119}
\ncline[]{n1}{n120}
\ncline[]{n1}{n122}
\ncline[]{n1}{n123}
\ncline[]{n1}{n126}
\ncline[]{n1}{n128}
\ncline[]{n1}{n129}
\ncline[]{n1}{n131}
\ncline[]{n1}{n132}
\ncline[]{n1}{n137}
\ncline[]{n1}{n138}
\ncline[]{n1}{n139}
\ncline[]{n1}{n140}
\ncline[]{n1}{n141}
\ncline[]{n1}{n142}
\ncline[]{n1}{n143}
\ncline[]{n1}{n144}
\ncline[]{n1}{n146}
\ncline[]{n1}{n147}
\ncline[]{n1}{n148}
\ncline[]{n2}{n4}
\ncline[]{n2}{n6}
\ncline[]{n2}{n7}
\ncline[]{n2}{n8}
\ncline[]{n2}{n9}
\ncline[]{n2}{n12}
\ncline[]{n2}{n13}
\ncline[]{n2}{n17}
\ncline[]{n2}{n19}
\ncline[]{n2}{n20}
\ncline[]{n2}{n21}
\ncline[]{n2}{n23}
\ncline[]{n2}{n24}
\ncline[]{n2}{n26}
\ncline[]{n2}{n28}
\ncline[]{n2}{n29}
\ncline[]{n2}{n32}
\ncline[]{n2}{n33}
\ncline[]{n2}{n35}
\ncline[]{n2}{n36}
\ncline[]{n2}{n38}
\ncline[]{n2}{n39}
\ncline[]{n2}{n40}
\ncline[]{n2}{n41}
\ncline[]{n2}{n42}
\ncline[]{n2}{n43}
\ncline[]{n2}{n44}
\ncline[]{n2}{n46}
\ncline[]{n2}{n47}
\ncline[]{n2}{n50}
\ncline[]{n2}{n51}
\ncline[]{n2}{n53}
\ncline[]{n2}{n55}
\ncline[]{n2}{n56}
\ncline[]{n2}{n57}
\ncline[]{n2}{n58}
\ncline[]{n2}{n59}
\ncline[]{n2}{n60}
\ncline[]{n2}{n61}
\ncline[]{n2}{n64}
\ncline[]{n2}{n65}
\ncline[]{n2}{n66}
\ncline[]{n2}{n70}
\ncline[]{n2}{n71}
\ncline[]{n2}{n72}
\ncline[]{n2}{n77}
\ncline[]{n2}{n79}
\ncline[]{n2}{n80}
\ncline[]{n2}{n82}
\ncline[]{n2}{n83}
\ncline[]{n2}{n84}
\ncline[]{n2}{n86}
\ncline[]{n2}{n87}
\ncline[]{n2}{n89}
\ncline[]{n2}{n92}
\ncline[]{n2}{n93}
\ncline[]{n2}{n94}
\ncline[]{n2}{n95}
\ncline[]{n2}{n97}
\ncline[]{n2}{n98}
\ncline[]{n2}{n99}
\ncline[]{n2}{n100}
\ncline[]{n2}{n101}
\ncline[]{n2}{n102}
\ncline[]{n2}{n103}
\ncline[]{n2}{n104}
\ncline[]{n2}{n107}
\ncline[]{n2}{n110}
\ncline[]{n2}{n112}
\ncline[]{n2}{n114}
\ncline[]{n2}{n116}
\ncline[]{n2}{n117}
\ncline[]{n2}{n118}
\ncline[]{n2}{n119}
\ncline[]{n2}{n120}
\ncline[]{n2}{n122}
\ncline[]{n2}{n123}
\ncline[]{n2}{n126}
\ncline[]{n2}{n128}
\ncline[]{n2}{n129}
\ncline[]{n2}{n131}
\ncline[]{n2}{n132}
\ncline[]{n2}{n137}
\ncline[]{n2}{n138}
\ncline[]{n2}{n139}
\ncline[]{n2}{n140}
\ncline[]{n2}{n141}
\ncline[]{n2}{n142}
\ncline[]{n2}{n143}
\ncline[]{n2}{n144}
\ncline[]{n2}{n146}
\ncline[]{n2}{n147}
\ncline[]{n2}{n148}
\ncline[]{n3}{n5}
\ncline[]{n3}{n14}
\ncline[]{n3}{n27}
\ncline[]{n3}{n34}
\ncline[]{n3}{n45}
\ncline[]{n3}{n52}
\ncline[]{n3}{n67}
\ncline[]{n3}{n74}
\ncline[]{n3}{n75}
\ncline[]{n3}{n76}
\ncline[]{n3}{n78}
\ncline[]{n3}{n81}
\ncline[]{n3}{n85}
\ncline[]{n3}{n90}
\ncline[]{n3}{n91}
\ncline[]{n3}{n96}
\ncline[]{n3}{n105}
\ncline[]{n3}{n106}
\ncline[]{n3}{n108}
\ncline[]{n3}{n109}
\ncline[]{n3}{n121}
\ncline[]{n3}{n125}
\ncline[]{n3}{n134}
\ncline[]{n4}{n6}
\ncline[]{n4}{n7}
\ncline[]{n4}{n8}
\ncline[]{n4}{n9}
\ncline[]{n4}{n12}
\ncline[]{n4}{n13}
\ncline[]{n4}{n17}
\ncline[]{n4}{n19}
\ncline[]{n4}{n20}
\ncline[]{n4}{n21}
\ncline[]{n4}{n23}
\ncline[]{n4}{n24}
\ncline[]{n4}{n26}
\ncline[]{n4}{n28}
\ncline[]{n4}{n29}
\ncline[]{n4}{n32}
\ncline[]{n4}{n33}
\ncline[]{n4}{n35}
\ncline[]{n4}{n36}
\ncline[]{n4}{n38}
\ncline[]{n4}{n39}
\ncline[]{n4}{n40}
\ncline[]{n4}{n41}
\ncline[]{n4}{n42}
\ncline[]{n4}{n43}
\ncline[]{n4}{n44}
\ncline[]{n4}{n46}
\ncline[]{n4}{n47}
\ncline[]{n4}{n50}
\ncline[]{n4}{n51}
\ncline[]{n4}{n53}
\ncline[]{n4}{n55}
\ncline[]{n4}{n56}
\ncline[]{n4}{n57}
\ncline[]{n4}{n58}
\ncline[]{n4}{n59}
\ncline[]{n4}{n60}
\ncline[]{n4}{n61}
\ncline[]{n4}{n64}
\ncline[]{n4}{n65}
\ncline[]{n4}{n66}
\ncline[]{n4}{n70}
\ncline[]{n4}{n71}
\ncline[]{n4}{n72}
\ncline[]{n4}{n77}
\ncline[]{n4}{n79}
\ncline[]{n4}{n80}
\ncline[]{n4}{n82}
\ncline[]{n4}{n83}
\ncline[]{n4}{n84}
\ncline[]{n4}{n86}
\ncline[]{n4}{n87}
\ncline[]{n4}{n89}
\ncline[]{n4}{n92}
\ncline[]{n4}{n93}
\ncline[]{n4}{n94}
\ncline[]{n4}{n95}
\ncline[]{n4}{n97}
\ncline[]{n4}{n98}
\ncline[]{n4}{n99}
\ncline[]{n4}{n100}
\ncline[]{n4}{n101}
\ncline[]{n4}{n102}
\ncline[]{n4}{n103}
\ncline[]{n4}{n104}
\ncline[]{n4}{n107}
\ncline[]{n4}{n110}
\ncline[]{n4}{n112}
\ncline[]{n4}{n114}
\ncline[]{n4}{n116}
\ncline[]{n4}{n117}
\ncline[]{n4}{n118}
\ncline[]{n4}{n119}
\ncline[]{n4}{n120}
\ncline[]{n4}{n122}
\ncline[]{n4}{n123}
\ncline[]{n4}{n126}
\ncline[]{n4}{n128}
\ncline[]{n4}{n129}
\ncline[]{n4}{n131}
\ncline[]{n4}{n132}
\ncline[]{n4}{n137}
\ncline[]{n4}{n138}
\ncline[]{n4}{n139}
\ncline[]{n4}{n140}
\ncline[]{n4}{n141}
\ncline[]{n4}{n142}
\ncline[]{n4}{n143}
\ncline[]{n4}{n144}
\ncline[]{n4}{n146}
\ncline[]{n4}{n147}
\ncline[]{n4}{n148}
\ncline[]{n5}{n14}
\ncline[]{n5}{n27}
\ncline[]{n5}{n34}
\ncline[]{n5}{n45}
\ncline[]{n5}{n52}
\ncline[]{n5}{n67}
\ncline[]{n5}{n74}
\ncline[]{n5}{n75}
\ncline[]{n5}{n76}
\ncline[]{n5}{n78}
\ncline[]{n5}{n81}
\ncline[]{n5}{n85}
\ncline[]{n5}{n90}
\ncline[]{n5}{n91}
\ncline[]{n5}{n96}
\ncline[]{n5}{n105}
\ncline[]{n5}{n106}
\ncline[]{n5}{n108}
\ncline[]{n5}{n109}
\ncline[]{n5}{n121}
\ncline[]{n5}{n125}
\ncline[]{n5}{n134}
\ncline[]{n6}{n7}
\ncline[]{n6}{n8}
\ncline[]{n6}{n9}
\ncline[]{n6}{n12}
\ncline[]{n6}{n13}
\ncline[]{n6}{n17}
\ncline[]{n6}{n19}
\ncline[]{n6}{n20}
\ncline[]{n6}{n21}
\ncline[]{n6}{n23}
\ncline[]{n6}{n24}
\ncline[]{n6}{n26}
\ncline[]{n6}{n28}
\ncline[]{n6}{n29}
\ncline[]{n6}{n32}
\ncline[]{n6}{n33}
\ncline[]{n6}{n35}
\ncline[]{n6}{n36}
\ncline[]{n6}{n38}
\ncline[]{n6}{n39}
\ncline[]{n6}{n40}
\ncline[]{n6}{n41}
\ncline[]{n6}{n42}
\ncline[]{n6}{n43}
\ncline[]{n6}{n44}
\ncline[]{n6}{n46}
\ncline[]{n6}{n47}
\ncline[]{n6}{n50}
\ncline[]{n6}{n51}
\ncline[]{n6}{n53}
\ncline[]{n6}{n55}
\ncline[]{n6}{n56}
\ncline[]{n6}{n57}
\ncline[]{n6}{n58}
\ncline[]{n6}{n59}
\ncline[]{n6}{n60}
\ncline[]{n6}{n61}
\ncline[]{n6}{n64}
\ncline[]{n6}{n65}
\ncline[]{n6}{n66}
\ncline[]{n6}{n70}
\ncline[]{n6}{n71}
\ncline[]{n6}{n72}
\ncline[]{n6}{n77}
\ncline[]{n6}{n79}
\ncline[]{n6}{n80}
\ncline[]{n6}{n82}
\ncline[]{n6}{n83}
\ncline[]{n6}{n84}
\ncline[]{n6}{n86}
\ncline[]{n6}{n87}
\ncline[]{n6}{n89}
\ncline[]{n6}{n92}
\ncline[]{n6}{n93}
\ncline[]{n6}{n94}
\ncline[]{n6}{n95}
\ncline[]{n6}{n97}
\ncline[]{n6}{n98}
\ncline[]{n6}{n99}
\ncline[]{n6}{n100}
\ncline[]{n6}{n101}
\ncline[]{n6}{n102}
\ncline[]{n6}{n103}
\ncline[]{n6}{n104}
\ncline[]{n6}{n107}
\ncline[]{n6}{n110}
\ncline[]{n6}{n112}
\ncline[]{n6}{n114}
\ncline[]{n6}{n116}
\ncline[]{n6}{n117}
\ncline[]{n6}{n118}
\ncline[]{n6}{n119}
\ncline[]{n6}{n120}
\ncline[]{n6}{n122}
\ncline[]{n6}{n123}
\ncline[]{n6}{n126}
\ncline[]{n6}{n128}
\ncline[]{n6}{n129}
\ncline[]{n6}{n131}
\ncline[]{n6}{n132}
\ncline[]{n6}{n137}
\ncline[]{n6}{n138}
\ncline[]{n6}{n139}
\ncline[]{n6}{n140}
\ncline[]{n6}{n141}
\ncline[]{n6}{n142}
\ncline[]{n6}{n143}
\ncline[]{n6}{n144}
\ncline[]{n6}{n146}
\ncline[]{n6}{n147}
\ncline[]{n6}{n148}
\ncline[]{n7}{n8}
\ncline[]{n7}{n9}
\ncline[]{n7}{n12}
\ncline[]{n7}{n13}
\ncline[]{n7}{n17}
\ncline[]{n7}{n19}
\ncline[]{n7}{n20}
\ncline[]{n7}{n21}
\ncline[]{n7}{n23}
\ncline[]{n7}{n24}
\ncline[]{n7}{n26}
\ncline[]{n7}{n28}
\ncline[]{n7}{n29}
\ncline[]{n7}{n32}
\ncline[]{n7}{n33}
\ncline[]{n7}{n35}
\ncline[]{n7}{n36}
\ncline[]{n7}{n38}
\ncline[]{n7}{n39}
\ncline[]{n7}{n40}
\ncline[]{n7}{n41}
\ncline[]{n7}{n42}
\ncline[]{n7}{n43}
\ncline[]{n7}{n44}
\ncline[]{n7}{n46}
\ncline[]{n7}{n47}
\ncline[]{n7}{n50}
\ncline[]{n7}{n51}
\ncline[]{n7}{n53}
\ncline[]{n7}{n55}
\ncline[]{n7}{n56}
\ncline[]{n7}{n57}
\ncline[]{n7}{n58}
\ncline[]{n7}{n59}
\ncline[]{n7}{n60}
\ncline[]{n7}{n61}
\ncline[]{n7}{n64}
\ncline[]{n7}{n65}
\ncline[]{n7}{n66}
\ncline[]{n7}{n70}
\ncline[]{n7}{n71}
\ncline[]{n7}{n72}
\ncline[]{n7}{n77}
\ncline[]{n7}{n79}
\ncline[]{n7}{n80}
\ncline[]{n7}{n82}
\ncline[]{n7}{n83}
\ncline[]{n7}{n84}
\ncline[]{n7}{n86}
\ncline[]{n7}{n87}
\ncline[]{n7}{n89}
\ncline[]{n7}{n92}
\ncline[]{n7}{n93}
\ncline[]{n7}{n94}
\ncline[]{n7}{n95}
\ncline[]{n7}{n97}
\ncline[]{n7}{n98}
\ncline[]{n7}{n99}
\ncline[]{n7}{n100}
\ncline[]{n7}{n101}
\ncline[]{n7}{n102}
\ncline[]{n7}{n103}
\ncline[]{n7}{n104}
\ncline[]{n7}{n107}
\ncline[]{n7}{n110}
\ncline[]{n7}{n112}
\ncline[]{n7}{n114}
\ncline[]{n7}{n116}
\ncline[]{n7}{n117}
\ncline[]{n7}{n118}
\ncline[]{n7}{n119}
\ncline[]{n7}{n120}
\ncline[]{n7}{n122}
\ncline[]{n7}{n123}
\ncline[]{n7}{n126}
\ncline[]{n7}{n128}
\ncline[]{n7}{n129}
\ncline[]{n7}{n131}
\ncline[]{n7}{n132}
\ncline[]{n7}{n137}
\ncline[]{n7}{n138}
\ncline[]{n7}{n139}
\ncline[]{n7}{n140}
\ncline[]{n7}{n141}
\ncline[]{n7}{n142}
\ncline[]{n7}{n143}
\ncline[]{n7}{n144}
\ncline[]{n7}{n146}
\ncline[]{n7}{n147}
\ncline[]{n7}{n148}
\ncline[]{n8}{n9}
\ncline[]{n8}{n12}
\ncline[]{n8}{n13}
\ncline[]{n8}{n17}
\ncline[]{n8}{n19}
\ncline[]{n8}{n20}
\ncline[]{n8}{n21}
\ncline[]{n8}{n23}
\ncline[]{n8}{n24}
\ncline[]{n8}{n26}
\ncline[]{n8}{n28}
\ncline[]{n8}{n29}
\ncline[]{n8}{n32}
\ncline[]{n8}{n33}
\ncline[]{n8}{n35}
\ncline[]{n8}{n36}
\ncline[]{n8}{n38}
\ncline[]{n8}{n39}
\ncline[]{n8}{n40}
\ncline[]{n8}{n41}
\ncline[]{n8}{n42}
\ncline[]{n8}{n43}
\ncline[]{n8}{n44}
\ncline[]{n8}{n46}
\ncline[]{n8}{n47}
\ncline[]{n8}{n50}
\ncline[]{n8}{n51}
\ncline[]{n8}{n53}
\ncline[]{n8}{n55}
\ncline[]{n8}{n56}
\ncline[]{n8}{n57}
\ncline[]{n8}{n58}
\ncline[]{n8}{n59}
\ncline[]{n8}{n60}
\ncline[]{n8}{n61}
\ncline[]{n8}{n64}
\ncline[]{n8}{n65}
\ncline[]{n8}{n66}
\ncline[]{n8}{n70}
\ncline[]{n8}{n71}
\ncline[]{n8}{n72}
\ncline[]{n8}{n77}
\ncline[]{n8}{n79}
\ncline[]{n8}{n80}
\ncline[]{n8}{n82}
\ncline[]{n8}{n83}
\ncline[]{n8}{n84}
\ncline[]{n8}{n86}
\ncline[]{n8}{n87}
\ncline[]{n8}{n89}
\ncline[]{n8}{n92}
\ncline[]{n8}{n93}
\ncline[]{n8}{n94}
\ncline[]{n8}{n95}
\ncline[]{n8}{n97}
\ncline[]{n8}{n98}
\ncline[]{n8}{n99}
\ncline[]{n8}{n100}
\ncline[]{n8}{n101}
\ncline[]{n8}{n102}
\ncline[]{n8}{n103}
\ncline[]{n8}{n104}
\ncline[]{n8}{n107}
\ncline[]{n8}{n110}
\ncline[]{n8}{n112}
\ncline[]{n8}{n114}
\ncline[]{n8}{n116}
\ncline[]{n8}{n117}
\ncline[]{n8}{n118}
\ncline[]{n8}{n119}
\ncline[]{n8}{n120}
\ncline[]{n8}{n122}
\ncline[]{n8}{n123}
\ncline[]{n8}{n126}
\ncline[]{n8}{n128}
\ncline[]{n8}{n129}
\ncline[]{n8}{n131}
\ncline[]{n8}{n132}
\ncline[]{n8}{n137}
\ncline[]{n8}{n138}
\ncline[]{n8}{n139}
\ncline[]{n8}{n140}
\ncline[]{n8}{n141}
\ncline[]{n8}{n142}
\ncline[]{n8}{n143}
\ncline[]{n8}{n144}
\ncline[]{n8}{n146}
\ncline[]{n8}{n147}
\ncline[]{n8}{n148}
\ncline[]{n9}{n12}
\ncline[]{n9}{n13}
\ncline[]{n9}{n17}
\ncline[]{n9}{n19}
\ncline[]{n9}{n20}
\ncline[]{n9}{n21}
\ncline[]{n9}{n23}
\ncline[]{n9}{n24}
\ncline[]{n9}{n26}
\ncline[]{n9}{n28}
\ncline[]{n9}{n29}
\ncline[]{n9}{n32}
\ncline[]{n9}{n33}
\ncline[]{n9}{n35}
\ncline[]{n9}{n36}
\ncline[]{n9}{n38}
\ncline[]{n9}{n39}
\ncline[]{n9}{n40}
\ncline[]{n9}{n41}
\ncline[]{n9}{n42}
\ncline[]{n9}{n43}
\ncline[]{n9}{n44}
\ncline[]{n9}{n46}
\ncline[]{n9}{n47}
\ncline[]{n9}{n50}
\ncline[]{n9}{n51}
\ncline[]{n9}{n53}
\ncline[]{n9}{n55}
\ncline[]{n9}{n56}
\ncline[]{n9}{n57}
\ncline[]{n9}{n58}
\ncline[]{n9}{n59}
\ncline[]{n9}{n60}
\ncline[]{n9}{n61}
\ncline[]{n9}{n64}
\ncline[]{n9}{n65}
\ncline[]{n9}{n66}
\ncline[]{n9}{n70}
\ncline[]{n9}{n71}
\ncline[]{n9}{n72}
\ncline[]{n9}{n77}
\ncline[]{n9}{n79}
\ncline[]{n9}{n80}
\ncline[]{n9}{n82}
\ncline[]{n9}{n83}
\ncline[]{n9}{n84}
\ncline[]{n9}{n86}
\ncline[]{n9}{n87}
\ncline[]{n9}{n89}
\ncline[]{n9}{n92}
\ncline[]{n9}{n93}
\ncline[]{n9}{n94}
\ncline[]{n9}{n95}
\ncline[]{n9}{n97}
\ncline[]{n9}{n98}
\ncline[]{n9}{n99}
\ncline[]{n9}{n100}
\ncline[]{n9}{n101}
\ncline[]{n9}{n102}
\ncline[]{n9}{n103}
\ncline[]{n9}{n104}
\ncline[]{n9}{n107}
\ncline[]{n9}{n110}
\ncline[]{n9}{n112}
\ncline[]{n9}{n114}
\ncline[]{n9}{n116}
\ncline[]{n9}{n117}
\ncline[]{n9}{n118}
\ncline[]{n9}{n119}
\ncline[]{n9}{n120}
\ncline[]{n9}{n122}
\ncline[]{n9}{n123}
\ncline[]{n9}{n126}
\ncline[]{n9}{n128}
\ncline[]{n9}{n129}
\ncline[]{n9}{n131}
\ncline[]{n9}{n132}
\ncline[]{n9}{n137}
\ncline[]{n9}{n138}
\ncline[]{n9}{n139}
\ncline[]{n9}{n140}
\ncline[]{n9}{n141}
\ncline[]{n9}{n142}
\ncline[]{n9}{n143}
\ncline[]{n9}{n144}
\ncline[]{n9}{n146}
\ncline[]{n9}{n147}
\ncline[]{n9}{n148}
\ncline[]{n10}{n11}
\ncline[]{n10}{n15}
\ncline[]{n10}{n16}
\ncline[]{n10}{n18}
\ncline[]{n10}{n22}
\ncline[]{n10}{n25}
\ncline[]{n10}{n30}
\ncline[]{n10}{n31}
\ncline[]{n10}{n37}
\ncline[]{n10}{n48}
\ncline[]{n10}{n49}
\ncline[]{n10}{n54}
\ncline[]{n10}{n62}
\ncline[]{n10}{n63}
\ncline[]{n10}{n68}
\ncline[]{n10}{n69}
\ncline[]{n10}{n73}
\ncline[]{n10}{n88}
\ncline[]{n10}{n111}
\ncline[]{n10}{n113}
\ncline[]{n10}{n115}
\ncline[]{n10}{n124}
\ncline[]{n10}{n127}
\ncline[]{n10}{n130}
\ncline[]{n10}{n133}
\ncline[]{n10}{n135}
\ncline[]{n10}{n136}
\ncline[]{n10}{n145}
\ncline[]{n10}{n149}
\ncline[]{n11}{n15}
\ncline[]{n11}{n16}
\ncline[]{n11}{n18}
\ncline[]{n11}{n22}
\ncline[]{n11}{n25}
\ncline[]{n11}{n30}
\ncline[]{n11}{n31}
\ncline[]{n11}{n37}
\ncline[]{n11}{n48}
\ncline[]{n11}{n49}
\ncline[]{n11}{n54}
\ncline[]{n11}{n62}
\ncline[]{n11}{n63}
\ncline[]{n11}{n68}
\ncline[]{n11}{n69}
\ncline[]{n11}{n73}
\ncline[]{n11}{n88}
\ncline[]{n11}{n111}
\ncline[]{n11}{n113}
\ncline[]{n11}{n115}
\ncline[]{n11}{n124}
\ncline[]{n11}{n127}
\ncline[]{n11}{n130}
\ncline[]{n11}{n133}
\ncline[]{n11}{n135}
\ncline[]{n11}{n136}
\ncline[]{n11}{n145}
\ncline[]{n11}{n149}
\ncline[]{n12}{n13}
\ncline[]{n12}{n17}
\ncline[]{n12}{n19}
\ncline[]{n12}{n20}
\ncline[]{n12}{n21}
\ncline[]{n12}{n23}
\ncline[]{n12}{n24}
\ncline[]{n12}{n26}
\ncline[]{n12}{n28}
\ncline[]{n12}{n29}
\ncline[]{n12}{n32}
\ncline[]{n12}{n33}
\ncline[]{n12}{n35}
\ncline[]{n12}{n36}
\ncline[]{n12}{n38}
\ncline[]{n12}{n39}
\ncline[]{n12}{n40}
\ncline[]{n12}{n41}
\ncline[]{n12}{n42}
\ncline[]{n12}{n43}
\ncline[]{n12}{n44}
\ncline[]{n12}{n46}
\ncline[]{n12}{n47}
\ncline[]{n12}{n50}
\ncline[]{n12}{n51}
\ncline[]{n12}{n53}
\ncline[]{n12}{n55}
\ncline[]{n12}{n56}
\ncline[]{n12}{n57}
\ncline[]{n12}{n58}
\ncline[]{n12}{n59}
\ncline[]{n12}{n60}
\ncline[]{n12}{n61}
\ncline[]{n12}{n64}
\ncline[]{n12}{n65}
\ncline[]{n12}{n66}
\ncline[]{n12}{n70}
\ncline[]{n12}{n71}
\ncline[]{n12}{n72}
\ncline[]{n12}{n77}
\ncline[]{n12}{n79}
\ncline[]{n12}{n80}
\ncline[]{n12}{n82}
\ncline[]{n12}{n83}
\ncline[]{n12}{n84}
\ncline[]{n12}{n86}
\ncline[]{n12}{n87}
\ncline[]{n12}{n89}
\ncline[]{n12}{n92}
\ncline[]{n12}{n93}
\ncline[]{n12}{n94}
\ncline[]{n12}{n95}
\ncline[]{n12}{n97}
\ncline[]{n12}{n98}
\ncline[]{n12}{n99}
\ncline[]{n12}{n100}
\ncline[]{n12}{n101}
\ncline[]{n12}{n102}
\ncline[]{n12}{n103}
\ncline[]{n12}{n104}
\ncline[]{n12}{n107}
\ncline[]{n12}{n110}
\ncline[]{n12}{n112}
\ncline[]{n12}{n114}
\ncline[]{n12}{n116}
\ncline[]{n12}{n117}
\ncline[]{n12}{n118}
\ncline[]{n12}{n119}
\ncline[]{n12}{n120}
\ncline[]{n12}{n122}
\ncline[]{n12}{n123}
\ncline[]{n12}{n126}
\ncline[]{n12}{n128}
\ncline[]{n12}{n129}
\ncline[]{n12}{n131}
\ncline[]{n12}{n132}
\ncline[]{n12}{n137}
\ncline[]{n12}{n138}
\ncline[]{n12}{n139}
\ncline[]{n12}{n140}
\ncline[]{n12}{n141}
\ncline[]{n12}{n142}
\ncline[]{n12}{n143}
\ncline[]{n12}{n144}
\ncline[]{n12}{n146}
\ncline[]{n12}{n147}
\ncline[]{n12}{n148}
\ncline[]{n13}{n17}
\ncline[]{n13}{n19}
\ncline[]{n13}{n20}
\ncline[]{n13}{n21}
\ncline[]{n13}{n23}
\ncline[]{n13}{n24}
\ncline[]{n13}{n26}
\ncline[]{n13}{n28}
\ncline[]{n13}{n29}
\ncline[]{n13}{n32}
\ncline[]{n13}{n33}
\ncline[]{n13}{n35}
\ncline[]{n13}{n36}
\ncline[]{n13}{n38}
\ncline[]{n13}{n39}
\ncline[]{n13}{n40}
\ncline[]{n13}{n41}
\ncline[]{n13}{n42}
\ncline[]{n13}{n43}
\ncline[]{n13}{n44}
\ncline[]{n13}{n46}
\ncline[]{n13}{n47}
\ncline[]{n13}{n50}
\ncline[]{n13}{n51}
\ncline[]{n13}{n53}
\ncline[]{n13}{n55}
\ncline[]{n13}{n56}
\ncline[]{n13}{n57}
\ncline[]{n13}{n58}
\ncline[]{n13}{n59}
\ncline[]{n13}{n60}
\ncline[]{n13}{n61}
\ncline[]{n13}{n64}
\ncline[]{n13}{n65}
\ncline[]{n13}{n66}
\ncline[]{n13}{n70}
\ncline[]{n13}{n71}
\ncline[]{n13}{n72}
\ncline[]{n13}{n77}
\ncline[]{n13}{n79}
\ncline[]{n13}{n80}
\ncline[]{n13}{n82}
\ncline[]{n13}{n83}
\ncline[]{n13}{n84}
\ncline[]{n13}{n86}
\ncline[]{n13}{n87}
\ncline[]{n13}{n89}
\ncline[]{n13}{n92}
\ncline[]{n13}{n93}
\ncline[]{n13}{n94}
\ncline[]{n13}{n95}
\ncline[]{n13}{n97}
\ncline[]{n13}{n98}
\ncline[]{n13}{n99}
\ncline[]{n13}{n100}
\ncline[]{n13}{n101}
\ncline[]{n13}{n102}
\ncline[]{n13}{n103}
\ncline[]{n13}{n104}
\ncline[]{n13}{n107}
\ncline[]{n13}{n110}
\ncline[]{n13}{n112}
\ncline[]{n13}{n114}
\ncline[]{n13}{n116}
\ncline[]{n13}{n117}
\ncline[]{n13}{n118}
\ncline[]{n13}{n119}
\ncline[]{n13}{n120}
\ncline[]{n13}{n122}
\ncline[]{n13}{n123}
\ncline[]{n13}{n126}
\ncline[]{n13}{n128}
\ncline[]{n13}{n129}
\ncline[]{n13}{n131}
\ncline[]{n13}{n132}
\ncline[]{n13}{n137}
\ncline[]{n13}{n138}
\ncline[]{n13}{n139}
\ncline[]{n13}{n140}
\ncline[]{n13}{n141}
\ncline[]{n13}{n142}
\ncline[]{n13}{n143}
\ncline[]{n13}{n144}
\ncline[]{n13}{n146}
\ncline[]{n13}{n147}
\ncline[]{n13}{n148}
\ncline[]{n14}{n27}
\ncline[]{n14}{n34}
\ncline[]{n14}{n45}
\ncline[]{n14}{n52}
\ncline[]{n14}{n67}
\ncline[]{n14}{n74}
\ncline[]{n14}{n75}
\ncline[]{n14}{n76}
\ncline[]{n14}{n78}
\ncline[]{n14}{n81}
\ncline[]{n14}{n85}
\ncline[]{n14}{n90}
\ncline[]{n14}{n91}
\ncline[]{n14}{n96}
\ncline[]{n14}{n105}
\ncline[]{n14}{n106}
\ncline[]{n14}{n108}
\ncline[]{n14}{n109}
\ncline[]{n14}{n121}
\ncline[]{n14}{n125}
\ncline[]{n14}{n134}
\ncline[]{n15}{n16}
\ncline[]{n15}{n18}
\ncline[]{n15}{n22}
\ncline[]{n15}{n25}
\ncline[]{n15}{n30}
\ncline[]{n15}{n31}
\ncline[]{n15}{n37}
\ncline[]{n15}{n48}
\ncline[]{n15}{n49}
\ncline[]{n15}{n54}
\ncline[]{n15}{n62}
\ncline[]{n15}{n63}
\ncline[]{n15}{n68}
\ncline[]{n15}{n69}
\ncline[]{n15}{n73}
\ncline[]{n15}{n88}
\ncline[]{n15}{n111}
\ncline[]{n15}{n113}
\ncline[]{n15}{n115}
\ncline[]{n15}{n124}
\ncline[]{n15}{n127}
\ncline[]{n15}{n130}
\ncline[]{n15}{n133}
\ncline[]{n15}{n135}
\ncline[]{n15}{n136}
\ncline[]{n15}{n145}
\ncline[]{n15}{n149}
\ncline[]{n16}{n18}
\ncline[]{n16}{n22}
\ncline[]{n16}{n25}
\ncline[]{n16}{n30}
\ncline[]{n16}{n31}
\ncline[]{n16}{n37}
\ncline[]{n16}{n48}
\ncline[]{n16}{n49}
\ncline[]{n16}{n54}
\ncline[]{n16}{n62}
\ncline[]{n16}{n63}
\ncline[]{n16}{n68}
\ncline[]{n16}{n69}
\ncline[]{n16}{n73}
\ncline[]{n16}{n88}
\ncline[]{n16}{n111}
\ncline[]{n16}{n113}
\ncline[]{n16}{n115}
\ncline[]{n16}{n124}
\ncline[]{n16}{n127}
\ncline[]{n16}{n130}
\ncline[]{n16}{n133}
\ncline[]{n16}{n135}
\ncline[]{n16}{n136}
\ncline[]{n16}{n145}
\ncline[]{n16}{n149}
\ncline[]{n17}{n19}
\ncline[]{n17}{n20}
\ncline[]{n17}{n21}
\ncline[]{n17}{n23}
\ncline[]{n17}{n24}
\ncline[]{n17}{n26}
\ncline[]{n17}{n28}
\ncline[]{n17}{n29}
\ncline[]{n17}{n32}
\ncline[]{n17}{n33}
\ncline[]{n17}{n35}
\ncline[]{n17}{n36}
\ncline[]{n17}{n38}
\ncline[]{n17}{n39}
\ncline[]{n17}{n40}
\ncline[]{n17}{n41}
\ncline[]{n17}{n42}
\ncline[]{n17}{n43}
\ncline[]{n17}{n44}
\ncline[]{n17}{n46}
\ncline[]{n17}{n47}
\ncline[]{n17}{n50}
\ncline[]{n17}{n51}
\ncline[]{n17}{n53}
\ncline[]{n17}{n55}
\ncline[]{n17}{n56}
\ncline[]{n17}{n57}
\ncline[]{n17}{n58}
\ncline[]{n17}{n59}
\ncline[]{n17}{n60}
\ncline[]{n17}{n61}
\ncline[]{n17}{n64}
\ncline[]{n17}{n65}
\ncline[]{n17}{n66}
\ncline[]{n17}{n70}
\ncline[]{n17}{n71}
\ncline[]{n17}{n72}
\ncline[]{n17}{n77}
\ncline[]{n17}{n79}
\ncline[]{n17}{n80}
\ncline[]{n17}{n82}
\ncline[]{n17}{n83}
\ncline[]{n17}{n84}
\ncline[]{n17}{n86}
\ncline[]{n17}{n87}
\ncline[]{n17}{n89}
\ncline[]{n17}{n92}
\ncline[]{n17}{n93}
\ncline[]{n17}{n94}
\ncline[]{n17}{n95}
\ncline[]{n17}{n97}
\ncline[]{n17}{n98}
\ncline[]{n17}{n99}
\ncline[]{n17}{n100}
\ncline[]{n17}{n101}
\ncline[]{n17}{n102}
\ncline[]{n17}{n103}
\ncline[]{n17}{n104}
\ncline[]{n17}{n107}
\ncline[]{n17}{n110}
\ncline[]{n17}{n112}
\ncline[]{n17}{n114}
\ncline[]{n17}{n116}
\ncline[]{n17}{n117}
\ncline[]{n17}{n118}
\ncline[]{n17}{n119}
\ncline[]{n17}{n120}
\ncline[]{n17}{n122}
\ncline[]{n17}{n123}
\ncline[]{n17}{n126}
\ncline[]{n17}{n128}
\ncline[]{n17}{n129}
\ncline[]{n17}{n131}
\ncline[]{n17}{n132}
\ncline[]{n17}{n137}
\ncline[]{n17}{n138}
\ncline[]{n17}{n139}
\ncline[]{n17}{n140}
\ncline[]{n17}{n141}
\ncline[]{n17}{n142}
\ncline[]{n17}{n143}
\ncline[]{n17}{n144}
\ncline[]{n17}{n146}
\ncline[]{n17}{n147}
\ncline[]{n17}{n148}
\ncline[]{n18}{n22}
\ncline[]{n18}{n25}
\ncline[]{n18}{n30}
\ncline[]{n18}{n31}
\ncline[]{n18}{n37}
\ncline[]{n18}{n48}
\ncline[]{n18}{n49}
\ncline[]{n18}{n54}
\ncline[]{n18}{n62}
\ncline[]{n18}{n63}
\ncline[]{n18}{n68}
\ncline[]{n18}{n69}
\ncline[]{n18}{n73}
\ncline[]{n18}{n88}
\ncline[]{n18}{n111}
\ncline[]{n18}{n113}
\ncline[]{n18}{n115}
\ncline[]{n18}{n124}
\ncline[]{n18}{n127}
\ncline[]{n18}{n130}
\ncline[]{n18}{n133}
\ncline[]{n18}{n135}
\ncline[]{n18}{n136}
\ncline[]{n18}{n145}
\ncline[]{n18}{n149}
\ncline[]{n19}{n20}
\ncline[]{n19}{n21}
\ncline[]{n19}{n23}
\ncline[]{n19}{n24}
\ncline[]{n19}{n26}
\ncline[]{n19}{n28}
\ncline[]{n19}{n29}
\ncline[]{n19}{n32}
\ncline[]{n19}{n33}
\ncline[]{n19}{n35}
\ncline[]{n19}{n36}
\ncline[]{n19}{n38}
\ncline[]{n19}{n39}
\ncline[]{n19}{n40}
\ncline[]{n19}{n41}
\ncline[]{n19}{n42}
\ncline[]{n19}{n43}
\ncline[]{n19}{n44}
\ncline[]{n19}{n46}
\ncline[]{n19}{n47}
\ncline[]{n19}{n50}
\ncline[]{n19}{n51}
\ncline[]{n19}{n53}
\ncline[]{n19}{n55}
\ncline[]{n19}{n56}
\ncline[]{n19}{n57}
\ncline[]{n19}{n58}
\ncline[]{n19}{n59}
\ncline[]{n19}{n60}
\ncline[]{n19}{n61}
\ncline[]{n19}{n64}
\ncline[]{n19}{n65}
\ncline[]{n19}{n66}
\ncline[]{n19}{n70}
\ncline[]{n19}{n71}
\ncline[]{n19}{n72}
\ncline[]{n19}{n77}
\ncline[]{n19}{n79}
\ncline[]{n19}{n80}
\ncline[]{n19}{n82}
\ncline[]{n19}{n83}
\ncline[]{n19}{n84}
\ncline[]{n19}{n86}
\ncline[]{n19}{n87}
\ncline[]{n19}{n89}
\ncline[]{n19}{n92}
\ncline[]{n19}{n93}
\ncline[]{n19}{n94}
\ncline[]{n19}{n95}
\ncline[]{n19}{n97}
\ncline[]{n19}{n98}
\ncline[]{n19}{n99}
\ncline[]{n19}{n100}
\ncline[]{n19}{n101}
\ncline[]{n19}{n102}
\ncline[]{n19}{n103}
\ncline[]{n19}{n104}
\ncline[]{n19}{n107}
\ncline[]{n19}{n110}
\ncline[]{n19}{n112}
\ncline[]{n19}{n114}
\ncline[]{n19}{n116}
\ncline[]{n19}{n117}
\ncline[]{n19}{n118}
\ncline[]{n19}{n119}
\ncline[]{n19}{n120}
\ncline[]{n19}{n122}
\ncline[]{n19}{n123}
\ncline[]{n19}{n126}
\ncline[]{n19}{n128}
\ncline[]{n19}{n129}
\ncline[]{n19}{n131}
\ncline[]{n19}{n132}
\ncline[]{n19}{n137}
\ncline[]{n19}{n138}
\ncline[]{n19}{n139}
\ncline[]{n19}{n140}
\ncline[]{n19}{n141}
\ncline[]{n19}{n142}
\ncline[]{n19}{n143}
\ncline[]{n19}{n144}
\ncline[]{n19}{n146}
\ncline[]{n19}{n147}
\ncline[]{n19}{n148}
\ncline[]{n20}{n21}
\ncline[]{n20}{n23}
\ncline[]{n20}{n24}
\ncline[]{n20}{n26}
\ncline[]{n20}{n28}
\ncline[]{n20}{n29}
\ncline[]{n20}{n32}
\ncline[]{n20}{n33}
\ncline[]{n20}{n35}
\ncline[]{n20}{n36}
\ncline[]{n20}{n38}
\ncline[]{n20}{n39}
\ncline[]{n20}{n40}
\ncline[]{n20}{n41}
\ncline[]{n20}{n42}
\ncline[]{n20}{n43}
\ncline[]{n20}{n44}
\ncline[]{n20}{n46}
\ncline[]{n20}{n47}
\ncline[]{n20}{n50}
\ncline[]{n20}{n51}
\ncline[]{n20}{n53}
\ncline[]{n20}{n55}
\ncline[]{n20}{n56}
\ncline[]{n20}{n57}
\ncline[]{n20}{n58}
\ncline[]{n20}{n59}
\ncline[]{n20}{n60}
\ncline[]{n20}{n61}
\ncline[]{n20}{n64}
\ncline[]{n20}{n65}
\ncline[]{n20}{n66}
\ncline[]{n20}{n70}
\ncline[]{n20}{n71}
\ncline[]{n20}{n72}
\ncline[]{n20}{n77}
\ncline[]{n20}{n79}
\ncline[]{n20}{n80}
\ncline[]{n20}{n82}
\ncline[]{n20}{n83}
\ncline[]{n20}{n84}
\ncline[]{n20}{n86}
\ncline[]{n20}{n87}
\ncline[]{n20}{n89}
\ncline[]{n20}{n92}
\ncline[]{n20}{n93}
\ncline[]{n20}{n94}
\ncline[]{n20}{n95}
\ncline[]{n20}{n97}
\ncline[]{n20}{n98}
\ncline[]{n20}{n99}
\ncline[]{n20}{n100}
\ncline[]{n20}{n101}
\ncline[]{n20}{n102}
\ncline[]{n20}{n103}
\ncline[]{n20}{n104}
\ncline[]{n20}{n107}
\ncline[]{n20}{n110}
\ncline[]{n20}{n112}
\ncline[]{n20}{n114}
\ncline[]{n20}{n116}
\ncline[]{n20}{n117}
\ncline[]{n20}{n118}
\ncline[]{n20}{n119}
\ncline[]{n20}{n120}
\ncline[]{n20}{n122}
\ncline[]{n20}{n123}
\ncline[]{n20}{n126}
\ncline[]{n20}{n128}
\ncline[]{n20}{n129}
\ncline[]{n20}{n131}
\ncline[]{n20}{n132}
\ncline[]{n20}{n137}
\ncline[]{n20}{n138}
\ncline[]{n20}{n139}
\ncline[]{n20}{n140}
\ncline[]{n20}{n141}
\ncline[]{n20}{n142}
\ncline[]{n20}{n143}
\ncline[]{n20}{n144}
\ncline[]{n20}{n146}
\ncline[]{n20}{n147}
\ncline[]{n20}{n148}
\ncline[]{n21}{n23}
\ncline[]{n21}{n24}
\ncline[]{n21}{n26}
\ncline[]{n21}{n28}
\ncline[]{n21}{n29}
\ncline[]{n21}{n32}
\ncline[]{n21}{n33}
\ncline[]{n21}{n35}
\ncline[]{n21}{n36}
\ncline[]{n21}{n38}
\ncline[]{n21}{n39}
\ncline[]{n21}{n40}
\ncline[]{n21}{n41}
\ncline[]{n21}{n42}
\ncline[]{n21}{n43}
\ncline[]{n21}{n44}
\ncline[]{n21}{n46}
\ncline[]{n21}{n47}
\ncline[]{n21}{n50}
\ncline[]{n21}{n51}
\ncline[]{n21}{n53}
\ncline[]{n21}{n55}
\ncline[]{n21}{n56}
\ncline[]{n21}{n57}
\ncline[]{n21}{n58}
\ncline[]{n21}{n59}
\ncline[]{n21}{n60}
\ncline[]{n21}{n61}
\ncline[]{n21}{n64}
\ncline[]{n21}{n65}
\ncline[]{n21}{n66}
\ncline[]{n21}{n70}
\ncline[]{n21}{n71}
\ncline[]{n21}{n72}
\ncline[]{n21}{n77}
\ncline[]{n21}{n79}
\ncline[]{n21}{n80}
\ncline[]{n21}{n82}
\ncline[]{n21}{n83}
\ncline[]{n21}{n84}
\ncline[]{n21}{n86}
\ncline[]{n21}{n87}
\ncline[]{n21}{n89}
\ncline[]{n21}{n92}
\ncline[]{n21}{n93}
\ncline[]{n21}{n94}
\ncline[]{n21}{n95}
\ncline[]{n21}{n97}
\ncline[]{n21}{n98}
\ncline[]{n21}{n99}
\ncline[]{n21}{n100}
\ncline[]{n21}{n101}
\ncline[]{n21}{n102}
\ncline[]{n21}{n103}
\ncline[]{n21}{n104}
\ncline[]{n21}{n107}
\ncline[]{n21}{n110}
\ncline[]{n21}{n112}
\ncline[]{n21}{n114}
\ncline[]{n21}{n116}
\ncline[]{n21}{n117}
\ncline[]{n21}{n118}
\ncline[]{n21}{n119}
\ncline[]{n21}{n120}
\ncline[]{n21}{n122}
\ncline[]{n21}{n123}
\ncline[]{n21}{n126}
\ncline[]{n21}{n128}
\ncline[]{n21}{n129}
\ncline[]{n21}{n131}
\ncline[]{n21}{n132}
\ncline[]{n21}{n137}
\ncline[]{n21}{n138}
\ncline[]{n21}{n139}
\ncline[]{n21}{n140}
\ncline[]{n21}{n141}
\ncline[]{n21}{n142}
\ncline[]{n21}{n143}
\ncline[]{n21}{n144}
\ncline[]{n21}{n146}
\ncline[]{n21}{n147}
\ncline[]{n21}{n148}
\ncline[]{n22}{n25}
\ncline[]{n22}{n30}
\ncline[]{n22}{n31}
\ncline[]{n22}{n37}
\ncline[]{n22}{n48}
\ncline[]{n22}{n49}
\ncline[]{n22}{n54}
\ncline[]{n22}{n62}
\ncline[]{n22}{n63}
\ncline[]{n22}{n68}
\ncline[]{n22}{n69}
\ncline[]{n22}{n73}
\ncline[]{n22}{n88}
\ncline[]{n22}{n111}
\ncline[]{n22}{n113}
\ncline[]{n22}{n115}
\ncline[]{n22}{n124}
\ncline[]{n22}{n127}
\ncline[]{n22}{n130}
\ncline[]{n22}{n133}
\ncline[]{n22}{n135}
\ncline[]{n22}{n136}
\ncline[]{n22}{n145}
\ncline[]{n22}{n149}
\ncline[]{n23}{n24}
\ncline[]{n23}{n26}
\ncline[]{n23}{n28}
\ncline[]{n23}{n29}
\ncline[]{n23}{n32}
\ncline[]{n23}{n33}
\ncline[]{n23}{n35}
\ncline[]{n23}{n36}
\ncline[]{n23}{n38}
\ncline[]{n23}{n39}
\ncline[]{n23}{n40}
\ncline[]{n23}{n41}
\ncline[]{n23}{n42}
\ncline[]{n23}{n43}
\ncline[]{n23}{n44}
\ncline[]{n23}{n46}
\ncline[]{n23}{n47}
\ncline[]{n23}{n50}
\ncline[]{n23}{n51}
\ncline[]{n23}{n53}
\ncline[]{n23}{n55}
\ncline[]{n23}{n56}
\ncline[]{n23}{n57}
\ncline[]{n23}{n58}
\ncline[]{n23}{n59}
\ncline[]{n23}{n60}
\ncline[]{n23}{n61}
\ncline[]{n23}{n64}
\ncline[]{n23}{n65}
\ncline[]{n23}{n66}
\ncline[]{n23}{n70}
\ncline[]{n23}{n71}
\ncline[]{n23}{n72}
\ncline[]{n23}{n77}
\ncline[]{n23}{n79}
\ncline[]{n23}{n80}
\ncline[]{n23}{n82}
\ncline[]{n23}{n83}
\ncline[]{n23}{n84}
\ncline[]{n23}{n86}
\ncline[]{n23}{n87}
\ncline[]{n23}{n89}
\ncline[]{n23}{n92}
\ncline[]{n23}{n93}
\ncline[]{n23}{n94}
\ncline[]{n23}{n95}
\ncline[]{n23}{n97}
\ncline[]{n23}{n98}
\ncline[]{n23}{n99}
\ncline[]{n23}{n100}
\ncline[]{n23}{n101}
\ncline[]{n23}{n102}
\ncline[]{n23}{n103}
\ncline[]{n23}{n104}
\ncline[]{n23}{n107}
\ncline[]{n23}{n110}
\ncline[]{n23}{n112}
\ncline[]{n23}{n114}
\ncline[]{n23}{n116}
\ncline[]{n23}{n117}
\ncline[]{n23}{n118}
\ncline[]{n23}{n119}
\ncline[]{n23}{n120}
\ncline[]{n23}{n122}
\ncline[]{n23}{n123}
\ncline[]{n23}{n126}
\ncline[]{n23}{n128}
\ncline[]{n23}{n129}
\ncline[]{n23}{n131}
\ncline[]{n23}{n132}
\ncline[]{n23}{n137}
\ncline[]{n23}{n138}
\ncline[]{n23}{n139}
\ncline[]{n23}{n140}
\ncline[]{n23}{n141}
\ncline[]{n23}{n142}
\ncline[]{n23}{n143}
\ncline[]{n23}{n144}
\ncline[]{n23}{n146}
\ncline[]{n23}{n147}
\ncline[]{n23}{n148}
\ncline[]{n24}{n26}
\ncline[]{n24}{n28}
\ncline[]{n24}{n29}
\ncline[]{n24}{n32}
\ncline[]{n24}{n33}
\ncline[]{n24}{n35}
\ncline[]{n24}{n36}
\ncline[]{n24}{n38}
\ncline[]{n24}{n39}
\ncline[]{n24}{n40}
\ncline[]{n24}{n41}
\ncline[]{n24}{n42}
\ncline[]{n24}{n43}
\ncline[]{n24}{n44}
\ncline[]{n24}{n46}
\ncline[]{n24}{n47}
\ncline[]{n24}{n50}
\ncline[]{n24}{n51}
\ncline[]{n24}{n53}
\ncline[]{n24}{n55}
\ncline[]{n24}{n56}
\ncline[]{n24}{n57}
\ncline[]{n24}{n58}
\ncline[]{n24}{n59}
\ncline[]{n24}{n60}
\ncline[]{n24}{n61}
\ncline[]{n24}{n64}
\ncline[]{n24}{n65}
\ncline[]{n24}{n66}
\ncline[]{n24}{n70}
\ncline[]{n24}{n71}
\ncline[]{n24}{n72}
\ncline[]{n24}{n77}
\ncline[]{n24}{n79}
\ncline[]{n24}{n80}
\ncline[]{n24}{n82}
\ncline[]{n24}{n83}
\ncline[]{n24}{n84}
\ncline[]{n24}{n86}
\ncline[]{n24}{n87}
\ncline[]{n24}{n89}
\ncline[]{n24}{n92}
\ncline[]{n24}{n93}
\ncline[]{n24}{n94}
\ncline[]{n24}{n95}
\ncline[]{n24}{n97}
\ncline[]{n24}{n98}
\ncline[]{n24}{n99}
\ncline[]{n24}{n100}
\ncline[]{n24}{n101}
\ncline[]{n24}{n102}
\ncline[]{n24}{n103}
\ncline[]{n24}{n104}
\ncline[]{n24}{n107}
\ncline[]{n24}{n110}
\ncline[]{n24}{n112}
\ncline[]{n24}{n114}
\ncline[]{n24}{n116}
\ncline[]{n24}{n117}
\ncline[]{n24}{n118}
\ncline[]{n24}{n119}
\ncline[]{n24}{n120}
\ncline[]{n24}{n122}
\ncline[]{n24}{n123}
\ncline[]{n24}{n126}
\ncline[]{n24}{n128}
\ncline[]{n24}{n129}
\ncline[]{n24}{n131}
\ncline[]{n24}{n132}
\ncline[]{n24}{n137}
\ncline[]{n24}{n138}
\ncline[]{n24}{n139}
\ncline[]{n24}{n140}
\ncline[]{n24}{n141}
\ncline[]{n24}{n142}
\ncline[]{n24}{n143}
\ncline[]{n24}{n144}
\ncline[]{n24}{n146}
\ncline[]{n24}{n147}
\ncline[]{n24}{n148}
\ncline[]{n25}{n30}
\ncline[]{n25}{n31}
\ncline[]{n25}{n37}
\ncline[]{n25}{n48}
\ncline[]{n25}{n49}
\ncline[]{n25}{n54}
\ncline[]{n25}{n62}
\ncline[]{n25}{n63}
\ncline[]{n25}{n68}
\ncline[]{n25}{n69}
\ncline[]{n25}{n73}
\ncline[]{n25}{n88}
\ncline[]{n25}{n111}
\ncline[]{n25}{n113}
\ncline[]{n25}{n115}
\ncline[]{n25}{n124}
\ncline[]{n25}{n127}
\ncline[]{n25}{n130}
\ncline[]{n25}{n133}
\ncline[]{n25}{n135}
\ncline[]{n25}{n136}
\ncline[]{n25}{n145}
\ncline[]{n25}{n149}
\ncline[]{n26}{n28}
\ncline[]{n26}{n29}
\ncline[]{n26}{n32}
\ncline[]{n26}{n33}
\ncline[]{n26}{n35}
\ncline[]{n26}{n36}
\ncline[]{n26}{n38}
\ncline[]{n26}{n39}
\ncline[]{n26}{n40}
\ncline[]{n26}{n41}
\ncline[]{n26}{n42}
\ncline[]{n26}{n43}
\ncline[]{n26}{n44}
\ncline[]{n26}{n46}
\ncline[]{n26}{n47}
\ncline[]{n26}{n50}
\ncline[]{n26}{n51}
\ncline[]{n26}{n53}
\ncline[]{n26}{n55}
\ncline[]{n26}{n56}
\ncline[]{n26}{n57}
\ncline[]{n26}{n58}
\ncline[]{n26}{n59}
\ncline[]{n26}{n60}
\ncline[]{n26}{n61}
\ncline[]{n26}{n64}
\ncline[]{n26}{n65}
\ncline[]{n26}{n66}
\ncline[]{n26}{n70}
\ncline[]{n26}{n71}
\ncline[]{n26}{n72}
\ncline[]{n26}{n77}
\ncline[]{n26}{n79}
\ncline[]{n26}{n80}
\ncline[]{n26}{n82}
\ncline[]{n26}{n83}
\ncline[]{n26}{n84}
\ncline[]{n26}{n86}
\ncline[]{n26}{n87}
\ncline[]{n26}{n89}
\ncline[]{n26}{n92}
\ncline[]{n26}{n93}
\ncline[]{n26}{n94}
\ncline[]{n26}{n95}
\ncline[]{n26}{n97}
\ncline[]{n26}{n98}
\ncline[]{n26}{n99}
\ncline[]{n26}{n100}
\ncline[]{n26}{n101}
\ncline[]{n26}{n102}
\ncline[]{n26}{n103}
\ncline[]{n26}{n104}
\ncline[]{n26}{n107}
\ncline[]{n26}{n110}
\ncline[]{n26}{n112}
\ncline[]{n26}{n114}
\ncline[]{n26}{n116}
\ncline[]{n26}{n117}
\ncline[]{n26}{n118}
\ncline[]{n26}{n119}
\ncline[]{n26}{n120}
\ncline[]{n26}{n122}
\ncline[]{n26}{n123}
\ncline[]{n26}{n126}
\ncline[]{n26}{n128}
\ncline[]{n26}{n129}
\ncline[]{n26}{n131}
\ncline[]{n26}{n132}
\ncline[]{n26}{n137}
\ncline[]{n26}{n138}
\ncline[]{n26}{n139}
\ncline[]{n26}{n140}
\ncline[]{n26}{n141}
\ncline[]{n26}{n142}
\ncline[]{n26}{n143}
\ncline[]{n26}{n144}
\ncline[]{n26}{n146}
\ncline[]{n26}{n147}
\ncline[]{n26}{n148}
\ncline[]{n27}{n34}
\ncline[]{n27}{n45}
\ncline[]{n27}{n52}
\ncline[]{n27}{n67}
\ncline[]{n27}{n74}
\ncline[]{n27}{n75}
\ncline[]{n27}{n76}
\ncline[]{n27}{n78}
\ncline[]{n27}{n81}
\ncline[]{n27}{n85}
\ncline[]{n27}{n90}
\ncline[]{n27}{n91}
\ncline[]{n27}{n96}
\ncline[]{n27}{n105}
\ncline[]{n27}{n106}
\ncline[]{n27}{n108}
\ncline[]{n27}{n109}
\ncline[]{n27}{n121}
\ncline[]{n27}{n125}
\ncline[]{n27}{n134}
\ncline[]{n28}{n29}
\ncline[]{n28}{n32}
\ncline[]{n28}{n33}
\ncline[]{n28}{n35}
\ncline[]{n28}{n36}
\ncline[]{n28}{n38}
\ncline[]{n28}{n39}
\ncline[]{n28}{n40}
\ncline[]{n28}{n41}
\ncline[]{n28}{n42}
\ncline[]{n28}{n43}
\ncline[]{n28}{n44}
\ncline[]{n28}{n46}
\ncline[]{n28}{n47}
\ncline[]{n28}{n50}
\ncline[]{n28}{n51}
\ncline[]{n28}{n53}
\ncline[]{n28}{n55}
\ncline[]{n28}{n56}
\ncline[]{n28}{n57}
\ncline[]{n28}{n58}
\ncline[]{n28}{n59}
\ncline[]{n28}{n60}
\ncline[]{n28}{n61}
\ncline[]{n28}{n64}
\ncline[]{n28}{n65}
\ncline[]{n28}{n66}
\ncline[]{n28}{n70}
\ncline[]{n28}{n71}
\ncline[]{n28}{n72}
\ncline[]{n28}{n77}
\ncline[]{n28}{n79}
\ncline[]{n28}{n80}
\ncline[]{n28}{n82}
\ncline[]{n28}{n83}
\ncline[]{n28}{n84}
\ncline[]{n28}{n86}
\ncline[]{n28}{n87}
\ncline[]{n28}{n89}
\ncline[]{n28}{n92}
\ncline[]{n28}{n93}
\ncline[]{n28}{n94}
\ncline[]{n28}{n95}
\ncline[]{n28}{n97}
\ncline[]{n28}{n98}
\ncline[]{n28}{n99}
\ncline[]{n28}{n100}
\ncline[]{n28}{n101}
\ncline[]{n28}{n102}
\ncline[]{n28}{n103}
\ncline[]{n28}{n104}
\ncline[]{n28}{n107}
\ncline[]{n28}{n110}
\ncline[]{n28}{n112}
\ncline[]{n28}{n114}
\ncline[]{n28}{n116}
\ncline[]{n28}{n117}
\ncline[]{n28}{n118}
\ncline[]{n28}{n119}
\ncline[]{n28}{n120}
\ncline[]{n28}{n122}
\ncline[]{n28}{n123}
\ncline[]{n28}{n126}
\ncline[]{n28}{n128}
\ncline[]{n28}{n129}
\ncline[]{n28}{n131}
\ncline[]{n28}{n132}
\ncline[]{n28}{n137}
\ncline[]{n28}{n138}
\ncline[]{n28}{n139}
\ncline[]{n28}{n140}
\ncline[]{n28}{n141}
\ncline[]{n28}{n142}
\ncline[]{n28}{n143}
\ncline[]{n28}{n144}
\ncline[]{n28}{n146}
\ncline[]{n28}{n147}
\ncline[]{n28}{n148}
\ncline[]{n29}{n32}
\ncline[]{n29}{n33}
\ncline[]{n29}{n35}
\ncline[]{n29}{n36}
\ncline[]{n29}{n38}
\ncline[]{n29}{n39}
\ncline[]{n29}{n40}
\ncline[]{n29}{n41}
\ncline[]{n29}{n42}
\ncline[]{n29}{n43}
\ncline[]{n29}{n44}
\ncline[]{n29}{n46}
\ncline[]{n29}{n47}
\ncline[]{n29}{n50}
\ncline[]{n29}{n51}
\ncline[]{n29}{n53}
\ncline[]{n29}{n55}
\ncline[]{n29}{n56}
\ncline[]{n29}{n57}
\ncline[]{n29}{n58}
\ncline[]{n29}{n59}
\ncline[]{n29}{n60}
\ncline[]{n29}{n61}
\ncline[]{n29}{n64}
\ncline[]{n29}{n65}
\ncline[]{n29}{n66}
\ncline[]{n29}{n70}
\ncline[]{n29}{n71}
\ncline[]{n29}{n72}
\ncline[]{n29}{n77}
\ncline[]{n29}{n79}
\ncline[]{n29}{n80}
\ncline[]{n29}{n82}
\ncline[]{n29}{n83}
\ncline[]{n29}{n84}
\ncline[]{n29}{n86}
\ncline[]{n29}{n87}
\ncline[]{n29}{n89}
\ncline[]{n29}{n92}
\ncline[]{n29}{n93}
\ncline[]{n29}{n94}
\ncline[]{n29}{n95}
\ncline[]{n29}{n97}
\ncline[]{n29}{n98}
\ncline[]{n29}{n99}
\ncline[]{n29}{n100}
\ncline[]{n29}{n101}
\ncline[]{n29}{n102}
\ncline[]{n29}{n103}
\ncline[]{n29}{n104}
\ncline[]{n29}{n107}
\ncline[]{n29}{n110}
\ncline[]{n29}{n112}
\ncline[]{n29}{n114}
\ncline[]{n29}{n116}
\ncline[]{n29}{n117}
\ncline[]{n29}{n118}
\ncline[]{n29}{n119}
\ncline[]{n29}{n120}
\ncline[]{n29}{n122}
\ncline[]{n29}{n123}
\ncline[]{n29}{n126}
\ncline[]{n29}{n128}
\ncline[]{n29}{n129}
\ncline[]{n29}{n131}
\ncline[]{n29}{n132}
\ncline[]{n29}{n137}
\ncline[]{n29}{n138}
\ncline[]{n29}{n139}
\ncline[]{n29}{n140}
\ncline[]{n29}{n141}
\ncline[]{n29}{n142}
\ncline[]{n29}{n143}
\ncline[]{n29}{n144}
\ncline[]{n29}{n146}
\ncline[]{n29}{n147}
\ncline[]{n29}{n148}
\ncline[]{n30}{n31}
\ncline[]{n30}{n37}
\ncline[]{n30}{n48}
\ncline[]{n30}{n49}
\ncline[]{n30}{n54}
\ncline[]{n30}{n62}
\ncline[]{n30}{n63}
\ncline[]{n30}{n68}
\ncline[]{n30}{n69}
\ncline[]{n30}{n73}
\ncline[]{n30}{n88}
\ncline[]{n30}{n111}
\ncline[]{n30}{n113}
\ncline[]{n30}{n115}
\ncline[]{n30}{n124}
\ncline[]{n30}{n127}
\ncline[]{n30}{n130}
\ncline[]{n30}{n133}
\ncline[]{n30}{n135}
\ncline[]{n30}{n136}
\ncline[]{n30}{n145}
\ncline[]{n30}{n149}
\ncline[]{n31}{n37}
\ncline[]{n31}{n48}
\ncline[]{n31}{n49}
\ncline[]{n31}{n54}
\ncline[]{n31}{n62}
\ncline[]{n31}{n63}
\ncline[]{n31}{n68}
\ncline[]{n31}{n69}
\ncline[]{n31}{n73}
\ncline[]{n31}{n88}
\ncline[]{n31}{n111}
\ncline[]{n31}{n113}
\ncline[]{n31}{n115}
\ncline[]{n31}{n124}
\ncline[]{n31}{n127}
\ncline[]{n31}{n130}
\ncline[]{n31}{n133}
\ncline[]{n31}{n135}
\ncline[]{n31}{n136}
\ncline[]{n31}{n145}
\ncline[]{n31}{n149}
\ncline[]{n32}{n33}
\ncline[]{n32}{n35}
\ncline[]{n32}{n36}
\ncline[]{n32}{n38}
\ncline[]{n32}{n39}
\ncline[]{n32}{n40}
\ncline[]{n32}{n41}
\ncline[]{n32}{n42}
\ncline[]{n32}{n43}
\ncline[]{n32}{n44}
\ncline[]{n32}{n46}
\ncline[]{n32}{n47}
\ncline[]{n32}{n50}
\ncline[]{n32}{n51}
\ncline[]{n32}{n53}
\ncline[]{n32}{n55}
\ncline[]{n32}{n56}
\ncline[]{n32}{n57}
\ncline[]{n32}{n58}
\ncline[]{n32}{n59}
\ncline[]{n32}{n60}
\ncline[]{n32}{n61}
\ncline[]{n32}{n64}
\ncline[]{n32}{n65}
\ncline[]{n32}{n66}
\ncline[]{n32}{n70}
\ncline[]{n32}{n71}
\ncline[]{n32}{n72}
\ncline[]{n32}{n77}
\ncline[]{n32}{n79}
\ncline[]{n32}{n80}
\ncline[]{n32}{n82}
\ncline[]{n32}{n83}
\ncline[]{n32}{n84}
\ncline[]{n32}{n86}
\ncline[]{n32}{n87}
\ncline[]{n32}{n89}
\ncline[]{n32}{n92}
\ncline[]{n32}{n93}
\ncline[]{n32}{n94}
\ncline[]{n32}{n95}
\ncline[]{n32}{n97}
\ncline[]{n32}{n98}
\ncline[]{n32}{n99}
\ncline[]{n32}{n100}
\ncline[]{n32}{n101}
\ncline[]{n32}{n102}
\ncline[]{n32}{n103}
\ncline[]{n32}{n104}
\ncline[]{n32}{n107}
\ncline[]{n32}{n110}
\ncline[]{n32}{n112}
\ncline[]{n32}{n114}
\ncline[]{n32}{n116}
\ncline[]{n32}{n117}
\ncline[]{n32}{n118}
\ncline[]{n32}{n119}
\ncline[]{n32}{n120}
\ncline[]{n32}{n122}
\ncline[]{n32}{n123}
\ncline[]{n32}{n126}
\ncline[]{n32}{n128}
\ncline[]{n32}{n129}
\ncline[]{n32}{n131}
\ncline[]{n32}{n132}
\ncline[]{n32}{n137}
\ncline[]{n32}{n138}
\ncline[]{n32}{n139}
\ncline[]{n32}{n140}
\ncline[]{n32}{n141}
\ncline[]{n32}{n142}
\ncline[]{n32}{n143}
\ncline[]{n32}{n144}
\ncline[]{n32}{n146}
\ncline[]{n32}{n147}
\ncline[]{n32}{n148}
\ncline[]{n33}{n35}
\ncline[]{n33}{n36}
\ncline[]{n33}{n38}
\ncline[]{n33}{n39}
\ncline[]{n33}{n40}
\ncline[]{n33}{n41}
\ncline[]{n33}{n42}
\ncline[]{n33}{n43}
\ncline[]{n33}{n44}
\ncline[]{n33}{n46}
\ncline[]{n33}{n47}
\ncline[]{n33}{n50}
\ncline[]{n33}{n51}
\ncline[]{n33}{n53}
\ncline[]{n33}{n55}
\ncline[]{n33}{n56}
\ncline[]{n33}{n57}
\ncline[]{n33}{n58}
\ncline[]{n33}{n59}
\ncline[]{n33}{n60}
\ncline[]{n33}{n61}
\ncline[]{n33}{n64}
\ncline[]{n33}{n65}
\ncline[]{n33}{n66}
\ncline[]{n33}{n70}
\ncline[]{n33}{n71}
\ncline[]{n33}{n72}
\ncline[]{n33}{n77}
\ncline[]{n33}{n79}
\ncline[]{n33}{n80}
\ncline[]{n33}{n82}
\ncline[]{n33}{n83}
\ncline[]{n33}{n84}
\ncline[]{n33}{n86}
\ncline[]{n33}{n87}
\ncline[]{n33}{n89}
\ncline[]{n33}{n92}
\ncline[]{n33}{n93}
\ncline[]{n33}{n94}
\ncline[]{n33}{n95}
\ncline[]{n33}{n97}
\ncline[]{n33}{n98}
\ncline[]{n33}{n99}
\ncline[]{n33}{n100}
\ncline[]{n33}{n101}
\ncline[]{n33}{n102}
\ncline[]{n33}{n103}
\ncline[]{n33}{n104}
\ncline[]{n33}{n107}
\ncline[]{n33}{n110}
\ncline[]{n33}{n112}
\ncline[]{n33}{n114}
\ncline[]{n33}{n116}
\ncline[]{n33}{n117}
\ncline[]{n33}{n118}
\ncline[]{n33}{n119}
\ncline[]{n33}{n120}
\ncline[]{n33}{n122}
\ncline[]{n33}{n123}
\ncline[]{n33}{n126}
\ncline[]{n33}{n128}
\ncline[]{n33}{n129}
\ncline[]{n33}{n131}
\ncline[]{n33}{n132}
\ncline[]{n33}{n137}
\ncline[]{n33}{n138}
\ncline[]{n33}{n139}
\ncline[]{n33}{n140}
\ncline[]{n33}{n141}
\ncline[]{n33}{n142}
\ncline[]{n33}{n143}
\ncline[]{n33}{n144}
\ncline[]{n33}{n146}
\ncline[]{n33}{n147}
\ncline[]{n33}{n148}
\ncline[]{n34}{n45}
\ncline[]{n34}{n52}
\ncline[]{n34}{n67}
\ncline[]{n34}{n74}
\ncline[]{n34}{n75}
\ncline[]{n34}{n76}
\ncline[]{n34}{n78}
\ncline[]{n34}{n81}
\ncline[]{n34}{n85}
\ncline[]{n34}{n90}
\ncline[]{n34}{n91}
\ncline[]{n34}{n96}
\ncline[]{n34}{n105}
\ncline[]{n34}{n106}
\ncline[]{n34}{n108}
\ncline[]{n34}{n109}
\ncline[]{n34}{n121}
\ncline[]{n34}{n125}
\ncline[]{n34}{n134}
\ncline[]{n35}{n36}
\ncline[]{n35}{n38}
\ncline[]{n35}{n39}
\ncline[]{n35}{n40}
\ncline[]{n35}{n41}
\ncline[]{n35}{n42}
\ncline[]{n35}{n43}
\ncline[]{n35}{n44}
\ncline[]{n35}{n46}
\ncline[]{n35}{n47}
\ncline[]{n35}{n50}
\ncline[]{n35}{n51}
\ncline[]{n35}{n53}
\ncline[]{n35}{n55}
\ncline[]{n35}{n56}
\ncline[]{n35}{n57}
\ncline[]{n35}{n58}
\ncline[]{n35}{n59}
\ncline[]{n35}{n60}
\ncline[]{n35}{n61}
\ncline[]{n35}{n64}
\ncline[]{n35}{n65}
\ncline[]{n35}{n66}
\ncline[]{n35}{n70}
\ncline[]{n35}{n71}
\ncline[]{n35}{n72}
\ncline[]{n35}{n77}
\ncline[]{n35}{n79}
\ncline[]{n35}{n80}
\ncline[]{n35}{n82}
\ncline[]{n35}{n83}
\ncline[]{n35}{n84}
\ncline[]{n35}{n86}
\ncline[]{n35}{n87}
\ncline[]{n35}{n89}
\ncline[]{n35}{n92}
\ncline[]{n35}{n93}
\ncline[]{n35}{n94}
\ncline[]{n35}{n95}
\ncline[]{n35}{n97}
\ncline[]{n35}{n98}
\ncline[]{n35}{n99}
\ncline[]{n35}{n100}
\ncline[]{n35}{n101}
\ncline[]{n35}{n102}
\ncline[]{n35}{n103}
\ncline[]{n35}{n104}
\ncline[]{n35}{n107}
\ncline[]{n35}{n110}
\ncline[]{n35}{n112}
\ncline[]{n35}{n114}
\ncline[]{n35}{n116}
\ncline[]{n35}{n117}
\ncline[]{n35}{n118}
\ncline[]{n35}{n119}
\ncline[]{n35}{n120}
\ncline[]{n35}{n122}
\ncline[]{n35}{n123}
\ncline[]{n35}{n126}
\ncline[]{n35}{n128}
\ncline[]{n35}{n129}
\ncline[]{n35}{n131}
\ncline[]{n35}{n132}
\ncline[]{n35}{n137}
\ncline[]{n35}{n138}
\ncline[]{n35}{n139}
\ncline[]{n35}{n140}
\ncline[]{n35}{n141}
\ncline[]{n35}{n142}
\ncline[]{n35}{n143}
\ncline[]{n35}{n144}
\ncline[]{n35}{n146}
\ncline[]{n35}{n147}
\ncline[]{n35}{n148}
\ncline[]{n36}{n38}
\ncline[]{n36}{n39}
\ncline[]{n36}{n40}
\ncline[]{n36}{n41}
\ncline[]{n36}{n42}
\ncline[]{n36}{n43}
\ncline[]{n36}{n44}
\ncline[]{n36}{n46}
\ncline[]{n36}{n47}
\ncline[]{n36}{n50}
\ncline[]{n36}{n51}
\ncline[]{n36}{n53}
\ncline[]{n36}{n55}
\ncline[]{n36}{n56}
\ncline[]{n36}{n57}
\ncline[]{n36}{n58}
\ncline[]{n36}{n59}
\ncline[]{n36}{n60}
\ncline[]{n36}{n61}
\ncline[]{n36}{n64}
\ncline[]{n36}{n65}
\ncline[]{n36}{n66}
\ncline[]{n36}{n70}
\ncline[]{n36}{n71}
\ncline[]{n36}{n72}
\ncline[]{n36}{n77}
\ncline[]{n36}{n79}
\ncline[]{n36}{n80}
\ncline[]{n36}{n82}
\ncline[]{n36}{n83}
\ncline[]{n36}{n84}
\ncline[]{n36}{n86}
\ncline[]{n36}{n87}
\ncline[]{n36}{n89}
\ncline[]{n36}{n92}
\ncline[]{n36}{n93}
\ncline[]{n36}{n94}
\ncline[]{n36}{n95}
\ncline[]{n36}{n97}
\ncline[]{n36}{n98}
\ncline[]{n36}{n99}
\ncline[]{n36}{n100}
\ncline[]{n36}{n101}
\ncline[]{n36}{n102}
\ncline[]{n36}{n103}
\ncline[]{n36}{n104}
\ncline[]{n36}{n107}
\ncline[]{n36}{n110}
\ncline[]{n36}{n112}
\ncline[]{n36}{n114}
\ncline[]{n36}{n116}
\ncline[]{n36}{n117}
\ncline[]{n36}{n118}
\ncline[]{n36}{n119}
\ncline[]{n36}{n120}
\ncline[]{n36}{n122}
\ncline[]{n36}{n123}
\ncline[]{n36}{n126}
\ncline[]{n36}{n128}
\ncline[]{n36}{n129}
\ncline[]{n36}{n131}
\ncline[]{n36}{n132}
\ncline[]{n36}{n137}
\ncline[]{n36}{n138}
\ncline[]{n36}{n139}
\ncline[]{n36}{n140}
\ncline[]{n36}{n141}
\ncline[]{n36}{n142}
\ncline[]{n36}{n143}
\ncline[]{n36}{n144}
\ncline[]{n36}{n146}
\ncline[]{n36}{n147}
\ncline[]{n36}{n148}
\ncline[]{n37}{n48}
\ncline[]{n37}{n49}
\ncline[]{n37}{n54}
\ncline[]{n37}{n62}
\ncline[]{n37}{n63}
\ncline[]{n37}{n68}
\ncline[]{n37}{n69}
\ncline[]{n37}{n73}
\ncline[]{n37}{n88}
\ncline[]{n37}{n111}
\ncline[]{n37}{n113}
\ncline[]{n37}{n115}
\ncline[]{n37}{n124}
\ncline[]{n37}{n127}
\ncline[]{n37}{n130}
\ncline[]{n37}{n133}
\ncline[]{n37}{n135}
\ncline[]{n37}{n136}
\ncline[]{n37}{n145}
\ncline[]{n37}{n149}
\ncline[]{n38}{n39}
\ncline[]{n38}{n40}
\ncline[]{n38}{n41}
\ncline[]{n38}{n42}
\ncline[]{n38}{n43}
\ncline[]{n38}{n44}
\ncline[]{n38}{n46}
\ncline[]{n38}{n47}
\ncline[]{n38}{n50}
\ncline[]{n38}{n51}
\ncline[]{n38}{n53}
\ncline[]{n38}{n55}
\ncline[]{n38}{n56}
\ncline[]{n38}{n57}
\ncline[]{n38}{n58}
\ncline[]{n38}{n59}
\ncline[]{n38}{n60}
\ncline[]{n38}{n61}
\ncline[]{n38}{n64}
\ncline[]{n38}{n65}
\ncline[]{n38}{n66}
\ncline[]{n38}{n70}
\ncline[]{n38}{n71}
\ncline[]{n38}{n72}
\ncline[]{n38}{n77}
\ncline[]{n38}{n79}
\ncline[]{n38}{n80}
\ncline[]{n38}{n82}
\ncline[]{n38}{n83}
\ncline[]{n38}{n84}
\ncline[]{n38}{n86}
\ncline[]{n38}{n87}
\ncline[]{n38}{n89}
\ncline[]{n38}{n92}
\ncline[]{n38}{n93}
\ncline[]{n38}{n94}
\ncline[]{n38}{n95}
\ncline[]{n38}{n97}
\ncline[]{n38}{n98}
\ncline[]{n38}{n99}
\ncline[]{n38}{n100}
\ncline[]{n38}{n101}
\ncline[]{n38}{n102}
\ncline[]{n38}{n103}
\ncline[]{n38}{n104}
\ncline[]{n38}{n107}
\ncline[]{n38}{n110}
\ncline[]{n38}{n112}
\ncline[]{n38}{n114}
\ncline[]{n38}{n116}
\ncline[]{n38}{n117}
\ncline[]{n38}{n118}
\ncline[]{n38}{n119}
\ncline[]{n38}{n120}
\ncline[]{n38}{n122}
\ncline[]{n38}{n123}
\ncline[]{n38}{n126}
\ncline[]{n38}{n128}
\ncline[]{n38}{n129}
\ncline[]{n38}{n131}
\ncline[]{n38}{n132}
\ncline[]{n38}{n137}
\ncline[]{n38}{n138}
\ncline[]{n38}{n139}
\ncline[]{n38}{n140}
\ncline[]{n38}{n141}
\ncline[]{n38}{n142}
\ncline[]{n38}{n143}
\ncline[]{n38}{n144}
\ncline[]{n38}{n146}
\ncline[]{n38}{n147}
\ncline[]{n38}{n148}
\ncline[]{n39}{n40}
\ncline[]{n39}{n41}
\ncline[]{n39}{n42}
\ncline[]{n39}{n43}
\ncline[]{n39}{n44}
\ncline[]{n39}{n46}
\ncline[]{n39}{n47}
\ncline[]{n39}{n50}
\ncline[]{n39}{n51}
\ncline[]{n39}{n53}
\ncline[]{n39}{n55}
\ncline[]{n39}{n56}
\ncline[]{n39}{n57}
\ncline[]{n39}{n58}
\ncline[]{n39}{n59}
\ncline[]{n39}{n60}
\ncline[]{n39}{n61}
\ncline[]{n39}{n64}
\ncline[]{n39}{n65}
\ncline[]{n39}{n66}
\ncline[]{n39}{n70}
\ncline[]{n39}{n71}
\ncline[]{n39}{n72}
\ncline[]{n39}{n77}
\ncline[]{n39}{n79}
\ncline[]{n39}{n80}
\ncline[]{n39}{n82}
\ncline[]{n39}{n83}
\ncline[]{n39}{n84}
\ncline[]{n39}{n86}
\ncline[]{n39}{n87}
\ncline[]{n39}{n89}
\ncline[]{n39}{n92}
\ncline[]{n39}{n93}
\ncline[]{n39}{n94}
\ncline[]{n39}{n95}
\ncline[]{n39}{n97}
\ncline[]{n39}{n98}
\ncline[]{n39}{n99}
\ncline[]{n39}{n100}
\ncline[]{n39}{n101}
\ncline[]{n39}{n102}
\ncline[]{n39}{n103}
\ncline[]{n39}{n104}
\ncline[]{n39}{n107}
\ncline[]{n39}{n110}
\ncline[]{n39}{n112}
\ncline[]{n39}{n114}
\ncline[]{n39}{n116}
\ncline[]{n39}{n117}
\ncline[]{n39}{n118}
\ncline[]{n39}{n119}
\ncline[]{n39}{n120}
\ncline[]{n39}{n122}
\ncline[]{n39}{n123}
\ncline[]{n39}{n126}
\ncline[]{n39}{n128}
\ncline[]{n39}{n129}
\ncline[]{n39}{n131}
\ncline[]{n39}{n132}
\ncline[]{n39}{n137}
\ncline[]{n39}{n138}
\ncline[]{n39}{n139}
\ncline[]{n39}{n140}
\ncline[]{n39}{n141}
\ncline[]{n39}{n142}
\ncline[]{n39}{n143}
\ncline[]{n39}{n144}
\ncline[]{n39}{n146}
\ncline[]{n39}{n147}
\ncline[]{n39}{n148}
\ncline[]{n40}{n41}
\ncline[]{n40}{n42}
\ncline[]{n40}{n43}
\ncline[]{n40}{n44}
\ncline[]{n40}{n46}
\ncline[]{n40}{n47}
\ncline[]{n40}{n50}
\ncline[]{n40}{n51}
\ncline[]{n40}{n53}
\ncline[]{n40}{n55}
\ncline[]{n40}{n56}
\ncline[]{n40}{n57}
\ncline[]{n40}{n58}
\ncline[]{n40}{n59}
\ncline[]{n40}{n60}
\ncline[]{n40}{n61}
\ncline[]{n40}{n64}
\ncline[]{n40}{n65}
\ncline[]{n40}{n66}
\ncline[]{n40}{n70}
\ncline[]{n40}{n71}
\ncline[]{n40}{n72}
\ncline[]{n40}{n77}
\ncline[]{n40}{n79}
\ncline[]{n40}{n80}
\ncline[]{n40}{n82}
\ncline[]{n40}{n83}
\ncline[]{n40}{n84}
\ncline[]{n40}{n86}
\ncline[]{n40}{n87}
\ncline[]{n40}{n89}
\ncline[]{n40}{n92}
\ncline[]{n40}{n93}
\ncline[]{n40}{n94}
\ncline[]{n40}{n95}
\ncline[]{n40}{n97}
\ncline[]{n40}{n98}
\ncline[]{n40}{n99}
\ncline[]{n40}{n100}
\ncline[]{n40}{n101}
\ncline[]{n40}{n102}
\ncline[]{n40}{n103}
\ncline[]{n40}{n104}
\ncline[]{n40}{n107}
\ncline[]{n40}{n110}
\ncline[]{n40}{n112}
\ncline[]{n40}{n114}
\ncline[]{n40}{n116}
\ncline[]{n40}{n117}
\ncline[]{n40}{n118}
\ncline[]{n40}{n119}
\ncline[]{n40}{n120}
\ncline[]{n40}{n122}
\ncline[]{n40}{n123}
\ncline[]{n40}{n126}
\ncline[]{n40}{n128}
\ncline[]{n40}{n129}
\ncline[]{n40}{n131}
\ncline[]{n40}{n132}
\ncline[]{n40}{n137}
\ncline[]{n40}{n138}
\ncline[]{n40}{n139}
\ncline[]{n40}{n140}
\ncline[]{n40}{n141}
\ncline[]{n40}{n142}
\ncline[]{n40}{n143}
\ncline[]{n40}{n144}
\ncline[]{n40}{n146}
\ncline[]{n40}{n147}
\ncline[]{n40}{n148}
\ncline[]{n41}{n42}
\ncline[]{n41}{n43}
\ncline[]{n41}{n44}
\ncline[]{n41}{n46}
\ncline[]{n41}{n47}
\ncline[]{n41}{n50}
\ncline[]{n41}{n51}
\ncline[]{n41}{n53}
\ncline[]{n41}{n55}
\ncline[]{n41}{n56}
\ncline[]{n41}{n57}
\ncline[]{n41}{n58}
\ncline[]{n41}{n59}
\ncline[]{n41}{n60}
\ncline[]{n41}{n61}
\ncline[]{n41}{n64}
\ncline[]{n41}{n65}
\ncline[]{n41}{n66}
\ncline[]{n41}{n70}
\ncline[]{n41}{n71}
\ncline[]{n41}{n72}
\ncline[]{n41}{n77}
\ncline[]{n41}{n79}
\ncline[]{n41}{n80}
\ncline[]{n41}{n82}
\ncline[]{n41}{n83}
\ncline[]{n41}{n84}
\ncline[]{n41}{n86}
\ncline[]{n41}{n87}
\ncline[]{n41}{n89}
\ncline[]{n41}{n92}
\ncline[]{n41}{n93}
\ncline[]{n41}{n94}
\ncline[]{n41}{n95}
\ncline[]{n41}{n97}
\ncline[]{n41}{n98}
\ncline[]{n41}{n99}
\ncline[]{n41}{n100}
\ncline[]{n41}{n101}
\ncline[]{n41}{n102}
\ncline[]{n41}{n103}
\ncline[]{n41}{n104}
\ncline[]{n41}{n107}
\ncline[]{n41}{n110}
\ncline[]{n41}{n112}
\ncline[]{n41}{n114}
\ncline[]{n41}{n116}
\ncline[]{n41}{n117}
\ncline[]{n41}{n118}
\ncline[]{n41}{n119}
\ncline[]{n41}{n120}
\ncline[]{n41}{n122}
\ncline[]{n41}{n123}
\ncline[]{n41}{n126}
\ncline[]{n41}{n128}
\ncline[]{n41}{n129}
\ncline[]{n41}{n131}
\ncline[]{n41}{n132}
\ncline[]{n41}{n137}
\ncline[]{n41}{n138}
\ncline[]{n41}{n139}
\ncline[]{n41}{n140}
\ncline[]{n41}{n141}
\ncline[]{n41}{n142}
\ncline[]{n41}{n143}
\ncline[]{n41}{n144}
\ncline[]{n41}{n146}
\ncline[]{n41}{n147}
\ncline[]{n41}{n148}
\ncline[]{n42}{n43}
\ncline[]{n42}{n44}
\ncline[]{n42}{n46}
\ncline[]{n42}{n47}
\ncline[]{n42}{n50}
\ncline[]{n42}{n51}
\ncline[]{n42}{n53}
\ncline[]{n42}{n55}
\ncline[]{n42}{n56}
\ncline[]{n42}{n57}
\ncline[]{n42}{n58}
\ncline[]{n42}{n59}
\ncline[]{n42}{n60}
\ncline[]{n42}{n61}
\ncline[]{n42}{n64}
\ncline[]{n42}{n65}
\ncline[]{n42}{n66}
\ncline[]{n42}{n70}
\ncline[]{n42}{n71}
\ncline[]{n42}{n72}
\ncline[]{n42}{n77}
\ncline[]{n42}{n79}
\ncline[]{n42}{n80}
\ncline[]{n42}{n82}
\ncline[]{n42}{n83}
\ncline[]{n42}{n84}
\ncline[]{n42}{n86}
\ncline[]{n42}{n87}
\ncline[]{n42}{n89}
\ncline[]{n42}{n92}
\ncline[]{n42}{n93}
\ncline[]{n42}{n94}
\ncline[]{n42}{n95}
\ncline[]{n42}{n97}
\ncline[]{n42}{n98}
\ncline[]{n42}{n99}
\ncline[]{n42}{n100}
\ncline[]{n42}{n101}
\ncline[]{n42}{n102}
\ncline[]{n42}{n103}
\ncline[]{n42}{n104}
\ncline[]{n42}{n107}
\ncline[]{n42}{n110}
\ncline[]{n42}{n112}
\ncline[]{n42}{n114}
\ncline[]{n42}{n116}
\ncline[]{n42}{n117}
\ncline[]{n42}{n118}
\ncline[]{n42}{n119}
\ncline[]{n42}{n120}
\ncline[]{n42}{n122}
\ncline[]{n42}{n123}
\ncline[]{n42}{n126}
\ncline[]{n42}{n128}
\ncline[]{n42}{n129}
\ncline[]{n42}{n131}
\ncline[]{n42}{n132}
\ncline[]{n42}{n137}
\ncline[]{n42}{n138}
\ncline[]{n42}{n139}
\ncline[]{n42}{n140}
\ncline[]{n42}{n141}
\ncline[]{n42}{n142}
\ncline[]{n42}{n143}
\ncline[]{n42}{n144}
\ncline[]{n42}{n146}
\ncline[]{n42}{n147}
\ncline[]{n42}{n148}
\ncline[]{n43}{n44}
\ncline[]{n43}{n46}
\ncline[]{n43}{n47}
\ncline[]{n43}{n50}
\ncline[]{n43}{n51}
\ncline[]{n43}{n53}
\ncline[]{n43}{n55}
\ncline[]{n43}{n56}
\ncline[]{n43}{n57}
\ncline[]{n43}{n58}
\ncline[]{n43}{n59}
\ncline[]{n43}{n60}
\ncline[]{n43}{n61}
\ncline[]{n43}{n64}
\ncline[]{n43}{n65}
\ncline[]{n43}{n66}
\ncline[]{n43}{n70}
\ncline[]{n43}{n71}
\ncline[]{n43}{n72}
\ncline[]{n43}{n77}
\ncline[]{n43}{n79}
\ncline[]{n43}{n80}
\ncline[]{n43}{n82}
\ncline[]{n43}{n83}
\ncline[]{n43}{n84}
\ncline[]{n43}{n86}
\ncline[]{n43}{n87}
\ncline[]{n43}{n89}
\ncline[]{n43}{n92}
\ncline[]{n43}{n93}
\ncline[]{n43}{n94}
\ncline[]{n43}{n95}
\ncline[]{n43}{n97}
\ncline[]{n43}{n98}
\ncline[]{n43}{n99}
\ncline[]{n43}{n100}
\ncline[]{n43}{n101}
\ncline[]{n43}{n102}
\ncline[]{n43}{n103}
\ncline[]{n43}{n104}
\ncline[]{n43}{n107}
\ncline[]{n43}{n110}
\ncline[]{n43}{n112}
\ncline[]{n43}{n114}
\ncline[]{n43}{n116}
\ncline[]{n43}{n117}
\ncline[]{n43}{n118}
\ncline[]{n43}{n119}
\ncline[]{n43}{n120}
\ncline[]{n43}{n122}
\ncline[]{n43}{n123}
\ncline[]{n43}{n126}
\ncline[]{n43}{n128}
\ncline[]{n43}{n129}
\ncline[]{n43}{n131}
\ncline[]{n43}{n132}
\ncline[]{n43}{n137}
\ncline[]{n43}{n138}
\ncline[]{n43}{n139}
\ncline[]{n43}{n140}
\ncline[]{n43}{n141}
\ncline[]{n43}{n142}
\ncline[]{n43}{n143}
\ncline[]{n43}{n144}
\ncline[]{n43}{n146}
\ncline[]{n43}{n147}
\ncline[]{n43}{n148}
\ncline[]{n44}{n46}
\ncline[]{n44}{n47}
\ncline[]{n44}{n50}
\ncline[]{n44}{n51}
\ncline[]{n44}{n53}
\ncline[]{n44}{n55}
\ncline[]{n44}{n56}
\ncline[]{n44}{n57}
\ncline[]{n44}{n58}
\ncline[]{n44}{n59}
\ncline[]{n44}{n60}
\ncline[]{n44}{n61}
\ncline[]{n44}{n64}
\ncline[]{n44}{n65}
\ncline[]{n44}{n66}
\ncline[]{n44}{n70}
\ncline[]{n44}{n71}
\ncline[]{n44}{n72}
\ncline[]{n44}{n77}
\ncline[]{n44}{n79}
\ncline[]{n44}{n80}
\ncline[]{n44}{n82}
\ncline[]{n44}{n83}
\ncline[]{n44}{n84}
\ncline[]{n44}{n86}
\ncline[]{n44}{n87}
\ncline[]{n44}{n89}
\ncline[]{n44}{n92}
\ncline[]{n44}{n93}
\ncline[]{n44}{n94}
\ncline[]{n44}{n95}
\ncline[]{n44}{n97}
\ncline[]{n44}{n98}
\ncline[]{n44}{n99}
\ncline[]{n44}{n100}
\ncline[]{n44}{n101}
\ncline[]{n44}{n102}
\ncline[]{n44}{n103}
\ncline[]{n44}{n104}
\ncline[]{n44}{n107}
\ncline[]{n44}{n110}
\ncline[]{n44}{n112}
\ncline[]{n44}{n114}
\ncline[]{n44}{n116}
\ncline[]{n44}{n117}
\ncline[]{n44}{n118}
\ncline[]{n44}{n119}
\ncline[]{n44}{n120}
\ncline[]{n44}{n122}
\ncline[]{n44}{n123}
\ncline[]{n44}{n126}
\ncline[]{n44}{n128}
\ncline[]{n44}{n129}
\ncline[]{n44}{n131}
\ncline[]{n44}{n132}
\ncline[]{n44}{n137}
\ncline[]{n44}{n138}
\ncline[]{n44}{n139}
\ncline[]{n44}{n140}
\ncline[]{n44}{n141}
\ncline[]{n44}{n142}
\ncline[]{n44}{n143}
\ncline[]{n44}{n144}
\ncline[]{n44}{n146}
\ncline[]{n44}{n147}
\ncline[]{n44}{n148}
\ncline[]{n45}{n52}
\ncline[]{n45}{n67}
\ncline[]{n45}{n74}
\ncline[]{n45}{n75}
\ncline[]{n45}{n76}
\ncline[]{n45}{n78}
\ncline[]{n45}{n81}
\ncline[]{n45}{n85}
\ncline[]{n45}{n90}
\ncline[]{n45}{n91}
\ncline[]{n45}{n96}
\ncline[]{n45}{n105}
\ncline[]{n45}{n106}
\ncline[]{n45}{n108}
\ncline[]{n45}{n109}
\ncline[]{n45}{n121}
\ncline[]{n45}{n125}
\ncline[]{n45}{n134}
\ncline[]{n46}{n47}
\ncline[]{n46}{n50}
\ncline[]{n46}{n51}
\ncline[]{n46}{n53}
\ncline[]{n46}{n55}
\ncline[]{n46}{n56}
\ncline[]{n46}{n57}
\ncline[]{n46}{n58}
\ncline[]{n46}{n59}
\ncline[]{n46}{n60}
\ncline[]{n46}{n61}
\ncline[]{n46}{n64}
\ncline[]{n46}{n65}
\ncline[]{n46}{n66}
\ncline[]{n46}{n70}
\ncline[]{n46}{n71}
\ncline[]{n46}{n72}
\ncline[]{n46}{n77}
\ncline[]{n46}{n79}
\ncline[]{n46}{n80}
\ncline[]{n46}{n82}
\ncline[]{n46}{n83}
\ncline[]{n46}{n84}
\ncline[]{n46}{n86}
\ncline[]{n46}{n87}
\ncline[]{n46}{n89}
\ncline[]{n46}{n92}
\ncline[]{n46}{n93}
\ncline[]{n46}{n94}
\ncline[]{n46}{n95}
\ncline[]{n46}{n97}
\ncline[]{n46}{n98}
\ncline[]{n46}{n99}
\ncline[]{n46}{n100}
\ncline[]{n46}{n101}
\ncline[]{n46}{n102}
\ncline[]{n46}{n103}
\ncline[]{n46}{n104}
\ncline[]{n46}{n107}
\ncline[]{n46}{n110}
\ncline[]{n46}{n112}
\ncline[]{n46}{n114}
\ncline[]{n46}{n116}
\ncline[]{n46}{n117}
\ncline[]{n46}{n118}
\ncline[]{n46}{n119}
\ncline[]{n46}{n120}
\ncline[]{n46}{n122}
\ncline[]{n46}{n123}
\ncline[]{n46}{n126}
\ncline[]{n46}{n128}
\ncline[]{n46}{n129}
\ncline[]{n46}{n131}
\ncline[]{n46}{n132}
\ncline[]{n46}{n137}
\ncline[]{n46}{n138}
\ncline[]{n46}{n139}
\ncline[]{n46}{n140}
\ncline[]{n46}{n141}
\ncline[]{n46}{n142}
\ncline[]{n46}{n143}
\ncline[]{n46}{n144}
\ncline[]{n46}{n146}
\ncline[]{n46}{n147}
\ncline[]{n46}{n148}
\ncline[]{n47}{n50}
\ncline[]{n47}{n51}
\ncline[]{n47}{n53}
\ncline[]{n47}{n55}
\ncline[]{n47}{n56}
\ncline[]{n47}{n57}
\ncline[]{n47}{n58}
\ncline[]{n47}{n59}
\ncline[]{n47}{n60}
\ncline[]{n47}{n61}
\ncline[]{n47}{n64}
\ncline[]{n47}{n65}
\ncline[]{n47}{n66}
\ncline[]{n47}{n70}
\ncline[]{n47}{n71}
\ncline[]{n47}{n72}
\ncline[]{n47}{n77}
\ncline[]{n47}{n79}
\ncline[]{n47}{n80}
\ncline[]{n47}{n82}
\ncline[]{n47}{n83}
\ncline[]{n47}{n84}
\ncline[]{n47}{n86}
\ncline[]{n47}{n87}
\ncline[]{n47}{n89}
\ncline[]{n47}{n92}
\ncline[]{n47}{n93}
\ncline[]{n47}{n94}
\ncline[]{n47}{n95}
\ncline[]{n47}{n97}
\ncline[]{n47}{n98}
\ncline[]{n47}{n99}
\ncline[]{n47}{n100}
\ncline[]{n47}{n101}
\ncline[]{n47}{n102}
\ncline[]{n47}{n103}
\ncline[]{n47}{n104}
\ncline[]{n47}{n107}
\ncline[]{n47}{n110}
\ncline[]{n47}{n112}
\ncline[]{n47}{n114}
\ncline[]{n47}{n116}
\ncline[]{n47}{n117}
\ncline[]{n47}{n118}
\ncline[]{n47}{n119}
\ncline[]{n47}{n120}
\ncline[]{n47}{n122}
\ncline[]{n47}{n123}
\ncline[]{n47}{n126}
\ncline[]{n47}{n128}
\ncline[]{n47}{n129}
\ncline[]{n47}{n131}
\ncline[]{n47}{n132}
\ncline[]{n47}{n137}
\ncline[]{n47}{n138}
\ncline[]{n47}{n139}
\ncline[]{n47}{n140}
\ncline[]{n47}{n141}
\ncline[]{n47}{n142}
\ncline[]{n47}{n143}
\ncline[]{n47}{n144}
\ncline[]{n47}{n146}
\ncline[]{n47}{n147}
\ncline[]{n47}{n148}
\ncline[]{n48}{n49}
\ncline[]{n48}{n54}
\ncline[]{n48}{n62}
\ncline[]{n48}{n63}
\ncline[]{n48}{n68}
\ncline[]{n48}{n69}
\ncline[]{n48}{n73}
\ncline[]{n48}{n88}
\ncline[]{n48}{n111}
\ncline[]{n48}{n113}
\ncline[]{n48}{n115}
\ncline[]{n48}{n124}
\ncline[]{n48}{n127}
\ncline[]{n48}{n130}
\ncline[]{n48}{n133}
\ncline[]{n48}{n135}
\ncline[]{n48}{n136}
\ncline[]{n48}{n145}
\ncline[]{n48}{n149}
\ncline[]{n49}{n54}
\ncline[]{n49}{n62}
\ncline[]{n49}{n63}
\ncline[]{n49}{n68}
\ncline[]{n49}{n69}
\ncline[]{n49}{n73}
\ncline[]{n49}{n88}
\ncline[]{n49}{n111}
\ncline[]{n49}{n113}
\ncline[]{n49}{n115}
\ncline[]{n49}{n124}
\ncline[]{n49}{n127}
\ncline[]{n49}{n130}
\ncline[]{n49}{n133}
\ncline[]{n49}{n135}
\ncline[]{n49}{n136}
\ncline[]{n49}{n145}
\ncline[]{n49}{n149}
\ncline[]{n50}{n51}
\ncline[]{n50}{n53}
\ncline[]{n50}{n55}
\ncline[]{n50}{n56}
\ncline[]{n50}{n57}
\ncline[]{n50}{n58}
\ncline[]{n50}{n59}
\ncline[]{n50}{n60}
\ncline[]{n50}{n61}
\ncline[]{n50}{n64}
\ncline[]{n50}{n65}
\ncline[]{n50}{n66}
\ncline[]{n50}{n70}
\ncline[]{n50}{n71}
\ncline[]{n50}{n72}
\ncline[]{n50}{n77}
\ncline[]{n50}{n79}
\ncline[]{n50}{n80}
\ncline[]{n50}{n82}
\ncline[]{n50}{n83}
\ncline[]{n50}{n84}
\ncline[]{n50}{n86}
\ncline[]{n50}{n87}
\ncline[]{n50}{n89}
\ncline[]{n50}{n92}
\ncline[]{n50}{n93}
\ncline[]{n50}{n94}
\ncline[]{n50}{n95}
\ncline[]{n50}{n97}
\ncline[]{n50}{n98}
\ncline[]{n50}{n99}
\ncline[]{n50}{n100}
\ncline[]{n50}{n101}
\ncline[]{n50}{n102}
\ncline[]{n50}{n103}
\ncline[]{n50}{n104}
\ncline[]{n50}{n107}
\ncline[]{n50}{n110}
\ncline[]{n50}{n112}
\ncline[]{n50}{n114}
\ncline[]{n50}{n116}
\ncline[]{n50}{n117}
\ncline[]{n50}{n118}
\ncline[]{n50}{n119}
\ncline[]{n50}{n120}
\ncline[]{n50}{n122}
\ncline[]{n50}{n123}
\ncline[]{n50}{n126}
\ncline[]{n50}{n128}
\ncline[]{n50}{n129}
\ncline[]{n50}{n131}
\ncline[]{n50}{n132}
\ncline[]{n50}{n137}
\ncline[]{n50}{n138}
\ncline[]{n50}{n139}
\ncline[]{n50}{n140}
\ncline[]{n50}{n141}
\ncline[]{n50}{n142}
\ncline[]{n50}{n143}
\ncline[]{n50}{n144}
\ncline[]{n50}{n146}
\ncline[]{n50}{n147}
\ncline[]{n50}{n148}
\ncline[]{n51}{n53}
\ncline[]{n51}{n55}
\ncline[]{n51}{n56}
\ncline[]{n51}{n57}
\ncline[]{n51}{n58}
\ncline[]{n51}{n59}
\ncline[]{n51}{n60}
\ncline[]{n51}{n61}
\ncline[]{n51}{n64}
\ncline[]{n51}{n65}
\ncline[]{n51}{n66}
\ncline[]{n51}{n70}
\ncline[]{n51}{n71}
\ncline[]{n51}{n72}
\ncline[]{n51}{n77}
\ncline[]{n51}{n79}
\ncline[]{n51}{n80}
\ncline[]{n51}{n82}
\ncline[]{n51}{n83}
\ncline[]{n51}{n84}
\ncline[]{n51}{n86}
\ncline[]{n51}{n87}
\ncline[]{n51}{n89}
\ncline[]{n51}{n92}
\ncline[]{n51}{n93}
\ncline[]{n51}{n94}
\ncline[]{n51}{n95}
\ncline[]{n51}{n97}
\ncline[]{n51}{n98}
\ncline[]{n51}{n99}
\ncline[]{n51}{n100}
\ncline[]{n51}{n101}
\ncline[]{n51}{n102}
\ncline[]{n51}{n103}
\ncline[]{n51}{n104}
\ncline[]{n51}{n107}
\ncline[]{n51}{n110}
\ncline[]{n51}{n112}
\ncline[]{n51}{n114}
\ncline[]{n51}{n116}
\ncline[]{n51}{n117}
\ncline[]{n51}{n118}
\ncline[]{n51}{n119}
\ncline[]{n51}{n120}
\ncline[]{n51}{n122}
\ncline[]{n51}{n123}
\ncline[]{n51}{n126}
\ncline[]{n51}{n128}
\ncline[]{n51}{n129}
\ncline[]{n51}{n131}
\ncline[]{n51}{n132}
\ncline[]{n51}{n137}
\ncline[]{n51}{n138}
\ncline[]{n51}{n139}
\ncline[]{n51}{n140}
\ncline[]{n51}{n141}
\ncline[]{n51}{n142}
\ncline[]{n51}{n143}
\ncline[]{n51}{n144}
\ncline[]{n51}{n146}
\ncline[]{n51}{n147}
\ncline[]{n51}{n148}
\ncline[]{n52}{n67}
\ncline[]{n52}{n74}
\ncline[]{n52}{n75}
\ncline[]{n52}{n76}
\ncline[]{n52}{n78}
\ncline[]{n52}{n81}
\ncline[]{n52}{n85}
\ncline[]{n52}{n90}
\ncline[]{n52}{n91}
\ncline[]{n52}{n96}
\ncline[]{n52}{n105}
\ncline[]{n52}{n106}
\ncline[]{n52}{n108}
\ncline[]{n52}{n109}
\ncline[]{n52}{n121}
\ncline[]{n52}{n125}
\ncline[]{n52}{n134}
\ncline[]{n53}{n55}
\ncline[]{n53}{n56}
\ncline[]{n53}{n57}
\ncline[]{n53}{n58}
\ncline[]{n53}{n59}
\ncline[]{n53}{n60}
\ncline[]{n53}{n61}
\ncline[]{n53}{n64}
\ncline[]{n53}{n65}
\ncline[]{n53}{n66}
\ncline[]{n53}{n70}
\ncline[]{n53}{n71}
\ncline[]{n53}{n72}
\ncline[]{n53}{n77}
\ncline[]{n53}{n79}
\ncline[]{n53}{n80}
\ncline[]{n53}{n82}
\ncline[]{n53}{n83}
\ncline[]{n53}{n84}
\ncline[]{n53}{n86}
\ncline[]{n53}{n87}
\ncline[]{n53}{n89}
\ncline[]{n53}{n92}
\ncline[]{n53}{n93}
\ncline[]{n53}{n94}
\ncline[]{n53}{n95}
\ncline[]{n53}{n97}
\ncline[]{n53}{n98}
\ncline[]{n53}{n99}
\ncline[]{n53}{n100}
\ncline[]{n53}{n101}
\ncline[]{n53}{n102}
\ncline[]{n53}{n103}
\ncline[]{n53}{n104}
\ncline[]{n53}{n107}
\ncline[]{n53}{n110}
\ncline[]{n53}{n112}
\ncline[]{n53}{n114}
\ncline[]{n53}{n116}
\ncline[]{n53}{n117}
\ncline[]{n53}{n118}
\ncline[]{n53}{n119}
\ncline[]{n53}{n120}
\ncline[]{n53}{n122}
\ncline[]{n53}{n123}
\ncline[]{n53}{n126}
\ncline[]{n53}{n128}
\ncline[]{n53}{n129}
\ncline[]{n53}{n131}
\ncline[]{n53}{n132}
\ncline[]{n53}{n137}
\ncline[]{n53}{n138}
\ncline[]{n53}{n139}
\ncline[]{n53}{n140}
\ncline[]{n53}{n141}
\ncline[]{n53}{n142}
\ncline[]{n53}{n143}
\ncline[]{n53}{n144}
\ncline[]{n53}{n146}
\ncline[]{n53}{n147}
\ncline[]{n53}{n148}
\ncline[]{n54}{n62}
\ncline[]{n54}{n63}
\ncline[]{n54}{n68}
\ncline[]{n54}{n69}
\ncline[]{n54}{n73}
\ncline[]{n54}{n88}
\ncline[]{n54}{n111}
\ncline[]{n54}{n113}
\ncline[]{n54}{n115}
\ncline[]{n54}{n124}
\ncline[]{n54}{n127}
\ncline[]{n54}{n130}
\ncline[]{n54}{n133}
\ncline[]{n54}{n135}
\ncline[]{n54}{n136}
\ncline[]{n54}{n145}
\ncline[]{n54}{n149}
\ncline[]{n55}{n56}
\ncline[]{n55}{n57}
\ncline[]{n55}{n58}
\ncline[]{n55}{n59}
\ncline[]{n55}{n60}
\ncline[]{n55}{n61}
\ncline[]{n55}{n64}
\ncline[]{n55}{n65}
\ncline[]{n55}{n66}
\ncline[]{n55}{n70}
\ncline[]{n55}{n71}
\ncline[]{n55}{n72}
\ncline[]{n55}{n77}
\ncline[]{n55}{n79}
\ncline[]{n55}{n80}
\ncline[]{n55}{n82}
\ncline[]{n55}{n83}
\ncline[]{n55}{n84}
\ncline[]{n55}{n86}
\ncline[]{n55}{n87}
\ncline[]{n55}{n89}
\ncline[]{n55}{n92}
\ncline[]{n55}{n93}
\ncline[]{n55}{n94}
\ncline[]{n55}{n95}
\ncline[]{n55}{n97}
\ncline[]{n55}{n98}
\ncline[]{n55}{n99}
\ncline[]{n55}{n100}
\ncline[]{n55}{n101}
\ncline[]{n55}{n102}
\ncline[]{n55}{n103}
\ncline[]{n55}{n104}
\ncline[]{n55}{n107}
\ncline[]{n55}{n110}
\ncline[]{n55}{n112}
\ncline[]{n55}{n114}
\ncline[]{n55}{n116}
\ncline[]{n55}{n117}
\ncline[]{n55}{n118}
\ncline[]{n55}{n119}
\ncline[]{n55}{n120}
\ncline[]{n55}{n122}
\ncline[]{n55}{n123}
\ncline[]{n55}{n126}
\ncline[]{n55}{n128}
\ncline[]{n55}{n129}
\ncline[]{n55}{n131}
\ncline[]{n55}{n132}
\ncline[]{n55}{n137}
\ncline[]{n55}{n138}
\ncline[]{n55}{n139}
\ncline[]{n55}{n140}
\ncline[]{n55}{n141}
\ncline[]{n55}{n142}
\ncline[]{n55}{n143}
\ncline[]{n55}{n144}
\ncline[]{n55}{n146}
\ncline[]{n55}{n147}
\ncline[]{n55}{n148}
\ncline[]{n56}{n57}
\ncline[]{n56}{n58}
\ncline[]{n56}{n59}
\ncline[]{n56}{n60}
\ncline[]{n56}{n61}
\ncline[]{n56}{n64}
\ncline[]{n56}{n65}
\ncline[]{n56}{n66}
\ncline[]{n56}{n70}
\ncline[]{n56}{n71}
\ncline[]{n56}{n72}
\ncline[]{n56}{n77}
\ncline[]{n56}{n79}
\ncline[]{n56}{n80}
\ncline[]{n56}{n82}
\ncline[]{n56}{n83}
\ncline[]{n56}{n84}
\ncline[]{n56}{n86}
\ncline[]{n56}{n87}
\ncline[]{n56}{n89}
\ncline[]{n56}{n92}
\ncline[]{n56}{n93}
\ncline[]{n56}{n94}
\ncline[]{n56}{n95}
\ncline[]{n56}{n97}
\ncline[]{n56}{n98}
\ncline[]{n56}{n99}
\ncline[]{n56}{n100}
\ncline[]{n56}{n101}
\ncline[]{n56}{n102}
\ncline[]{n56}{n103}
\ncline[]{n56}{n104}
\ncline[]{n56}{n107}
\ncline[]{n56}{n110}
\ncline[]{n56}{n112}
\ncline[]{n56}{n114}
\ncline[]{n56}{n116}
\ncline[]{n56}{n117}
\ncline[]{n56}{n118}
\ncline[]{n56}{n119}
\ncline[]{n56}{n120}
\ncline[]{n56}{n122}
\ncline[]{n56}{n123}
\ncline[]{n56}{n126}
\ncline[]{n56}{n128}
\ncline[]{n56}{n129}
\ncline[]{n56}{n131}
\ncline[]{n56}{n132}
\ncline[]{n56}{n137}
\ncline[]{n56}{n138}
\ncline[]{n56}{n139}
\ncline[]{n56}{n140}
\ncline[]{n56}{n141}
\ncline[]{n56}{n142}
\ncline[]{n56}{n143}
\ncline[]{n56}{n144}
\ncline[]{n56}{n146}
\ncline[]{n56}{n147}
\ncline[]{n56}{n148}
\ncline[]{n57}{n58}
\ncline[]{n57}{n59}
\ncline[]{n57}{n60}
\ncline[]{n57}{n61}
\ncline[]{n57}{n64}
\ncline[]{n57}{n65}
\ncline[]{n57}{n66}
\ncline[]{n57}{n70}
\ncline[]{n57}{n71}
\ncline[]{n57}{n72}
\ncline[]{n57}{n77}
\ncline[]{n57}{n79}
\ncline[]{n57}{n80}
\ncline[]{n57}{n82}
\ncline[]{n57}{n83}
\ncline[]{n57}{n84}
\ncline[]{n57}{n86}
\ncline[]{n57}{n87}
\ncline[]{n57}{n89}
\ncline[]{n57}{n92}
\ncline[]{n57}{n93}
\ncline[]{n57}{n94}
\ncline[]{n57}{n95}
\ncline[]{n57}{n97}
\ncline[]{n57}{n98}
\ncline[]{n57}{n99}
\ncline[]{n57}{n100}
\ncline[]{n57}{n101}
\ncline[]{n57}{n102}
\ncline[]{n57}{n103}
\ncline[]{n57}{n104}
\ncline[]{n57}{n107}
\ncline[]{n57}{n110}
\ncline[]{n57}{n112}
\ncline[]{n57}{n114}
\ncline[]{n57}{n116}
\ncline[]{n57}{n117}
\ncline[]{n57}{n118}
\ncline[]{n57}{n119}
\ncline[]{n57}{n120}
\ncline[]{n57}{n122}
\ncline[]{n57}{n123}
\ncline[]{n57}{n126}
\ncline[]{n57}{n128}
\ncline[]{n57}{n129}
\ncline[]{n57}{n131}
\ncline[]{n57}{n132}
\ncline[]{n57}{n137}
\ncline[]{n57}{n138}
\ncline[]{n57}{n139}
\ncline[]{n57}{n140}
\ncline[]{n57}{n141}
\ncline[]{n57}{n142}
\ncline[]{n57}{n143}
\ncline[]{n57}{n144}
\ncline[]{n57}{n146}
\ncline[]{n57}{n147}
\ncline[]{n57}{n148}
\ncline[]{n58}{n59}
\ncline[]{n58}{n60}
\ncline[]{n58}{n61}
\ncline[]{n58}{n64}
\ncline[]{n58}{n65}
\ncline[]{n58}{n66}
\ncline[]{n58}{n70}
\ncline[]{n58}{n71}
\ncline[]{n58}{n72}
\ncline[]{n58}{n77}
\ncline[]{n58}{n79}
\ncline[]{n58}{n80}
\ncline[]{n58}{n82}
\ncline[]{n58}{n83}
\ncline[]{n58}{n84}
\ncline[]{n58}{n86}
\ncline[]{n58}{n87}
\ncline[]{n58}{n89}
\ncline[]{n58}{n92}
\ncline[]{n58}{n93}
\ncline[]{n58}{n94}
\ncline[]{n58}{n95}
\ncline[]{n58}{n97}
\ncline[]{n58}{n98}
\ncline[]{n58}{n99}
\ncline[]{n58}{n100}
\ncline[]{n58}{n101}
\ncline[]{n58}{n102}
\ncline[]{n58}{n103}
\ncline[]{n58}{n104}
\ncline[]{n58}{n107}
\ncline[]{n58}{n110}
\ncline[]{n58}{n112}
\ncline[]{n58}{n114}
\ncline[]{n58}{n116}
\ncline[]{n58}{n117}
\ncline[]{n58}{n118}
\ncline[]{n58}{n119}
\ncline[]{n58}{n120}
\ncline[]{n58}{n122}
\ncline[]{n58}{n123}
\ncline[]{n58}{n126}
\ncline[]{n58}{n128}
\ncline[]{n58}{n129}
\ncline[]{n58}{n131}
\ncline[]{n58}{n132}
\ncline[]{n58}{n137}
\ncline[]{n58}{n138}
\ncline[]{n58}{n139}
\ncline[]{n58}{n140}
\ncline[]{n58}{n141}
\ncline[]{n58}{n142}
\ncline[]{n58}{n143}
\ncline[]{n58}{n144}
\ncline[]{n58}{n146}
\ncline[]{n58}{n147}
\ncline[]{n58}{n148}
\ncline[]{n59}{n60}
\ncline[]{n59}{n61}
\ncline[]{n59}{n64}
\ncline[]{n59}{n65}
\ncline[]{n59}{n66}
\ncline[]{n59}{n70}
\ncline[]{n59}{n71}
\ncline[]{n59}{n72}
\ncline[]{n59}{n77}
\ncline[]{n59}{n79}
\ncline[]{n59}{n80}
\ncline[]{n59}{n82}
\ncline[]{n59}{n83}
\ncline[]{n59}{n84}
\ncline[]{n59}{n86}
\ncline[]{n59}{n87}
\ncline[]{n59}{n89}
\ncline[]{n59}{n92}
\ncline[]{n59}{n93}
\ncline[]{n59}{n94}
\ncline[]{n59}{n95}
\ncline[]{n59}{n97}
\ncline[]{n59}{n98}
\ncline[]{n59}{n99}
\ncline[]{n59}{n100}
\ncline[]{n59}{n101}
\ncline[]{n59}{n102}
\ncline[]{n59}{n103}
\ncline[]{n59}{n104}
\ncline[]{n59}{n107}
\ncline[]{n59}{n110}
\ncline[]{n59}{n112}
\ncline[]{n59}{n114}
\ncline[]{n59}{n116}
\ncline[]{n59}{n117}
\ncline[]{n59}{n118}
\ncline[]{n59}{n119}
\ncline[]{n59}{n120}
\ncline[]{n59}{n122}
\ncline[]{n59}{n123}
\ncline[]{n59}{n126}
\ncline[]{n59}{n128}
\ncline[]{n59}{n129}
\ncline[]{n59}{n131}
\ncline[]{n59}{n132}
\ncline[]{n59}{n137}
\ncline[]{n59}{n138}
\ncline[]{n59}{n139}
\ncline[]{n59}{n140}
\ncline[]{n59}{n141}
\ncline[]{n59}{n142}
\ncline[]{n59}{n143}
\ncline[]{n59}{n144}
\ncline[]{n59}{n146}
\ncline[]{n59}{n147}
\ncline[]{n59}{n148}
\ncline[]{n60}{n61}
\ncline[]{n60}{n64}
\ncline[]{n60}{n65}
\ncline[]{n60}{n66}
\ncline[]{n60}{n70}
\ncline[]{n60}{n71}
\ncline[]{n60}{n72}
\ncline[]{n60}{n77}
\ncline[]{n60}{n79}
\ncline[]{n60}{n80}
\ncline[]{n60}{n82}
\ncline[]{n60}{n83}
\ncline[]{n60}{n84}
\ncline[]{n60}{n86}
\ncline[]{n60}{n87}
\ncline[]{n60}{n89}
\ncline[]{n60}{n92}
\ncline[]{n60}{n93}
\ncline[]{n60}{n94}
\ncline[]{n60}{n95}
\ncline[]{n60}{n97}
\ncline[]{n60}{n98}
\ncline[]{n60}{n99}
\ncline[]{n60}{n100}
\ncline[]{n60}{n101}
\ncline[]{n60}{n102}
\ncline[]{n60}{n103}
\ncline[]{n60}{n104}
\ncline[]{n60}{n107}
\ncline[]{n60}{n110}
\ncline[]{n60}{n112}
\ncline[]{n60}{n114}
\ncline[]{n60}{n116}
\ncline[]{n60}{n117}
\ncline[]{n60}{n118}
\ncline[]{n60}{n119}
\ncline[]{n60}{n120}
\ncline[]{n60}{n122}
\ncline[]{n60}{n123}
\ncline[]{n60}{n126}
\ncline[]{n60}{n128}
\ncline[]{n60}{n129}
\ncline[]{n60}{n131}
\ncline[]{n60}{n132}
\ncline[]{n60}{n137}
\ncline[]{n60}{n138}
\ncline[]{n60}{n139}
\ncline[]{n60}{n140}
\ncline[]{n60}{n141}
\ncline[]{n60}{n142}
\ncline[]{n60}{n143}
\ncline[]{n60}{n144}
\ncline[]{n60}{n146}
\ncline[]{n60}{n147}
\ncline[]{n60}{n148}
\ncline[]{n61}{n64}
\ncline[]{n61}{n65}
\ncline[]{n61}{n66}
\ncline[]{n61}{n70}
\ncline[]{n61}{n71}
\ncline[]{n61}{n72}
\ncline[]{n61}{n77}
\ncline[]{n61}{n79}
\ncline[]{n61}{n80}
\ncline[]{n61}{n82}
\ncline[]{n61}{n83}
\ncline[]{n61}{n84}
\ncline[]{n61}{n86}
\ncline[]{n61}{n87}
\ncline[]{n61}{n89}
\ncline[]{n61}{n92}
\ncline[]{n61}{n93}
\ncline[]{n61}{n94}
\ncline[]{n61}{n95}
\ncline[]{n61}{n97}
\ncline[]{n61}{n98}
\ncline[]{n61}{n99}
\ncline[]{n61}{n100}
\ncline[]{n61}{n101}
\ncline[]{n61}{n102}
\ncline[]{n61}{n103}
\ncline[]{n61}{n104}
\ncline[]{n61}{n107}
\ncline[]{n61}{n110}
\ncline[]{n61}{n112}
\ncline[]{n61}{n114}
\ncline[]{n61}{n116}
\ncline[]{n61}{n117}
\ncline[]{n61}{n118}
\ncline[]{n61}{n119}
\ncline[]{n61}{n120}
\ncline[]{n61}{n122}
\ncline[]{n61}{n123}
\ncline[]{n61}{n126}
\ncline[]{n61}{n128}
\ncline[]{n61}{n129}
\ncline[]{n61}{n131}
\ncline[]{n61}{n132}
\ncline[]{n61}{n137}
\ncline[]{n61}{n138}
\ncline[]{n61}{n139}
\ncline[]{n61}{n140}
\ncline[]{n61}{n141}
\ncline[]{n61}{n142}
\ncline[]{n61}{n143}
\ncline[]{n61}{n144}
\ncline[]{n61}{n146}
\ncline[]{n61}{n147}
\ncline[]{n61}{n148}
\ncline[]{n62}{n63}
\ncline[]{n62}{n68}
\ncline[]{n62}{n69}
\ncline[]{n62}{n73}
\ncline[]{n62}{n88}
\ncline[]{n62}{n111}
\ncline[]{n62}{n113}
\ncline[]{n62}{n115}
\ncline[]{n62}{n124}
\ncline[]{n62}{n127}
\ncline[]{n62}{n130}
\ncline[]{n62}{n133}
\ncline[]{n62}{n135}
\ncline[]{n62}{n136}
\ncline[]{n62}{n145}
\ncline[]{n62}{n149}
\ncline[]{n63}{n68}
\ncline[]{n63}{n69}
\ncline[]{n63}{n73}
\ncline[]{n63}{n88}
\ncline[]{n63}{n111}
\ncline[]{n63}{n113}
\ncline[]{n63}{n115}
\ncline[]{n63}{n124}
\ncline[]{n63}{n127}
\ncline[]{n63}{n130}
\ncline[]{n63}{n133}
\ncline[]{n63}{n135}
\ncline[]{n63}{n136}
\ncline[]{n63}{n145}
\ncline[]{n63}{n149}
\ncline[]{n64}{n65}
\ncline[]{n64}{n66}
\ncline[]{n64}{n70}
\ncline[]{n64}{n71}
\ncline[]{n64}{n72}
\ncline[]{n64}{n77}
\ncline[]{n64}{n79}
\ncline[]{n64}{n80}
\ncline[]{n64}{n82}
\ncline[]{n64}{n83}
\ncline[]{n64}{n84}
\ncline[]{n64}{n86}
\ncline[]{n64}{n87}
\ncline[]{n64}{n89}
\ncline[]{n64}{n92}
\ncline[]{n64}{n93}
\ncline[]{n64}{n94}
\ncline[]{n64}{n95}
\ncline[]{n64}{n97}
\ncline[]{n64}{n98}
\ncline[]{n64}{n99}
\ncline[]{n64}{n100}
\ncline[]{n64}{n101}
\ncline[]{n64}{n102}
\ncline[]{n64}{n103}
\ncline[]{n64}{n104}
\ncline[]{n64}{n107}
\ncline[]{n64}{n110}
\ncline[]{n64}{n112}
\ncline[]{n64}{n114}
\ncline[]{n64}{n116}
\ncline[]{n64}{n117}
\ncline[]{n64}{n118}
\ncline[]{n64}{n119}
\ncline[]{n64}{n120}
\ncline[]{n64}{n122}
\ncline[]{n64}{n123}
\ncline[]{n64}{n126}
\ncline[]{n64}{n128}
\ncline[]{n64}{n129}
\ncline[]{n64}{n131}
\ncline[]{n64}{n132}
\ncline[]{n64}{n137}
\ncline[]{n64}{n138}
\ncline[]{n64}{n139}
\ncline[]{n64}{n140}
\ncline[]{n64}{n141}
\ncline[]{n64}{n142}
\ncline[]{n64}{n143}
\ncline[]{n64}{n144}
\ncline[]{n64}{n146}
\ncline[]{n64}{n147}
\ncline[]{n64}{n148}
\ncline[]{n65}{n66}
\ncline[]{n65}{n70}
\ncline[]{n65}{n71}
\ncline[]{n65}{n72}
\ncline[]{n65}{n77}
\ncline[]{n65}{n79}
\ncline[]{n65}{n80}
\ncline[]{n65}{n82}
\ncline[]{n65}{n83}
\ncline[]{n65}{n84}
\ncline[]{n65}{n86}
\ncline[]{n65}{n87}
\ncline[]{n65}{n89}
\ncline[]{n65}{n92}
\ncline[]{n65}{n93}
\ncline[]{n65}{n94}
\ncline[]{n65}{n95}
\ncline[]{n65}{n97}
\ncline[]{n65}{n98}
\ncline[]{n65}{n99}
\ncline[]{n65}{n100}
\ncline[]{n65}{n101}
\ncline[]{n65}{n102}
\ncline[]{n65}{n103}
\ncline[]{n65}{n104}
\ncline[]{n65}{n107}
\ncline[]{n65}{n110}
\ncline[]{n65}{n112}
\ncline[]{n65}{n114}
\ncline[]{n65}{n116}
\ncline[]{n65}{n117}
\ncline[]{n65}{n118}
\ncline[]{n65}{n119}
\ncline[]{n65}{n120}
\ncline[]{n65}{n122}
\ncline[]{n65}{n123}
\ncline[]{n65}{n126}
\ncline[]{n65}{n128}
\ncline[]{n65}{n129}
\ncline[]{n65}{n131}
\ncline[]{n65}{n132}
\ncline[]{n65}{n137}
\ncline[]{n65}{n138}
\ncline[]{n65}{n139}
\ncline[]{n65}{n140}
\ncline[]{n65}{n141}
\ncline[]{n65}{n142}
\ncline[]{n65}{n143}
\ncline[]{n65}{n144}
\ncline[]{n65}{n146}
\ncline[]{n65}{n147}
\ncline[]{n65}{n148}
\ncline[]{n66}{n70}
\ncline[]{n66}{n71}
\ncline[]{n66}{n72}
\ncline[]{n66}{n77}
\ncline[]{n66}{n79}
\ncline[]{n66}{n80}
\ncline[]{n66}{n82}
\ncline[]{n66}{n83}
\ncline[]{n66}{n84}
\ncline[]{n66}{n86}
\ncline[]{n66}{n87}
\ncline[]{n66}{n89}
\ncline[]{n66}{n92}
\ncline[]{n66}{n93}
\ncline[]{n66}{n94}
\ncline[]{n66}{n95}
\ncline[]{n66}{n97}
\ncline[]{n66}{n98}
\ncline[]{n66}{n99}
\ncline[]{n66}{n100}
\ncline[]{n66}{n101}
\ncline[]{n66}{n102}
\ncline[]{n66}{n103}
\ncline[]{n66}{n104}
\ncline[]{n66}{n107}
\ncline[]{n66}{n110}
\ncline[]{n66}{n112}
\ncline[]{n66}{n114}
\ncline[]{n66}{n116}
\ncline[]{n66}{n117}
\ncline[]{n66}{n118}
\ncline[]{n66}{n119}
\ncline[]{n66}{n120}
\ncline[]{n66}{n122}
\ncline[]{n66}{n123}
\ncline[]{n66}{n126}
\ncline[]{n66}{n128}
\ncline[]{n66}{n129}
\ncline[]{n66}{n131}
\ncline[]{n66}{n132}
\ncline[]{n66}{n137}
\ncline[]{n66}{n138}
\ncline[]{n66}{n139}
\ncline[]{n66}{n140}
\ncline[]{n66}{n141}
\ncline[]{n66}{n142}
\ncline[]{n66}{n143}
\ncline[]{n66}{n144}
\ncline[]{n66}{n146}
\ncline[]{n66}{n147}
\ncline[]{n66}{n148}
\ncline[]{n67}{n74}
\ncline[]{n67}{n75}
\ncline[]{n67}{n76}
\ncline[]{n67}{n78}
\ncline[]{n67}{n81}
\ncline[]{n67}{n85}
\ncline[]{n67}{n90}
\ncline[]{n67}{n91}
\ncline[]{n67}{n96}
\ncline[]{n67}{n105}
\ncline[]{n67}{n106}
\ncline[]{n67}{n108}
\ncline[]{n67}{n109}
\ncline[]{n67}{n121}
\ncline[]{n67}{n125}
\ncline[]{n67}{n134}
\ncline[]{n68}{n69}
\ncline[]{n68}{n73}
\ncline[]{n68}{n88}
\ncline[]{n68}{n111}
\ncline[]{n68}{n113}
\ncline[]{n68}{n115}
\ncline[]{n68}{n124}
\ncline[]{n68}{n127}
\ncline[]{n68}{n130}
\ncline[]{n68}{n133}
\ncline[]{n68}{n135}
\ncline[]{n68}{n136}
\ncline[]{n68}{n145}
\ncline[]{n68}{n149}
\ncline[]{n69}{n73}
\ncline[]{n69}{n88}
\ncline[]{n69}{n111}
\ncline[]{n69}{n113}
\ncline[]{n69}{n115}
\ncline[]{n69}{n124}
\ncline[]{n69}{n127}
\ncline[]{n69}{n130}
\ncline[]{n69}{n133}
\ncline[]{n69}{n135}
\ncline[]{n69}{n136}
\ncline[]{n69}{n145}
\ncline[]{n69}{n149}
\ncline[]{n70}{n71}
\ncline[]{n70}{n72}
\ncline[]{n70}{n77}
\ncline[]{n70}{n79}
\ncline[]{n70}{n80}
\ncline[]{n70}{n82}
\ncline[]{n70}{n83}
\ncline[]{n70}{n84}
\ncline[]{n70}{n86}
\ncline[]{n70}{n87}
\ncline[]{n70}{n89}
\ncline[]{n70}{n92}
\ncline[]{n70}{n93}
\ncline[]{n70}{n94}
\ncline[]{n70}{n95}
\ncline[]{n70}{n97}
\ncline[]{n70}{n98}
\ncline[]{n70}{n99}
\ncline[]{n70}{n100}
\ncline[]{n70}{n101}
\ncline[]{n70}{n102}
\ncline[]{n70}{n103}
\ncline[]{n70}{n104}
\ncline[]{n70}{n107}
\ncline[]{n70}{n110}
\ncline[]{n70}{n112}
\ncline[]{n70}{n114}
\ncline[]{n70}{n116}
\ncline[]{n70}{n117}
\ncline[]{n70}{n118}
\ncline[]{n70}{n119}
\ncline[]{n70}{n120}
\ncline[]{n70}{n122}
\ncline[]{n70}{n123}
\ncline[]{n70}{n126}
\ncline[]{n70}{n128}
\ncline[]{n70}{n129}
\ncline[]{n70}{n131}
\ncline[]{n70}{n132}
\ncline[]{n70}{n137}
\ncline[]{n70}{n138}
\ncline[]{n70}{n139}
\ncline[]{n70}{n140}
\ncline[]{n70}{n141}
\ncline[]{n70}{n142}
\ncline[]{n70}{n143}
\ncline[]{n70}{n144}
\ncline[]{n70}{n146}
\ncline[]{n70}{n147}
\ncline[]{n70}{n148}
\ncline[]{n71}{n72}
\ncline[]{n71}{n77}
\ncline[]{n71}{n79}
\ncline[]{n71}{n80}
\ncline[]{n71}{n82}
\ncline[]{n71}{n83}
\ncline[]{n71}{n84}
\ncline[]{n71}{n86}
\ncline[]{n71}{n87}
\ncline[]{n71}{n89}
\ncline[]{n71}{n92}
\ncline[]{n71}{n93}
\ncline[]{n71}{n94}
\ncline[]{n71}{n95}
\ncline[]{n71}{n97}
\ncline[]{n71}{n98}
\ncline[]{n71}{n99}
\ncline[]{n71}{n100}
\ncline[]{n71}{n101}
\ncline[]{n71}{n102}
\ncline[]{n71}{n103}
\ncline[]{n71}{n104}
\ncline[]{n71}{n107}
\ncline[]{n71}{n110}
\ncline[]{n71}{n112}
\ncline[]{n71}{n114}
\ncline[]{n71}{n116}
\ncline[]{n71}{n117}
\ncline[]{n71}{n118}
\ncline[]{n71}{n119}
\ncline[]{n71}{n120}
\ncline[]{n71}{n122}
\ncline[]{n71}{n123}
\ncline[]{n71}{n126}
\ncline[]{n71}{n128}
\ncline[]{n71}{n129}
\ncline[]{n71}{n131}
\ncline[]{n71}{n132}
\ncline[]{n71}{n137}
\ncline[]{n71}{n138}
\ncline[]{n71}{n139}
\ncline[]{n71}{n140}
\ncline[]{n71}{n141}
\ncline[]{n71}{n142}
\ncline[]{n71}{n143}
\ncline[]{n71}{n144}
\ncline[]{n71}{n146}
\ncline[]{n71}{n147}
\ncline[]{n71}{n148}
\ncline[]{n72}{n77}
\ncline[]{n72}{n79}
\ncline[]{n72}{n80}
\ncline[]{n72}{n82}
\ncline[]{n72}{n83}
\ncline[]{n72}{n84}
\ncline[]{n72}{n86}
\ncline[]{n72}{n87}
\ncline[]{n72}{n89}
\ncline[]{n72}{n92}
\ncline[]{n72}{n93}
\ncline[]{n72}{n94}
\ncline[]{n72}{n95}
\ncline[]{n72}{n97}
\ncline[]{n72}{n98}
\ncline[]{n72}{n99}
\ncline[]{n72}{n100}
\ncline[]{n72}{n101}
\ncline[]{n72}{n102}
\ncline[]{n72}{n103}
\ncline[]{n72}{n104}
\ncline[]{n72}{n107}
\ncline[]{n72}{n110}
\ncline[]{n72}{n112}
\ncline[]{n72}{n114}
\ncline[]{n72}{n116}
\ncline[]{n72}{n117}
\ncline[]{n72}{n118}
\ncline[]{n72}{n119}
\ncline[]{n72}{n120}
\ncline[]{n72}{n122}
\ncline[]{n72}{n123}
\ncline[]{n72}{n126}
\ncline[]{n72}{n128}
\ncline[]{n72}{n129}
\ncline[]{n72}{n131}
\ncline[]{n72}{n132}
\ncline[]{n72}{n137}
\ncline[]{n72}{n138}
\ncline[]{n72}{n139}
\ncline[]{n72}{n140}
\ncline[]{n72}{n141}
\ncline[]{n72}{n142}
\ncline[]{n72}{n143}
\ncline[]{n72}{n144}
\ncline[]{n72}{n146}
\ncline[]{n72}{n147}
\ncline[]{n72}{n148}
\ncline[]{n73}{n88}
\ncline[]{n73}{n111}
\ncline[]{n73}{n113}
\ncline[]{n73}{n115}
\ncline[]{n73}{n124}
\ncline[]{n73}{n127}
\ncline[]{n73}{n130}
\ncline[]{n73}{n133}
\ncline[]{n73}{n135}
\ncline[]{n73}{n136}
\ncline[]{n73}{n145}
\ncline[]{n73}{n149}
\ncline[]{n74}{n75}
\ncline[]{n74}{n76}
\ncline[]{n74}{n78}
\ncline[]{n74}{n81}
\ncline[]{n74}{n85}
\ncline[]{n74}{n90}
\ncline[]{n74}{n91}
\ncline[]{n74}{n96}
\ncline[]{n74}{n105}
\ncline[]{n74}{n106}
\ncline[]{n74}{n108}
\ncline[]{n74}{n109}
\ncline[]{n74}{n121}
\ncline[]{n74}{n125}
\ncline[]{n74}{n134}
\ncline[]{n75}{n76}
\ncline[]{n75}{n78}
\ncline[]{n75}{n81}
\ncline[]{n75}{n85}
\ncline[]{n75}{n90}
\ncline[]{n75}{n91}
\ncline[]{n75}{n96}
\ncline[]{n75}{n105}
\ncline[]{n75}{n106}
\ncline[]{n75}{n108}
\ncline[]{n75}{n109}
\ncline[]{n75}{n121}
\ncline[]{n75}{n125}
\ncline[]{n75}{n134}
\ncline[]{n76}{n78}
\ncline[]{n76}{n81}
\ncline[]{n76}{n85}
\ncline[]{n76}{n90}
\ncline[]{n76}{n91}
\ncline[]{n76}{n96}
\ncline[]{n76}{n105}
\ncline[]{n76}{n106}
\ncline[]{n76}{n108}
\ncline[]{n76}{n109}
\ncline[]{n76}{n121}
\ncline[]{n76}{n125}
\ncline[]{n76}{n134}
\ncline[]{n77}{n79}
\ncline[]{n77}{n80}
\ncline[]{n77}{n82}
\ncline[]{n77}{n83}
\ncline[]{n77}{n84}
\ncline[]{n77}{n86}
\ncline[]{n77}{n87}
\ncline[]{n77}{n89}
\ncline[]{n77}{n92}
\ncline[]{n77}{n93}
\ncline[]{n77}{n94}
\ncline[]{n77}{n95}
\ncline[]{n77}{n97}
\ncline[]{n77}{n98}
\ncline[]{n77}{n99}
\ncline[]{n77}{n100}
\ncline[]{n77}{n101}
\ncline[]{n77}{n102}
\ncline[]{n77}{n103}
\ncline[]{n77}{n104}
\ncline[]{n77}{n107}
\ncline[]{n77}{n110}
\ncline[]{n77}{n112}
\ncline[]{n77}{n114}
\ncline[]{n77}{n116}
\ncline[]{n77}{n117}
\ncline[]{n77}{n118}
\ncline[]{n77}{n119}
\ncline[]{n77}{n120}
\ncline[]{n77}{n122}
\ncline[]{n77}{n123}
\ncline[]{n77}{n126}
\ncline[]{n77}{n128}
\ncline[]{n77}{n129}
\ncline[]{n77}{n131}
\ncline[]{n77}{n132}
\ncline[]{n77}{n137}
\ncline[]{n77}{n138}
\ncline[]{n77}{n139}
\ncline[]{n77}{n140}
\ncline[]{n77}{n141}
\ncline[]{n77}{n142}
\ncline[]{n77}{n143}
\ncline[]{n77}{n144}
\ncline[]{n77}{n146}
\ncline[]{n77}{n147}
\ncline[]{n77}{n148}
\ncline[]{n78}{n81}
\ncline[]{n78}{n85}
\ncline[]{n78}{n90}
\ncline[]{n78}{n91}
\ncline[]{n78}{n96}
\ncline[]{n78}{n105}
\ncline[]{n78}{n106}
\ncline[]{n78}{n108}
\ncline[]{n78}{n109}
\ncline[]{n78}{n121}
\ncline[]{n78}{n125}
\ncline[]{n78}{n134}
\ncline[]{n79}{n80}
\ncline[]{n79}{n82}
\ncline[]{n79}{n83}
\ncline[]{n79}{n84}
\ncline[]{n79}{n86}
\ncline[]{n79}{n87}
\ncline[]{n79}{n89}
\ncline[]{n79}{n92}
\ncline[]{n79}{n93}
\ncline[]{n79}{n94}
\ncline[]{n79}{n95}
\ncline[]{n79}{n97}
\ncline[]{n79}{n98}
\ncline[]{n79}{n99}
\ncline[]{n79}{n100}
\ncline[]{n79}{n101}
\ncline[]{n79}{n102}
\ncline[]{n79}{n103}
\ncline[]{n79}{n104}
\ncline[]{n79}{n107}
\ncline[]{n79}{n110}
\ncline[]{n79}{n112}
\ncline[]{n79}{n114}
\ncline[]{n79}{n116}
\ncline[]{n79}{n117}
\ncline[]{n79}{n118}
\ncline[]{n79}{n119}
\ncline[]{n79}{n120}
\ncline[]{n79}{n122}
\ncline[]{n79}{n123}
\ncline[]{n79}{n126}
\ncline[]{n79}{n128}
\ncline[]{n79}{n129}
\ncline[]{n79}{n131}
\ncline[]{n79}{n132}
\ncline[]{n79}{n137}
\ncline[]{n79}{n138}
\ncline[]{n79}{n139}
\ncline[]{n79}{n140}
\ncline[]{n79}{n141}
\ncline[]{n79}{n142}
\ncline[]{n79}{n143}
\ncline[]{n79}{n144}
\ncline[]{n79}{n146}
\ncline[]{n79}{n147}
\ncline[]{n79}{n148}
\ncline[]{n80}{n82}
\ncline[]{n80}{n83}
\ncline[]{n80}{n84}
\ncline[]{n80}{n86}
\ncline[]{n80}{n87}
\ncline[]{n80}{n89}
\ncline[]{n80}{n92}
\ncline[]{n80}{n93}
\ncline[]{n80}{n94}
\ncline[]{n80}{n95}
\ncline[]{n80}{n97}
\ncline[]{n80}{n98}
\ncline[]{n80}{n99}
\ncline[]{n80}{n100}
\ncline[]{n80}{n101}
\ncline[]{n80}{n102}
\ncline[]{n80}{n103}
\ncline[]{n80}{n104}
\ncline[]{n80}{n107}
\ncline[]{n80}{n110}
\ncline[]{n80}{n112}
\ncline[]{n80}{n114}
\ncline[]{n80}{n116}
\ncline[]{n80}{n117}
\ncline[]{n80}{n118}
\ncline[]{n80}{n119}
\ncline[]{n80}{n120}
\ncline[]{n80}{n122}
\ncline[]{n80}{n123}
\ncline[]{n80}{n126}
\ncline[]{n80}{n128}
\ncline[]{n80}{n129}
\ncline[]{n80}{n131}
\ncline[]{n80}{n132}
\ncline[]{n80}{n137}
\ncline[]{n80}{n138}
\ncline[]{n80}{n139}
\ncline[]{n80}{n140}
\ncline[]{n80}{n141}
\ncline[]{n80}{n142}
\ncline[]{n80}{n143}
\ncline[]{n80}{n144}
\ncline[]{n80}{n146}
\ncline[]{n80}{n147}
\ncline[]{n80}{n148}
\ncline[]{n81}{n85}
\ncline[]{n81}{n90}
\ncline[]{n81}{n91}
\ncline[]{n81}{n96}
\ncline[]{n81}{n105}
\ncline[]{n81}{n106}
\ncline[]{n81}{n108}
\ncline[]{n81}{n109}
\ncline[]{n81}{n121}
\ncline[]{n81}{n125}
\ncline[]{n81}{n134}
\ncline[]{n82}{n83}
\ncline[]{n82}{n84}
\ncline[]{n82}{n86}
\ncline[]{n82}{n87}
\ncline[]{n82}{n89}
\ncline[]{n82}{n92}
\ncline[]{n82}{n93}
\ncline[]{n82}{n94}
\ncline[]{n82}{n95}
\ncline[]{n82}{n97}
\ncline[]{n82}{n98}
\ncline[]{n82}{n99}
\ncline[]{n82}{n100}
\ncline[]{n82}{n101}
\ncline[]{n82}{n102}
\ncline[]{n82}{n103}
\ncline[]{n82}{n104}
\ncline[]{n82}{n107}
\ncline[]{n82}{n110}
\ncline[]{n82}{n112}
\ncline[]{n82}{n114}
\ncline[]{n82}{n116}
\ncline[]{n82}{n117}
\ncline[]{n82}{n118}
\ncline[]{n82}{n119}
\ncline[]{n82}{n120}
\ncline[]{n82}{n122}
\ncline[]{n82}{n123}
\ncline[]{n82}{n126}
\ncline[]{n82}{n128}
\ncline[]{n82}{n129}
\ncline[]{n82}{n131}
\ncline[]{n82}{n132}
\ncline[]{n82}{n137}
\ncline[]{n82}{n138}
\ncline[]{n82}{n139}
\ncline[]{n82}{n140}
\ncline[]{n82}{n141}
\ncline[]{n82}{n142}
\ncline[]{n82}{n143}
\ncline[]{n82}{n144}
\ncline[]{n82}{n146}
\ncline[]{n82}{n147}
\ncline[]{n82}{n148}
\ncline[]{n83}{n84}
\ncline[]{n83}{n86}
\ncline[]{n83}{n87}
\ncline[]{n83}{n89}
\ncline[]{n83}{n92}
\ncline[]{n83}{n93}
\ncline[]{n83}{n94}
\ncline[]{n83}{n95}
\ncline[]{n83}{n97}
\ncline[]{n83}{n98}
\ncline[]{n83}{n99}
\ncline[]{n83}{n100}
\ncline[]{n83}{n101}
\ncline[]{n83}{n102}
\ncline[]{n83}{n103}
\ncline[]{n83}{n104}
\ncline[]{n83}{n107}
\ncline[]{n83}{n110}
\ncline[]{n83}{n112}
\ncline[]{n83}{n114}
\ncline[]{n83}{n116}
\ncline[]{n83}{n117}
\ncline[]{n83}{n118}
\ncline[]{n83}{n119}
\ncline[]{n83}{n120}
\ncline[]{n83}{n122}
\ncline[]{n83}{n123}
\ncline[]{n83}{n126}
\ncline[]{n83}{n128}
\ncline[]{n83}{n129}
\ncline[]{n83}{n131}
\ncline[]{n83}{n132}
\ncline[]{n83}{n137}
\ncline[]{n83}{n138}
\ncline[]{n83}{n139}
\ncline[]{n83}{n140}
\ncline[]{n83}{n141}
\ncline[]{n83}{n142}
\ncline[]{n83}{n143}
\ncline[]{n83}{n144}
\ncline[]{n83}{n146}
\ncline[]{n83}{n147}
\ncline[]{n83}{n148}
\ncline[]{n84}{n86}
\ncline[]{n84}{n87}
\ncline[]{n84}{n89}
\ncline[]{n84}{n92}
\ncline[]{n84}{n93}
\ncline[]{n84}{n94}
\ncline[]{n84}{n95}
\ncline[]{n84}{n97}
\ncline[]{n84}{n98}
\ncline[]{n84}{n99}
\ncline[]{n84}{n100}
\ncline[]{n84}{n101}
\ncline[]{n84}{n102}
\ncline[]{n84}{n103}
\ncline[]{n84}{n104}
\ncline[]{n84}{n107}
\ncline[]{n84}{n110}
\ncline[]{n84}{n112}
\ncline[]{n84}{n114}
\ncline[]{n84}{n116}
\ncline[]{n84}{n117}
\ncline[]{n84}{n118}
\ncline[]{n84}{n119}
\ncline[]{n84}{n120}
\ncline[]{n84}{n122}
\ncline[]{n84}{n123}
\ncline[]{n84}{n126}
\ncline[]{n84}{n128}
\ncline[]{n84}{n129}
\ncline[]{n84}{n131}
\ncline[]{n84}{n132}
\ncline[]{n84}{n137}
\ncline[]{n84}{n138}
\ncline[]{n84}{n139}
\ncline[]{n84}{n140}
\ncline[]{n84}{n141}
\ncline[]{n84}{n142}
\ncline[]{n84}{n143}
\ncline[]{n84}{n144}
\ncline[]{n84}{n146}
\ncline[]{n84}{n147}
\ncline[]{n84}{n148}
\ncline[]{n85}{n90}
\ncline[]{n85}{n91}
\ncline[]{n85}{n96}
\ncline[]{n85}{n105}
\ncline[]{n85}{n106}
\ncline[]{n85}{n108}
\ncline[]{n85}{n109}
\ncline[]{n85}{n121}
\ncline[]{n85}{n125}
\ncline[]{n85}{n134}
\ncline[]{n86}{n87}
\ncline[]{n86}{n89}
\ncline[]{n86}{n92}
\ncline[]{n86}{n93}
\ncline[]{n86}{n94}
\ncline[]{n86}{n95}
\ncline[]{n86}{n97}
\ncline[]{n86}{n98}
\ncline[]{n86}{n99}
\ncline[]{n86}{n100}
\ncline[]{n86}{n101}
\ncline[]{n86}{n102}
\ncline[]{n86}{n103}
\ncline[]{n86}{n104}
\ncline[]{n86}{n107}
\ncline[]{n86}{n110}
\ncline[]{n86}{n112}
\ncline[]{n86}{n114}
\ncline[]{n86}{n116}
\ncline[]{n86}{n117}
\ncline[]{n86}{n118}
\ncline[]{n86}{n119}
\ncline[]{n86}{n120}
\ncline[]{n86}{n122}
\ncline[]{n86}{n123}
\ncline[]{n86}{n126}
\ncline[]{n86}{n128}
\ncline[]{n86}{n129}
\ncline[]{n86}{n131}
\ncline[]{n86}{n132}
\ncline[]{n86}{n137}
\ncline[]{n86}{n138}
\ncline[]{n86}{n139}
\ncline[]{n86}{n140}
\ncline[]{n86}{n141}
\ncline[]{n86}{n142}
\ncline[]{n86}{n143}
\ncline[]{n86}{n144}
\ncline[]{n86}{n146}
\ncline[]{n86}{n147}
\ncline[]{n86}{n148}
\ncline[]{n87}{n89}
\ncline[]{n87}{n92}
\ncline[]{n87}{n93}
\ncline[]{n87}{n94}
\ncline[]{n87}{n95}
\ncline[]{n87}{n97}
\ncline[]{n87}{n98}
\ncline[]{n87}{n99}
\ncline[]{n87}{n100}
\ncline[]{n87}{n101}
\ncline[]{n87}{n102}
\ncline[]{n87}{n103}
\ncline[]{n87}{n104}
\ncline[]{n87}{n107}
\ncline[]{n87}{n110}
\ncline[]{n87}{n112}
\ncline[]{n87}{n114}
\ncline[]{n87}{n116}
\ncline[]{n87}{n117}
\ncline[]{n87}{n118}
\ncline[]{n87}{n119}
\ncline[]{n87}{n120}
\ncline[]{n87}{n122}
\ncline[]{n87}{n123}
\ncline[]{n87}{n126}
\ncline[]{n87}{n128}
\ncline[]{n87}{n129}
\ncline[]{n87}{n131}
\ncline[]{n87}{n132}
\ncline[]{n87}{n137}
\ncline[]{n87}{n138}
\ncline[]{n87}{n139}
\ncline[]{n87}{n140}
\ncline[]{n87}{n141}
\ncline[]{n87}{n142}
\ncline[]{n87}{n143}
\ncline[]{n87}{n144}
\ncline[]{n87}{n146}
\ncline[]{n87}{n147}
\ncline[]{n87}{n148}
\ncline[]{n88}{n111}
\ncline[]{n88}{n113}
\ncline[]{n88}{n115}
\ncline[]{n88}{n124}
\ncline[]{n88}{n127}
\ncline[]{n88}{n130}
\ncline[]{n88}{n133}
\ncline[]{n88}{n135}
\ncline[]{n88}{n136}
\ncline[]{n88}{n145}
\ncline[]{n88}{n149}
\ncline[]{n89}{n92}
\ncline[]{n89}{n93}
\ncline[]{n89}{n94}
\ncline[]{n89}{n95}
\ncline[]{n89}{n97}
\ncline[]{n89}{n98}
\ncline[]{n89}{n99}
\ncline[]{n89}{n100}
\ncline[]{n89}{n101}
\ncline[]{n89}{n102}
\ncline[]{n89}{n103}
\ncline[]{n89}{n104}
\ncline[]{n89}{n107}
\ncline[]{n89}{n110}
\ncline[]{n89}{n112}
\ncline[]{n89}{n114}
\ncline[]{n89}{n116}
\ncline[]{n89}{n117}
\ncline[]{n89}{n118}
\ncline[]{n89}{n119}
\ncline[]{n89}{n120}
\ncline[]{n89}{n122}
\ncline[]{n89}{n123}
\ncline[]{n89}{n126}
\ncline[]{n89}{n128}
\ncline[]{n89}{n129}
\ncline[]{n89}{n131}
\ncline[]{n89}{n132}
\ncline[]{n89}{n137}
\ncline[]{n89}{n138}
\ncline[]{n89}{n139}
\ncline[]{n89}{n140}
\ncline[]{n89}{n141}
\ncline[]{n89}{n142}
\ncline[]{n89}{n143}
\ncline[]{n89}{n144}
\ncline[]{n89}{n146}
\ncline[]{n89}{n147}
\ncline[]{n89}{n148}
\ncline[]{n90}{n91}
\ncline[]{n90}{n96}
\ncline[]{n90}{n105}
\ncline[]{n90}{n106}
\ncline[]{n90}{n108}
\ncline[]{n90}{n109}
\ncline[]{n90}{n121}
\ncline[]{n90}{n125}
\ncline[]{n90}{n134}
\ncline[]{n91}{n96}
\ncline[]{n91}{n105}
\ncline[]{n91}{n106}
\ncline[]{n91}{n108}
\ncline[]{n91}{n109}
\ncline[]{n91}{n121}
\ncline[]{n91}{n125}
\ncline[]{n91}{n134}
\ncline[]{n92}{n93}
\ncline[]{n92}{n94}
\ncline[]{n92}{n95}
\ncline[]{n92}{n97}
\ncline[]{n92}{n98}
\ncline[]{n92}{n99}
\ncline[]{n92}{n100}
\ncline[]{n92}{n101}
\ncline[]{n92}{n102}
\ncline[]{n92}{n103}
\ncline[]{n92}{n104}
\ncline[]{n92}{n107}
\ncline[]{n92}{n110}
\ncline[]{n92}{n112}
\ncline[]{n92}{n114}
\ncline[]{n92}{n116}
\ncline[]{n92}{n117}
\ncline[]{n92}{n118}
\ncline[]{n92}{n119}
\ncline[]{n92}{n120}
\ncline[]{n92}{n122}
\ncline[]{n92}{n123}
\ncline[]{n92}{n126}
\ncline[]{n92}{n128}
\ncline[]{n92}{n129}
\ncline[]{n92}{n131}
\ncline[]{n92}{n132}
\ncline[]{n92}{n137}
\ncline[]{n92}{n138}
\ncline[]{n92}{n139}
\ncline[]{n92}{n140}
\ncline[]{n92}{n141}
\ncline[]{n92}{n142}
\ncline[]{n92}{n143}
\ncline[]{n92}{n144}
\ncline[]{n92}{n146}
\ncline[]{n92}{n147}
\ncline[]{n92}{n148}
\ncline[]{n93}{n94}
\ncline[]{n93}{n95}
\ncline[]{n93}{n97}
\ncline[]{n93}{n98}
\ncline[]{n93}{n99}
\ncline[]{n93}{n100}
\ncline[]{n93}{n101}
\ncline[]{n93}{n102}
\ncline[]{n93}{n103}
\ncline[]{n93}{n104}
\ncline[]{n93}{n107}
\ncline[]{n93}{n110}
\ncline[]{n93}{n112}
\ncline[]{n93}{n114}
\ncline[]{n93}{n116}
\ncline[]{n93}{n117}
\ncline[]{n93}{n118}
\ncline[]{n93}{n119}
\ncline[]{n93}{n120}
\ncline[]{n93}{n122}
\ncline[]{n93}{n123}
\ncline[]{n93}{n126}
\ncline[]{n93}{n128}
\ncline[]{n93}{n129}
\ncline[]{n93}{n131}
\ncline[]{n93}{n132}
\ncline[]{n93}{n137}
\ncline[]{n93}{n138}
\ncline[]{n93}{n139}
\ncline[]{n93}{n140}
\ncline[]{n93}{n141}
\ncline[]{n93}{n142}
\ncline[]{n93}{n143}
\ncline[]{n93}{n144}
\ncline[]{n93}{n146}
\ncline[]{n93}{n147}
\ncline[]{n93}{n148}
\ncline[]{n94}{n95}
\ncline[]{n94}{n97}
\ncline[]{n94}{n98}
\ncline[]{n94}{n99}
\ncline[]{n94}{n100}
\ncline[]{n94}{n101}
\ncline[]{n94}{n102}
\ncline[]{n94}{n103}
\ncline[]{n94}{n104}
\ncline[]{n94}{n107}
\ncline[]{n94}{n110}
\ncline[]{n94}{n112}
\ncline[]{n94}{n114}
\ncline[]{n94}{n116}
\ncline[]{n94}{n117}
\ncline[]{n94}{n118}
\ncline[]{n94}{n119}
\ncline[]{n94}{n120}
\ncline[]{n94}{n122}
\ncline[]{n94}{n123}
\ncline[]{n94}{n126}
\ncline[]{n94}{n128}
\ncline[]{n94}{n129}
\ncline[]{n94}{n131}
\ncline[]{n94}{n132}
\ncline[]{n94}{n137}
\ncline[]{n94}{n138}
\ncline[]{n94}{n139}
\ncline[]{n94}{n140}
\ncline[]{n94}{n141}
\ncline[]{n94}{n142}
\ncline[]{n94}{n143}
\ncline[]{n94}{n144}
\ncline[]{n94}{n146}
\ncline[]{n94}{n147}
\ncline[]{n94}{n148}
\ncline[]{n95}{n97}
\ncline[]{n95}{n98}
\ncline[]{n95}{n99}
\ncline[]{n95}{n100}
\ncline[]{n95}{n101}
\ncline[]{n95}{n102}
\ncline[]{n95}{n103}
\ncline[]{n95}{n104}
\ncline[]{n95}{n107}
\ncline[]{n95}{n110}
\ncline[]{n95}{n112}
\ncline[]{n95}{n114}
\ncline[]{n95}{n116}
\ncline[]{n95}{n117}
\ncline[]{n95}{n118}
\ncline[]{n95}{n119}
\ncline[]{n95}{n120}
\ncline[]{n95}{n122}
\ncline[]{n95}{n123}
\ncline[]{n95}{n126}
\ncline[]{n95}{n128}
\ncline[]{n95}{n129}
\ncline[]{n95}{n131}
\ncline[]{n95}{n132}
\ncline[]{n95}{n137}
\ncline[]{n95}{n138}
\ncline[]{n95}{n139}
\ncline[]{n95}{n140}
\ncline[]{n95}{n141}
\ncline[]{n95}{n142}
\ncline[]{n95}{n143}
\ncline[]{n95}{n144}
\ncline[]{n95}{n146}
\ncline[]{n95}{n147}
\ncline[]{n95}{n148}
\ncline[]{n96}{n105}
\ncline[]{n96}{n106}
\ncline[]{n96}{n108}
\ncline[]{n96}{n109}
\ncline[]{n96}{n121}
\ncline[]{n96}{n125}
\ncline[]{n96}{n134}
\ncline[]{n97}{n98}
\ncline[]{n97}{n99}
\ncline[]{n97}{n100}
\ncline[]{n97}{n101}
\ncline[]{n97}{n102}
\ncline[]{n97}{n103}
\ncline[]{n97}{n104}
\ncline[]{n97}{n107}
\ncline[]{n97}{n110}
\ncline[]{n97}{n112}
\ncline[]{n97}{n114}
\ncline[]{n97}{n116}
\ncline[]{n97}{n117}
\ncline[]{n97}{n118}
\ncline[]{n97}{n119}
\ncline[]{n97}{n120}
\ncline[]{n97}{n122}
\ncline[]{n97}{n123}
\ncline[]{n97}{n126}
\ncline[]{n97}{n128}
\ncline[]{n97}{n129}
\ncline[]{n97}{n131}
\ncline[]{n97}{n132}
\ncline[]{n97}{n137}
\ncline[]{n97}{n138}
\ncline[]{n97}{n139}
\ncline[]{n97}{n140}
\ncline[]{n97}{n141}
\ncline[]{n97}{n142}
\ncline[]{n97}{n143}
\ncline[]{n97}{n144}
\ncline[]{n97}{n146}
\ncline[]{n97}{n147}
\ncline[]{n97}{n148}
\ncline[]{n98}{n99}
\ncline[]{n98}{n100}
\ncline[]{n98}{n101}
\ncline[]{n98}{n102}
\ncline[]{n98}{n103}
\ncline[]{n98}{n104}
\ncline[]{n98}{n107}
\ncline[]{n98}{n110}
\ncline[]{n98}{n112}
\ncline[]{n98}{n114}
\ncline[]{n98}{n116}
\ncline[]{n98}{n117}
\ncline[]{n98}{n118}
\ncline[]{n98}{n119}
\ncline[]{n98}{n120}
\ncline[]{n98}{n122}
\ncline[]{n98}{n123}
\ncline[]{n98}{n126}
\ncline[]{n98}{n128}
\ncline[]{n98}{n129}
\ncline[]{n98}{n131}
\ncline[]{n98}{n132}
\ncline[]{n98}{n137}
\ncline[]{n98}{n138}
\ncline[]{n98}{n139}
\ncline[]{n98}{n140}
\ncline[]{n98}{n141}
\ncline[]{n98}{n142}
\ncline[]{n98}{n143}
\ncline[]{n98}{n144}
\ncline[]{n98}{n146}
\ncline[]{n98}{n147}
\ncline[]{n98}{n148}
\ncline[]{n99}{n100}
\ncline[]{n99}{n101}
\ncline[]{n99}{n102}
\ncline[]{n99}{n103}
\ncline[]{n99}{n104}
\ncline[]{n99}{n107}
\ncline[]{n99}{n110}
\ncline[]{n99}{n112}
\ncline[]{n99}{n114}
\ncline[]{n99}{n116}
\ncline[]{n99}{n117}
\ncline[]{n99}{n118}
\ncline[]{n99}{n119}
\ncline[]{n99}{n120}
\ncline[]{n99}{n122}
\ncline[]{n99}{n123}
\ncline[]{n99}{n126}
\ncline[]{n99}{n128}
\ncline[]{n99}{n129}
\ncline[]{n99}{n131}
\ncline[]{n99}{n132}
\ncline[]{n99}{n137}
\ncline[]{n99}{n138}
\ncline[]{n99}{n139}
\ncline[]{n99}{n140}
\ncline[]{n99}{n141}
\ncline[]{n99}{n142}
\ncline[]{n99}{n143}
\ncline[]{n99}{n144}
\ncline[]{n99}{n146}
\ncline[]{n99}{n147}
\ncline[]{n99}{n148}
\ncline[]{n100}{n101}
\ncline[]{n100}{n102}
\ncline[]{n100}{n103}
\ncline[]{n100}{n104}
\ncline[]{n100}{n107}
\ncline[]{n100}{n110}
\ncline[]{n100}{n112}
\ncline[]{n100}{n114}
\ncline[]{n100}{n116}
\ncline[]{n100}{n117}
\ncline[]{n100}{n118}
\ncline[]{n100}{n119}
\ncline[]{n100}{n120}
\ncline[]{n100}{n122}
\ncline[]{n100}{n123}
\ncline[]{n100}{n126}
\ncline[]{n100}{n128}
\ncline[]{n100}{n129}
\ncline[]{n100}{n131}
\ncline[]{n100}{n132}
\ncline[]{n100}{n137}
\ncline[]{n100}{n138}
\ncline[]{n100}{n139}
\ncline[]{n100}{n140}
\ncline[]{n100}{n141}
\ncline[]{n100}{n142}
\ncline[]{n100}{n143}
\ncline[]{n100}{n144}
\ncline[]{n100}{n146}
\ncline[]{n100}{n147}
\ncline[]{n100}{n148}
\ncline[]{n101}{n102}
\ncline[]{n101}{n103}
\ncline[]{n101}{n104}
\ncline[]{n101}{n107}
\ncline[]{n101}{n110}
\ncline[]{n101}{n112}
\ncline[]{n101}{n114}
\ncline[]{n101}{n116}
\ncline[]{n101}{n117}
\ncline[]{n101}{n118}
\ncline[]{n101}{n119}
\ncline[]{n101}{n120}
\ncline[]{n101}{n122}
\ncline[]{n101}{n123}
\ncline[]{n101}{n126}
\ncline[]{n101}{n128}
\ncline[]{n101}{n129}
\ncline[]{n101}{n131}
\ncline[]{n101}{n132}
\ncline[]{n101}{n137}
\ncline[]{n101}{n138}
\ncline[]{n101}{n139}
\ncline[]{n101}{n140}
\ncline[]{n101}{n141}
\ncline[]{n101}{n142}
\ncline[]{n101}{n143}
\ncline[]{n101}{n144}
\ncline[]{n101}{n146}
\ncline[]{n101}{n147}
\ncline[]{n101}{n148}
\ncline[]{n102}{n103}
\ncline[]{n102}{n104}
\ncline[]{n102}{n107}
\ncline[]{n102}{n110}
\ncline[]{n102}{n112}
\ncline[]{n102}{n114}
\ncline[]{n102}{n116}
\ncline[]{n102}{n117}
\ncline[]{n102}{n118}
\ncline[]{n102}{n119}
\ncline[]{n102}{n120}
\ncline[]{n102}{n122}
\ncline[]{n102}{n123}
\ncline[]{n102}{n126}
\ncline[]{n102}{n128}
\ncline[]{n102}{n129}
\ncline[]{n102}{n131}
\ncline[]{n102}{n132}
\ncline[]{n102}{n137}
\ncline[]{n102}{n138}
\ncline[]{n102}{n139}
\ncline[]{n102}{n140}
\ncline[]{n102}{n141}
\ncline[]{n102}{n142}
\ncline[]{n102}{n143}
\ncline[]{n102}{n144}
\ncline[]{n102}{n146}
\ncline[]{n102}{n147}
\ncline[]{n102}{n148}
\ncline[]{n103}{n104}
\ncline[]{n103}{n107}
\ncline[]{n103}{n110}
\ncline[]{n103}{n112}
\ncline[]{n103}{n114}
\ncline[]{n103}{n116}
\ncline[]{n103}{n117}
\ncline[]{n103}{n118}
\ncline[]{n103}{n119}
\ncline[]{n103}{n120}
\ncline[]{n103}{n122}
\ncline[]{n103}{n123}
\ncline[]{n103}{n126}
\ncline[]{n103}{n128}
\ncline[]{n103}{n129}
\ncline[]{n103}{n131}
\ncline[]{n103}{n132}
\ncline[]{n103}{n137}
\ncline[]{n103}{n138}
\ncline[]{n103}{n139}
\ncline[]{n103}{n140}
\ncline[]{n103}{n141}
\ncline[]{n103}{n142}
\ncline[]{n103}{n143}
\ncline[]{n103}{n144}
\ncline[]{n103}{n146}
\ncline[]{n103}{n147}
\ncline[]{n103}{n148}
\ncline[]{n104}{n107}
\ncline[]{n104}{n110}
\ncline[]{n104}{n112}
\ncline[]{n104}{n114}
\ncline[]{n104}{n116}
\ncline[]{n104}{n117}
\ncline[]{n104}{n118}
\ncline[]{n104}{n119}
\ncline[]{n104}{n120}
\ncline[]{n104}{n122}
\ncline[]{n104}{n123}
\ncline[]{n104}{n126}
\ncline[]{n104}{n128}
\ncline[]{n104}{n129}
\ncline[]{n104}{n131}
\ncline[]{n104}{n132}
\ncline[]{n104}{n137}
\ncline[]{n104}{n138}
\ncline[]{n104}{n139}
\ncline[]{n104}{n140}
\ncline[]{n104}{n141}
\ncline[]{n104}{n142}
\ncline[]{n104}{n143}
\ncline[]{n104}{n144}
\ncline[]{n104}{n146}
\ncline[]{n104}{n147}
\ncline[]{n104}{n148}
\ncline[]{n105}{n106}
\ncline[]{n105}{n108}
\ncline[]{n105}{n109}
\ncline[]{n105}{n121}
\ncline[]{n105}{n125}
\ncline[]{n105}{n134}
\ncline[]{n106}{n108}
\ncline[]{n106}{n109}
\ncline[]{n106}{n121}
\ncline[]{n106}{n125}
\ncline[]{n106}{n134}
\ncline[]{n107}{n110}
\ncline[]{n107}{n112}
\ncline[]{n107}{n114}
\ncline[]{n107}{n116}
\ncline[]{n107}{n117}
\ncline[]{n107}{n118}
\ncline[]{n107}{n119}
\ncline[]{n107}{n120}
\ncline[]{n107}{n122}
\ncline[]{n107}{n123}
\ncline[]{n107}{n126}
\ncline[]{n107}{n128}
\ncline[]{n107}{n129}
\ncline[]{n107}{n131}
\ncline[]{n107}{n132}
\ncline[]{n107}{n137}
\ncline[]{n107}{n138}
\ncline[]{n107}{n139}
\ncline[]{n107}{n140}
\ncline[]{n107}{n141}
\ncline[]{n107}{n142}
\ncline[]{n107}{n143}
\ncline[]{n107}{n144}
\ncline[]{n107}{n146}
\ncline[]{n107}{n147}
\ncline[]{n107}{n148}
\ncline[]{n108}{n109}
\ncline[]{n108}{n121}
\ncline[]{n108}{n125}
\ncline[]{n108}{n134}
\ncline[]{n109}{n121}
\ncline[]{n109}{n125}
\ncline[]{n109}{n134}
\ncline[]{n110}{n112}
\ncline[]{n110}{n114}
\ncline[]{n110}{n116}
\ncline[]{n110}{n117}
\ncline[]{n110}{n118}
\ncline[]{n110}{n119}
\ncline[]{n110}{n120}
\ncline[]{n110}{n122}
\ncline[]{n110}{n123}
\ncline[]{n110}{n126}
\ncline[]{n110}{n128}
\ncline[]{n110}{n129}
\ncline[]{n110}{n131}
\ncline[]{n110}{n132}
\ncline[]{n110}{n137}
\ncline[]{n110}{n138}
\ncline[]{n110}{n139}
\ncline[]{n110}{n140}
\ncline[]{n110}{n141}
\ncline[]{n110}{n142}
\ncline[]{n110}{n143}
\ncline[]{n110}{n144}
\ncline[]{n110}{n146}
\ncline[]{n110}{n147}
\ncline[]{n110}{n148}
\ncline[]{n111}{n113}
\ncline[]{n111}{n115}
\ncline[]{n111}{n124}
\ncline[]{n111}{n127}
\ncline[]{n111}{n130}
\ncline[]{n111}{n133}
\ncline[]{n111}{n135}
\ncline[]{n111}{n136}
\ncline[]{n111}{n145}
\ncline[]{n111}{n149}
\ncline[]{n112}{n114}
\ncline[]{n112}{n116}
\ncline[]{n112}{n117}
\ncline[]{n112}{n118}
\ncline[]{n112}{n119}
\ncline[]{n112}{n120}
\ncline[]{n112}{n122}
\ncline[]{n112}{n123}
\ncline[]{n112}{n126}
\ncline[]{n112}{n128}
\ncline[]{n112}{n129}
\ncline[]{n112}{n131}
\ncline[]{n112}{n132}
\ncline[]{n112}{n137}
\ncline[]{n112}{n138}
\ncline[]{n112}{n139}
\ncline[]{n112}{n140}
\ncline[]{n112}{n141}
\ncline[]{n112}{n142}
\ncline[]{n112}{n143}
\ncline[]{n112}{n144}
\ncline[]{n112}{n146}
\ncline[]{n112}{n147}
\ncline[]{n112}{n148}
\ncline[]{n113}{n115}
\ncline[]{n113}{n124}
\ncline[]{n113}{n127}
\ncline[]{n113}{n130}
\ncline[]{n113}{n133}
\ncline[]{n113}{n135}
\ncline[]{n113}{n136}
\ncline[]{n113}{n145}
\ncline[]{n113}{n149}
\ncline[]{n114}{n116}
\ncline[]{n114}{n117}
\ncline[]{n114}{n118}
\ncline[]{n114}{n119}
\ncline[]{n114}{n120}
\ncline[]{n114}{n122}
\ncline[]{n114}{n123}
\ncline[]{n114}{n126}
\ncline[]{n114}{n128}
\ncline[]{n114}{n129}
\ncline[]{n114}{n131}
\ncline[]{n114}{n132}
\ncline[]{n114}{n137}
\ncline[]{n114}{n138}
\ncline[]{n114}{n139}
\ncline[]{n114}{n140}
\ncline[]{n114}{n141}
\ncline[]{n114}{n142}
\ncline[]{n114}{n143}
\ncline[]{n114}{n144}
\ncline[]{n114}{n146}
\ncline[]{n114}{n147}
\ncline[]{n114}{n148}
\ncline[]{n115}{n124}
\ncline[]{n115}{n127}
\ncline[]{n115}{n130}
\ncline[]{n115}{n133}
\ncline[]{n115}{n135}
\ncline[]{n115}{n136}
\ncline[]{n115}{n145}
\ncline[]{n115}{n149}
\ncline[]{n116}{n117}
\ncline[]{n116}{n118}
\ncline[]{n116}{n119}
\ncline[]{n116}{n120}
\ncline[]{n116}{n122}
\ncline[]{n116}{n123}
\ncline[]{n116}{n126}
\ncline[]{n116}{n128}
\ncline[]{n116}{n129}
\ncline[]{n116}{n131}
\ncline[]{n116}{n132}
\ncline[]{n116}{n137}
\ncline[]{n116}{n138}
\ncline[]{n116}{n139}
\ncline[]{n116}{n140}
\ncline[]{n116}{n141}
\ncline[]{n116}{n142}
\ncline[]{n116}{n143}
\ncline[]{n116}{n144}
\ncline[]{n116}{n146}
\ncline[]{n116}{n147}
\ncline[]{n116}{n148}
\ncline[]{n117}{n118}
\ncline[]{n117}{n119}
\ncline[]{n117}{n120}
\ncline[]{n117}{n122}
\ncline[]{n117}{n123}
\ncline[]{n117}{n126}
\ncline[]{n117}{n128}
\ncline[]{n117}{n129}
\ncline[]{n117}{n131}
\ncline[]{n117}{n132}
\ncline[]{n117}{n137}
\ncline[]{n117}{n138}
\ncline[]{n117}{n139}
\ncline[]{n117}{n140}
\ncline[]{n117}{n141}
\ncline[]{n117}{n142}
\ncline[]{n117}{n143}
\ncline[]{n117}{n144}
\ncline[]{n117}{n146}
\ncline[]{n117}{n147}
\ncline[]{n117}{n148}
\ncline[]{n118}{n119}
\ncline[]{n118}{n120}
\ncline[]{n118}{n122}
\ncline[]{n118}{n123}
\ncline[]{n118}{n126}
\ncline[]{n118}{n128}
\ncline[]{n118}{n129}
\ncline[]{n118}{n131}
\ncline[]{n118}{n132}
\ncline[]{n118}{n137}
\ncline[]{n118}{n138}
\ncline[]{n118}{n139}
\ncline[]{n118}{n140}
\ncline[]{n118}{n141}
\ncline[]{n118}{n142}
\ncline[]{n118}{n143}
\ncline[]{n118}{n144}
\ncline[]{n118}{n146}
\ncline[]{n118}{n147}
\ncline[]{n118}{n148}
\ncline[]{n119}{n120}
\ncline[]{n119}{n122}
\ncline[]{n119}{n123}
\ncline[]{n119}{n126}
\ncline[]{n119}{n128}
\ncline[]{n119}{n129}
\ncline[]{n119}{n131}
\ncline[]{n119}{n132}
\ncline[]{n119}{n137}
\ncline[]{n119}{n138}
\ncline[]{n119}{n139}
\ncline[]{n119}{n140}
\ncline[]{n119}{n141}
\ncline[]{n119}{n142}
\ncline[]{n119}{n143}
\ncline[]{n119}{n144}
\ncline[]{n119}{n146}
\ncline[]{n119}{n147}
\ncline[]{n119}{n148}
\ncline[]{n120}{n122}
\ncline[]{n120}{n123}
\ncline[]{n120}{n126}
\ncline[]{n120}{n128}
\ncline[]{n120}{n129}
\ncline[]{n120}{n131}
\ncline[]{n120}{n132}
\ncline[]{n120}{n137}
\ncline[]{n120}{n138}
\ncline[]{n120}{n139}
\ncline[]{n120}{n140}
\ncline[]{n120}{n141}
\ncline[]{n120}{n142}
\ncline[]{n120}{n143}
\ncline[]{n120}{n144}
\ncline[]{n120}{n146}
\ncline[]{n120}{n147}
\ncline[]{n120}{n148}
\ncline[]{n121}{n125}
\ncline[]{n121}{n134}
\ncline[]{n122}{n123}
\ncline[]{n122}{n126}
\ncline[]{n122}{n128}
\ncline[]{n122}{n129}
\ncline[]{n122}{n131}
\ncline[]{n122}{n132}
\ncline[]{n122}{n137}
\ncline[]{n122}{n138}
\ncline[]{n122}{n139}
\ncline[]{n122}{n140}
\ncline[]{n122}{n141}
\ncline[]{n122}{n142}
\ncline[]{n122}{n143}
\ncline[]{n122}{n144}
\ncline[]{n122}{n146}
\ncline[]{n122}{n147}
\ncline[]{n122}{n148}
\ncline[]{n123}{n126}
\ncline[]{n123}{n128}
\ncline[]{n123}{n129}
\ncline[]{n123}{n131}
\ncline[]{n123}{n132}
\ncline[]{n123}{n137}
\ncline[]{n123}{n138}
\ncline[]{n123}{n139}
\ncline[]{n123}{n140}
\ncline[]{n123}{n141}
\ncline[]{n123}{n142}
\ncline[]{n123}{n143}
\ncline[]{n123}{n144}
\ncline[]{n123}{n146}
\ncline[]{n123}{n147}
\ncline[]{n123}{n148}
\ncline[]{n124}{n127}
\ncline[]{n124}{n130}
\ncline[]{n124}{n133}
\ncline[]{n124}{n135}
\ncline[]{n124}{n136}
\ncline[]{n124}{n145}
\ncline[]{n124}{n149}
\ncline[]{n125}{n134}
\ncline[]{n126}{n128}
\ncline[]{n126}{n129}
\ncline[]{n126}{n131}
\ncline[]{n126}{n132}
\ncline[]{n126}{n137}
\ncline[]{n126}{n138}
\ncline[]{n126}{n139}
\ncline[]{n126}{n140}
\ncline[]{n126}{n141}
\ncline[]{n126}{n142}
\ncline[]{n126}{n143}
\ncline[]{n126}{n144}
\ncline[]{n126}{n146}
\ncline[]{n126}{n147}
\ncline[]{n126}{n148}
\ncline[]{n127}{n130}
\ncline[]{n127}{n133}
\ncline[]{n127}{n135}
\ncline[]{n127}{n136}
\ncline[]{n127}{n145}
\ncline[]{n127}{n149}
\ncline[]{n128}{n129}
\ncline[]{n128}{n131}
\ncline[]{n128}{n132}
\ncline[]{n128}{n137}
\ncline[]{n128}{n138}
\ncline[]{n128}{n139}
\ncline[]{n128}{n140}
\ncline[]{n128}{n141}
\ncline[]{n128}{n142}
\ncline[]{n128}{n143}
\ncline[]{n128}{n144}
\ncline[]{n128}{n146}
\ncline[]{n128}{n147}
\ncline[]{n128}{n148}
\ncline[]{n129}{n131}
\ncline[]{n129}{n132}
\ncline[]{n129}{n137}
\ncline[]{n129}{n138}
\ncline[]{n129}{n139}
\ncline[]{n129}{n140}
\ncline[]{n129}{n141}
\ncline[]{n129}{n142}
\ncline[]{n129}{n143}
\ncline[]{n129}{n144}
\ncline[]{n129}{n146}
\ncline[]{n129}{n147}
\ncline[]{n129}{n148}
\ncline[]{n130}{n133}
\ncline[]{n130}{n135}
\ncline[]{n130}{n136}
\ncline[]{n130}{n145}
\ncline[]{n130}{n149}
\ncline[]{n131}{n132}
\ncline[]{n131}{n137}
\ncline[]{n131}{n138}
\ncline[]{n131}{n139}
\ncline[]{n131}{n140}
\ncline[]{n131}{n141}
\ncline[]{n131}{n142}
\ncline[]{n131}{n143}
\ncline[]{n131}{n144}
\ncline[]{n131}{n146}
\ncline[]{n131}{n147}
\ncline[]{n131}{n148}
\ncline[]{n132}{n137}
\ncline[]{n132}{n138}
\ncline[]{n132}{n139}
\ncline[]{n132}{n140}
\ncline[]{n132}{n141}
\ncline[]{n132}{n142}
\ncline[]{n132}{n143}
\ncline[]{n132}{n144}
\ncline[]{n132}{n146}
\ncline[]{n132}{n147}
\ncline[]{n132}{n148}
\ncline[]{n133}{n135}
\ncline[]{n133}{n136}
\ncline[]{n133}{n145}
\ncline[]{n133}{n149}
\ncline[]{n135}{n136}
\ncline[]{n135}{n145}
\ncline[]{n135}{n149}
\ncline[]{n136}{n145}
\ncline[]{n136}{n149}
\ncline[]{n137}{n138}
\ncline[]{n137}{n139}
\ncline[]{n137}{n140}
\ncline[]{n137}{n141}
\ncline[]{n137}{n142}
\ncline[]{n137}{n143}
\ncline[]{n137}{n144}
\ncline[]{n137}{n146}
\ncline[]{n137}{n147}
\ncline[]{n137}{n148}
\ncline[]{n138}{n139}
\ncline[]{n138}{n140}
\ncline[]{n138}{n141}
\ncline[]{n138}{n142}
\ncline[]{n138}{n143}
\ncline[]{n138}{n144}
\ncline[]{n138}{n146}
\ncline[]{n138}{n147}
\ncline[]{n138}{n148}
\ncline[]{n139}{n140}
\ncline[]{n139}{n141}
\ncline[]{n139}{n142}
\ncline[]{n139}{n143}
\ncline[]{n139}{n144}
\ncline[]{n139}{n146}
\ncline[]{n139}{n147}
\ncline[]{n139}{n148}
\ncline[]{n140}{n141}
\ncline[]{n140}{n142}
\ncline[]{n140}{n143}
\ncline[]{n140}{n144}
\ncline[]{n140}{n146}
\ncline[]{n140}{n147}
\ncline[]{n140}{n148}
\ncline[]{n141}{n142}
\ncline[]{n141}{n143}
\ncline[]{n141}{n144}
\ncline[]{n141}{n146}
\ncline[]{n141}{n147}
\ncline[]{n141}{n148}
\ncline[]{n142}{n143}
\ncline[]{n142}{n144}
\ncline[]{n142}{n146}
\ncline[]{n142}{n147}
\ncline[]{n142}{n148}
\ncline[]{n143}{n144}
\ncline[]{n143}{n146}
\ncline[]{n143}{n147}
\ncline[]{n143}{n148}
\ncline[]{n144}{n146}
\ncline[]{n144}{n147}
\ncline[]{n144}{n148}
\ncline[]{n145}{n149}
\ncline[]{n146}{n147}
\ncline[]{n146}{n148}
\ncline[]{n147}{n148}
\psset{fillcolor=gray, linecolor=black}
\dotnode[dotstyle=Btriangle](-0.511, -0.102){n0}
\dotnode[dotstyle=Btriangle](-1.464, 0.504){n1}
\dotnode[dotstyle=Btriangle](-1.043, 0.229){n2}
\dotnode[dotstyle=Bo](2.840, -0.221){n3}
\dotnode[dotstyle=Btriangle](-0.262, -0.548){n4}
\dotnode[dotstyle=Bo](2.890, -0.137){n5}
\dotnode[dotstyle=Btriangle](-2.350, -0.042){n6}
\dotnode[dotstyle=Btriangle](-1.414, -0.575){n7}
\dotnode[dotstyle=Btriangle](-2.314, 0.183){n8}
\dotnode[dotstyle=Btriangle](-2.320, -0.246){n9}
\dotnode[dotstyle=Bsquare](2.543, 0.440){n10}
\dotnode[dotstyle=Bsquare](2.587, 0.520){n11}
\dotnode[dotstyle=Btriangle](-0.331, -0.211){n12}
\dotnode[dotstyle=Btriangle](-1.291, -0.116){n13}
\dotnode[dotstyle=Bo](2.674, -0.107){n14}
\dotnode[dotstyle=Bsquare](2.469, 0.138){n15}
\dotnode[dotstyle=Bsquare](2.626, 0.170){n16}
\dotnode[dotstyle=Btriangle](-0.298, -0.347){n17}
\dotnode[dotstyle=Bsquare](2.704, 0.115){n18}
\dotnode[dotstyle=Btriangle](-2.614, 0.558){n19}
\dotnode[dotstyle=Btriangle](-1.390, -0.283){n20}
\dotnode[dotstyle=Btriangle](-1.764, 0.079){n21}
\dotnode[dotstyle=Bsquare](2.384, 1.345){n22}
\dotnode[dotstyle=Btriangle](-0.165, -0.680){n23}
\dotnode[dotstyle=Btriangle](-0.519, -1.191){n24}
\dotnode[dotstyle=Bsquare](2.406, 0.196){n25}
\dotnode[dotstyle=Btriangle](-0.181, -0.826){n26}
\dotnode[dotstyle=Bo](2.746, -0.311){n27}
\dotnode[dotstyle=Btriangle](-2.388, 0.463){n28}
\dotnode[dotstyle=Btriangle](-2.563, 0.276){n29}
\dotnode[dotstyle=Bsquare](2.623, 0.818){n30}
\dotnode[dotstyle=Bsquare](2.867, 0.077){n31}
\dotnode[dotstyle=Btriangle](-1.345, -0.776){n32}
\dotnode[dotstyle=Btriangle](-0.229, -0.402){n33}
\dotnode[dotstyle=Bo](0.908, -0.752){n34}
\dotnode[dotstyle=Btriangle](-0.043, -0.581){n35}
\dotnode[dotstyle=Btriangle](-0.235, -0.332){n36}
\dotnode[dotstyle=Bsquare](2.209, 0.443){n37}
\dotnode[dotstyle=Btriangle](-0.813, -0.371){n38}
\dotnode[dotstyle=Btriangle](-1.526, -0.375){n39}
\dotnode[dotstyle=Btriangle](-0.660, -0.352){n40}
\dotnode[dotstyle=Btriangle](-0.890, -0.034){n41}
\dotnode[dotstyle=Btriangle](-2.165, 0.215){n42}
\dotnode[dotstyle=Btriangle](-2.932, 0.352){n43}
\dotnode[dotstyle=Btriangle](-0.356, -0.503){n44}
\dotnode[dotstyle=Bo](2.787, -0.228){n45}
\dotnode[dotstyle=Btriangle](-2.616, 0.342){n46}
\dotnode[dotstyle=Btriangle](0.307, -0.365){n47}
\dotnode[dotstyle=Bsquare](2.543, 0.586){n48}
\dotnode[dotstyle=Bsquare](2.199, 0.879){n49}
\dotnode[dotstyle=Btriangle](-0.375, -0.292){n50}
\dotnode[dotstyle=Btriangle](-1.198, -0.606){n51}
\dotnode[dotstyle=Bo](2.982, -0.480){n52}
\dotnode[dotstyle=Btriangle](-2.532, -0.012){n53}
\dotnode[dotstyle=Bsquare](2.410, 0.418){n54}
\dotnode[dotstyle=Btriangle](-2.145, 0.139){n55}
\dotnode[dotstyle=Btriangle](-2.108, 0.371){n56}
\dotnode[dotstyle=Btriangle](-3.077, 0.686){n57}
\dotnode[dotstyle=Btriangle](-0.921, -0.182){n58}
\dotnode[dotstyle=Btriangle](0.175, -0.252){n59}
\dotnode[dotstyle=Btriangle](-1.780, -0.501){n60}
\dotnode[dotstyle=Btriangle](-1.802, -0.216){n61}
\dotnode[dotstyle=Bsquare](2.770, 0.271){n62}
\dotnode[dotstyle=Bsquare](2.303, 0.106){n63}
\dotnode[dotstyle=Btriangle](-0.245, -0.267){n64}
\dotnode[dotstyle=Btriangle](-0.587, -0.483){n65}
\dotnode[dotstyle=Btriangle](-1.585, -0.539){n66}
\dotnode[dotstyle=Bo](2.674, -0.107){n67}
\dotnode[dotstyle=Bsquare](3.216, 0.142){n68}
\dotnode[dotstyle=Bsquare](2.640, 0.319){n69}
\dotnode[dotstyle=Btriangle](-3.232, 1.371){n70}
\dotnode[dotstyle=Btriangle](-3.795, 0.253){n71}
\dotnode[dotstyle=Btriangle](-0.357, -0.067){n72}
\dotnode[dotstyle=Bsquare](2.625, 0.607){n73}
\dotnode[dotstyle=Bo](2.821, -0.082){n74}
\dotnode[dotstyle=Bo](2.633, -0.190){n75}
\dotnode[dotstyle=Bo](2.888, -0.571){n76}
\dotnode[dotstyle=Btriangle](-1.258, -0.179){n77}
\dotnode[dotstyle=Bo](2.716, -0.243){n78}
\dotnode[dotstyle=Btriangle](-0.812, -0.162){n79}
\dotnode[dotstyle=Btriangle](-1.902, 0.116){n80}
\dotnode[dotstyle=Bo](0.707, -1.008){n81}
\dotnode[dotstyle=Btriangle](-0.932, 0.319){n82}
\dotnode[dotstyle=Btriangle](-1.971, -0.181){n83}
\dotnode[dotstyle=Btriangle](-1.557, 0.267){n84}
\dotnode[dotstyle=Bo](0.511, -1.262){n85}
\dotnode[dotstyle=Btriangle](-1.116, -0.084){n86}
\dotnode[dotstyle=Btriangle](-2.419, 0.304){n87}
\dotnode[dotstyle=Bsquare](2.280, 0.748){n88}
\dotnode[dotstyle=Btriangle](-1.388, -0.204){n89}
\dotnode[dotstyle=Bo](2.613, 0.022){n90}
\dotnode[dotstyle=Bo](2.998, -0.334){n91}
\dotnode[dotstyle=Btriangle](-1.905, 0.119){n92}
\dotnode[dotstyle=Btriangle](-0.942, -0.542){n93}
\dotnode[dotstyle=Btriangle](-1.298, -0.761){n94}
\dotnode[dotstyle=Btriangle](-2.841, 0.373){n95}
\dotnode[dotstyle=Bo](0.751, -1.001){n96}
\dotnode[dotstyle=Btriangle](-1.285, 0.685){n97}
\dotnode[dotstyle=Btriangle](0.192, -0.677){n98}
\dotnode[dotstyle=Btriangle](-1.095, 0.284){n99}
\dotnode[dotstyle=Btriangle](-1.331, 0.245){n100}
\dotnode[dotstyle=Btriangle](-0.984, -0.124){n101}
\dotnode[dotstyle=Btriangle](-1.662, 0.242){n102}
\dotnode[dotstyle=Btriangle](-2.276, 0.333){n103}
\dotnode[dotstyle=Btriangle](-0.928, 0.468){n104}
\dotnode[dotstyle=Bo](2.356, -0.031){n105}
\dotnode[dotstyle=Bo](2.715, -0.170){n106}
\dotnode[dotstyle=Btriangle](-2.428, 0.377){n107}
\dotnode[dotstyle=Bo](2.852, -0.933){n108}
\dotnode[dotstyle=Bo](3.225, -0.503){n109}
\dotnode[dotstyle=Btriangle](0.010, -0.721){n110}
\dotnode[dotstyle=Bsquare](2.729, 0.334){n111}
\dotnode[dotstyle=Btriangle](-0.714, 0.150){n112}
\dotnode[dotstyle=Bsquare](2.562, 0.375){n113}
\dotnode[dotstyle=Btriangle](-0.135, -0.312){n114}
\dotnode[dotstyle=Bsquare](2.597, 1.100){n115}
\dotnode[dotstyle=Btriangle](-3.397, 0.547){n116}
\dotnode[dotstyle=Btriangle](-1.443, -0.144){n117}
\dotnode[dotstyle=Btriangle](-1.905, 0.048){n118}
\dotnode[dotstyle=Btriangle](-1.296, -0.328){n119}
\dotnode[dotstyle=Btriangle](-1.414, -0.575){n120}
\dotnode[dotstyle=Bo](2.508, -0.139){n121}
\dotnode[dotstyle=Btriangle](-1.220, 0.408){n122}
\dotnode[dotstyle=Btriangle](-1.379, -0.421){n123}
\dotnode[dotstyle=Bsquare](2.684, 0.327){n124}
\dotnode[dotstyle=Bo](2.588, -0.197){n125}
\dotnode[dotstyle=Btriangle](-0.640, -0.417){n126}
\dotnode[dotstyle=Bsquare](2.538, 0.510){n127}
\dotnode[dotstyle=Btriangle](-0.900, 0.330){n128}
\dotnode[dotstyle=Btriangle](-2.123, -0.211){n129}
\dotnode[dotstyle=Bsquare](2.648, 0.820){n130}
\dotnode[dotstyle=Btriangle](-2.159, -0.218){n131}
\dotnode[dotstyle=Btriangle](-3.499, 0.457){n132}
\dotnode[dotstyle=Bsquare](2.644, 1.186){n133}
\dotnode[dotstyle=Bo](2.674, -0.107){n134}
\dotnode[dotstyle=Bsquare](2.507, 0.652){n135}
\dotnode[dotstyle=Bsquare](2.648, 0.319){n136}
\dotnode[dotstyle=Btriangle](-0.807, 0.195){n137}
\dotnode[dotstyle=Btriangle](-1.949, 0.041){n138}
\dotnode[dotstyle=Btriangle](0.070, -0.703){n139}
\dotnode[dotstyle=Btriangle](-2.918, 0.780){n140}
\dotnode[dotstyle=Btriangle](-1.922, 0.409){n141}
\dotnode[dotstyle=Btriangle](-0.642, 0.019){n142}
\dotnode[dotstyle=Btriangle](-1.087, 0.075){n143}
\dotnode[dotstyle=Btriangle](-1.169, -0.165){n144}
\dotnode[dotstyle=Bsquare](2.311, 0.398){n145}
\dotnode[dotstyle=Btriangle](-0.463, -0.670){n146}
\dotnode[dotstyle=Btriangle](-1.944, 0.187){n147}
\dotnode[dotstyle=Btriangle](-3.489, 1.172){n148}
\dotnode[dotstyle=Bsquare](2.590, 0.236){n149}

        \endpsgraph
		}}}
\end{figure}
  \begin{center}
  \scalebox{0.8}{%
\begin{tabular}{|l|cccc|ccccc|}
  \hline
  & \multicolumn{4}{c|}{Lower better} &
  \multicolumn{5}{c|}{Higher better}\\\cline{2-10}
 & $BetaCV$ &
 $Cindex$ & $Q$ & $\mathit{DB}$ & $\mathit{NC}$  & $Dunn$  & $\mathit{SC}$ & $\Gamma$ & $\Gamma_n$\\
\hline
(a) Good & 0.24 & 0.034 & $-$0.23 & 0.65 & 2.67  & 0.08
     & 0.60 & 8.19&  0.92\\
(b) Bad & 0.33 & 0.08 & $-$0.20 & 1.11 & 2.56  & 0.03  & 0.55 & 7.32
& 0.83\\
\hline
\end{tabular}
}
\end{center}
\end{frame}




\begin{frame}{Relative Measures: Silhouette Coeff\/{i}cient}
The silhouette coeff\/{i}cient for each
point $s_{j}$, and the average SC value
can be used to estimate the number of clusters
in the data. 

\medskip
The approach consists of plotting the $s_{j}$ values in
descending order for each cluster, and to note the overall $SC$ value
for a particular value of $k$, as well as clusterwise SC values:
\begin{align*}
  SC_i = \frac{1}{n_i}\sum_{\bx_{j} \in C_i} s_{j}
\end{align*}
We then pick the value $k$ that yields the best clustering, with
many points having high $s_{j}$ values within each cluster, as well as
high values for $SC$ and $SC_i$ ($1 \le i \le k$).
\end{frame}



\readdata{\dataKa}{CLUST/eval/figs/irisKmeans-K2-Sil.txt}
\readdata{\dataKb}{CLUST/eval/figs/irisKmeans-K3-Sil.txt}
\readdata{\dataKc}{CLUST/eval/figs/irisKmeans-K4-Sil.txt}
\begin{frame}[fragile]{Iris K-means: Silhouette Coeff\/{i}cient Plot
  ($k=2$)}
\setcounter{subfigure}{0}
\begin{figure}
\captionsetup[subfloat]{captionskip=0.5in}
\centering %\small
\psset{xAxisLabel=$~$,yAxisLabel=silhouette coeff\/{i}cient,%
xAxisLabelPos={c,-0.05},yAxisLabelPos={-20,c}} \psset{xLabels={}}
\centerline{ \subfloat[$k=2$, $SC=0.706$]{
\scalebox{0.9}{
\label{fig:clust:eval:silplotK2}
\begin{psgraph}[Dy=0.1,labels=y,ticks=y]{->}(0,0)(150,1.1){4in}{1.75in}
    \listplot[%plotstyle=bar, barwidth=0.01cm,
    %fillcolor=lightgray, fillstyle=solid,
    plotstyle=dots, dotscale=0.75,
    plotNoMax=2, plotNo=1]{\dataKa}
    \psline{<->}(0,-0.05)(96,-0.05)
    \uput[d](50,-0.05){\scriptsize
    $\begin{array}{c}
        SC_1=0.706\\
        n_1=97
      \end{array}$
      }
    \psline[](96.5,-0.05)(96.5,0.95)
    \psline{<->}(98,-0.05)(150,-0.05)
    \uput[d](120,-0.05){\scriptsize
      $\begin{array}{c}
        SC_2=0.662\\
        n_2=53
      \end{array}$
      }
\end{psgraph}
}} }
\end{figure}
 $k=2$ yields the highest silhouette coeff\/{i}cient, with the two
 clusters essentially well separated.
\end{frame}

\begin{frame}[fragile]{Iris K-means: Silhouette Coeff\/{i}cient Plot
  ($k=3$)}
\begin{figure}
\captionsetup[subfloat]{captionskip=0.5in}

\centerline{ \subfloat[$k=3$, $SC=0.598$]{
\scalebox{0.9}{
\label{fig:clust:eval:silplotK3}
\begin{psgraph}[Dy=0.1,labels=y,ticks=y]{->}(0,0)(150,1){4in}{1.75in}
    \listplot[%plotstyle=bar, barwidth=0.01cm,
    %fillcolor=lightgray, fillstyle=solid,
    plotstyle=dots, dotscale=0.75,
    plotNoMax=2, plotNo=1]{\dataKb}
    \psline{<->}(0,-0.05)(60,-0.05)
    \uput[d](30,-0.05){\scriptsize
      $\begin{array}{c}
        SC_1=0.466\\
        n_1=61
      \end{array}$
      }
    \psline[](60.5,-0.05)(60.5,0.95)
    \psline{<->}(62,-0.05)(110,-0.05)
    \uput[d](85,-0.05){\scriptsize
      $\begin{array}{c}
        SC_2=0.818\\
        n_2=50
      \end{array}$
      }
    \psline[](110.5,-0.05)(110.5,0.95)
    \psline{<->}(111,-0.05)(150,-0.05)
    \uput[d](130,-0.05){\scriptsize
      $\begin{array}{c}
        SC_3=0.52\\
        n_3=39
      \end{array}$
      }
\end{psgraph}
} } }
\end{figure}
\end{frame}

\begin{frame}[fragile]{Iris K-means: Silhouette Coeff\/{i}cient Plot
  ($k=4$)}
\begin{figure}
\captionsetup[subfloat]{captionskip=0.5in}
\centerline{ \subfloat[$k=4$, $SC=0.559$]{
\scalebox{0.9}{
\label{fig:clust:eval:silplotK4}
\begin{psgraph}[Dy=0.1,labels=y,ticks=y]{->}(0,0)(150,1){4in}{1.75in}
    \listplot[%plotstyle=bar, barwidth=0.01cm,
    %fillcolor=lightgray, fillstyle=solid,
    plotstyle=dots,dotscale=0.75,
    plotNoMax=2, plotNo=1]{\dataKc}
    \psline{<->}(0,-0.05)(48,-0.05)
    \uput[d](20,-0.05){\scriptsize
      $\begin{array}{c}
        SC_1=0.376\\
        n_1=49
      \end{array}$
      }
    \psline[](48.5,-0.05)(48.5,0.95)
    \psline{<->}(49,-0.05)(76,-0.05)
    \uput[d](65,-0.05){\scriptsize
      $\begin{array}{c}
        SC_2=0.534\\
        n_2=28
      \end{array}$
      }
    \psline[](76.5,-0.05)(76.5,0.95)
    \psline{<->}(77,-0.05)(126,-0.05)
    \uput[d](100,-0.05){\scriptsize
      $\begin{array}{c}
        SC_3=0.787\\
        n_3=50
      \end{array}$
      }
    \psline[](126.5,-0.05)(126.5,0.95)
    \psline{<->}(127,-0.05)(150,-0.05)
    \uput[d](140,-0.05){\scriptsize
      $\begin{array}{c}
        SC_4=0.484\\
        n_4=23
      \end{array}$
      }
\end{psgraph}
}}} 
\end{figure}
\end{frame}



\begin{frame}{Relative Measures: Calinski--Harabasz Index}
  Given the dataset $\bD =\{\bx_i\}_{i=1}^n$,
the scatter matrix for $\bD$ is given as
\begin{align*}
  \bS = n\cov = \sum_{j=1}^n \lB(\bx_{j} - \bmu\rB)
  \lB(\bx_{j} - \bmu\rB)^T
\end{align*}
where $\bmu = \tfrac{1}{n} \sum_{j=1}^n \bx_{j} $ is the mean and $\cov$
is the covariance matrix.  The scatter matrix can
be decomposed into two matrices $\bS = \bS_W + \bS_B$,
where $\bS_W$ is the within-cluster scatter
matrix and $\bS_B$ is the between-cluster scatter matrix, given
as
\begin{align*}
  \bS_W & = \sum_{i=1}^k \sum_{\bx_{j} \in C_i}
  \lB(\bx_{j} - \bmu_i\rB) \lB(\bx_{j} - \bmu_i\rB)^T\\
  \bS_B & = \sum_{i=1}^k n_i
  \lB(\bmu_i - \bmu\rB) \lB(\bmu_i - \bmu\rB)^T
\end{align*}
where $\bmu_i = \tfrac{1}{n_i} \sum_{\bx_{j} \in C_i} \bx_{j}$ is
the mean for cluster $C_i$.
\end{frame}


\begin{frame}{Relative Measures: Calinski--Harabasz Index}
The Calinski--Harabasz (CH) variance ratio criterion for a given value of $k$
is def\/{i}ned as follows:
\begin{align*}
\tcbhighmath{
  \mathit{CH}(k) = \frac{tr(\bS_B)/(k-1)}{tr(\bS_W)/(n-k)} =
  \frac{n-k}{k-1} \cdot \frac{tr(\bS_B)}{tr(\bS_W)}
}
\end{align*}
where $tr$ is the trace of the matrix.

\medskip
We plot the $\mathit{CH}$ values and look for a large increase in the
value followed by little or no gain. We choose
the value $k>3$ that minimizes the term
\begin{align*}
  \Delta(k) = \Bigl(\mathit{CH}(k+1)-\mathit{CH}(k)\Bigr) - \Bigl(\mathit{CH}(k)-\mathit{CH}(k-1)\Bigr)
\end{align*}
The intuition is that we want to f\/{i}nd the value of $k$ for which $\mathit{CH}(k)$ is much higher than
$\mathit{CH}(k-1)$ and there is only a
little improvement or a decrease in the
$\mathit{CH}(k + 1)$ value.

\end{frame}



\readdata{\dataCH}{CLUST/eval/figs/CHvals.txt}
\begin{frame}[fragile]{Calinski--Harabasz Variance Ratio}
  \small
  CH ratio for various values
  of $k$ on the Iris principal components data, using the K-means
  algorithm, with the best results chosen from 200 runs.
\begin{figure}
    \centering
    \def\pshlabel#1{ {\footnotesize #1}}
    \def\psvlabel#1{ {\footnotesize #1}}
\psset{dotscale=2,fillcolor=lightgray}
\psset{xAxisLabel=$k$,yAxisLabel=$\mathit{CH}$,%
xAxisLabelPos={c,-40},yAxisLabelPos={-1.25,c}}
\scalebox{0.8}{
\begin{psgraph}[Dy=50,Oy=550,Ox=1,showorigin=false]{->}(0,0)(9,250){3.5in}{2in}
\pstScalePoints(1,1){-1 add}{-550 add}
  \listplot[showpoints=true, dotstyle=Bsquare]{\dataCH}
\end{psgraph}
}
\end{figure}
\vspace{0.2in}
  The successive
  $\mathit{CH}(k)$ and $\Delta(k)$ values are as follows:
  \begin{align*}
    \begin{array}{c|cccccccc}
    k & 2 & 3 & 4 & 5 & 6 & 7 & 8 & 9\\
    \hline
    \mathit{CH}(k) & 570.25 & 692.40 & 717.79 & 683.14 & 708.26 & 700.17 & 738.05
    & 728.63\\
    \Delta(k) & \text{--} & -96.78 & -60.03 & 59.78 & -33.22 & 45.97 &
    -47.30 & \text{--}\\
    \end{array}
  \end{align*}
$\Delta(k)$ suggests $k=3$ as the best (lowest) value.
\end{frame}


\begin{frame}{Relative Measures: Gap Statistic}
The gap statistic compares the sum of
intracluster
weights $W_{in}$ 
for different values of $k$ with their expected values assuming
no apparent clustering structure, which forms the null hypothesis.

\bigskip
Let $\cC_k$ be the clustering obtained for a specif\/{i}ed value of $k$.
Let $W_{in}^k(\bD)$ denote
the sum of intracluster weights (over all clusters) for $\cC_k$ on the
input dataset $\bD$.

\bigskip
We would like to compute the probability of the observed
$W_{in}^k$ value under the null hypothesis.
To obtain an empirical distribution for $W_{in}$, we resort to Monte
Carlo simulations of the sampling process.  
\end{frame}

\begin{frame}{Relative Measures: Gap Statistic}
We generate $t$
random samples comprising $n$ points.
Let $\bR_i \in \setR^{n \times d}$, $1 \le i \le t$ denote the $i$th sample.
Let $W_{in}^k(\bR_i)$ denote the sum of intracluster weights for a
given clustering of $\bR_i$ into $k$ clusters.

\medskip
From each sample
dataset $\bR_i$, we generate clusterings for different values of $k$,
and record the intracluster values
$W_{in}^k(\bR_i)$.

\medskip
Let $\mu_W(k)$ and $\sigma_W(k)$ denote the
mean and standard deviation of these intracluster weights for each
value of $k$.
The {\em gap statistic} for a given $k$ is then def\/{i}ned as
\begin{align*}
\tcbhighmath{
  gap(k) = \mu_W(k) - \log  W_{in}^k(\bD)
}
\end{align*}

\medskip
Choose $k$ as follows:
\begin{align*}
  k^* = \arg\min_k \Bigl\{ gap(k) \ge gap(k+1) - \sigma_W(k+1)\Bigr\}
\end{align*}
\end{frame}


\readdata{\dataG}{CLUST/eval/figs/gapstatistic.txt}
\readdata{\dataGE}{CLUST/eval/figs/gapstatistic-errs.txt}
\begin{frame}[fragile]{Gap Statistic: Randomly Generated Data}
\setcounter{subfigure}{0}
\begin{figure}
%\begin{figure}[!t]%fig17.5
    \centering
    \def\pshlabel#1{ {\footnotesize $#1$}}
    \def\psvlabel#1{ {\footnotesize $#1$}}
\captionsetup[subfloat]{captionskip=0.4in}
\centering \psset{dotscale=2,fillcolor=lightgray} \centerline{
\subfloat[Randomly generated data ($k=3$)]{
\label{fig:clust:eval:gapdata}
  \pspicture[](-5,-2.25)(3.5,2.75)
\psset{xunit=0.5in,yunit=0.65in,dotscale=1.5,arrowscale=2,PointName=none}
  \psaxes[tickstyle=bottom,Dx=1,Ox=-4,Dy=0.5,Oy=-1.5]{->}(-4,-1.5)(3.5,1.5)
  \psset{dotstyle=Bsquare,fillcolor=lightgray}
  \psdot[](2.17,-0.72)
\psdot[](2.29,-1.22)
\psdot[](2.33,-0.19)
\psdot[](2.50,-0.01)
\psdot[](1.24,0.86)
\psdot[](1.56,-0.15)
\psdot[](3.04,0.22)
\psdot[](2.21,0.98)
\psdot[](1.30,-0.51)
\psdot[](2.06,-0.57)
\psdot[](2.16,0.17)
\psdot[](2.24,0.87)
\psdot[](1.06,-0.62)
\psdot[](1.85,0.35)
\psdot[](2.24,-1.15)
\psdot[](1.10,1.19)
\psdot[](2.34,-1.07)
\psdot[](3.13,1.30)
\psdot[](1.04,-1.01)
\psdot[](1.20,0.78)
\psdot[](1.15,-0.01)
\psdot[](2.34,0.01)
\psdot[](2.06,-1.03)
\psdot[](1.00,-0.20)
\psdot[](1.61,-0.20)
\psdot[](2.87,0.94)
\psdot[](2.26,-1.18)
\psdot[](2.82,-0.49)
\psdot[](1.25,-0.20)
\psdot[](2.64,-1.11)
\psdot[](1.15,0.32)
\psdot[](1.70,1.25)
\psdot[](2.02,-0.65)
\psdot[](1.14,0.26)
\psdot[](2.41,0.37)
\psdot[](2.14,-0.25)
\psdot[](2.06,0.34)
\psdot[](1.56,1.09)
\psdot[](2.87,0.33)

  \psset{dotstyle=Bo,fillcolor=lightgray}
  \psdot[](0.33,0.09)
\psdot[](-1.13,-0.73)
\psdot[](-1.03,0.39)
\psdot[](-0.88,-0.24)
\psdot[](-0.81,1.14)
\psdot[](0.12,1.31)
\psdot[](0.36,-0.80)
\psdot[](0.51,-1.23)
\psdot[](-1.03,1.09)
\psdot[](-0.36,-0.64)
\psdot[](-0.80,0.88)
\psdot[](0.75,-0.47)
\psdot[](0.28,-0.65)
\psdot[](0.72,-1.18)
\psdot[](0.40,0.95)
\psdot[](0.29,0.28)
\psdot[](-1.07,-0.14)
\psdot[](0.52,0.90)
\psdot[](0.93,0.87)
\psdot[](-0.48,0.92)
\psdot[](-0.35,0.15)
\psdot[](0.34,-0.99)
\psdot[](-0.05,-1.13)
\psdot[](0.21,0.32)
\psdot[](-1.18,0.23)
\psdot[](0.20,-1.01)
\psdot[](-1.05,0.68)
\psdot[](0.32,0.26)
\psdot[](-1.05,1.35)
\psdot[](0.30,0.32)
\psdot[](0.69,0.36)
\psdot[](0.72,0.48)
\psdot[](0.65,1.31)
\psdot[](0.74,0.47)
\psdot[](-1.12,0.91)
\psdot[](0.07,-0.25)
\psdot[](0.40,0.17)
\psdot[](-0.54,-0.85)
\psdot[](-0.31,1.20)
\psdot[](0.49,0.27)
\psdot[](0.32,-0.38)
\psdot[](-0.14,0.06)
\psdot[](-0.57,-1.20)
\psdot[](0.61,-0.35)
\psdot[](-0.91,-0.47)
\psdot[](-0.92,-0.19)
\psdot[](0.38,-0.88)
\psdot[](-0.16,1.03)
\psdot[](0.85,0.26)
\psdot[](0.55,0.42)
\psdot[](-0.69,1.15)
\psdot[](-0.76,0.03)
\psdot[](0.18,0.71)
\psdot[](0.04,-0.49)
\psdot[](0.47,0.09)
\psdot[](-0.94,0.97)
\psdot[](0.15,0.85)

  \psset{dotstyle=Btriangle,fillcolor=lightgray}
  \psdot[](-1.35,-1.11)
\psdot[](-2.14,1.19)
\psdot[](-2.57,0.85)
\psdot[](-2.56,1.30)
\psdot[](-2.45,0.31)
\psdot[](-2.22,-0.72)
\psdot[](-1.92,-1.01)
\psdot[](-3.66,-0.29)
\psdot[](-3.43,-0.51)
\psdot[](-3.29,-0.01)
\psdot[](-2.94,-0.06)
\psdot[](-2.16,-0.02)
\psdot[](-3.12,0.05)
\psdot[](-3.20,-0.60)
\psdot[](-3.71,-0.73)
\psdot[](-3.03,0.48)
\psdot[](-1.42,-0.40)
\psdot[](-1.33,-0.92)
\psdot[](-3.63,0.22)
\psdot[](-1.34,-0.19)
\psdot[](-3.50,-0.55)
\psdot[](-3.79,1.06)
\psdot[](-2.46,0.03)
\psdot[](-2.17,-0.57)
\psdot[](-2.33,-0.17)
\psdot[](-3.50,-1.23)
\psdot[](-2.64,0.36)
\psdot[](-3.21,1.12)
\psdot[](-1.66,0.36)
\psdot[](-2.68,0.46)
\psdot[](-3.52,0.44)
\psdot[](-3.04,0.75)
\psdot[](-3.05,0.31)
\psdot[](-2.13,-0.98)
\psdot[](-1.41,0.59)
\psdot[](-2.30,-0.32)
\psdot[](-2.98,-0.46)
\psdot[](-2.63,-0.04)
\psdot[](-3.25,-1.00)
\psdot[](-2.60,-0.30)
\psdot[](-3.44,-0.59)
\psdot[](-1.23,-0.95)
\psdot[](-1.92,-1.06)
\psdot[](-1.51,-0.60)
\psdot[](-3.05,1.23)
\psdot[](-1.97,-0.12)
\psdot[](-1.34,0.23)
\psdot[](-1.44,0.01)
\psdot[](-2.42,1.09)
\psdot[](-2.23,0.55)
\psdot[](-2.05,-0.88)
\psdot[](-2.35,0.68)
\psdot[](-2.53,0.29)
\psdot[](-1.55,0.96)

  \psset{fillcolor=black}
  \pstGeonode[PointSymbol=Bsquare, dotscale=2]%
      (1.95,-0.022){A}
  \pstGeonode[PointSymbol=Bo,dotscale=2]%
      (-0.078,0.15){B}
  \pstGeonode[PointSymbol=Btriangle,dotscale=2]%
      (-2.51,-0.028){C}
    \psclip{\psframe[](-4,-1.5)(3.5,1.5)}%
    {
    \psset{linestyle=none, PointSymbol=none}
    \pstMediatorAB{A}{B}{K}{KP}
    \pstMediatorAB{C}{A}{J}{JP}
    \pstMediatorAB{B}{C}{I}{IP}
    \pstInterLL[PointSymbol=none]{I}{IP}{K}{KP}{O}
    \psset{linewidth=1pt,linestyle=dashed}
    \pstGeonode[PointSymbol=none](-4,-1.5){a}(-4,1.5){b}(3.5,1.5){c}(3.5,-1.5){d}
    \pstInterLL[PointSymbol=none]{O}{I}{b}{c}{oi}
    \pstLineAB{O}{oi}
    %\pstInterLL[PointSymbol=none]{O}{J}{a}{d}{oj}
    %\pstLineAB{O}{oj}
    \pstInterLL[PointSymbol=none]{O}{K}{b}{c}{ok}
    \pstLineAB{O}{ok}
    }
    \endpsclip
    \endpspicture
  }}
\end{figure}
\end{frame}


\begin{frame}[fragile]{Gap Statistic: Intracluster Weights and Gap
  Values}
\begin{figure}
%\begin{figure}[!t]%fig17.5
    \centering
    \def\pshlabel#1{ {\footnotesize $#1$}}
    \def\psvlabel#1{ {\footnotesize $#1$}}
\captionsetup[subfloat]{captionskip=0.4in}
\centering \psset{dotscale=2,fillcolor=lightgray} 
\centerline{\hskip30pt\scalebox{0.85}{ \subfloat[Intracluster
weights]{ \label{fig:clust:eval:WK}
\psset{xAxisLabel=$k$,yAxisLabel=$\log_2 W_{in}^k$,%
xAxisLabelPos={c,-1},yAxisLabelPos={-2,c}}
\pslegend[rt]{\rule[1ex]{2em}{1pt}%
\psdot[dotstyle=Btriangle,dotscale=1.5](-0.6,0.2) & expected:
$\mu_W(k)$\\
\rule[1ex]{2em}{1pt}%
\psdot[dotstyle=Bo,dotscale=1.5](-0.6,0.2) & observed: $W_{in}^k$}
\begin{psgraph}[Dy=1,Oy=10]{->}(0,0)(10,6){2.2in}{2in}
    \pstScalePoints(1,1){}{-10 add}
    \listplot[showpoints=true, dotstyle=Bo,
    plotNoMax=6, plotNo=4]{\dataG}
    \listplot[showpoints=true, dotstyle=Btriangle,
    plotNoMax=6, plotNo=5]{\dataG}
\end{psgraph}
} \hspace{0.4in} \subfloat[Gap statistic]{
\label{fig:clust:eval:gap}
\psset{xAxisLabel=$k$,yAxisLabel=$gap(k)$,%
xAxisLabelPos={c,-0.15},yAxisLabelPos={-2,c}}
\begin{psgraph}[Dy=0.1]{->}(0,0)(10,1){2.2in}{2in}
    \listplot[showpoints=true, dotstyle=Bsquare, dotscale=1.5,
    plotNoMax=6, plotNo=1]{\dataG}
    \def\DoCoordinate#1#2{}\GetCoordinates{\dataGE}
\end{psgraph}
} }} 
\end{figure}
\end{frame}


\begin{frame}{Gap Statistic as a Function of $\textit{k}$}
% {\tabcolsep10pt
\renewcommand{\arraystretch}{1.1} 
\begin{center}
\begin{tabular}{|c|ccc|}
    \hline
$k$ &  $gap(k)$ & $\sigma_W(k)$ & $gap(k)-\sigma_W(k)$\\
\hline
1 &  0.093 &0.0456 &0.047\\
2 &  0.346 &0.0486 &0.297\\
3 &  0.679 &0.0529 &0.626\\
4 &  0.753 &0.0701 &0.682\\
5 &  0.586 &0.0711 &0.515\\
6 &  0.715 &0.0654 &0.650\\
7 &  0.808 &0.0611 &0.746\\
8 &  0.680 &0.0597 &0.620\\
9 &  0.632 &0.0606 &0.571\\
\hline
  \end{tabular}%}
\end{center}
The optimal value for the number of clusters is
$k=4$ because
$$gap(4) = 0.753 > gap(5)-\sigma_W(5) = 0.515$$

\medskip
However, if we
relax the gap test to be within two standard deviations, then the
optimal value is $k=3$ because
$$gap(3) = 0.679 >
gap(4)-2\sigma_W(4) = 0.753-2\cdot0.0701 = 0.613$$
\end{frame}



\begin{frame}{Cluster Stability}
The main idea behind cluster stability is that the clusterings obtained
from several datasets sampled from the same underlying distribution as
$\bD$ should be similar or ``stable.''

\medskip
Stability can be
used to f\/{i}nd  a good value for
$k$, the correct number of clusters.

\medskip
We generate $t$ samples of size $n$ by sampling from $\bD$
with replacement.
Let $\cC_k(\bD_i)$ denote the
clustering obtained from sample $\bD_i$, for a given value of $k$. 

\medskip
Next,
we compare the distance between all pairs of clusterings
$\cC_k(\bD_i)$ and $\cC_k(\bD_{j})$ using several 
of the external cluster evaluation measures.
From these values we compute the
expected pairwise distance for each value of $k$.  F{i}nally, the value
$k^*$ that exhibits the least deviation between the clusterings obtained
from the resampled datasets is the best choice for $k$ because it
exhibits the most stability.
\end{frame}



\newcommand{\StabClus}{{\textsc{ClusteringStability}}}
\begin{frame}[fragile]{Clustering Stability Algorithm}
\begin{tightalgo}[H]{\textwidth-18pt}
\SetKwInOut{Algorithm}{\StabClus\ ($A, t, k^{\max}, \bD$)}
\Algorithm{}
$n \gets \card{\bD}$\;
%\tcp{Generate $t$ samples}
\For{$i = 1,2,\ldots,t$}{
  $\bD_i \gets$ sample $n$ points from $\bD$ with replacement\;
}
%\tcp{Generate clusterings for different values of $k$}
\For{$i = 1,2,\ldots,t$}{
  \For{$k =2,3,\ldots,k^{\max}$}{
    $\cC_k(\bD_i) \assign$ cluster $\bD_i$ into $k$ clusters using
    algorithm $A$\;
  }
}
%\tcp{Compute mean difference between clusterings for each $k$}
\ForEach{pair $\bD_i, \bD_{j}$ with $j > i$ }{
  $\bD_{ij} \gets \bD_i \cap \bD_{j}$ \tcp{create common dataset}\;%using Eq.\nosp\eqref{eq:clus:eval:Dij}}\; 
\For{$k = 2,3,\ldots,k^{\max}$}{
  $d_{ij}(k) \gets d\bigl(\cC_k(\bD_i),\cC_k(\bD_{j}),\bD_{ij}\bigr)$ \tcp{distance between clusterings}\;
  }
}
\For{$k = 2,3,\ldots,k^{\max}$}{
  $\mu_d(k) \assign
  \frac{2}{t(t-1)} \sum_{i=1}^t \sum_{j>i} d_{ij}(k)$
  %\nllabel{alg:clust:eval:clusterstability:mud} \tcp{expected pairwise
  %distance}\;
}
%\tcp{Choose best $k$}
$k^* \gets \arg\min_k \bigl\{ \mu_d(k) \bigr\}$
\end{tightalgo}
\end{frame}


\readdata{\dataCS}{CLUST/eval/figs/stabilityvals.txt}
\begin{frame}[fragile]{Clustering Stability: Iris Data}
  \framesubtitle{$t=500$ bootstrap samples; best K-means from 100 runs}
\begin{figure}[H]
    \centering
    \def\pshlabel#1{ {\footnotesize #1}}
    \def\psvlabel#1{ {\footnotesize #1}}
\psset{dotscale=2,fillcolor=lightgray,arrowscale=2}
\centerline{
\begin{pspicture}(-3,-1)(3,5)
\psset{xAxisLabel=$k$,yAxisLabel=Expected Value,%
xAxisLabelPos={c,-0.15},yAxisLabelPos={-1.5,c}}
\pslegend[rb]{\rule[1ex]{2em}{1pt}%
\psdot[dotstyle=Bo,dotscale=1.5](-0.5,0.03) & $\mu_s(k): \mathit{FM}$\\
\rule[1ex]{2em}{1pt}%
\psdot[dotstyle=Btriangle,dotscale=1.5](-0.5,0.03) & $\mu_d(k): VI$}
\begin{psgraph}[Dy=0.1]{->}(0,0)(10,1){3in}{2in}
    \listplot[showpoints=true, dotstyle=Bo,
    plotNoMax=4, plotNo=3]{\dataCS}
    \listplot[showpoints=true, dotstyle=Btriangle,
    plotNoMax=4, plotNo=1]{\dataCS}
\end{psgraph}
\end{pspicture}
}
\end{figure}
The best choice is $k=2$.
\end{frame}



\begin{frame}{Clustering Tendency: Spatial Histogram}

Clustering tendency or clusterability
aims to determine whether the dataset $\bD$ has any
meaningful groups to begin with. 

\medskip
Let $X_1, X_2, \ldots, X_d$ denote the $d$ dimensions. Given
$b$, the number of bins for each dimension, we divide each dimension
$X_{j}$ into $b$ equi-width bins, and simply count how many points lie in
each of the $b^d$ $d$-dimensional cells.  

\medskip
From this spatial histogram, we
can obtain the empirical joint probability mass function (EPMF) for the
dataset $\bD$
\begin{align*}
  f(\bi) = P(\bx_{j} \in \text{cell }\bi) =
  \frac{\bigl|\{\bx_{j} \in \text{cell }\bi\}\bigr|}{n}
\end{align*}
where $\bi = (i_1,i_2,\ldots,i_d)$ denotes a cell index, with $i_{j}$
denoting
the bin index along dimension $X_{j}$.
\end{frame}


\begin{frame}{Clustering Tendency: Spatial Histogram}
We generate $t$ random samples, each comprising $n$ points within
the same $d$-dimensional space as the input dataset $\bD$.  
Let $\bR_{j}$
denote the $j$th such random sample. 
We then compute the
corresponding EPMF $g_{j}(\bi)$ for each $\bR_{j}$, $1\le j\le t$.

\medskip
We next compute how much the distribution $f$ differs from
$g_{j}$ (for $j=1,\ldots,t$),
using the Kullback--Leibler (KL) divergence from $f$ to
$g_{j}$, def\/{i}ned as
\index{Kullback-Leibler divergence}
\index{KL divergence|see{Kullback-Leibler divergence}}
\begin{align*}
  \mathit{KL}(f|g_{j}) = \sum_{\bi} f(\bi) \log \lB(\frac{f(\bi)}{g_{j}(\bi)} \rB)
\end{align*}
The KL divergence is zero only when $f$ and $g_{j}$ are the same
distributions. Using these divergence values, we can compute how much
the dataset $\bD$ differs from a random dataset.
\end{frame}



\readdata{\dataHtwoD}{CLUST/eval/figs/tendency2dhist.txt}
\readdata{\dataKLtwoD}{CLUST/eval/figs/kl2d.txt}
\begin{frame}[fragile]{Spatial Histogram: Iris Data versus Uniform}
\setcounter{subfigure}{0}
\begin{figure}
%\begin{figure}[!p]\vspace*{12pt}%fig17.7
  \centering
  \captionsetup[subfloat]{captionskip=35pt}
    \psset{xAxisLabel=$\bu_1$,yAxisLabel= $\bu_2$}
    \def\pshlabel#1{ {\footnotesize $#1$}}
    \def\psvlabel#1{ {\footnotesize $#1$}}
    \centerline{
        \psset{xunit=0.3in,yunit=0.6in,
          dotscale=1.5,dotstyle=Bo,fillcolor=lightgray,
          arrowscale=2}
    \subfloat[Iris: spatial cells]{
    \label{fig:clust:eval:IrisSpatialCells}
    \scalebox{0.8}{%
        %\pspicture[](-3,-1.25)(3,1.5)
        %\psaxes[tickstyle=bottom,Dx=1,Ox=-4,Dy=0.5,Oy=-1.5]{->}%
        %  (-4,-1.5)(3.5,1.5)
        \psgraph[tickstyle=bottom,Dx=1,Ox=-4,Dy=0.5,Oy=-1.5]{->}%
        (-4,-1.5)(3.5,1.5){2.5in}{2in}
        \input{CLUST/eval/irisPCgood-C1}
        \input{CLUST/eval/irisPCgood-W1}
        \input{CLUST/eval/irisPCgood-C2}
        \input{CLUST/eval/irisPCgood-W2}
        \input{CLUST/eval/irisPCgood-C3}
        \psdot[](-1.46,0.50)
\psdot[](-1.56,0.27)
\psdot[](-1.28,0.69)

        \psset{linestyle=dashed}
        \multido{\ni=-3.795+1.403}{6}{%
        \psline(\ni,-1.262)(\ni,1.371)}
        \multido{\ni=-1.262+0.527}{6}{%
        \psline(-3.795,\ni)(3.225,\ni)}
        \endpsgraph
        }}
    \hspace{0.5in}
  \subfloat[Uniform: spatial cells]{
  \label{fig:clust:eval:RandomSpatialCells}
  \scalebox{0.8}{%
  %\pspicture[](-3,-1.25)(3,1.5)
  %\psaxes[tickstyle=bottom,Dx=1,Ox=-4,Dy=0.5,Oy=-1.5]{->}%
%   (-4,-1.5)(3.5,1.5)
  \psgraph[tickstyle=bottom,Dx=1,Ox=-4,Dy=0.5,Oy=-1.5]{->}%
  (-4,-1.5)(3.5,1.5){2.5in}{2in}
  \psdot[](2.17,-0.72)
\psdot[](2.29,-1.22)
\psdot[](2.33,-0.19)
\psdot[](2.50,-0.01)
\psdot[](1.24,0.86)
\psdot[](1.56,-0.15)
\psdot[](3.04,0.22)
\psdot[](2.21,0.98)
\psdot[](1.30,-0.51)
\psdot[](2.06,-0.57)
\psdot[](2.16,0.17)
\psdot[](2.24,0.87)
\psdot[](1.06,-0.62)
\psdot[](1.85,0.35)
\psdot[](2.24,-1.15)
\psdot[](1.10,1.19)
\psdot[](2.34,-1.07)
\psdot[](3.13,1.30)
\psdot[](1.04,-1.01)
\psdot[](1.20,0.78)
\psdot[](1.15,-0.01)
\psdot[](2.34,0.01)
\psdot[](2.06,-1.03)
\psdot[](1.00,-0.20)
\psdot[](1.61,-0.20)
\psdot[](2.87,0.94)
\psdot[](2.26,-1.18)
\psdot[](2.82,-0.49)
\psdot[](1.25,-0.20)
\psdot[](2.64,-1.11)
\psdot[](1.15,0.32)
\psdot[](1.70,1.25)
\psdot[](2.02,-0.65)
\psdot[](1.14,0.26)
\psdot[](2.41,0.37)
\psdot[](2.14,-0.25)
\psdot[](2.06,0.34)
\psdot[](1.56,1.09)
\psdot[](2.87,0.33)

  \psdot[](2.17,-0.72)
\psdot[](2.29,-1.22)
\psdot[](2.33,-0.19)
\psdot[](2.50,-0.01)
\psdot[](1.24,0.86)
\psdot[](1.56,-0.15)
\psdot[](3.04,0.22)
\psdot[](2.21,0.98)
\psdot[](1.30,-0.51)
\psdot[](2.06,-0.57)
\psdot[](2.16,0.17)
\psdot[](2.24,0.87)
\psdot[](1.06,-0.62)
\psdot[](1.85,0.35)
\psdot[](2.24,-1.15)
\psdot[](1.10,1.19)
\psdot[](2.34,-1.07)
\psdot[](3.13,1.30)
\psdot[](1.04,-1.01)
\psdot[](1.20,0.78)
\psdot[](1.15,-0.01)
\psdot[](2.34,0.01)
\psdot[](2.06,-1.03)
\psdot[](1.00,-0.20)
\psdot[](1.61,-0.20)
\psdot[](2.87,0.94)
\psdot[](2.26,-1.18)
\psdot[](2.82,-0.49)
\psdot[](1.25,-0.20)
\psdot[](2.64,-1.11)
\psdot[](1.15,0.32)
\psdot[](1.70,1.25)
\psdot[](2.02,-0.65)
\psdot[](1.14,0.26)
\psdot[](2.41,0.37)
\psdot[](2.14,-0.25)
\psdot[](2.06,0.34)
\psdot[](1.56,1.09)
\psdot[](2.87,0.33)

  \psdot[](0.33,0.09)
\psdot[](-1.13,-0.73)
\psdot[](-1.03,0.39)
\psdot[](-0.88,-0.24)
\psdot[](-0.81,1.14)
\psdot[](0.12,1.31)
\psdot[](0.36,-0.80)
\psdot[](0.51,-1.23)
\psdot[](-1.03,1.09)
\psdot[](-0.36,-0.64)
\psdot[](-0.80,0.88)
\psdot[](0.75,-0.47)
\psdot[](0.28,-0.65)
\psdot[](0.72,-1.18)
\psdot[](0.40,0.95)
\psdot[](0.29,0.28)
\psdot[](-1.07,-0.14)
\psdot[](0.52,0.90)
\psdot[](0.93,0.87)
\psdot[](-0.48,0.92)
\psdot[](-0.35,0.15)
\psdot[](0.34,-0.99)
\psdot[](-0.05,-1.13)
\psdot[](0.21,0.32)
\psdot[](-1.18,0.23)
\psdot[](0.20,-1.01)
\psdot[](-1.05,0.68)
\psdot[](0.32,0.26)
\psdot[](-1.05,1.35)
\psdot[](0.30,0.32)
\psdot[](0.69,0.36)
\psdot[](0.72,0.48)
\psdot[](0.65,1.31)
\psdot[](0.74,0.47)
\psdot[](-1.12,0.91)
\psdot[](0.07,-0.25)
\psdot[](0.40,0.17)
\psdot[](-0.54,-0.85)
\psdot[](-0.31,1.20)
\psdot[](0.49,0.27)
\psdot[](0.32,-0.38)
\psdot[](-0.14,0.06)
\psdot[](-0.57,-1.20)
\psdot[](0.61,-0.35)
\psdot[](-0.91,-0.47)
\psdot[](-0.92,-0.19)
\psdot[](0.38,-0.88)
\psdot[](-0.16,1.03)
\psdot[](0.85,0.26)
\psdot[](0.55,0.42)
\psdot[](-0.69,1.15)
\psdot[](-0.76,0.03)
\psdot[](0.18,0.71)
\psdot[](0.04,-0.49)
\psdot[](0.47,0.09)
\psdot[](-0.94,0.97)
\psdot[](0.15,0.85)

  \psdot[](-1.35,-1.11)
\psdot[](-2.14,1.19)
\psdot[](-2.57,0.85)
\psdot[](-2.56,1.30)
\psdot[](-2.45,0.31)
\psdot[](-2.22,-0.72)
\psdot[](-1.92,-1.01)
\psdot[](-3.66,-0.29)
\psdot[](-3.43,-0.51)
\psdot[](-3.29,-0.01)
\psdot[](-2.94,-0.06)
\psdot[](-2.16,-0.02)
\psdot[](-3.12,0.05)
\psdot[](-3.20,-0.60)
\psdot[](-3.71,-0.73)
\psdot[](-3.03,0.48)
\psdot[](-1.42,-0.40)
\psdot[](-1.33,-0.92)
\psdot[](-3.63,0.22)
\psdot[](-1.34,-0.19)
\psdot[](-3.50,-0.55)
\psdot[](-3.79,1.06)
\psdot[](-2.46,0.03)
\psdot[](-2.17,-0.57)
\psdot[](-2.33,-0.17)
\psdot[](-3.50,-1.23)
\psdot[](-2.64,0.36)
\psdot[](-3.21,1.12)
\psdot[](-1.66,0.36)
\psdot[](-2.68,0.46)
\psdot[](-3.52,0.44)
\psdot[](-3.04,0.75)
\psdot[](-3.05,0.31)
\psdot[](-2.13,-0.98)
\psdot[](-1.41,0.59)
\psdot[](-2.30,-0.32)
\psdot[](-2.98,-0.46)
\psdot[](-2.63,-0.04)
\psdot[](-3.25,-1.00)
\psdot[](-2.60,-0.30)
\psdot[](-3.44,-0.59)
\psdot[](-1.23,-0.95)
\psdot[](-1.92,-1.06)
\psdot[](-1.51,-0.60)
\psdot[](-3.05,1.23)
\psdot[](-1.97,-0.12)
\psdot[](-1.34,0.23)
\psdot[](-1.44,0.01)
\psdot[](-2.42,1.09)
\psdot[](-2.23,0.55)
\psdot[](-2.05,-0.88)
\psdot[](-2.35,0.68)
\psdot[](-2.53,0.29)
\psdot[](-1.55,0.96)

        \psset{linestyle=dashed}
        \multido{\ni=-3.795+1.403}{6}{%
        \psline(\ni,-1.262)(\ni,1.371)}
        \multido{\ni=-1.262+0.527}{6}{%
        \psline(-3.795,\ni)(3.225,\ni)}
    \endpsgraph
    }}}
\end{figure}
\end{frame}

\begin{frame}[fragile]{Spatial Histogram: Empirical PMF}
\begin{figure}
  \centering
  \captionsetup[subfloat]{captionskip=35pt}
    \psset{xAxisLabel=$\bu_1$,yAxisLabel= $\bu_2$}
    \def\pshlabel#1{ {\footnotesize $#1$}}
    \def\psvlabel#1{ {\footnotesize $#1$}}
  \centerline{
  \psset{arrowscale=2}
  \subfloat[Empirical probability mass function]{%
\def\pshlabel#1{\scriptsize {#1}}
\def\psvlabel#1{\scriptsize {#1}}
  \label{fig:clust:eval:irisSpatialdistr}
  \psset{xAxisLabel=Spatial Cells,yAxisLabel=Probability,%
  xAxisLabelPos={c,-0.04},yAxisLabelPos={-3.5,c}}
  \pslegend[rt]{%
  \psline[linewidth=3pt,linecolor=darkgray](0,0)(1.15,0) & $\qquad$Iris
  ($f$)\\
  \psline[linewidth=3pt,linecolor=lightgray](0,0)(1.15,0) &
  $\qquad$Uniform ($g_{j}$)}
  \begin{psgraph}[Dy=0.02,showorigin=true]{->}(0,0)(25,0.2){3.5in}{1.75in}
    \listplot[plotstyle=bar,barwidth=0.1cm,
          fillcolor=darkgray,fillstyle=solid,
          plotNoMax=2,plotNo=1]{\dataHtwoD}
          \pstScalePoints(1,1){0.25 add}{}
          \listplot[plotstyle=bar,barwidth=0.1cm,
          fillcolor=lightgray,fillstyle=solid,
          plotNoMax=2,plotNo=2]{\dataHtwoD}
          \listplot[showpoints=false,linestyle=dashed,
          plotNoMax=2,plotNo=2]{\dataHtwoD}
          \pstScalePoints(1,1){}{}
          \listplot[showpoints=false,linewidth=1.5pt,
          plotNoMax=2,plotNo=1]{\dataHtwoD}
  \end{psgraph}
  }}
\end{figure}
\end{frame}

\begin{frame}[fragile]{Spatial Histogram: KL Divergence Distribution}
\begin{figure}
  \centering
  \captionsetup[subfloat]{captionskip=35pt}
    \psset{xAxisLabel=$\bu_1$,yAxisLabel= $\bu_2$}
    \def\pshlabel#1{ {\footnotesize $#1$}}
    \def\psvlabel#1{ {\footnotesize $#1$}}
  \centerline{
  \psset{arrowscale=2}
    \subfloat[KL-divergence distribution]{
    \label{fig:clust:eval:irisSpatialKL}
    \psset{xAxisLabel=KL Divergence,yAxisLabel=Probability,%
  xAxisLabelPos={c,-0.07},yAxisLabelPos={-0.2,c}}
  \begin{psgraph}[Dy=0.05,Dx=0.15,Ox=0.65,showorigin=true]{->}(0,0)(1.2,0.3){3in}{1.5in}
    \pstScalePoints(1,1){-0.65 add}{}
    \listplot[plotstyle=bar,barwidth=0.2cm,
          fillcolor=lightgray,fillstyle=solid,
          plotNoMax=2,plotNoX=2,plotNo=2]{\dataKLtwoD}
  \end{psgraph}
}}
\end{figure}
  We generated $t=500$ random samples from the null
  distribution, and computed the KL divergence from $f$ to $g_{j}$ for
  each $1 \le j \le t$.

  \smallskip
 The mean KL value is
  $\mu_{\mathit{KL}} = 1.17$, with a standard deviation of $\sigma_{\mathit{KL}}=0.18$.
\end{frame}


\begin{frame}{Clustering Tendency: Distance Distribution}
We can compare the pairwise point
distances from $\bD$, with those from the randomly generated
samples $\bR_i$ from the null distribution. 

We create the EPMF from the proximity matrix $\bW$ for $\bD$ by binning the distances into $b$ bins:

\begin{align*}
  f(i) = P(w_{pq} \in \text{ bin } i \;|\; \bx_p, \bx_q \in \bD, p<q) =
  \frac{\bigl|\{w_{pq} \in \text{ bin } i\}\bigr|}{n(n-1)/2}
\end{align*}
Likewise, for each of the samples $\bR_{j}$, we determine the
EPMF for the pairwise distances, denoted $g_{j}$. 

\medskip
F{i}nally, we 
compute the KL divergences between $f$ and $g_{j}$.
The expected divergence
indicates the extent to which $\bD$ differs from the null (random)
distribution.
\end{frame}


\readdata{\dataHoneD}{CLUST/eval/figs/tendency1dhist.txt}
\readdata{\dataKLoneD}{CLUST/eval/figs/kl1d.txt}
\begin{frame}[fragile]{Iris Data: Distance Distribution}
\setcounter{subfigure}{0}
\begin{figure}
  \centering
  \captionsetup[subfloat]{captionskip=30pt}
    \def\pshlabel#1{ {\footnotesize #1}}
    \def\psvlabel#1{ {\footnotesize #1}}
  \centerline{
  \subfloat[]{
    \label{fig:clust:eval:irisDistdist}
  \scalebox{0.9}{%
  \psset{xAxisLabel=Pairwise distance,yAxisLabel=Probability,%
  xAxisLabelPos={c,-0.015},yAxisLabelPos={-0.9,c}}
  \pslegend[rt]{%
  \psline[linewidth=3pt,linecolor=darkgray](0,0)(0.3,0) & $\qquad$Iris
  ($f$)\\
  \psline[linewidth=3pt,linecolor=lightgray](0,0)(0.3,0) &
  $\qquad$Uniform ($g_{j}$)}
  \begin{psgraph}[Dy=0.01]{->}(0,0)(7,0.11){3.75in}{2in}
    \listplot[plotstyle=bar,barwidth=0.1cm,
          fillcolor=darkgray,fillstyle=solid,
          plotNoMax=3,plotNoX=2,plotNo=2]{\dataHoneD}
          \pstScalePoints(1,1){0.09 add}{}
          \listplot[plotstyle=bar,barwidth=0.1cm,
          fillcolor=lightgray,fillstyle=solid,
          plotNoMax=3,plotNoX=2,plotNo=3]{\dataHoneD}
          \listplot[showpoints=false,linestyle=dashed,
          plotNoMax=3,plotNoX=2,plotNo=3]{\dataHoneD}
          \pstScalePoints(1,1){}{}
          \listplot[showpoints=false,linewidth=1.5pt,
          plotNoMax=3,plotNoX=2,plotNo=2]{\dataHoneD}
  \end{psgraph}
  }}}
\end{figure}
\end{frame}

\begin{frame}[fragile]{Iris Data: Distance Distribution}
\begin{figure}
  \centering
  \captionsetup[subfloat]{captionskip=30pt}
    \def\pshlabel#1{ {\footnotesize #1}}
    \def\psvlabel#1{ {\footnotesize #1}}
    \centerline{
    \subfloat[]{
    \label{fig:clust:eval:randomKL}
  \scalebox{0.9}{%
  \psset{xAxisLabel=KL divergence,yAxisLabel=Probability,%
  xAxisLabelPos={c,-0.0475},yAxisLabelPos={-0.02,c}}
  \begin{psgraph}[Dy=0.05,Dx=0.02,Ox=0.12]{->}(0,0)(0.12,0.25){3.5in}{1.75in}
    \pstScalePoints(1,1){-0.12 add}{}
    \listplot[plotstyle=bar,barwidth=0.2cm,
          fillcolor=lightgray,fillstyle=solid,
          plotNoMax=2,plotNoX=2,plotNo=2]{\dataKLoneD}
  \end{psgraph}
  }}}
\end{figure}
\end{frame}



\begin{frame}{Clustering Tendency: Hopkins Statistic}
Given a dataset $\bD$ comprising $n$ points,
we generate $t$ uniform subsamples $\bR_i$ of $m$ points each, 
sampled from the same dataspace as $\bD$.

\medskip
We also generate $t$ subsamples of $m$ points
directly from $\bD$, using sampling without replacement. Let $\bD_i$
denote the $i$th direct subsample. 

\medskip
Next, we compute the minimum distance
between each point $\bx_{j} \in \bD_i$ and points in $\bD$
\begin{align*}
  \dist_{\min}(\bx_{j}) = \min_{\bx_i \in \bD, \bx_i \ne \bx_{j}}
  \Bigl\{\dist(\bx_{j}, \bx_i)\Bigr\}
\end{align*}
We
also compute the minimum distance $\dist_{\min}(\by_{j})$
between a point $\by_{j} \in \bR_i$ and
points in $\bD$.

\medskip
The Hopkins statistic (in $d$ dimensions) for the $i$th pair of samples $\bR_i$ and $\bD_i$ is then def\/{i}ned as
\begin{align*}
  \mathit{HS}_i = \frac{\sum_{\by_{j} \in \bR_i} \lB(\dist_{\min}(\by_{j})\rB)^d}
  {\sum_{\by_{j} \in \bR_i} \lB(\dist_{\min}(\by_{j})\rB)^d +
  \sum_{\bx_{j} \in \bD_i} \lB(\dist_{\min}(\bx_{j})\rB)^d}
\end{align*}
If the data is well clustered we expect $\dist_{\min}(\bx_{j})$ values to
be smaller compared to the $\dist_{\min}(\by_{j})$ values, and in this
case $\mathit{HS}_i$ tends to 1. 
\end{frame}


\readdata{\dataHS}{CLUST/eval/figs/hopkinsdat.txt}
\psset{arrowscale=2}
\def\pshlabel#1{ {\footnotesize #1}}
\def\psvlabel#1{ {\footnotesize #1}}
\begin{frame}{Iris Data: Hopkins Statistic Distribution}
\begin{figure}
\begin{center}
  \hspace*{-40pt}
    \begin{pspicture}(-3,-1)(3,5)
  \psset{xAxisLabel=Hopkins Statistic,yAxisLabel=Probability,%
  xAxisLabelPos={c,-0.025},yAxisLabelPos={-0.025,c}}
  \begin{psgraph}[Dy=0.05,Dx=0.02,Ox=0.82,showorigin=false]{->}(0,0)(0.18,0.15){4in}{2in}
    \pstScalePoints(1,1){-0.82 add}{}
    \listplot[plotstyle=bar,barwidth=0.1cm,
          fillcolor=lightgray,fillstyle=solid,
          plotNoMax=2,plotNoX=2,plotNo=2]{\dataHS}
  \end{psgraph}
  \end{pspicture}
\end{center}
\end{figure}
Number of sample pairs $t=500$, subsample size $m=30$.
  The mean of the Hopkins statistic is $\mu_{\mathit{HS}} = 0.935$, with a
  standard deviation of $\sigma_{\mathit{HS}}=0.025$. 
\end{frame}
